\chapter{Reference Guide}
\label{rg}


\Subsection*{Abbreviation}

\Subsubsection*{Description}

Abbreviation is a function used by OneLoop and PaVeReduce for generating smaller files when saving results to the hard disk. The
  convention is that a definition like GP \(=\) GluonPropagator should be accompanied by the definition Abbreviation[GluonPropagator]
  \(=\) HoldForm[GP].

See also: { }\${}Abbreviations, OneLoop, PaVeReduce, WriteOut, WriteOutPaVe, GluonPropagator, GluonVertex, QuarkPropagator.

\Subsubsection*{Examples}

\dispSFinmath{
\Muserfunction{GP}[p,\multsp \{\mu ,\multsp a\},\multsp \{\nu ,\multsp b\}]
}

\dispSFoutmath{
\Pi _{ab}^{\mu \nu }(p)
}

\Subsection*{Amplitude}

\Subsubsection*{Description}

Amplitude is a database of Feynman amplitudes. Amplitude["name"] returns the amplitude corresponding to the string "name". A list of all
  defined names is obtained with Amplitude[]. New amplitudes can be added to the file "Amplitude.m". It is strongly recommended to use
  names that reflect the process. The option Gauge \(\rightarrow \) 1 means `t Hooft Feynman gauge; Polarization \(\rightarrow \) 0 gives
  unpolarized OPE-type amplitudes, Polarization \(\rightarrow \) 1 the polarized ones.

\dispSFinmath{
\Mfunction{Options}[\Mvariable{Amplitude}]
}

\dispSFoutmath{
\{\Mvariable{Dimension}\rightarrow D,\Mvariable{Gauge}\rightarrow 1,\Mvariable{QuarkMass}\rightarrow 0,
    \Mvariable{Polarization}\rightarrow 1\}
}

See also:  FeynAmp.

\Subsubsection*{Examples}

\dispSFinmath{
\Muserfunction{Amplitude}[]//\Mfunction{Length}
}

\dispSFoutmath{
206
}

This is the amplitude of a gluon self-energy diagram.

\dispSFinmath{
\Muserfunction{Amplitude}["se1g1"]
}

\dispSFoutmath{
\Pi _{cd}^{\alpha \rho }(p-q)\multsp \Pi _{ef}^{\beta \sigma }(q)\multsp {V^{\nu \rho \sigma }}(-p,\multsp p-q,\multsp q)\multsp
   {V^{\mu \alpha \beta }}(p,\multsp q-p,\multsp -q)\multsp {f_{ace}}\multsp {f_{bdf}}
}

\dispSFinmath{
\Muserfunction{Explicit}[\%]
}

\dispSFoutmath{
\MathBegin{MathArray}{l}
-\frac{1}{{{(p-q)}^2}\multsp {q^2}}
    ({g^{\alpha \rho }}\multsp {g^{\beta \sigma }}\multsp
      (-{g_s}\multsp {p^{\alpha }}\multsp {g^{\beta \mu }}-{g_s}\multsp {q^{\alpha }}\multsp {g^{\beta \mu }}+
        2\multsp {g_s}\multsp {g^{\alpha \mu }}\multsp {p^{\beta }}-{g_s}\multsp {g^{\alpha \mu }}\multsp {q^{\beta }}-
        {g_s}\multsp {g^{\alpha \beta }}\multsp {p^{\mu }}+2\multsp {g_s}\multsp {g^{\alpha \beta }}\multsp {q^{\mu }})\multsp   \\
   \noalign{\vspace{1.02083ex}}
\hspace{4.em} ({g_s}\multsp {p^{\nu }}\multsp {g^{\rho \sigma }}-
      2\multsp {g_s}\multsp {q^{\nu }}\multsp {g^{\rho \sigma }}+{g_s}\multsp {g^{\nu \sigma }}\multsp {p^{\rho }}+
      {g_s}\multsp {g^{\nu \sigma }}\multsp {q^{\rho }}-2\multsp {g_s}\multsp {g^{\nu \rho }}\multsp {p^{\sigma }}+
      {g_s}\multsp {g^{\nu \rho }}\multsp {q^{\sigma }})\multsp {{\delta }_{cd}}\multsp {{\delta }_{ef}}\multsp {f_{ace}}\multsp
    {f_{bdf}})\\
\MathEnd{MathArray}
}

This is the amplitude for graph 6.2 from the paper Z.Phys C {\bfseries 70:}637-654, 1996{\bfseries .}

\dispSFinmath{
\Muserfunction{FeynAmp}[\Mvariable{q1},\Mvariable{q2},\Muserfunction{EpsEvaluate}[
     \Muserfunction{Trick}[\Muserfunction{Explicit}[\Muserfunction{Amplitude}["gg2"]]]]]
}

\dispSFoutmath{
\MathBegin{MathArray}{l}
\int {{\DifferentialD }^D}{q_1}\int {{\DifferentialD }^D}{q_2}(
   \frac{1}{q_{1}^{2}.q_{2}^{2}.q_{2}^{2}.{{({q_2}-p)}^2}.{{({q_1}-{q_2})}^2}.{{({q_1}-p)}^2}}  \\
\noalign{\vspace{1.08333ex}}
   \hspace{2.em} \big(2\multsp \ImaginaryI \multsp {{\epsilon }^{{{\lambda }_1}{{\lambda }_5}\Delta {q_2}}}\multsp
    \big(-{g_s}\multsp {p^{\nu }}\multsp {g^{{{\lambda }_7}{{\lambda }_{10}}}}+
      2\multsp {g_s}\multsp q_{1}^{\nu }\multsp {g^{{{\lambda }_7}{{\lambda }_{10}}}}+
      2\multsp {g_s}\multsp {g^{\nu {{\lambda }_{10}}}}\multsp {p^{{{\lambda }_7}}}-
      {g_s}\multsp {g^{\nu {{\lambda }_{10}}}}\multsp q_{1}^{{{\lambda }_7}}-
      {g_s}\multsp {g^{\nu {{\lambda }_7}}}\multsp {p^{{{\lambda }_{10}}}}-
      {g_s}\multsp {g^{\nu {{\lambda }_7}}}\multsp q_{1}^{{{\lambda }_{10}}}\big)\multsp   \\
\noalign{\vspace{0.697917ex}}
   \hspace{4.em} ({g_s}\multsp {p^{\mu }}\multsp {g^{{{\lambda }_1}{{\lambda }_{11}}}}-
     2\multsp {g_s}\multsp q_{2}^{\mu }\multsp {g^{{{\lambda }_1}{{\lambda }_{11}}}}-
     2\multsp {g_s}\multsp {g^{\mu {{\lambda }_{11}}}}\multsp {p^{{{\lambda }_1}}}+
     {g_s}\multsp {g^{\mu {{\lambda }_{11}}}}\multsp q_{2}^{{{\lambda }_1}}+
     {g_s}\multsp {g^{\mu {{\lambda }_1}}}\multsp {p^{{{\lambda }_{11}}}}+
     {g_s}\multsp {g^{\mu {{\lambda }_1}}}\multsp q_{2}^{{{\lambda }_{11}}})\multsp   \\
\noalign{\vspace{0.729167ex}}
   \hspace{4.em} \big(-2\multsp {g_s}\multsp q_{1}^{{{\lambda }_5}}\multsp {g^{{{\lambda }_7}{{\lambda }_{12}}}}+
     {g_s}\multsp q_{2}^{{{\lambda }_5}}\multsp {g^{{{\lambda }_7}{{\lambda }_{12}}}}+
     {g_s}\multsp {g^{{{\lambda }_5}{{\lambda }_{12}}}}\multsp q_{1}^{{{\lambda }_7}}-
     2\multsp {g_s}\multsp {g^{{{\lambda }_5}{{\lambda }_{12}}}}\multsp q_{2}^{{{\lambda }_7}}+
     {g_s}\multsp {g^{{{\lambda }_5}{{\lambda }_7}}}\multsp q_{1}^{{{\lambda }_{12}}}+
     {g_s}\multsp {g^{{{\lambda }_5}{{\lambda }_7}}}\multsp q_{2}^{{{\lambda }_{12}}}\big)\multsp   \\
\noalign{\vspace{0.729167ex}}
   \hspace{4.em} \big(-{g_s}\multsp {p^{{{\lambda }_{10}}}}\multsp {g^{{{\lambda }_{11}}{{\lambda }_{12}}}}-
    {g_s}\multsp q_{1}^{{{\lambda }_{10}}}\multsp {g^{{{\lambda }_{11}}{{\lambda }_{12}}}}+
    2\multsp {g_s}\multsp q_{2}^{{{\lambda }_{10}}}\multsp {g^{{{\lambda }_{11}}{{\lambda }_{12}}}}-
    {g_s}\multsp {g^{{{\lambda }_{10}}{{\lambda }_{12}}}}\multsp {p^{{{\lambda }_{11}}}}+
    2\multsp {g_s}\multsp {g^{{{\lambda }_{10}}{{\lambda }_{12}}}}\multsp q_{1}^{{{\lambda }_{11}}}-
    {g_s}\multsp {g^{{{\lambda }_{10}}{{\lambda }_{12}}}}\multsp q_{2}^{{{\lambda }_{11}}}+2\multsp {g_s}\multsp   \\
   \noalign{\vspace{0.729167ex}}
\hspace{7.em} {g^{{{\lambda }_{10}}{{\lambda }_{11}}}}\multsp {p^{{{\lambda }_{12}}}}-
        {g_s}\multsp {g^{{{\lambda }_{10}}{{\lambda }_{11}}}}\multsp q_{1}^{{{\lambda }_{12}}}-
        {g_s}\multsp {g^{{{\lambda }_{10}}{{\lambda }_{11}}}}\multsp q_{2}^{{{\lambda }_{12}}}\big)\multsp (1-{{(-1)}^m})\multsp
      {{(\Delta \cdot {q_2})}^{m-1}}\multsp {f_{a{c_5}{c_{11}}}}\multsp {f_{b{c_7}{c_{10}}}}\multsp {f_{{c_5}{c_7}{c_{12}}}}\multsp
      {f_{{c_{10}}{c_{11}}{c_{12}}}}\big))\\
\MathEnd{MathArray}
}

\Subsection*{Amputate}

\Subsubsection*{Description}

Amputate[exp,q1,q2,...] amputates Eps and DiracGamma. Amputate[exp,q1,q2,Pair\(\rightarrow \)\{p\}] amputates also p.q1 and p.q2;
  Pair\(\rightarrow \)All amputates all except OPEDelta.

\dispSFinmath{
\Mfunction{Options}[\Mvariable{Amputate}]
}

\dispSFoutmath{
\{\Mvariable{Dimension}\rightarrow D,\Mvariable{Pair}\rightarrow \{\},\Mvariable{Unique}\rightarrow \Mvariable{True}\}
}

See also: { }DiracGamma, DiracMatrix, DiracSimplify, DiracSlash, DiracTrick.

\Subsubsection*{Examples}

\dispSFinmath{
\Muserfunction{DiracSlash}[p].\Muserfunction{DiracSlash}[q]
}

\dispSFoutmath{
(\gamma \cdot p).(\gamma \cdot q)
}

\dispSFinmath{
\Muserfunction{Amputate}[\%,q]
}

\dispSFoutmath{
(\gamma \cdot p).{{\gamma }^{\$AL\$27(1)}}\multsp {q^{\$AL\$27(1)}}
}

\Subsection*{AnomalousDimension}

\Subsubsection*{Description}

AnomalousDimension[name] is a database of anomalous dimensions of twist 2 operators.

\dispSFinmath{
\Mfunction{Options}[\Mvariable{AnomalousDimension}]
}

\dispSFoutmath{
\{\Mvariable{Polarization}\rightarrow 1,\Mvariable{Simplify}\rightarrow \Mvariable{FullSimplify}\}
}

See also: SplittingFunction, SumS, SumT.

\Subsubsection*{Examples}

Polarized case

\dispSFinmath{
\Mfunction{SetOptions}[\multsp \Mvariable{AnomalousDimension},\Mvariable{Polarization}\rightarrow 1]
}

\dispSFoutmath{
\{\Mvariable{Polarization}\rightarrow 1,\Mvariable{Simplify}\rightarrow \Mvariable{FullSimplify}\}
}

\(\gamma _{\Mvariable{NS},\Mvariable{qq}\multsp }^{(0)\multsp }\)polarized

\dispSFinmath{
\Muserfunction{AnomalousDimension}[\Mvariable{gnsqq0}]
}

\dispSFoutmath{
{C_F}\multsp \Big(8\multsp {S_1}(m-1)+\frac{4}{m}+\frac{4}{m+1}-6\Big)
}

\(\gamma _{S,\Mvariable{qg}\multsp }^{(0)\multsp }\)polarized

\dispSFinmath{
\Muserfunction{AnomalousDimension}[\Mvariable{gsqg0}]
}

\dispSFoutmath{
\Big(\frac{8}{m}-\frac{16}{m+1}\Big)\multsp {T_f}
}

\(\gamma _{S,\Mvariable{gq}\multsp }^{(0)\multsp }\)polarized

\dispSFinmath{
\Muserfunction{AnomalousDimension}[\Mvariable{gsgq0}]
}

\dispSFoutmath{
{C_F}\multsp \Big(\frac{4}{m+1}-\frac{8}{m}\Big)
}

\(\gamma _{S,\Mvariable{gg}\multsp }^{(0)\multsp }\)polarized

\dispSFinmath{
\Muserfunction{AnomalousDimension}[\Mvariable{gsgg0}]
}

\dispSFoutmath{
\frac{8\multsp {T_f}}{3}+{C_A}\multsp \Big(8\multsp {S_1}(m-1)-\frac{8}{m}+\frac{16}{m+1}-\frac{22}{3}\Big)
}

\(\gamma _{\Mvariable{PS},\Mvariable{qq}\multsp }^{(0)\multsp }\)polarized

\dispSFinmath{
\Muserfunction{AnomalousDimension}[\Mvariable{gpsqq1}]
}

\dispSFoutmath{
16\multsp {C_F}\multsp \bigg(\frac{1}{m+1}+\frac{3}{{{(m+1)}^2}}+\frac{2}{{{(m+1)}^3}}-\frac{1}{m}-\frac{1}{{m^2}}+\frac{2}{{m^3}}\bigg)
   \multsp {T_f}
}

\(\gamma _{\Mvariable{NS},\Mvariable{qq}\multsp }^{(1)\multsp }\)polarized

\dispSFinmath{
\Muserfunction{AnomalousDimension}[\Mvariable{gnsqq1}]
}

\dispSFoutmath{
\MathBegin{MathArray}{l}
-\bigg(\frac{16\multsp {S_1}(m-1)}{{m^2}}+\frac{16\multsp {S_1}(m-1)}{{{(m+1)}^2}}+
     \frac{16\multsp {S_2}(m-1)}{m}+\frac{16\multsp {S_2}(m-1)}{m+1}-  \\
\noalign{\vspace{1.4375ex}}
\hspace{5.em} 24\multsp {S_2}(m-1)+
   32\multsp {S_{12}}(m-1)+32\multsp {S_{21}}(m-1)+\frac{32\multsp {{\left( \overvar{S}{\RawTilde } \right) }_2}(m-1)}{m}+
   \frac{32\multsp {{\left( \overvar{S}{\RawTilde } \right) }_2}(m-1)}{m+1}-  \\
\noalign{\vspace{1.42708ex}}
\hspace{5.em} 32\multsp
        {{\left( \overvar{S}{\RawTilde } \right) }_3}(m-1)+64\multsp {{\left( \overvar{S}{\RawTilde } \right) }_{12}}(m-1)-\frac{40}{m}+
       \frac{40}{m+1}+\frac{16}{{{(m+1)}^2}}+\frac{8}{{m^3}}+\frac{40}{{{(m+1)}^3}}+3\bigg)\multsp C_{F}^{2}-  \\
\noalign{\vspace{
   1.4375ex}}
\hspace{1.em} {N_f}\multsp \bigg(\frac{80}{9}\multsp {S_1}(m-1)-\frac{16}{3}\multsp {S_2}(m-1)-\frac{8}{9\multsp m}+
      \frac{88}{9\multsp (m+1)}-\frac{8}{3\multsp {m^2}}-\frac{8}{3\multsp {{(m+1)}^2}}-\frac{2}{3}\bigg)\multsp {C_F}-  \\
   \noalign{\vspace{1.45833ex}}
\hspace{1.em} {C_A}\multsp
   \bigg(-\frac{536}{9}\multsp {S_1}(m-1)+\frac{88}{3}\multsp {S_2}(m-1)-16\multsp {S_3}(m-1)-
     \frac{16\multsp {{\left( \overvar{S}{\RawTilde } \right) }_2}(m-1)}{m}-
     \frac{16\multsp {{\left( \overvar{S}{\RawTilde } \right) }_2}(m-1)}{m+1}+  \\
\noalign{\vspace{1.6875ex}}
\hspace{4.em} 16\multsp
      {{\left( \overvar{S}{\RawTilde } \right) }_3}(m-1)-32\multsp {{\left( \overvar{S}{\RawTilde } \right) }_{12}}(m-1)+
     \frac{212}{9\multsp m}-\frac{748}{9\multsp (m+1)}+\frac{44}{3\multsp {m^2}}-\frac{4}{3\multsp {{(m+1)}^2}}-\frac{16}{{{(m+1)}^3}}+
     \frac{17}{3}\bigg)\multsp {C_F}\\
\MathEnd{MathArray}
}

\(\gamma _{S,\Mvariable{qg}\multsp }^{(1)\multsp }\)polarized

\dispSFinmath{
\Muserfunction{AnomalousDimension}[\Mvariable{gsqg1}]
}

\dispSFoutmath{
\MathBegin{MathArray}{l}
8\multsp {C_F}\multsp {T_f}\multsp
   \bigg(\frac{2\multsp S_{1}^{2}(m-1)}{m}-\frac{4\multsp S_{1}^{2}(m-1)}{m+1}-\frac{2\multsp {S_2}(m-1)}{m}+  \\
\noalign{\vspace{
   1.33333ex}}
\hspace{4.em} \frac{4\multsp {S_2}(m-1)}{m+1}+\frac{14}{m}-\frac{19}{m+1}-\frac{1}{{m^2}}-\frac{8}{{{(m+1)}^2}}-
      \frac{2}{{m^3}}+\frac{4}{{{(m+1)}^3}}\bigg)+  \\
\noalign{\vspace{1.54167ex}}
\hspace{1.em} 16\multsp {C_A}\multsp {T_f}\multsp
   \bigg(-\frac{S_{1}^{2}(m-1)}{m}+\frac{2\multsp S_{1}^{2}(m-1)}{m+1}-\frac{2\multsp {S_1}(m-1)}{{m^2}}+
     \frac{4\multsp {S_1}(m-1)}{{{(m+1)}^2}}-\frac{{S_2}(m-1)}{m}+\frac{2\multsp {S_2}(m-1)}{m+1}-  \\
\noalign{\vspace{1.60417ex}}
   \hspace{4.em} \frac{2\multsp {{\left( \overvar{S}{\RawTilde } \right) }_2}(m-1)}{m}+
    \frac{4\multsp {{\left( \overvar{S}{\RawTilde } \right) }_2}(m-1)}{m+1}-\frac{4}{m}+\frac{3}{m+1}-\frac{3}{{m^2}}+
    \frac{8}{{{(m+1)}^2}}+\frac{2}{{m^3}}+\frac{12}{{{(m+1)}^3}}\bigg)\\
\MathEnd{MathArray}
}

\(\gamma _{S,\Mvariable{gq}\multsp }^{(1)\multsp }\)polarized

\dispSFinmath{
\Muserfunction{AnomalousDimension}[\Mvariable{gsgq1}]
}

\dispSFoutmath{
\MathBegin{MathArray}{l}
4\multsp \bigg(\frac{4\multsp S_{1}^{2}(m-1)}{m}-\frac{2\multsp S_{1}^{2}(m-1)}{m+1}-
     \frac{8\multsp {S_1}(m-1)}{m}+\frac{2\multsp {S_1}(m-1)}{m+1}+\frac{8\multsp {S_1}(m-1)}{{m^2}}-
     \frac{4\multsp {S_1}(m-1)}{{{(m+1)}^2}}+  \\
\noalign{\vspace{1.52083ex}}
\hspace{4.em} \frac{4\multsp {S_2}(m-1)}{m}-
      \frac{2\multsp {S_2}(m-1)}{m+1}+\frac{15}{m}-\frac{6}{m+1}-\frac{12}{{m^2}}+\frac{3}{{{(m+1)}^2}}+\frac{4}{{m^3}}-
      \frac{2}{{{(m+1)}^3}}\bigg)\multsp C_{F}^{2}+  \\
\noalign{\vspace{1.45833ex}}
\hspace{1.em} 32\multsp {T_f}\multsp
    \bigg(-\frac{2\multsp {S_1}(m-1)}{3\multsp m}+\frac{{S_1}(m-1)}{3\multsp (m+1)}+\frac{7}{9\multsp m}-\frac{2}{9\multsp (m+1)}-
      \frac{2}{3\multsp {m^2}}+\frac{1}{3\multsp {{(m+1)}^2}}\bigg)\multsp {C_F}+  \\
\noalign{\vspace{1.54167ex}}
\hspace{1.em} 8
   \multsp {C_A}\multsp \bigg(-\frac{2\multsp S_{1}^{2}(m-1)}{m}+\frac{S_{1}^{2}(m-1)}{m+1}+\frac{16\multsp {S_1}(m-1)}{3\multsp m}-
     \frac{5\multsp {S_1}(m-1)}{3\multsp (m+1)}+\frac{2\multsp {S_2}(m-1)}{m}-\frac{{S_2}(m-1)}{m+1}+  \\
\noalign{\vspace{
   1.60417ex}}
\hspace{4.em} \frac{4\multsp {{\left( \overvar{S}{\RawTilde } \right) }_2}(m-1)}{m}-
     \frac{2\multsp {{\left( \overvar{S}{\RawTilde } \right) }_2}(m-1)}{m+1}-\frac{56}{9\multsp m}-\frac{20}{9\multsp (m+1)}+
     \frac{28}{3\multsp {m^2}}-\frac{38}{3\multsp {{(m+1)}^2}}-\frac{4}{{m^3}}-\frac{6}{{{(m+1)}^3}}\bigg)\multsp {C_F}\\
   \MathEnd{MathArray}
}

\(\gamma _{S,\Mvariable{gg}\multsp }^{(1)\multsp }\)polarized

\dispSFinmath{
\Muserfunction{AnomalousDimension}[\Mvariable{gsgg1}]
}

\dispSFoutmath{
\MathBegin{MathArray}{l}
4\multsp \bigg(\frac{8\multsp {S_1}(m-1)}{{m^2}}-\frac{16\multsp {S_1}(m-1)}{{{(m+1)}^2}}+
     \frac{134}{9}\multsp {S_1}(m-1)+\frac{8\multsp {S_2}(m-1)}{m}-\frac{16\multsp {S_2}(m-1)}{m+1}+  \\
\noalign{\vspace{1.4375ex}}
   \hspace{4.em} 4\multsp {S_3}(m-1)-8\multsp {S_{12}}(m-1)-8\multsp {S_{21}}(m-1)+
   \frac{8\multsp {{\left( \overvar{S}{\RawTilde } \right) }_2}(m-1)}{m}-
   \frac{16\multsp {{\left( \overvar{S}{\RawTilde } \right) }_2}(m-1)}{m+1}+4\multsp {{\left( \overvar{S}{\RawTilde } \right) }_3}(m-1)-
   \\
\noalign{\vspace{1.42708ex}}
\hspace{4.em} 8\multsp {{\left( \overvar{S}{\RawTilde } \right) }_{12}}(m-1)-\frac{107}{9\multsp m}+
      \frac{241}{9\multsp (m+1)}+\frac{58}{3\multsp {m^2}}-\frac{86}{3\multsp {{(m+1)}^2}}-\frac{8}{{m^3}}-\frac{48}{{{(m+1)}^3}}-
      \frac{16}{3}\bigg)\multsp C_{A}^{2}+  \\
\noalign{\vspace{1.4375ex}}
\hspace{1.em} 32\multsp {T_f}\multsp
    \bigg(-\frac{5}{9}\multsp {S_1}(m-1)+\frac{14}{9\multsp m}-\frac{19}{9\multsp (m+1)}-\frac{1}{3\multsp {m^2}}-
      \frac{1}{3\multsp {{(m+1)}^2}}+\frac{1}{3}\bigg)\multsp {C_A}+  \\
\noalign{\vspace{1.45833ex}}
\hspace{1.em} 8\multsp {C_F}
   \multsp \bigg(-\frac{10}{m+1}+\frac{2}{{{(m+1)}^2}}+\frac{4}{{{(m+1)}^3}}+1+\frac{10}{m}-\frac{10}{{m^2}}+\frac{4}{{m^3}}\bigg)
   \multsp {T_f}\\
\MathEnd{MathArray}
}

\(\gamma _{S,\Mvariable{gg}\multsp }^{(1)\multsp }\)polarized (different representation)

\dispSFinmath{
\Muserfunction{AnomalousDimension}[\Mvariable{GSGG1}]
}

\dispSFoutmath{
\MathBegin{MathArray}{l}
4\multsp \bigg(-\frac{m\multsp (m\multsp (m\multsp (m\multsp (48\multsp m\multsp (m+3)+469)+698)+7)+258)+144}
       {9\multsp {m^3}\multsp {{(m+1)}^3}}+\frac{8\multsp S_{2}^{'}\NoBreak \big(\NoBreak \frac{m}{2}\NoBreak \big)}{m\multsp (m+1)}-  \\
   \noalign{\vspace{1.6875ex}}
\hspace{4.em} S_{3}^{'}\NoBreak \big(\NoBreak \frac{m}{2}\NoBreak \big)+
      \frac{2\multsp \big(m\multsp \big(67\multsp m\multsp {{(m+1)}^2}+144\big)+72\big)\multsp {S_1}(m)}
       {9\multsp {m^2}\multsp {{(m+1)}^2}}-4\multsp S_{2}^{'}\NoBreak \big(\NoBreak \frac{m}{2}\NoBreak \big)\multsp {S_1}(m)+
      8\multsp \overvar{S}{\RawTilde }(m)\bigg)\multsp C_{A}^{2}+  \\
\noalign{\vspace{1.45833ex}}
\hspace{1.em} 32\multsp {T_f}\multsp
    \bigg(\frac{m\multsp (m+1)\multsp (3\multsp m\multsp (m+1)+13)-3}{9\multsp {m^2}\multsp {{(m+1)}^2}}-\frac{5\multsp {S_1}(m)}{9}
     \bigg)\multsp {C_A}+\frac{8\multsp {C_F}\multsp (m\multsp (m\multsp (m\multsp (m\multsp (m\multsp (m+3)+5)+1)-8)+2)+4)\multsp {T_f}}
    {{m^3}\multsp {{(m+1)}^3}}\\
\MathEnd{MathArray}
}

Check that all odd moments give the same for the two representations of \(\gamma _{S,\Mvariable{gg}\multsp }^{(1)\multsp }.\)

\dispSFinmath{
\Mfunction{Table}[\%-\%\%/.\Mvariable{OPEm}\rightarrow \Mvariable{ij},\{\Mvariable{ij},1,17,2\}]
}

\dispSFoutmath{
\{0,0,0,0,0,0,0,0,0\}
}

\Subsection*{AntiCommutator}

\Subsubsection*{Description}

AntiCommutator[x, y] \(=\) c defines the anti-commutator of the non commuting objects x and y.

See also:  Commutator, CommutatorExplicit, DeclareNonCommutative, DotSimplify.

\Subsubsection*{Examples}

This declares {\ttfamily a} and {\ttfamily b} as noncommutative variables.

\dispSFinmath{
\Muserfunction{DeclareNonCommutative}[a,b]
}

\dispSFinmath{
\Muserfunction{AntiCommutator}[a,b]
}

\dispSFoutmath{
\{a,\> b\}
}

\dispSFinmath{
\Muserfunction{CommutatorExplicit}[\%]
}

\dispSFoutmath{
a.b+b.a
}

\dispSFinmath{
\Muserfunction{CommutatorExplicit}[\Muserfunction{AntiCommutator}[a+b,a-2b\multsp ]]
}

\dispSFoutmath{
(a-2\multsp b).(a+b)+(a+b).(a-2\multsp b)
}

\dispSFinmath{
\Muserfunction{DotSimplify}[\Muserfunction{AntiCommutator}[a+b,a-2b\multsp ]]
}

\dispSFoutmath{
2\multsp a.a-a.b-b.a-4\multsp b.b
}

\dispSFinmath{
\Muserfunction{DeclareNonCommutative}\big[c,d,\overvar{c}{\RawTilde },\overvar{d}{\RawTilde }\big]
}

Defining \{c,d\} \(=\) z results in replacements of c.d by z-d.c.

\dispSFinmath{
\Muserfunction{AntiCommutator}[c,d]\multsp =\multsp z
}

\dispSFoutmath{
z
}

\dispSFinmath{
\Muserfunction{DotSimplify}[\multsp d\multsp .\multsp c\multsp .\multsp d\multsp ]
}

\dispSFoutmath{
d\multsp z-d.d.c
}

\dispSFinmath{
\Muserfunction{AntiCommutator}\big[\overvar{d}{\RawTilde },\overvar{c}{\RawTilde }\big]\multsp =\multsp \overvar{z}{\RawTilde }
}

\dispSFoutmath{
\overvar{z}{\RawTilde }
}

\dispSFinmath{
\Muserfunction{DotSimplify}\big[\multsp \overvar{d}{\RawTilde }\multsp .\multsp \overvar{c}{\RawTilde }\multsp .\multsp
    \overvar{d}{\RawTilde }\multsp \big]
}

\dispSFoutmath{
\overvar{d}{\RawTilde }\multsp \overvar{z}{\RawTilde }-\overvar{c}{\RawTilde }.\overvar{d}{\RawTilde }.\overvar{d}{\RawTilde }
}

\dispSFinmath{
\Muserfunction{UnDeclareNonCommutative}\big[a,b,c,d,\overvar{c}{\RawTilde },\overvar{d}{\RawTilde }\big]
}

\dispSFinmath{
\Mfunction{Unset}[\Muserfunction{AntiCommutator}[c,d]]
}

\dispSFinmath{
\Mfunction{Unset}\big[\Muserfunction{AntiCommutator}\big[\overvar{d}{\RawTilde },\overvar{c}{\RawTilde }\big]\big]
}

\Subsection*{AntiQuarkField}

\Subsubsection*{Description}

AntiQuarkField is the name of a fermionic field. AntiQuarkField is just a name with no functional properties. Only typeset rules are
  attached.

See also:  QuantumField, QuarkField.

\Subsubsection*{Examples}

\dispSFinmath{
\Mvariable{AntiQuarkField}
}

\dispSFoutmath{
\overvar{\psi }{\_}
}

\Subsection*{AntiSymmetrize}

\Subsubsection*{Description}

AntiSymmetrize[expr, \{a1, a2, ...\}] antisymmetrizes expr with respect to the variables a1,a2, ...

See also: Symmetrize.

\Subsubsection*{Examples}

\dispSFinmath{
\Muserfunction{AntiSymmetrize}[f[a,b],\{a,b\}]
}

\dispSFoutmath{
\frac{1}{2}\multsp (f(a,b)-f(b,a))
}

\dispSFinmath{
\Muserfunction{AntiSymmetrize}[f[x,y,z],\{x,y,z\}]
}

\dispSFoutmath{
\frac{1}{6}\multsp (f(x,y,z)-f(x,z,y)-f(y,x,z)+f(y,z,x)+f(z,x,y)-f(z,y,x))
}

\Subsection*{Anti5}

\Subsubsection*{Description}

Anti5[exp] anticommutes all \({{\gamma }^5}\)in exp to the right. Anti5[exp, n] anticommutes all \({{\gamma }^{5\multsp }}\)n times to the right.
Anti5[exp, -n] anticommutes all \({{\gamma }^5}\) n times to the left.

The naive \({{\gamma }^5}\)scheme is used.

See also:  DiracOrder, DiracSimplify, DiracTrick.

\Subsubsection*{Examples}

\dispSFinmath{
\Muserfunction{DiracMatrix}[5,\mu ]\multsp
}

\dispSFoutmath{
{{\gamma }^5}{{\gamma }^{\mu }}
}

\dispSFinmath{
\Muserfunction{Anti5}[\%]
}

\dispSFoutmath{
-{{\gamma }^{\mu }}.{{\gamma }^5}
}

\dispSFinmath{
\Muserfunction{Anti5}[\%,-1]
}

\dispSFoutmath{
{{\gamma }^5}.{{\gamma }^{\mu }}
}

\dispSFinmath{
\Muserfunction{DiracMatrix}[5,\alpha ,\beta ,\gamma ,\delta ]
}

\dispSFoutmath{
{{\gamma }^5}{{\gamma }^{\alpha }}{{\gamma }^{\beta }}{{\gamma }^{\gamma }}{{\gamma }^{\delta }}
}

\dispSFinmath{
\Muserfunction{Anti5}[\%,2]
}

\dispSFoutmath{
{{\gamma }^{\alpha }}.{{\gamma }^{\beta }}.{{\gamma }^5}.{{\gamma }^{\gamma }}.{{\gamma }^{\delta }}
}

\dispSFinmath{
\Muserfunction{Anti5}[\%\%,\infty ]
}

\dispSFoutmath{
{{\gamma }^{\alpha }}.{{\gamma }^{\beta }}.{{\gamma }^{\gamma }}.{{\gamma }^{\delta }}.{{\gamma }^5}
}

\dispSFinmath{
\Muserfunction{Anti5}[\%,-\infty ]
}

\dispSFoutmath{
{{\gamma }^5}.{{\gamma }^{\alpha }}.{{\gamma }^{\beta }}.{{\gamma }^{\gamma }}.{{\gamma }^{\delta }}
}

In the naive \({{\gamma }^5}\)- scheme D-dimensional \(\gamma \)-matrices anticommute with \({{\gamma }^5}\).

\dispSFinmath{
\Muserfunction{GAD}[5,\mu ]
}

\dispSFoutmath{
{{\gamma }^5}.{{\gamma }^{\mu }}
}

\dispSFinmath{
\Muserfunction{Anti5}[\%]
}

\dispSFoutmath{
-{{\gamma }^{\mu }}.{{\gamma }^5}
}

\Subsection*{Apart2}

\Subsubsection*{Description}

Apart2[expr] partial fractions FeynAmpDenominators (and FAD's).

See also:  FAD, FeynAmpDenominator.

\Subsubsection*{Examples}

\dispSFinmath{
\Muserfunction{FAD}[\{q,m\},\{q,M\},q-p]
}

\dispSFoutmath{
\frac{1}{([ {{(q-p)}^2} ])\multsp
     ([ {q^2} - {m^2} ])\multsp
     ([ {q^2} - {M^2} ])}
}

\dispSFinmath{
\Muserfunction{Apart2}[\%]
}

\dispSFoutmath{
\frac{\frac{1}{({q^2}-{m^2}).{{(q-p)}^2}}-\frac{1}{({q^2}-{M^2}).{{(q-p)}^2}}}{{m^2}-{M^2}}
}

\dispSFinmath{
\Mfunction{StandardForm}[\Muserfunction{FCE}[\%]]
}

\dispSFoutmath{
\frac{\Muserfunction{FAD}[\{q,m\},-p+q]-\Muserfunction{FAD}[\{q,M\},-p+q]}{{m^2}-{M^2}}
}

\Subsection*{A0}

\Subsubsection*{Description}

\(\MathBegin{MathArray}{l}
\Muserfunction{A0}[{m^2}]\multsp \Mvariable{is}\Mvariable{the}\Mvariable{Passarino}-  \\
\noalign{\vspace{
   0.666667ex}}
\hspace{1.em} \Mvariable{Veltman}\Mvariable{one}-
   \Mvariable{point}\multsp \multsp \Mvariable{integral}\multsp \multsp {A_{0.}}\\
\MathEnd{MathArray}\)

\dispSFinmath{
\Mfunction{Options}[\multsp \Mvariable{A0}]
}

\dispSFoutmath{
\{\Mvariable{A0ToB0}\rightarrow \Mvariable{False}\}
}

See also:  B0, C0, D0, PaVe.

\Subsubsection*{Examples}

By default \({A_0}\) is not expressed in terms of \({B_0}\).

\dispSFinmath{
\Muserfunction{A0}\big[{m^2}\big]
}

\dispSFoutmath{
{A_0}({m^2})
}

\dispSFinmath{
\Mfunction{SetOptions}[\Mvariable{A0},\Mvariable{A0ToB0}\rightarrow \Mvariable{True}];
}

\dispSFinmath{
\Muserfunction{A0}\big[{m^2}\big]
}

\dispSFoutmath{
{B_0}(0,{m^2},{m^2})\multsp {m^2}+{m^2}
}

\dispSFinmath{
\Mfunction{SetOptions}[\Mvariable{A0},\Mvariable{A0ToB0}\rightarrow \Mvariable{False}];
}

According to the rules of dimensional regularization \({A_0}(0)\) is set to 0.

\dispSFinmath{
\Muserfunction{A0}[0]
}

\dispSFoutmath{
0
}

\dispSFinmath{
\Muserfunction{A0}[\Muserfunction{SmallVariable}[M\RawWedge 2]]
}

\dispSFoutmath{
0
}

\Subsection*{A0ToB0}

\Subsubsection*{Description}

A0ToB0 is an option for A0. If set to True, A0[\({m^2}\)] is expressed by (1 \(+\) B0[0, \({m^2}\), \({m^2}\)]) \({m^2}\).

\dispSFinmath{
\Mfunction{Options}[\multsp \Mvariable{A0}]
}

\dispSFoutmath{
\{\Mvariable{A0ToB0}\rightarrow \Mvariable{False}\}
}

See also:  A0, B0, C0, D0, PaVe.

\Subsection*{BackgroundGluonVertex}

\Subsubsection*{Description}

BackgroundGluonVertex[\{p,mu,a\}, \{q,nu,b\}, \{k,la,c\}] or BackgroundGluonVertex[ p,mu,a , q,nu,b , k,la,c ] yields the 3-gluon vertex
  in the background field gauge, where the first set of arguments corresponds to the external background field.
  BackgroundGluonVertex[\{p,mu,a\}, \{q,nu,b\}, \{k,la,c\}, \{s,si,d\}] or BackgroundGluonVertex[\{mu,a\}, \{nu,b\}, \{la,c\}, \{si,d\}] or BackgroundGluonVertex[p,mu,a
, q,nu,b , k,la,c , s,si,d] or BackgroundGluonVertex[ mu,a , nu,b , la,c , si,d ] yields the
  4-gluon vertex, with \{p,mu,a\} and \{k,la,c\} denoting the external background fields. The gauge, dimension and the name of the
  coupling constant are determined by the options Gauge, Dimension and CouplingConstant. The Feynman rules are taken from L.Abbot {\bfseries NPB
}185 (1981), 189-203; except that all momenta are incoming. Note that Abbots coupling constant convention is consistent with the default
  setting of GluonVertex.

\dispSFinmath{
\Mfunction{Options}[\Mvariable{BackgroundGluonVertex}]
}

\dispSFoutmath{
\{\Mvariable{Dimension}\rightarrow D,\Mvariable{CouplingConstant}\rightarrow {g_s},\Mvariable{Gauge}\rightarrow 1\}
}

See also:  GluonVertex.

\Subsubsection*{Examples}

\dispSFinmath{
\Muserfunction{BackgroundGluonVertex}[\{p,\mu ,a\},\{q,\nu ,b\},\{k,\lambda ,c\}]
}

\dispSFoutmath{
{g_s}\multsp \big({{(-k+p-q)}^{\lambda }}\multsp {g^{\mu \nu }}+{g^{\lambda \nu }}\multsp {{(q-k)}^{\mu }}+
     {g^{\lambda \mu }}\multsp {{(k-p+q)}^{\nu }}\big)\multsp {f_{abc}}
}

\dispSFinmath{
\Muserfunction{BackgroundGluonVertex}[\{p,\mu ,a\},\{q,\nu ,b\},\{k,\lambda ,c\},\{s,\sigma ,d\}]
}

\dispSFoutmath{
\MathBegin{MathArray}{l}
-\ImaginaryI \multsp g_{s}^{2}\multsp
   (({g^{\lambda \sigma }}\multsp {g^{\mu \nu }}-{g^{\lambda \nu }}\multsp {g^{\mu \sigma }}-{g^{\lambda \mu }}\multsp {g^{\nu \sigma }})
      \multsp {f_{ad\Mvariable{u6}}}\multsp {f_{bc\Mvariable{u6}}}+  \\
\noalign{\vspace{0.5625ex}}
\hspace{3.em} (
      {g^{\lambda \sigma }}\multsp {g^{\mu \nu }}-{g^{\lambda \nu }}\multsp {g^{\mu \sigma }})\multsp {f_{ac\Mvariable{u6}}}\multsp
     {f_{bd\Mvariable{u6}}}+({g^{\lambda \sigma }}\multsp {g^{\mu \nu }}-{g^{\lambda \nu }}\multsp {g^{\mu \sigma }}+
       {g^{\lambda \mu }}\multsp {g^{\nu \sigma }})\multsp {f_{ab\Mvariable{u6}}}\multsp {f_{cd\Mvariable{u6}}})\\
\MathEnd{MathArray}
}

\dispSFinmath{
\Muserfunction{BackgroundGluonVertex}[\{p,\mu ,a\},\{q,\nu ,b\},\{k,\lambda ,c\},\Mvariable{Gauge}\rightarrow \alpha ]
}

\dispSFoutmath{
{g_s}\multsp \bigg({{\Big(p-q-\frac{k}{\alpha }\Big)}^{\lambda }}\multsp {g^{\mu \nu }}+{g^{\lambda \nu }}\multsp {{(q-k)}^{\mu }}+
     {g^{\lambda \mu }}\multsp {{\big(k-p+\frac{q}{\alpha }\big)}^{\nu }}\bigg)\multsp {f_{abc}}
}

\dispSFinmath{
\Muserfunction{BackgroundGluonVertex}[\{p,\mu ,a\},\{q,\nu ,b\},\{k,\lambda ,c\},\{s,\sigma ,d\},\Mvariable{Gauge}\rightarrow \alpha ]
}

\dispSFoutmath{
\MathBegin{MathArray}{l}
-\ImaginaryI \multsp g_{s}^{2}\multsp
   \bigg(\bigg({g^{\lambda \sigma }}\multsp {g^{\mu \nu }}-\frac{{g^{\lambda \nu }}\multsp {g^{\mu \sigma }}}{\alpha }-
        {g^{\lambda \mu }}\multsp {g^{\nu \sigma }}\bigg)\multsp {f_{ad\Mvariable{u7}}}\multsp {f_{bc\Mvariable{u7}}}+  \\
   \noalign{\vspace{1.33333ex}}
\hspace{3.em} ({g^{\lambda \sigma }}\multsp {g^{\mu \nu }}-{g^{\lambda \nu }}\multsp {g^{\mu \sigma }})
     \multsp {f_{ac\Mvariable{u7}}}\multsp {f_{bd\Mvariable{u7}}}+
    \bigg(\frac{{g^{\lambda \sigma }}\multsp {g^{\mu \nu }}}{\alpha }-{g^{\lambda \nu }}\multsp {g^{\mu \sigma }}+
       {g^{\lambda \mu }}\multsp {g^{\nu \sigma }}\bigg)\multsp {f_{ab\Mvariable{u7}}}\multsp {f_{cd\Mvariable{u7}}}\bigg)\\
   \MathEnd{MathArray}
}

\Subsection*{Bracket}

\Subsubsection*{Description}

Bracket is an option of Convolute.

See also:  Convolute.

\Subsection*{BReduce}

\Subsubsection*{Description}

BReduce is an option for B0, B00, B1, B11 determining whether reductions to A0 and B0 will be done.

See also:  A0, B0, B00, B1, B11.

\Subsubsection*{Examples}

By default \({B_0}\) is not expressed in terms of \({A_0}\).

\dispSFinmath{
\Muserfunction{B0}[0,s,s]
}

\dispSFoutmath{
{B_0}(0,s,s)
}

\dispSFinmath{
\Muserfunction{B0}//\Mfunction{Options}
}

\dispSFoutmath{
\{\Mvariable{BReduce}\rightarrow \Mvariable{False},\Mvariable{B0Unique}\rightarrow \Mvariable{True},
    \Mvariable{B0Real}\rightarrow \Mvariable{False}\}
}

With BReduce\(\rightarrow \)True, transformation is done.

\dispSFinmath{
\Muserfunction{B0}[0,s,s,\Mvariable{BReduce}\rightarrow \Mvariable{True}]
}

\dispSFoutmath{
\frac{{A_0}(s)}{s}-1
}

\Subsection*{B0}

\Subsubsection*{Description}

B0[pp, ma\(\RawWedge\)2, mb\(\RawWedge\)2] is the Passarino-Veltman two-point integral \({B_0}\). All arguments are scalars and have dimension mass\(\RawWedge\)2.
If the option BReduce is set to True certain B0's are reduced to A0's.
  Setting the option B0Unique to True simplifies B0[a,0,a] and B0[0,0,a].

\dispSFinmath{
\Mfunction{Options}[\Mvariable{B0}]
}

\dispSFoutmath{
\{\Mvariable{BReduce}\rightarrow \Mvariable{False},\Mvariable{B0Unique}\rightarrow \Mvariable{True},
    \Mvariable{B0Real}\rightarrow \Mvariable{False}\}
}

See also:  B1, B00, B11, PaVe.

\Subsubsection*{Examples}

\dispSFinmath{
\Muserfunction{B0}\big[\Muserfunction{SP}[p,p],{m^2},{m^2}\big]
}

\dispSFoutmath{
{B_0}({p^2},{m^2},{m^2})
}

\dispSFinmath{
\Mfunction{SetOptions}[\Mvariable{B0},\Mvariable{B0Unique}\rightarrow \Mvariable{True},\Mvariable{B0Real}\rightarrow \Mvariable{True}];
}

\dispSFinmath{
\Muserfunction{B0}\big[0,0,{m^2}\big]
}

\dispSFoutmath{
{B_0}(0,{m^2},{m^2})+1
}

\dispSFinmath{
\Muserfunction{B0}\big[{m^2},0,{m^2}\big]
}

\dispSFoutmath{
{B_0}(0,{m^2},{m^2})+2
}

\dispSFinmath{
\Muserfunction{B0}[0,m\RawWedge 2,m\RawWedge 2]
}

\dispSFoutmath{
{B_0}(0,{m^2},{m^2})
}

\Subsection*{B0Real}

\Subsubsection*{Description}

B0Real is an option of B0 (default False). If set to True, B0 is assumed to be real and the relation B0[a,0,a] \(=\) 2 \(+\) B0[0,a,a] {
  }is applied.

See also: B0.

\Subsubsection*{Examples}

By default the arguments are not assumed real.

\dispSFinmath{
\Muserfunction{B0}[s,0,s]
}

\dispSFoutmath{
{B_0}(0,s,s)+2
}

\dispSFinmath{
\Muserfunction{B0}//\Mfunction{Options}
}

\dispSFoutmath{
\{\Mvariable{BReduce}\rightarrow \Mvariable{False},\Mvariable{B0Unique}\rightarrow \Mvariable{True},
    \Mvariable{B0Real}\rightarrow \Mvariable{True}\}
}

With B0Real\(\rightarrow \)True, transformation is done.

\dispSFinmath{
\Muserfunction{B0}[s,0,s,\Mvariable{B0Real}\rightarrow \Mvariable{True}]
}

\dispSFoutmath{
{B_0}(0,s,s)+2
}

\Subsection*{B0Unique}

\Subsubsection*{Description}

B0Unique is an option of B0. If set to True, B0[0,0,m2] is replaced with (B0[0,m2,m2]\(+\)1) and B0[m2,0,m2] simplifies to
  (B0[0,m2,m2]\(+\)2).

See also: B0.

\Subsubsection*{Examples}

By default transformation is done.

\dispSFinmath{
\Muserfunction{B0}[0,0,s]
}

\dispSFoutmath{
{B_0}(0,s,s)+1
}

\dispSFinmath{
\Muserfunction{B0}//\Mfunction{Options}
}

\dispSFoutmath{
\{\Mvariable{BReduce}\rightarrow \Mvariable{False},\Mvariable{B0Unique}\rightarrow \Mvariable{True},
    \Mvariable{B0Real}\rightarrow \Mvariable{True}\}
}

With B0Real\(\rightarrow \)False, nothing happens.

\dispSFinmath{
\Muserfunction{B0}[0,0,s,\Mvariable{B0Unique}\rightarrow \Mvariable{False}]
}

\dispSFoutmath{
{B_0}(0,0,s)
}

\Subsection*{B00}

\Subsubsection*{Description}

B00[pp, ma\(\RawWedge\)2,mb\(\RawWedge\)2] is the Passarino-Veltman \({B_0}\)-function, i.e., the coefficient function of the metric tensor. All
arguments are scalars and have dimension mass\(\RawWedge\)2.

\dispSFinmath{
\Mfunction{Options}[\Mvariable{B00}]
}

\dispSFoutmath{
\{\Mvariable{BReduce}\rightarrow \Mvariable{True}\}
}

See also:  B0, B1, PaVe.

\Subsubsection*{Examples}

Remember that SP[p] is a short hand input for ScalarProduct[p,p], i.e., \({p^2}\).

\dispSFinmath{
\Muserfunction{SP}[p]
}

\dispSFoutmath{
{p^2}
}

\dispSFinmath{
\Muserfunction{B00}\big[\Muserfunction{SP}[p],{m^2},{M^2}\big]
}

\dispSFoutmath{
\MathBegin{MathArray}{l}
\frac{1}{3}\multsp {B_0}({p^2},{m^2},{M^2})\multsp {m^2}+
   \frac{1}{18}\multsp (3\multsp {m^2}+3\multsp {M^2}-{p^2})+  \\
\noalign{\vspace{1.40625ex}}
\hspace{1.em} \frac{1}{6}\multsp
   \bigg({A_0}({M^2})+\bigg(\frac{({M^2}-{m^2})\multsp ({B_0}({p^2},{m^2},{M^2})-{B_0}(0,{m^2},{M^2}))}{2\multsp {p^2}}-
        \frac{1}{2}\multsp {B_0}({p^2},{m^2},{M^2})\bigg)\multsp ({m^2}-{M^2}+{p^2})\bigg)\\
\MathEnd{MathArray}
}

\dispSFinmath{
\Muserfunction{B00}\big[\Muserfunction{SP}[p],{m^2},{m^2}\big]
}

\dispSFoutmath{
\frac{1}{3}\multsp {B_0}({p^2},{m^2},{m^2})\multsp {m^2}+\frac{1}{18}\multsp (6\multsp {m^2}-{p^2})+
   \frac{1}{6}\multsp \Big({A_0}({m^2})-\frac{1}{2}\multsp {B_0}({p^2},{m^2},{m^2})\multsp {p^2}\Big)
}

\dispSFinmath{
\Muserfunction{B00}\big[\Muserfunction{SP}[p],{m^2},{M^2},\Mvariable{BReduce}\rightarrow \Mvariable{False}\big]
}

\dispSFoutmath{
{B_0}({p^2},\multsp {m^2},\multsp {M^2})
}

\dispSFinmath{
\Muserfunction{B00}\big[0,{m^2},{m^2}\big]
}

\dispSFoutmath{
\frac{1}{3}\multsp {B_0}(0,{m^2},{m^2})\multsp {m^2}+\frac{{m^2}}{3}+\frac{1}{6}\multsp {A_0}({m^2})
}

\dispSFinmath{
\Muserfunction{B00}\big[\Muserfunction{SmallVariable}\big[{M^2}\big],{m^2},{m^2}\big]
}

\dispSFoutmath{
\frac{1}{3}\multsp {B_0}({M^2},{m^2},{m^2})\multsp {m^2}+\frac{{m^2}}{3}+\frac{1}{6}\multsp {A_0}({m^2})
}

\Subsection*{B1}

\Subsubsection*{Description}

B1[pp, ma\(\RawWedge\)2, mb\(\RawWedge\)2] the Passarino-Veltman \({B_1}\)-function. All arguments are scalars and have dimension mass\(\RawWedge\)2.

\dispSFinmath{
\Mfunction{Options}[\Mvariable{B1}]
}

\dispSFoutmath{
\{\Mvariable{BReduce}\rightarrow \Mvariable{True}\}
}

See also:  B0, B00, B11, PaVe, PaVeReduce.

\Subsubsection*{Examples}

\dispSFinmath{
\Muserfunction{B1}\big[\Muserfunction{SP}[p],{m^2},{M^2}\big]
}

\dispSFoutmath{
\frac{({M^2}-{m^2})\multsp ({B_0}({p^2},{m^2},{M^2})-{B_0}(0,{m^2},{M^2}))}{2\multsp {p^2}}-\frac{1}{2}\multsp {B_0}({p^2},{m^2},{M^2})
}

\dispSFinmath{
\Muserfunction{B1}\big[\Muserfunction{SP}[p],{m^2},{M^2},\Mvariable{BReduce}\rightarrow \Mvariable{False}\big]
}

\dispSFoutmath{
{B_1}({p^2},{m^2},{M^2})
}

\dispSFinmath{
\Mfunction{SetOptions}[\Mvariable{B1},\Mvariable{BReduce}\rightarrow \Mvariable{True}];
}

\dispSFinmath{
\Muserfunction{B1}\big[\Muserfunction{SP}[p],{m^2},{m^2}\big]
}

\dispSFoutmath{
-\frac{1}{2}\multsp {B_0}({p^2},{m^2},{m^2})
}

\dispSFinmath{
\Muserfunction{B1}\big[{m^2},{m^2},0\big]
}

\dispSFoutmath{
\frac{1}{2}\multsp (-{B_0}(0,{m^2},{m^2})-2)-\frac{1}{2}
}

\dispSFinmath{
\Muserfunction{B1}\big[0,0,{m^2}\big]
}

\dispSFoutmath{
\frac{1}{2}\multsp (-{B_0}(0,{m^2},{m^2})-1)+\frac{1}{4}
}

\dispSFinmath{
\Muserfunction{B1}\big[\Mvariable{pp},\Muserfunction{SmallVariable}\big[m_{e}^{2}\big],m_{2}^{2}\big]
}

\dispSFoutmath{
\frac{({B_0}(\Mvariable{pp},m_{e}^{2},m_{2}^{2})-{B_0}(0,m_{e}^{2},m_{2}^{2}))\multsp m_{2}^{2}}{2\multsp \Mvariable{pp}}-
   \frac{1}{2}\multsp {B_0}(\Mvariable{pp},m_{e}^{2},m_{2}^{2})
}

\dispSFinmath{
\Muserfunction{B1}\big[\Muserfunction{SmallVariable}\big[m_{e}^{2}\big],\Muserfunction{SmallVariable}\big[m_{e}^{2}\big],0\big]
}

\dispSFoutmath{
\frac{1}{2}\multsp (-{B_0}(0,m_{e}^{2},m_{e}^{2})-2)-\frac{1}{2}
}

\Subsection*{B11}

\Subsubsection*{Description}

B11[pp, ma\(\RawWedge\)2, mb\(\RawWedge\)2] is the Passarino-Veltman \({B_{11}}\)-function, i.e., the coefficient function of \({p^{\mu }}\multsp
{p^{\nu }}\). All arguments are scalars and have dimension mass\({{ }^2}\).

\dispSFinmath{
\Mfunction{Options}[\Mvariable{B11}]
}

\dispSFoutmath{
\{\Mvariable{BReduce}\rightarrow \Mvariable{True}\}
}

See also:  B0, B00, B1, PaVe.

\Subsubsection*{Examples}

Remember that SP[p] is a short hand input for ScalarProduct[p,p], i.e. \({p^2}.\)

\dispSFinmath{
\Muserfunction{SP}[p]
}

\dispSFoutmath{
{p^2}
}

\dispSFinmath{
\Muserfunction{B11}\big[\Muserfunction{SP}[p],{m^2},{M^2}\big]
}

\dispSFoutmath{
\MathBegin{MathArray}{l}
\frac{1}{3\multsp {p^2}}\bigg(
    -{B_0}({p^2},{m^2},{M^2})\multsp {m^2}+{A_0}({M^2})+\frac{1}{6}\multsp (-3\multsp {m^2}-3\multsp {M^2}+{p^2})-  \\
   \noalign{\vspace{1.40625ex}}
\hspace{3.em} 2\multsp
     \bigg(\frac{({M^2}-{m^2})\multsp ({B_0}({p^2},{m^2},{M^2})-{B_0}(0,{m^2},{M^2}))}{2\multsp {p^2}}-
       \frac{1}{2}\multsp {B_0}({p^2},{m^2},{M^2})\bigg)\multsp ({m^2}-{M^2}+{p^2})\bigg)\\
\MathEnd{MathArray}
}

\dispSFinmath{
\Muserfunction{B11}\big[\Muserfunction{SP}[p],{m^2},{m^2}\big]
}

\dispSFoutmath{
\frac{{A_0}({m^2})+\frac{1}{6}\multsp ({p^2}-6\multsp {m^2})+{B_0}({p^2},{m^2},{m^2})\multsp ({p^2}-{m^2})}{3\multsp {p^2}}
}

\dispSFinmath{
\Mfunction{SetOptions}[\Mvariable{B11},\Mvariable{BReduce}\rightarrow \Mvariable{False}]
}

\dispSFoutmath{
\{\Mvariable{BReduce}\rightarrow \Mvariable{False}\}
}

\dispSFinmath{
\Muserfunction{B11}\big[\Muserfunction{SP}[p],{m^2},{M^2}\big]
}

\dispSFoutmath{
{B_{11}}({p^2},\multsp {m^2},\multsp {M^2})
}

\dispSFinmath{
\Mfunction{SetOptions}[\Mvariable{B11},\Mvariable{BReduce}\rightarrow \Mvariable{True}]
}

\dispSFoutmath{
\{\Mvariable{BReduce}\rightarrow \Mvariable{True}\}
}

\dispSFinmath{
\Muserfunction{B11}\big[0,{m^2},{m^2}\big]
}

\dispSFoutmath{
\frac{1}{3}\multsp {B_0}(0,{m^2},{m^2})
}

\dispSFinmath{
\Muserfunction{B11}\big[\Muserfunction{SmallVariable}\big[{M^2}\big],{m^2},{m^2}\big]
}

\dispSFoutmath{
\frac{1}{3}\multsp {B_0}({M^2},{m^2},{m^2})
}

\Subsection*{CA}

\Subsubsection*{Description}

CA is one of the Casimir operator eigenvalues of SU({\itshape N}) (CA \(=\) {\itshape N}).

See also:  CF, SUNSimplify.

\Subsubsection*{Examples}

\dispSFinmath{
\Mvariable{CA}
}

\dispSFoutmath{
{C_A}
}

\dispSFinmath{
\Muserfunction{SUNSimplify}[\Mvariable{CA},\Mvariable{SUNNToCACF}\rightarrow \Mvariable{False}]
}

\dispSFoutmath{
N
}

\dispSFinmath{
\Mvariable{SUNN}
}

\dispSFoutmath{
N
}

\Subsection*{Calc}

\Subsubsection*{Description}

Calc[exp] performs several basic simplifications. Calc[exp] is the same as DotSimplify[DiracSimplify[Contract[DiracSimplify[Explicit[
  SUNSimplify[Trick[exp], Explicit\(\rightarrow \)False] ]]]]].

See also:  DiracSimplify, DiracTrick, Trick.

\Subsubsection*{\(\Mvariable{Examples}\)}

This calculates \({{\gamma }^{\mu }}{{\gamma }_{\mu }}\) in 4 dimensions and \(g_{\nu }^{\nu }\) in D dimensions.

\dispSFinmath{
\Muserfunction{Calc}[\{\Muserfunction{GA}[\mu ,\mu ],\multsp \Muserfunction{MTD}[\nu ,\nu ]\}]
}

\dispSFoutmath{
\{D,D\}
}

This simplifies \({f_{\Mvariable{abc}}}\multsp {f_{\Mvariable{abe}}}.\)

\dispSFinmath{
\Muserfunction{Calc}[\Muserfunction{SUNF}[a,b,c]\multsp \Muserfunction{SUNF}[a,b,e]]
}

\dispSFoutmath{
{C_A}\multsp {{\delta }_{ce}}
}

\dispSFinmath{
\Muserfunction{FV}[p+r,\mu ]\multsp \Muserfunction{MT}[\mu ,\nu ]\multsp \Muserfunction{FV}[q-p,\nu ]
}

\dispSFoutmath{
({{q-p}^{\nu }})\multsp ({{p+r}^{\mu }})\multsp {g^{\mu \nu }}
}

\dispSFinmath{
\Muserfunction{Calc}[\%]
}

\dispSFoutmath{
-{p^2}+p\cdot q-p\cdot r+q\cdot r
}

\dispSFinmath{
\Muserfunction{GluonVertex}[p,1,q,2,-p-q,3]
}

\dispSFoutmath{
{V^{\Mvariable{li1}\Mvariable{li2}\Mvariable{li3}}}(p,\multsp q,\multsp -p-q)\multsp {f_{\Mvariable{ci1}\Mvariable{ci2}\Mvariable{ci3}}}
}

\dispSFinmath{
\Muserfunction{Calc}[\%\multsp \Muserfunction{FVD}[p,\Mvariable{li1}]\multsp \Muserfunction{FVD}[q,\Mvariable{li2}]\multsp
    \Muserfunction{FVD}[-p-q,\Mvariable{li3}]]
}

\dispSFoutmath{
0
}

\Subsection*{CalculateCounterTerm ***unfinished***}

\Subsubsection*{Description}

CalculateCounterTerm[exp, k] calculates the residue of exp.

\Subsubsection*{\(\Mvariable{Examples}\)}

\dispSFinmath{
?\Mvariable{CalculateCounterTerm}
}

\Print{\(\Mvariable{CalculateCounterTerm[exp,\multsp k]\multsp calculates\multsp the\multsp residue\multsp of\multsp exp.}\)}

\Subsection*{CancelQP}

\Subsubsection*{Description}

CancelQP is an option for OneLoop. If set to True, cancelation of all {\itshape q}.{\itshape p}'s and \({q^2}\) is performed.

\Subsubsection*{\(\Mvariable{Examples}\)}

\dispSFinmath{
t=\Muserfunction{SPD}[p+2\Mvariable{q2},p+\Mvariable{q2}]\multsp
     \Muserfunction{FAD}[\{\Mvariable{q2}-p,m\},\{\Mvariable{q2},0\},\{\Mvariable{q2},M\}]//\Muserfunction{FCI}
}

\dispSFoutmath{
\frac{(p+{q_2})\cdot (p+2\multsp {q_2})}{({{({q_2}-p)}^2}-{m^2}).q_{2}^{2}.(q_{2}^{2}-{M^2})}
}

\dispSFinmath{
\Muserfunction{OneLoop}[\Mvariable{q2},t,\Mvariable{CancelQP}\rightarrow \Mvariable{True}]
}

\dispSFoutmath{
\MathBegin{MathArray}{l}
\ImaginaryI \multsp {{\pi }^2}\multsp
   \bigg(\frac{3\multsp {B_0}({p^2},0,{m^2})\multsp {m^2}}{2\multsp {M^2}}-\frac{3\multsp {A_0}({M^2})}{2\multsp {M^2}}-  \\
   \noalign{\vspace{1.33333ex}}
\hspace{3.em} \frac{(3\multsp {m^2}-7\multsp {M^2})\multsp {B_0}({p^2},{m^2},{M^2})}{2\multsp {M^2}}-
    \frac{5\multsp {B_0}({p^2},0,{m^2})\multsp {p^2}}{2\multsp {M^2}}+
    \frac{5\multsp {B_0}({p^2},{m^2},{M^2})\multsp {p^2}}{2\multsp {M^2}}\bigg)\\
\MathEnd{MathArray}
}

\dispSFinmath{
\Muserfunction{OneLoop}[\Mvariable{q2},t,\Mvariable{CancelQP}\rightarrow \Mvariable{False}]
}

\dispSFoutmath{
\MathBegin{MathArray}{l}
\ImaginaryI \multsp {{\pi }^2}\multsp
   \bigg(-\frac{3\multsp {B_0}(0,{m^2},{m^2})\multsp {m^2}}{2\multsp {M^2}}+
     \frac{7\multsp {B_0}(0,{m^2},{M^2})\multsp {m^2}}{2\multsp {M^2}}+\frac{3\multsp {B_0}({p^2},0,{m^2})\multsp {m^2}}{2\multsp {M^2}}-
     \frac{3\multsp {m^2}}{2\multsp {M^2}}-\frac{2\multsp {A_0}({m^2})}{{M^2}}+\frac{2\multsp {A_0}({M^2})}{{M^2}}-  \\
   \noalign{\vspace{1.33333ex}}
\hspace{3.em} \frac{7}{2}\multsp {B_0}(0,{m^2},{M^2})-
    \frac{(3\multsp {m^2}-7\multsp {M^2})\multsp {B_0}({p^2},{m^2},{M^2})}{2\multsp {M^2}}-
    \frac{5\multsp {B_0}({p^2},0,{m^2})\multsp {p^2}}{2\multsp {M^2}}+
    \frac{5\multsp {B_0}({p^2},{m^2},{M^2})\multsp {p^2}}{2\multsp {M^2}}\bigg)\\
\MathEnd{MathArray}
}

\dispSFinmath{
t=\Muserfunction{SPD}[\Mvariable{q2},p]\Muserfunction{SPD}[\Mvariable{q1},p]\multsp
     \Muserfunction{FAD}[\{\Mvariable{q1},m\},\{\Mvariable{q2},m\},\Mvariable{q1}-p,\Mvariable{q2}-p,\Mvariable{q2}-\Mvariable{q1}]//
    \Muserfunction{FCI}
}

\dispSFoutmath{
\frac{p\cdot {q_1}\multsp p\cdot {q_2}}{(q_{1}^{2}-{m^2}).(q_{2}^{2}-{m^2}).{{({q_1}-p)}^2}.{{({q_2}-p)}^2}.{{({q_2}-{q_1})}^2}}
}

\dispSFinmath{
\Muserfunction{OneLoop}[\Mvariable{q2},t,\Mvariable{CancelQP}\rightarrow \Mvariable{True}]
}

\dispSFoutmath{
\MathBegin{MathArray}{l}
\ImaginaryI \multsp {{\pi }^2}\multsp
   \bigg(-\frac{{C_0}({p^2},q_{1}^{2},{p^2}-2\multsp p\cdot {q_1}+q_{1}^{2},0,{m^2},0)\multsp p\cdot {q_1}\multsp {m^2}}
       {2\multsp ({m^2}-q_{1}^{2})\multsp ({p^2}-2\multsp p\cdot {q_1}+q_{1}^{2})}+
     \frac{{B_0}(q_{1}^{2},0,{m^2})\multsp p\cdot {q_1}}{2\multsp ({m^2}-q_{1}^{2})\multsp ({p^2}-2\multsp p\cdot {q_1}+q_{1}^{2})}-  \\
   \noalign{\vspace{1.5625ex}}
\hspace{3.em} \frac{{B_0}({p^2}-2\multsp p\cdot {q_1}+q_{1}^{2},0,0)\multsp p\cdot {q_1}}
     {2\multsp ({m^2}-q_{1}^{2})\multsp ({p^2}-2\multsp p\cdot {q_1}+q_{1}^{2})}-
    \frac{{C_0}({p^2},q_{1}^{2},{p^2}-2\multsp p\cdot {q_1}+q_{1}^{2},0,{m^2},0)\multsp {p^2}\multsp p\cdot {q_1}}
     {2\multsp ({m^2}-q_{1}^{2})\multsp ({p^2}-2\multsp p\cdot {q_1}+q_{1}^{2})}\bigg)\\
\MathEnd{MathArray}
}

\dispSFinmath{
\Muserfunction{OneLoop}[\Mvariable{q2},t,\Mvariable{CancelQP}\rightarrow \Mvariable{False}]
}

\dispSFoutmath{
\MathBegin{MathArray}{l}
\ImaginaryI \multsp {{\pi }^2}\multsp
   \bigg(\frac{\Muserfunction{PaVe}(2,\{{p^2},q_{1}^{2},{p^2}-2\multsp p\cdot {q_1}+q_{1}^{2}\},\{0,{m^2},0\})\multsp {{p\cdot {q_1}}^2}}
      {({m^2}-q_{1}^{2})\multsp ({p^2}-2\multsp p\cdot {q_1}+q_{1}^{2})}-  \\
\noalign{\vspace{1.5625ex}}
\hspace{3.em} \frac{{p^2}
      \multsp \Muserfunction{PaVe}(1,\{{p^2},q_{1}^{2},{p^2}-2\multsp p\cdot {q_1}+q_{1}^{2}\},\{0,{m^2},0\})\multsp p\cdot {q_1}}{(
       {m^2}-q_{1}^{2})\multsp ({p^2}-2\multsp p\cdot {q_1}+q_{1}^{2})}-  \\
\noalign{\vspace{1.5625ex}}
\hspace{3.em} \frac{{p^2}
      \multsp \Muserfunction{PaVe}(2,\{{p^2},q_{1}^{2},{p^2}-2\multsp p\cdot {q_1}+q_{1}^{2}\},\{0,{m^2},0\})\multsp p\cdot {q_1}}{(
       {m^2}-q_{1}^{2})\multsp ({p^2}-2\multsp p\cdot {q_1}+q_{1}^{2})}-  \\
\noalign{\vspace{1.5625ex}}
\hspace{3.em} \frac{{C_0}({p^2},
        q_{1}^{2},{p^2}-2\multsp p\cdot {q_1}+q_{1}^{2},0,{m^2},0)\multsp {p^2}\multsp p\cdot {q_1}}{({m^2}-q_{1}^{2})\multsp
       ({p^2}-2\multsp p\cdot {q_1}+q_{1}^{2})}\bigg)\\
\MathEnd{MathArray}
}

\dispSFinmath{
\Mfunction{Clear}[t]
}

\Subsection*{Cases2}

\Subsubsection*{Description}

Cases2[expr, f] returns a list of all objects in expr with head f. Cases2[expr,f] is equivalent to
  Cases2[\{expr\},f[\_{}\_{}\_{}],Infinity]//Union. Cases2[expr, f, g, ...] or Cases2[expr, \{f,g, ...\}] is equivalent to
  Cases[\{expr\},f[\_{}\_{}\_{}] \(|\) g[\_{}\_{}\_{}] ...] .

\Subsubsection*{\(\Mvariable{Examples}\)}

\dispSFinmath{
\Muserfunction{Cases2}\big[f[a]+{{f[b]}^2}+f[c,d],f\big]
}

\dispSFoutmath{
\{f(a),f(b),f(c,d)\}
}

\dispSFinmath{
\Muserfunction{Cases2}[\sin [x]\multsp \sin [y-z]+g[y],\sin ,g]
}

\dispSFoutmath{
\{g(y),\sin(x),\sin(y-z)\}
}

\dispSFinmath{
\Muserfunction{Cases2}[\sin [x]\multsp \sin [y-z]+g[x]+g[a,b,c],\{\sin ,g\}]
}

\dispSFoutmath{
\{g(x),g(a,b,c),\sin(x),\sin(y-z)\}
}

\dispSFinmath{
\Muserfunction{Cases2}[\Muserfunction{DiracSlash}[p].\Muserfunction{DiracSlash}[q]+\Muserfunction{ScalarProduct}[p,p],\Mvariable{Dot}]
}

\dispSFoutmath{
\{(\gamma \cdot p).(\gamma \cdot q)\}
}

\Subsection*{CF}

\Subsubsection*{Description}

CF is one of the Casimir operator eigenvalues of SU({\itshape N}) (CF \(=\) (\({N^2}\)-1)/(2 {\itshape N})).

See also:  CA, SUNSimplify.

\Subsubsection*{Examples}

\dispSFinmath{
\Mvariable{CF}
}

\dispSFoutmath{
{C_F}
}

\dispSFinmath{
\Muserfunction{SUNSimplify}[\Mvariable{CF},\multsp \Mvariable{SUNNToCACF}\rightarrow \Mvariable{False}]
}

\dispSFoutmath{
\frac{{N^2}-1}{2\multsp N}
}

\dispSFinmath{
\%//\Mfunction{InputForm}
}

\dispSFoutmath{
(-1\multsp +\multsp \Mvariable{SUNN}\RawWedge 2)/(2*\Mvariable{SUNN})
}

\dispSFinmath{
\Mvariable{SUNN}
}

\dispSFoutmath{
N
}

\dispSFinmath{
\Muserfunction{SUNSimplify}[\Mvariable{SUNN}\RawWedge 2-1,\multsp \Mvariable{SUNNToCACF}\multsp \rightarrow \Mvariable{True}]
}

\dispSFoutmath{
2\multsp {C_A}\multsp {C_F}
}

\Subsection*{ChangeDimension}

\Subsubsection*{Description}

ChangeDimension[exp, dim] changes all LorentzIndex and Momenta in exp to dimension dim (and also Levi-Civita-tensors, Dirac slashes and
  Dirac matrices).

See also:  LorentzIndex, Momentum, DiracGamma, Eps.

\Subsubsection*{Examples}

Remember that LorentzIndex[mu, 4] is simplified to LorentzIndex[mu] and Momentum[p, 4] to Momentum[p]. Thus the fullowing objects are
  defined in four dimensions.

\dispSFinmath{
\{\Muserfunction{LorentzIndex}[\mu ],\multsp \Muserfunction{Momentum}[p]\}
}

\dispSFoutmath{
\{\mu ,p\}
}

\dispSFinmath{
\Muserfunction{ChangeDimension}[\%,\multsp D]
}

\dispSFoutmath{
\{\mu ,p\}
}

\dispSFinmath{
\%//\Mfunction{StandardForm}
}

\dispSFoutmath{
\{\Muserfunction{LorentzIndex}[\mu ,D],\Muserfunction{Momentum}[p,D]\}
}

This changes all non-4-dimensional objects to 4-dimensional ones.

\dispSFinmath{
\Muserfunction{ChangeDimension}[\%\%,\multsp 4]\multsp //\Mfunction{\multsp }\Mvariable{StandardForm}
}

\dispSFoutmath{
\{\Muserfunction{LorentzIndex}[\mu ],\Muserfunction{Momentum}[p]\}
}

Consider the following list of 4- and D-dimensional object.

\dispSFinmath{
\{\Muserfunction{GA}[\mu ,\nu ]\multsp \Muserfunction{MT}[\mu ,\nu ],\multsp
    \Muserfunction{GAD}[\mu ,\nu ]\multsp \Muserfunction{MTD}[\mu ,\nu ]\multsp f[D]\}
}

\dispSFoutmath{
\{{{\gamma }^{\mu }}.{{\gamma }^{\nu }}\multsp {g^{\mu \nu }},{{\gamma }^{\mu }}.{{\gamma }^{\nu }}\multsp f(D)\multsp {g^{\mu \nu }}\}
}

\dispSFinmath{
\Muserfunction{DiracTrick}[\Muserfunction{Contract}[\%]]
}

\dispSFoutmath{
\{4,D\multsp f(D)\}
}

\dispSFinmath{
\Muserfunction{DiracTrick}[\Muserfunction{Contract}[\Muserfunction{ChangeDimension}[\%\%,n]]]
}

\dispSFoutmath{
\{n,n\multsp f(D)\}
}

Any explicit occurence of D (like in f(D)) is not replaced by ChangeDimension.

\dispSFinmath{
\Mfunction{SetOptions}[\Mvariable{LeviCivita},\Mvariable{Dimension}\rightarrow D];
}

The option Dimension of Eps must be changed too, since with the default setting Dimension\(\rightarrow \)4 the arguments of Eps are
  automatically changed to 4 dimensions.

\dispSFinmath{
\Mfunction{SetOptions}[\Mvariable{Eps},\Mvariable{Dimension}\rightarrow D];
}

\dispSFinmath{
\Mvariable{a1}\multsp =\multsp \Muserfunction{LeviCivita}[\mu ,\nu ,\rho ,\sigma ]
}

\dispSFoutmath{
{{\epsilon }^{\mu \nu \rho \sigma }}
}

\dispSFinmath{
\Mvariable{a2}\multsp =\multsp \Muserfunction{ChangeDimension}[\Mvariable{a1},4]
}

\dispSFoutmath{
{{\epsilon }^{\mu \nu \rho \sigma }}
}

\dispSFinmath{
\Muserfunction{Factor2}\big[\Muserfunction{Contract}\big[{{\Mvariable{a1}}^2}\big]\big]
}

\dispSFoutmath{
(1-D)\multsp (2-D)\multsp (3-D)\multsp D
}

\dispSFinmath{
\Muserfunction{Contract}\big[{{\Mvariable{a2}}^2}\big]
}

\dispSFoutmath{
-24
}

\dispSFinmath{
\Mfunction{SetOptions}[\Mvariable{Eps},\Mvariable{Dimension}\rightarrow 4];
}

\dispSFinmath{
\Mfunction{Clear}[\Mvariable{a1},\Mvariable{a2}];
}

\Subsection*{ChargeConjugationMatrix}

\Subsubsection*{Description}

ChargeConjugationMatrix denotes the charge conjugation matrix {\itshape C}.

See also:  ChargeConjugationMatrixInv.

\Subsubsection*{Examples}

\dispSFinmath{
\Mvariable{ChargeConjugationMatrix}.\Muserfunction{DiracMatrix}[\mu ].\Mvariable{ChargeConjugationMatrixInv}
}

\dispSFoutmath{
C.{{\gamma }^{\mu }}.(-C)
}

\dispSFinmath{
\Muserfunction{Calc}[\%]
}

\dispSFoutmath{
-C.{{\gamma }^{\mu }}.C
}

\dispSFinmath{
\Mvariable{ChargeConjugationMatrix}.\Muserfunction{DiracGamma}[5].\Mvariable{ChargeConjugationMatrixInv}
}

\dispSFoutmath{
C.{{\gamma }^5}.(-C)
}

\dispSFinmath{
\Muserfunction{Calc}[\%]
}

\dispSFoutmath{
\gamma _{5}^{T}
}

\dispSFinmath{
\Mvariable{ChargeConjugationMatrix}\RawWedge 2
}

\dispSFoutmath{
-1
}

\Subsection*{ChargeConjugationMatrixInv}

\Subsubsection*{Description}

ChargeConjugationMatrixInv is the inverse of ChargeConjugationMatrix. It is substituted immediately by -ChargeConjugationMatrix.

See also:  ChargeConjugationMatrix.

\Subsubsection*{Examples}

\dispSFinmath{
\Mvariable{ChargeConjugationMatrix}.\Muserfunction{DiracMatrix}[\mu ].\Mvariable{ChargeConjugationMatrixInv}
}

\dispSFoutmath{
C.{{\gamma }^{\mu }}.(-C)
}

\dispSFinmath{
\Muserfunction{Calc}[\%]
}

\dispSFoutmath{
-C.{{\gamma }^{\mu }}.C
}

\dispSFinmath{
\Mvariable{ChargeConjugationMatrix}.\Muserfunction{DiracGamma}[5].\Mvariable{ChargeConjugationMatrixInv}
}

\dispSFoutmath{
C.{{\gamma }^5}.(-C)
}

\dispSFinmath{
\Muserfunction{Calc}[\%]
}

\dispSFoutmath{
\gamma _{5}^{T}
}

\dispSFinmath{
\Mvariable{ChargeConjugationMatrix}.\Mvariable{ChargeConjugationMatrixInv}
}

\dispSFoutmath{
C.(-C)
}

\dispSFinmath{
\%//\Muserfunction{Calc}
}

\dispSFoutmath{
-C.C
}

\dispSFinmath{
\Mvariable{ChargeConjugationMatrixInv}\RawWedge 2
}

\dispSFoutmath{
-1
}

\Subsection*{CheckContext}

\Subsubsection*{Description}

CheckContext[string] yields True if the packaged associated with string is already loaded, and False otherwise.

See also:  MakeContext, \${}FeynCalcStuff.

\Subsubsection*{Examples}

If a FeynCalc symbol has not been used, the corresponding subpackage has not been loaded:

\dispSFinmath{
\Muserfunction{CheckContext}["CompleteSquare"]
}

\dispSFoutmath{
\Mvariable{False}
}

Using it makes FeynCalc load the subpackage:

\dispSFinmath{
\Mvariable{CompleteSquare}
}

\dispSFoutmath{
\Mvariable{CompleteSquare}
}

\dispSFinmath{
\Muserfunction{CheckContext}["CompleteSquare"]
}

\dispSFoutmath{
\Mvariable{True}
}

\Subsection*{ChiralityProjector}

\Subsubsection*{Description}

ChiralityProjector[\(+\)1] denotes \(1/2(1+{{\gamma }^5})\). ChiralityProjector[-1] denotes \(1/2(1+{{\gamma }^5})\).

See also:  DiracGamma, DiracMatrix, FCI.

\Subsubsection*{Examples}

\dispSFinmath{
\{\Muserfunction{ChiralityProjector}[+1],\Muserfunction{ChiralityProjector}[-1]\}
}

\dispSFoutmath{
\big\{{{\gamma }^6},{{\gamma }^7}\big\}
}

\dispSFinmath{
\Muserfunction{FCI}[\%]
}

\dispSFoutmath{
\big\{{{\gamma }^6},{{\gamma }^7}\big\}
}

\dispSFinmath{
\Muserfunction{DiracSimplify}[\%,\Mvariable{DiracSubstitute67}\rightarrow \Mvariable{True}]
}

\dispSFoutmath{
\big\{\frac{{{\gamma }^5}}{2}+\frac{1}{2},\frac{1}{2}-\frac{{{\gamma }^5}}{2}\big\}
}

\Subsection*{Chisholm}

\Subsubsection*{Description}

Chisholm[x] substitutes products of three Dirac matrices or slashes by the Chisholm identity.

See also:  EpsChisholm, ChisholmSpinor.

\Subsubsection*{Examples}

\dispSFinmath{
\Muserfunction{Chisholm}[\Muserfunction{GA}[\alpha ,\beta ,\mu ,\nu ]]
}

\dispSFoutmath{
-\ImaginaryI \multsp {{\gamma }^{\$MU\$79}}.{{\gamma }^{\nu }}.{{\gamma }^5}\multsp {{\epsilon }^{\alpha \beta \mu \$MU\$79}}+
   {{\gamma }^{\mu }}.{{\gamma }^{\nu }}\multsp {g^{\alpha \beta }}-{{\gamma }^{\beta }}.{{\gamma }^{\nu }}\multsp {g^{\alpha \mu }}+
   {{\gamma }^{\alpha }}.{{\gamma }^{\nu }}\multsp {g^{\beta \mu }}
}

\dispSFinmath{
\Mvariable{t1}=\Muserfunction{DiracMatrix}[\mu ,\nu ,\rho ]
}

\dispSFoutmath{
{{\gamma }^{\mu }}{{\gamma }^{\nu }}{{\gamma }^{\rho }}
}

\dispSFinmath{
\Mvariable{t2}\multsp =\Muserfunction{Chisholm}[\Mvariable{t1}]
}

\dispSFoutmath{
\ImaginaryI \multsp {{\gamma }^{\$MU\$84}}.{{\gamma }^5}\multsp {{\epsilon }^{\mu \nu \rho \$MU\$84}}+
   {{\gamma }^{\rho }}\multsp {g^{\mu \nu }}-{{\gamma }^{\nu }}\multsp {g^{\mu \rho }}+{{\gamma }^{\mu }}\multsp {g^{\nu \rho }}
}

The \${}MU\${} variables are unique indices.

\dispSFinmath{
\Muserfunction{Calc}[\Mvariable{t1}.\Mvariable{t1}]
}

\dispSFoutmath{
-{D^3}+6\multsp {D^2}-4\multsp D
}

\dispSFinmath{
\Mvariable{t3}\multsp =\Muserfunction{Chisholm}[\Mvariable{t1}]
}

\dispSFoutmath{
\ImaginaryI \multsp {{\gamma }^{\$MU\$91}}.{{\gamma }^5}\multsp {{\epsilon }^{\mu \nu \rho \$MU\$91}}+
   {{\gamma }^{\rho }}\multsp {g^{\mu \nu }}-{{\gamma }^{\nu }}\multsp {g^{\mu \rho }}+{{\gamma }^{\mu }}\multsp {g^{\nu \rho }}
}

\dispSFinmath{
\Muserfunction{Calc}[\Mvariable{t2}.\Mvariable{t3}]
}

\dispSFoutmath{
3\multsp {D^2}-2\multsp D-24
}

\dispSFinmath{
\Mvariable{t4}=\Muserfunction{DiracSlash}[a,b,c]
}

\dispSFoutmath{
(\gamma \cdot a).(\gamma \cdot b).(\gamma \cdot c)
}

\dispSFinmath{
\Muserfunction{Chisholm}[\Mvariable{t4}]
}

\dispSFoutmath{
-\ImaginaryI \multsp {{\gamma }^{\$MU\$100}}.{{\gamma }^5}\multsp {{\epsilon }^{\$MU\$100abc}}+\gamma \cdot c\multsp a\cdot b-
   \gamma \cdot b\multsp a\cdot c+\gamma \cdot a\multsp b\cdot c
}

\dispSFinmath{
\Mvariable{a1}=\Muserfunction{GA}[\mu ,\nu ,\rho ,\sigma ,\tau ,\kappa ]
}

\dispSFoutmath{
{{\gamma }^{\mu }}.{{\gamma }^{\nu }}.{{\gamma }^{\rho }}.{{\gamma }^{\sigma }}.{{\gamma }^{\tau }}.{{\gamma }^{\kappa }}
}

\dispSFinmath{
\Mvariable{a2}\multsp =\Muserfunction{Chisholm}[\Mvariable{a1}]
}

\dispSFoutmath{
\MathBegin{MathArray}{l}
-\ImaginaryI \multsp {{\gamma }^{\tau }}.{{\gamma }^{\kappa }}.{{\gamma }^5}\multsp
    {{\epsilon }^{\mu \nu \rho \sigma }}+\ImaginaryI \multsp {{\gamma }^{\sigma }}.{{\gamma }^{\kappa }}.{{\gamma }^5}\multsp
    {{\epsilon }^{\mu \nu \rho \tau }}-\ImaginaryI \multsp {{\gamma }^{\$MU\$111}}.{{\gamma }^{\kappa }}.{{\gamma }^5}\multsp
    {{\epsilon }^{\rho \sigma \tau \$MU\$111}}\multsp {g^{\mu \nu }}+  \\
\noalign{\vspace{0.604167ex}}
\hspace{1.em} \ImaginaryI
    \multsp {{\gamma }^{\$MU\$114}}.{{\gamma }^{\kappa }}.{{\gamma }^5}\multsp {{\epsilon }^{\nu \sigma \tau \$MU\$114}}\multsp
    {g^{\mu \rho }}-\ImaginaryI \multsp {{\gamma }^{\$MU\$117}}.{{\gamma }^{\kappa }}.{{\gamma }^5}\multsp
    {{\epsilon }^{\mu \sigma \tau \$MU\$117}}\multsp {g^{\nu \rho }}+
   {{\gamma }^{\tau }}.{{\gamma }^{\kappa }}\multsp {g^{\mu \sigma }}\multsp {g^{\nu \rho }}-  \\
\noalign{\vspace{0.666667ex}}
   \hspace{1.em} {{\gamma }^{\sigma }}.{{\gamma }^{\kappa }}\multsp {g^{\mu \tau }}\multsp {g^{\nu \rho }}-
   {{\gamma }^{\tau }}.{{\gamma }^{\kappa }}\multsp {g^{\mu \rho }}\multsp {g^{\nu \sigma }}+
   {{\gamma }^{\rho }}.{{\gamma }^{\kappa }}\multsp {g^{\mu \tau }}\multsp {g^{\nu \sigma }}+
   {{\gamma }^{\sigma }}.{{\gamma }^{\kappa }}\multsp {g^{\mu \rho }}\multsp {g^{\nu \tau }}-
   {{\gamma }^{\rho }}.{{\gamma }^{\kappa }}\multsp {g^{\mu \sigma }}\multsp {g^{\nu \tau }}+  \\
\noalign{\vspace{0.666667ex}}
   \hspace{1.em} {{\gamma }^{\tau }}.{{\gamma }^{\kappa }}\multsp {g^{\mu \nu }}\multsp {g^{\rho \sigma }}-
   {{\gamma }^{\nu }}.{{\gamma }^{\kappa }}\multsp {g^{\mu \tau }}\multsp {g^{\rho \sigma }}+
   {{\gamma }^{\mu }}.{{\gamma }^{\kappa }}\multsp {g^{\nu \tau }}\multsp {g^{\rho \sigma }}-
   {{\gamma }^{\sigma }}.{{\gamma }^{\kappa }}\multsp {g^{\mu \nu }}\multsp {g^{\rho \tau }}+
   {{\gamma }^{\nu }}.{{\gamma }^{\kappa }}\multsp {g^{\mu \sigma }}\multsp {g^{\rho \tau }}-
   {{\gamma }^{\mu }}.{{\gamma }^{\kappa }}\multsp {g^{\nu \sigma }}\multsp {g^{\rho \tau }}-  \\
\noalign{\vspace{0.604167ex}}
   \hspace{1.em} \ImaginaryI \multsp {{\gamma }^{\$MU\$105}}.{{\gamma }^{\kappa }}.{{\gamma }^5}\multsp
    {{\epsilon }^{\mu \nu \rho \$MU\$105}}\multsp {g^{\sigma \tau }}+
   {{\gamma }^{\rho }}.{{\gamma }^{\kappa }}\multsp {g^{\mu \nu }}\multsp {g^{\sigma \tau }}-
   {{\gamma }^{\nu }}.{{\gamma }^{\kappa }}\multsp {g^{\mu \rho }}\multsp {g^{\sigma \tau }}+
   {{\gamma }^{\mu }}.{{\gamma }^{\kappa }}\multsp {g^{\nu \rho }}\multsp {g^{\sigma \tau }}\\
\MathEnd{MathArray}
}

\dispSFinmath{
\Mvariable{a3}\multsp =\Muserfunction{Chisholm}[\Mvariable{a1}]
}

\dispSFoutmath{
\MathBegin{MathArray}{l}
-\ImaginaryI \multsp {{\gamma }^{\tau }}.{{\gamma }^{\kappa }}.{{\gamma }^5}\multsp
    {{\epsilon }^{\mu \nu \rho \sigma }}+\ImaginaryI \multsp {{\gamma }^{\sigma }}.{{\gamma }^{\kappa }}.{{\gamma }^5}\multsp
    {{\epsilon }^{\mu \nu \rho \tau }}-\ImaginaryI \multsp {{\gamma }^{\$MU\$128}}.{{\gamma }^{\kappa }}.{{\gamma }^5}\multsp
    {{\epsilon }^{\rho \sigma \tau \$MU\$128}}\multsp {g^{\mu \nu }}+  \\
\noalign{\vspace{0.604167ex}}
\hspace{1.em} \ImaginaryI
    \multsp {{\gamma }^{\$MU\$131}}.{{\gamma }^{\kappa }}.{{\gamma }^5}\multsp {{\epsilon }^{\nu \sigma \tau \$MU\$131}}\multsp
    {g^{\mu \rho }}-\ImaginaryI \multsp {{\gamma }^{\$MU\$134}}.{{\gamma }^{\kappa }}.{{\gamma }^5}\multsp
    {{\epsilon }^{\mu \sigma \tau \$MU\$134}}\multsp {g^{\nu \rho }}+
   {{\gamma }^{\tau }}.{{\gamma }^{\kappa }}\multsp {g^{\mu \sigma }}\multsp {g^{\nu \rho }}-  \\
\noalign{\vspace{0.666667ex}}
   \hspace{1.em} {{\gamma }^{\sigma }}.{{\gamma }^{\kappa }}\multsp {g^{\mu \tau }}\multsp {g^{\nu \rho }}-
   {{\gamma }^{\tau }}.{{\gamma }^{\kappa }}\multsp {g^{\mu \rho }}\multsp {g^{\nu \sigma }}+
   {{\gamma }^{\rho }}.{{\gamma }^{\kappa }}\multsp {g^{\mu \tau }}\multsp {g^{\nu \sigma }}+
   {{\gamma }^{\sigma }}.{{\gamma }^{\kappa }}\multsp {g^{\mu \rho }}\multsp {g^{\nu \tau }}-
   {{\gamma }^{\rho }}.{{\gamma }^{\kappa }}\multsp {g^{\mu \sigma }}\multsp {g^{\nu \tau }}+  \\
\noalign{\vspace{0.666667ex}}
   \hspace{1.em} {{\gamma }^{\tau }}.{{\gamma }^{\kappa }}\multsp {g^{\mu \nu }}\multsp {g^{\rho \sigma }}-
   {{\gamma }^{\nu }}.{{\gamma }^{\kappa }}\multsp {g^{\mu \tau }}\multsp {g^{\rho \sigma }}+
   {{\gamma }^{\mu }}.{{\gamma }^{\kappa }}\multsp {g^{\nu \tau }}\multsp {g^{\rho \sigma }}-
   {{\gamma }^{\sigma }}.{{\gamma }^{\kappa }}\multsp {g^{\mu \nu }}\multsp {g^{\rho \tau }}+
   {{\gamma }^{\nu }}.{{\gamma }^{\kappa }}\multsp {g^{\mu \sigma }}\multsp {g^{\rho \tau }}-
   {{\gamma }^{\mu }}.{{\gamma }^{\kappa }}\multsp {g^{\nu \sigma }}\multsp {g^{\rho \tau }}-  \\
\noalign{\vspace{0.604167ex}}
   \hspace{1.em} \ImaginaryI \multsp {{\gamma }^{\$MU\$122}}.{{\gamma }^{\kappa }}.{{\gamma }^5}\multsp
    {{\epsilon }^{\mu \nu \rho \$MU\$122}}\multsp {g^{\sigma \tau }}+
   {{\gamma }^{\rho }}.{{\gamma }^{\kappa }}\multsp {g^{\mu \nu }}\multsp {g^{\sigma \tau }}-
   {{\gamma }^{\nu }}.{{\gamma }^{\kappa }}\multsp {g^{\mu \rho }}\multsp {g^{\sigma \tau }}+
   {{\gamma }^{\mu }}.{{\gamma }^{\kappa }}\multsp {g^{\nu \rho }}\multsp {g^{\sigma \tau }}\\
\MathEnd{MathArray}
}

Check that both a1.a1 and a2.a3 give the same.

\dispSFinmath{
\Muserfunction{Calc}[\Mvariable{a1}.\Mvariable{a1}]
}

\dispSFoutmath{
-{D^6}+30\multsp {D^5}-260\multsp {D^4}+840\multsp {D^3}-1120\multsp {D^2}+512\multsp D
}

\dispSFinmath{
\Muserfunction{Calc}[\Mvariable{a2}.\Mvariable{a3}]
}

\dispSFoutmath{
-15\multsp {D^4}+60\multsp {D^3}-214\multsp {D^2}+380\multsp D-144
}

\dispSFinmath{
\Mfunction{Clear}[\Mvariable{t1},\Mvariable{t2},\Mvariable{t3},\Mvariable{t4},\Mvariable{a1},\Mvariable{a2},\Mvariable{a3}]
}

\Subsection*{ChisholmSpinor}

\Subsubsection*{Description}

ChisholmSpinor[x] uses the Chisholm identity on a DiraGamma between spinors. As an optional second argument 1 or 2 may be given,
  indicating that ChisholmSpinor should only act on the first resp. second part of a product of spinor chains.

See also:  EpsChisholm, Chisholm.

\Subsubsection*{Examples}

\dispSFinmath{
\Muserfunction{Spinor}[\Mvariable{p1},\Mvariable{m1}].\Muserfunction{DiracGamma}[\Muserfunction{LorentzIndex}[\mu ]].
   \Muserfunction{Spinor}[\Mvariable{p2},\Mvariable{m2}]
}

\dispSFoutmath{
\varphi ({p_1},\Mvariable{m1}).{{\gamma }^{\mu }}.\varphi ({p_2},\Mvariable{m2})
}

\dispSFinmath{
\%//\Muserfunction{ChisholmSpinor}
}

\dispSFoutmath{
\MathBegin{MathArray}{l}
-\frac{\Mvariable{m1}\multsp \Mvariable{m2}\multsp
       \varphi ({p_1},\Mvariable{m1}).{{\gamma }^{\mu }}.\varphi ({p_2},\Mvariable{m2})}{{p_1}\cdot {p_2}}+
   \frac{\ImaginaryI \multsp \varphi ({p_1},\Mvariable{m1}).{{\gamma }^{\Mvariable{alpha1}}}.{{\gamma }^5}.\varphi ({p_2},\Mvariable{m2})
      \multsp {{\epsilon }^{\Mvariable{alpha1}\mu {p_1}{p_2}}}}{{p_1}\cdot {p_2}}+  \\
\noalign{\vspace{1.59375ex}}
\hspace{1.em}
     \frac{\Mvariable{m2}\multsp \varphi ({p_1},\Mvariable{m1}).\varphi ({p_2},\Mvariable{m2})\multsp p_{1}^{\mu }}{{p_1}\cdot {p_2}}+
   \frac{\Mvariable{m1}\multsp \varphi ({p_1},\Mvariable{m1}).\varphi ({p_2},\Mvariable{m2})\multsp p_{2}^{\mu }}{{p_1}\cdot {p_2}}\\
   \MathEnd{MathArray}
}

\Subsection*{ClearScalarProducts}

\Subsubsection*{Description}

ClearScalarProducts removes all user-performed specific settings for ScalarProduct's.

See also: ScalarProduct, Pair, SP, SPD.

\Subsubsection*{Examples}

\dispSFinmath{
\Muserfunction{ScalarProduct}[p,p]=m\RawWedge 2
}

\dispSFoutmath{
{m^2}
}

\dispSFinmath{
\Muserfunction{Pair}[\Muserfunction{Momentum}[p],\Muserfunction{Momentum}[p]]
}

\dispSFoutmath{
{m^2}
}

\dispSFinmath{
\Mvariable{ClearScalarProducts}
}

\dispSFinmath{
\Muserfunction{Pair}[\Muserfunction{Momentum}[p],\Muserfunction{Momentum}[p]]
}

\dispSFoutmath{
{p^2}
}

\dispSFinmath{
\Muserfunction{ScalarProduct}[p,p]
}

\dispSFoutmath{
{p^2}
}

\Subsection*{Collecting}

\Subsubsection*{Description}

Collecting is an option of ScalarProductCancel, Series2, TID and related functions. Setting it to True will trigger some kind of
  collecting of the result.

See also: { }ScalarProductCancel, Series2, TID.

\Subsection*{Collect2}

\Subsubsection*{Description}

Collect2[expr, x] collects together terms which are not free of any occurrence of x. Collect2[expr, \{x1, x2, ...\}] (or also
  Collect2[expr, x1, x2, ...]) collects together terms which are not free of any occurrence of x1, x2, .... The coefficients are put over
  a common denominator. If expr is expanded before collecting depends on the option Factoring, which may be set to Factor, Factor2, or
  any other function, which is applied to the coefficients. If expr is already expanded with respect to x (x1,x2, ...), the option
  Expanding can be set to False.

\dispSFinmath{
\Mfunction{Options}[\Mvariable{Collect2}]
}

\dispSFoutmath{
\{\Mvariable{Denominator}\rightarrow \Mvariable{False},\Mvariable{Dot}\rightarrow \Mvariable{False},
    \Mvariable{Expanding}\rightarrow \Mvariable{True},\Mvariable{Factoring}\rightarrow \Mvariable{Factor2},
    \Mvariable{IsolateNames}\rightarrow \Mvariable{False}\}
}

See also:  Isolate.

\Subsubsection*{Examples}

\dispSFinmath{
\Muserfunction{Collect2}\big[\Mvariable{t1}=a+r\multsp a+{k^2}\multsp f[a]-k\multsp f[a]+\frac{x}{2}-\frac{y}{w},a\big]
}

\dispSFoutmath{
a\multsp (r+1)+\frac{w\multsp x-2\multsp y}{2\multsp w}-(1-k)\multsp k\multsp f(a)
}

\dispSFinmath{
\Muserfunction{Collect2}[\Mvariable{t1},a,\Mvariable{Factoring}\rightarrow \Mvariable{False}]
}

\dispSFoutmath{
a\multsp (r+1)+\frac{x}{2}-\frac{y}{w}+\big({k^2}-k\big)\multsp f(a)
}

\dispSFinmath{
\Muserfunction{Collect2}[\Mvariable{t1},a,\Mvariable{Factoring}\rightarrow \Mvariable{Factor}]
}

\dispSFoutmath{
a\multsp (r+1)+\frac{w\multsp x-2\multsp y}{2\multsp w}+(k-1)\multsp k\multsp f(a)
}

\dispSFinmath{
\Muserfunction{Collect2}\big[2\multsp a\multsp (b-a)\multsp (h-1)-{b^2}\multsp (e\multsp a-c)+{b^2},\{a,b\}\big]
}

\dispSFoutmath{
2\multsp (1-h)\multsp {a^2}-{b^2}\multsp e\multsp a-2\multsp b\multsp (1-h)\multsp a+{b^2}\multsp (c+1)
}

\dispSFinmath{
\Muserfunction{Collect2}\big[\Mfunction{Expand}\big[{{(a-b-c-d)}^5}\big],a,\Mvariable{IsolateNames}\rightarrow L\big]
}

\Message{\(\MathBegin{MathArray}{l}
\Muserfunction{L1}::\Mvariable{shdw}:\multsp
   \Mvariable{Symbol}\multsp \Mvariable{L1}\multsp \Mvariable{appears}\multsp \Mvariable{in}\multsp \Mvariable{multiple}\multsp
      \Mvariable{contexts}\multsp \{\Mvariable{Global`},\Mvariable{HighEnergyPhysics`Phi`Objects`}\};\multsp   \\
\noalign{\vspace{
   0.666667ex}}
\hspace{2.em} \Mvariable{definitions}\multsp \Mvariable{in}\multsp \Mvariable{context}\multsp \Mvariable{Global`}
   \multsp \Mvariable{may}\multsp \Mvariable{shadow}\multsp \Mvariable{or}\multsp \Mvariable{be}\multsp \Mvariable{shadowed}\multsp
     \Mvariable{by}\multsp \Mvariable{other}\multsp \Mvariable{definitions}.\\
\MathEnd{MathArray}\)}

\dispSFoutmath{
{a^5}-5\multsp L(1)\multsp {a^4}+10\multsp {{L(1)}^2}\multsp {a^3}-10\multsp {{L(1)}^3}\multsp {a^2}+5\multsp {{L(1)}^4}\multsp a-
   {{L(1)}^5}
}

\dispSFinmath{
\Mfunction{ReleaseHold}[\%]
}

\dispSFoutmath{
{a^5}-5\multsp (b+c+d)\multsp {a^4}+10\multsp {{(b+c+d)}^2}\multsp {a^3}-10\multsp {{(b+c+d)}^3}\multsp {a^2}+
   5\multsp {{(b+c+d)}^4}\multsp a-{{(b+c+d)}^5}
}

\dispSFinmath{
\Mfunction{Clear}[\Mvariable{t1},L]
}

\dispSFinmath{
\Muserfunction{Collect2}\big[\Mfunction{Expand}\big[{{(a-b-c)}^3}\big],a,\Mvariable{Factoring}\rightarrow \Mvariable{fun}\big]
}

\dispSFoutmath{
\Muserfunction{fun}(1)\multsp {a^3}+\Muserfunction{fun}(-3\multsp b-3\multsp c)\multsp {a^2}+
   \Muserfunction{fun}(3\multsp {b^2}+6\multsp c\multsp b+3\multsp {c^2})\multsp a+
   \Muserfunction{fun}(-{b^3}-3\multsp c\multsp {b^2}-3\multsp {c^2}\multsp b-{c^3})
}

\dispSFinmath{
\%\multsp /.\multsp \Mvariable{fun}\rightarrow \Mvariable{FactorTerms}
}

\dispSFoutmath{
{a^3}-3\multsp (b+c)\multsp {a^2}+3\multsp ({b^2}+2\multsp c\multsp b+{c^2})\multsp a-{b^3}-{c^3}-3\multsp b\multsp {c^2}-
   3\multsp {b^2}\multsp c
}

\Subsection*{Collect3}

\Subsubsection*{Description}

Collect3[expr, \{x, y, ...\}] collects terms involving the same powers of monomials \({x^{{n_1}}}\)\({y^{{n_2}}}\) ... An option Factor \(\rightarrow
\) True/False can be { }given, which factors the coefficients. The option Head (default Plus)
  determines the applied function to the list of monomials { }mulitplied by their coefficients.

\dispSFinmath{
\Mfunction{Options}[\Mvariable{Collect3}]
}

\dispSFoutmath{
\{\Mvariable{Factor}\rightarrow \Mvariable{False},\Mvariable{Head}\rightarrow \Mvariable{Plus}\}
}

See also:  Collect2, Isolate.

\Subsubsection*{Examples}

\dispSFinmath{
\Muserfunction{Collect3}\big[2\multsp a\multsp (b-a)\multsp (h-1)-{b^2}\multsp (e\multsp a-c)+{b^2},\{a,b\}\big]
}

\dispSFoutmath{
(2-2\multsp h)\multsp {a^2}-{b^2}\multsp e\multsp a+b\multsp (2\multsp h-2)\multsp a+{b^2}\multsp (c+1)
}

\dispSFinmath{
\Muserfunction{Collect3}\big[\Mfunction{Expand}\big[{{(a-b-c-d)}^5}\big],\{a\}\big]
}

\dispSFoutmath{
\MathBegin{MathArray}{l}
{a^5}+(-5\multsp b-5\multsp c-5\multsp d)\multsp {a^4}+
   (10\multsp {b^2}+20\multsp c\multsp b+20\multsp d\multsp b+10\multsp {c^2}+10\multsp {d^2}+20\multsp c\multsp d)\multsp {a^3}+  \\
   \noalign{\vspace{0.604167ex}}
\hspace{1.em} (-10\multsp {b^3}-30\multsp c\multsp {b^2}-30\multsp d\multsp {b^2}-
      30\multsp {c^2}\multsp b-30\multsp {d^2}\multsp b-60\multsp c\multsp d\multsp b-10\multsp {c^3}-10\multsp {d^3}-
      30\multsp c\multsp {d^2}-30\multsp {c^2}\multsp d)\multsp {a^2}+  \\
\noalign{\vspace{0.604167ex}}
\hspace{1.em} (
   5\multsp {b^4}+20\multsp c\multsp {b^3}+20\multsp d\multsp {b^3}+30\multsp {c^2}\multsp {b^2}+30\multsp {d^2}\multsp {b^2}+
    60\multsp c\multsp d\multsp {b^2}+20\multsp {c^3}\multsp b+20\multsp {d^3}\multsp b+  \\
\noalign{\vspace{0.604167ex}}
   \hspace{4.em} 60\multsp c\multsp {d^2}\multsp b+60\multsp {c^2}\multsp d\multsp b+5\multsp {c^4}+5\multsp {d^4}+
      20\multsp c\multsp {d^3}+30\multsp {c^2}\multsp {d^2}+20\multsp {c^3}\multsp d)\multsp a-{b^5}-{c^5}-{d^5}-5\multsp b\multsp {c^4}-
    \\
\noalign{\vspace{0.604167ex}}
\hspace{1.em} 5\multsp b\multsp {d^4}-5\multsp c\multsp {d^4}-10\multsp {b^2}\multsp {c^3}-
   10\multsp {b^2}\multsp {d^3}-10\multsp {c^2}\multsp {d^3}-20\multsp b\multsp c\multsp {d^3}-10\multsp {b^3}\multsp {c^2}-
   10\multsp {b^3}\multsp {d^2}-10\multsp {c^3}\multsp {d^2}-  \\
\noalign{\vspace{0.604167ex}}
\hspace{1.em} 30\multsp b\multsp {c^2}
    \multsp {d^2}-30\multsp {b^2}\multsp c\multsp {d^2}-5\multsp {b^4}\multsp c-5\multsp {b^4}\multsp d-5\multsp {c^4}\multsp d-
   20\multsp b\multsp {c^3}\multsp d-30\multsp {b^2}\multsp {c^2}\multsp d-20\multsp {b^3}\multsp c\multsp d\\
\MathEnd{MathArray}
}

\Subsection*{Combinations}

\Subsubsection*{Description}

Combinations[l, n] returns a list of all possible sets containing n elements from the list l. (this function is probably in the
  combinatorics package, but we have enough in memory already).

\Subsubsection*{Examples}

\dispSFinmath{
\Muserfunction{Combinations}[\{a,b,c,d\},3]//\Mfunction{TableForm}
}

\dispSFoutmath{
\Muserfunction{Combinations}(\{a,b,c,d\},3)
}

\Subsection*{Combine}

\Subsubsection*{Description}

Combine[expr]puts terms in a sum over a common denominator,and cancels factors in the result. Combine is similar to Together, but accepts
  the option Expanding and works usually better than Together for polynomials involving rationals with sums in the denominator.

\dispSFinmath{
\Mfunction{Options}[\Mvariable{Combine}]
}

\dispSFoutmath{
\{\Mvariable{Expanding}\rightarrow \Mvariable{False}\}
}

See also:  Factor2.

\Subsubsection*{Examples}

\dispSFinmath{
\Muserfunction{Combine}\big[\frac{(a-b)\multsp (c-d)}{e}+g\big]
}

\dispSFoutmath{
\frac{(a-b)\multsp (c-d)+e\multsp g}{e}
}

Here the result from Together where the numerator is automatically expanded.

\dispSFinmath{
\Mfunction{Together}\big[\frac{(a-b)\multsp (c-d)}{e}+g\big]
}

\dispSFoutmath{
\frac{a\multsp c-b\multsp c-a\multsp d+b\multsp d+e\multsp g}{e}
}

If the option Expanding is set to True, the result of Combine is the same as Together, but uses a slightly different algorithm.

\dispSFinmath{
\Muserfunction{Combine}\big[\frac{(a-b)\multsp (c-d)}{e}+g,\Mvariable{Expanding}\rightarrow \Mvariable{True}\big]
}

\dispSFoutmath{
\frac{a\multsp c-b\multsp c-a\multsp d+b\multsp d+e\multsp g}{e}
}

\Subsection*{CombineGraphs}

\Subsubsection*{Description}

CombineGraphs is an option for OneLoopSum.

See also: { }OneLoopSum.

\Subsection*{Commutator}

\Subsubsection*{Description}

Commutator[x, y] \(=\) c defines the commutator between the non-commuting objects x and y.

See also:  AntiCommutator, CommutatorExplicit, DeclareNonCommutative, DotSimplify.

\Subsubsection*{Examples}

\dispSFinmath{
\Muserfunction{DeclareNonCommutative}[a,b,c,d]
}

\dispSFinmath{
\Muserfunction{Commutator}[a,b]
}

\dispSFoutmath{
[\NoBreak a\NoBreak ,b\NoBreak ]
}

\dispSFinmath{
\Muserfunction{CommutatorExplicit}[\%]
}

\dispSFoutmath{
a.b-b.a
}

\dispSFinmath{
\Muserfunction{DotSimplify}[\Muserfunction{Commutator}[a+b,c+d]]\multsp
}

\dispSFoutmath{
a.c+a.d+b.c+b.d-c.a-c.b-d.a-d.b
}

\dispSFinmath{
\Muserfunction{UnDeclareNonCommutative}[a,b,c,d]
}

Verify the Jacobi identity.

\dispSFinmath{
\chi =\Mvariable{Commutator};\multsp \Muserfunction{DeclareNonCommutative}[x,y,z];
}

\dispSFinmath{
\chi [x,\chi [y,z]]+\chi [y,\chi [z,x]]+\chi [z,\chi [x,y]]
}

\dispSFoutmath{
[\NoBreak x\NoBreak ,[\NoBreak y\NoBreak ,z\NoBreak ]\NoBreak ]+[\NoBreak y\NoBreak ,[\NoBreak z\NoBreak ,x\NoBreak ]\NoBreak ]+
   [\NoBreak z\NoBreak ,[\NoBreak x\NoBreak ,y\NoBreak ]\NoBreak ]
}

\dispSFinmath{
\Muserfunction{DotSimplify}[\%]
}

\dispSFoutmath{
0
}

\dispSFinmath{
\Mfunction{Clear}[\chi ]
}

\dispSFinmath{
\Muserfunction{UnDeclareNonCommutative}[x,y,z]
}

\Subsection*{CommutatorExplicit}

\Subsubsection*{Description}

CommutatorExplicit[exp] substitutes any Commutator and AntiCommutator in exp by their definitions.

See also:  Calc, DotSimplify.

\Subsubsection*{Examples}

\dispSFinmath{
\Muserfunction{DeclareNonCommutative}[a,b,c,d]
}

\dispSFinmath{
\Muserfunction{Commutator}[a,b]
}

\dispSFoutmath{
[\NoBreak a\NoBreak ,b\NoBreak ]
}

\dispSFinmath{
\Muserfunction{CommutatorExplicit}[\%]
}

\dispSFoutmath{
a.b-b.a
}

\dispSFinmath{
\Muserfunction{AntiCommutator}[a-c,b-d]
}

\dispSFoutmath{
\{a-c,\> b-d\}
}

\dispSFinmath{
\Muserfunction{CommutatorExplicit}[\%]
}

\dispSFoutmath{
(a-c).(b-d)+(b-d).(a-c)
}

\dispSFinmath{
\Muserfunction{CommutatorExplicit}[\%\%]//\Muserfunction{DotSimplify}
}

\dispSFoutmath{
a.b-a.d+b.a-b.c-c.b+c.d-d.a+d.c
}

\dispSFinmath{
\Muserfunction{UnDeclareNonCommutative}[a,b,c,d]
}

\Subsection*{CompleteSquare}

\Subsubsection*{Description}

Completes the square of a second order polynomial in the momentum x. CompleteSquare[a \({p^2}\)\(+\)b p\(+\)c, p] \(\rightarrow \) -\({b^2}\)/(4
a)\(+\)c\(+\)a (b/(2 a)\(+\)x)\(\RawWedge\)2. CompleteSquare[a \({p^2}\)\(+\)b p\(+\)c, p, q] \(\rightarrow \) \{-\({b^2}\)/(4 a)\(+\)c\(+\)a \({q^2}\),
q\(\rightarrow \)b/(2 a)\(+\)p\}.

\Subsubsection*{Examples}

\dispSFinmath{
\Mvariable{t1}=5\Muserfunction{SP}[2p+3r,p+r]//\Muserfunction{FCI}
}

\dispSFoutmath{
5\multsp (p+r)\cdot (2\multsp p+3\multsp r)
}

\dispSFinmath{
\Mvariable{t2}=\Muserfunction{CompleteSquare}[\Mvariable{t1},p]
}

\dispSFoutmath{
10\multsp \Big(p+\frac{5\multsp r}{4}\Big).\Big(p+\frac{5\multsp r}{4}\Big)-\frac{5\multsp {r^2}}{8}
}

\dispSFinmath{
\Mvariable{t1}-\Mvariable{t2}//\Muserfunction{ScalarProductExpand}//\Mfunction{Expand}
}

\dispSFoutmath{
0
}

\dispSFinmath{
\Muserfunction{CompleteSquare}[\Mvariable{t1},p,q]
}

\dispSFoutmath{
\big\{10\multsp {q^2}-\frac{5\multsp {r^2}}{8},q\rightarrow p+\frac{5\multsp r}{4}\big\}
}

\dispSFinmath{
\Mfunction{Clear}[\Mvariable{t1},\Mvariable{t2}]
}

\Subsection*{ComplexConjugate}

\Subsubsection*{Description}

ComplexConjugate[expr] complex conjugates expr. It operates on fermion lines, i.e., products of Spinor[..] .DiracMatrix[..] . Spinor[..],
  and changes all occuring LorentzIndex[mu] into LorentzIndex[ComplexIndex[mu]]. For taking the spin sum (i.e. constructing the traces)
  use FermionSpinSum. WARNING: In expr should be NO explicit I's in denominators!

See also:  ComplexIndex, FermionSpinSum, LorentzIndex.

\Subsubsection*{Examples}

\dispSFinmath{
\Muserfunction{ComplexConjugate}[\Muserfunction{MetricTensor}[\mu ,\nu ]]
}

\dispSFoutmath{
{g^{\mu \nu }}
}

\dispSFinmath{
\Mfunction{StandardForm}[\%]
}

\dispSFoutmath{
\Muserfunction{Pair}[\Muserfunction{LorentzIndex}[\mu ],\Muserfunction{LorentzIndex}[\nu ]]
}

\dispSFinmath{
\Muserfunction{GA}[\mu ,\nu ,5]
}

\dispSFoutmath{
{{\gamma }^{\mu }}.{{\gamma }^{\nu }}.{{\gamma }^5}
}

\dispSFinmath{
\Muserfunction{ComplexConjugate}[\%]
}

\dispSFoutmath{
-{{\gamma }^5}.{{\gamma }^{\nu }}.{{\gamma }^{\mu }}
}

\dispSFinmath{
\Muserfunction{SUNTrace}[\Muserfunction{SUNT}[a,b,c]]
}

\dispSFoutmath{
\Muserfunction{tr}({T_a}.{T_b}.{T_c})
}

\dispSFinmath{
\Muserfunction{ComplexConjugate}[\%]
}

\dispSFoutmath{
\Muserfunction{tr}({T_c}.{T_b}.{T_a})
}

\dispSFinmath{
\Muserfunction{ComplexConjugate}[\Muserfunction{SUNF}[a,b,c]]
}

\dispSFoutmath{
{f_{abc}}
}

\dispSFinmath{
\Mfunction{StandardForm}[\Muserfunction{FCE}[\Muserfunction{ComplexConjugate}[\Muserfunction{MetricTensor}[\mu ,\nu ]]]]
}

\dispSFoutmath{
\Muserfunction{MT}[\mu ,\nu ]
}

\dispSFinmath{
\Muserfunction{SpinorUBar}[\Mvariable{k1},m].\Muserfunction{GA}[\lambda ].\Muserfunction{SpinorU}[\Mvariable{p1},m]
}

\dispSFoutmath{
\overvar{u}{\_}(\Mvariable{k1},m).{{\gamma }^{\lambda }}.u({p_1},m)
}

\dispSFinmath{
\Muserfunction{ComplexConjugate}[\%]
}

\dispSFoutmath{
\varphi ({p_1},m).{{\gamma }^{\lambda }}.\varphi (\Mvariable{k1},m)
}

Notice that SpinorUBar and SpinorU are only input functions. Internally they are converted to Spinor objects.

\Subsection*{Contract}

\Subsubsection*{Description}

Contract[expr] contracts pairs of Lorentz indices of metric tensors, four-vectors and (depending on the optionEpsContract) of Levi-Civita
  tensors in expr. For the contraction of Dirac matrices with each other use DiracSimplify. Contract[exp1, exp2] contracts (exp1*exp2),
  where exp1 and exp2 may be larger products of sums of metric tensors and 4-vectors.

\dispSFinmath{
\Mfunction{Options}[\Mvariable{Contract}]
}

\dispSFoutmath{
\MathBegin{MathArray}{l}
\{\Mvariable{Collecting}\rightarrow \Mvariable{True},\Mvariable{Contract3}\rightarrow \Mvariable{False},
    \Mvariable{EpsContract}\rightarrow \Mvariable{True},\Mvariable{Expanding}\rightarrow \Mvariable{True},
    \Mvariable{Factoring}\rightarrow \Mvariable{False},  \\
\noalign{\vspace{0.666667ex}}
\hspace{1.em} \Mvariable{FeynCalcInternal}
     \rightarrow \Mvariable{False},\Mvariable{MomentumCombine}\rightarrow \Mvariable{False},
    \Mvariable{Rename}\rightarrow \Mvariable{False},\Mvariable{Schouten}\rightarrow 0\}\\
\MathEnd{MathArray}
}

The option setting Contract3 can be set to True resulting in a faster algorithm to contract products of tensors.

See also:  Pair, DiracSimplify, MomentumCombine.

\Subsubsection*{Examples}

\dispSFinmath{
\Muserfunction{MetricTensor}[\mu ,\nu ]\multsp \Muserfunction{FourVector}[p,\mu ]
}

\dispSFoutmath{
{p_{\mu }}\multsp {g^{\mu \nu }}
}

\dispSFinmath{
\Muserfunction{Contract}[\%]
}

\dispSFoutmath{
{p^{\nu }}
}

\dispSFinmath{
\Muserfunction{FourVector}[p,\mu ]\Muserfunction{DiracMatrix}[\mu ]
}

\dispSFoutmath{
{{\gamma }^{\mu }}\multsp {p_{\mu }}
}

\dispSFinmath{
\Muserfunction{Contract}[\%]
}

\dispSFoutmath{
\gamma \cdot p
}

\dispSFinmath{
\Muserfunction{MetricTensor}[\mu ,\mu ]
}

\dispSFoutmath{
{g^{\mu \mu }}
}

The default dimension for MetricTensor is 4.

\dispSFinmath{
\Muserfunction{Contract}[\%]
}

\dispSFoutmath{
4
}

A short way to enter D-dimensional metric tensors is given by MTD. The "." as multiplication operator is not necessary but just
  convenient for typesetting.

\dispSFinmath{
\Muserfunction{MTD}[\mu ,\nu ]\multsp .\multsp \Muserfunction{MTD}[\mu ,\nu ]
}

\dispSFoutmath{
{g^{\mu \nu }}.{g^{\mu \nu }}
}

\dispSFinmath{
\Muserfunction{Contract}[\%]
}

\dispSFoutmath{
D
}

\dispSFinmath{
\Muserfunction{MTD}[\mu ,\nu ]\multsp .\multsp \Muserfunction{MTD}[\mu ,\nu ]
}

\dispSFoutmath{
{g^{\mu \nu }}.{g^{\mu \nu }}
}

\dispSFinmath{
\Muserfunction{FourVector}[p,\mu ]\multsp \Muserfunction{FourVector}[q,\mu ]
}

\dispSFoutmath{
{p_{\mu }}\multsp {q_{\mu }}
}

\dispSFinmath{
\Muserfunction{Contract}[\%\multsp ]
}

\dispSFoutmath{
p\cdot q
}

\dispSFinmath{
\Muserfunction{FourVector}[p-q,\mu ]\multsp \Muserfunction{FourVector}[a-b,\mu ]
}

\dispSFoutmath{
{{(a-b)}_{\mu }}\multsp {{(p-q)}_{\mu }}
}

\dispSFinmath{
\Muserfunction{Contract}[\%]
}

\dispSFoutmath{
a\cdot p-a\cdot q-b\cdot p+b\cdot q
}

\dispSFinmath{
\Muserfunction{LeviCivita}[\mu ,\nu ,\alpha ,\sigma ]\multsp \Muserfunction{FourVector}[p,\sigma ]
}

\dispSFoutmath{
{{\epsilon }^{\mu \nu \alpha \sigma }}\multsp {p_{\sigma }}
}

\dispSFinmath{
\Muserfunction{Contract}[\%]
}

\dispSFoutmath{
{{\epsilon }^{\alpha \mu \nu p}}
}

\dispSFinmath{
\Muserfunction{LeviCivita}[\mu ,\nu ,\alpha ,\beta ]\multsp \Muserfunction{LeviCivita}[\mu ,\nu ,\alpha ,\sigma ]\multsp
}

\dispSFoutmath{
{{\epsilon }^{\mu \nu \alpha \beta }}\multsp {{\epsilon }^{\mu \nu \alpha \sigma }}
}

\dispSFinmath{
\Muserfunction{Contract}[\%]
}

\dispSFoutmath{
-6\multsp {g^{\beta \sigma }}
}

\dispSFinmath{
\Mfunction{SetOptions}[\Mvariable{Eps},\Mvariable{Dimension}\rightarrow D];
   \Muserfunction{LCD}[\mu ,\nu ,\alpha ,\beta ]\multsp \Muserfunction{LCD}[\mu ,\nu ,\alpha ,\sigma ]
}

\dispSFoutmath{
{{\epsilon }^{\mu \nu \alpha \beta }}\multsp {{\epsilon }^{\mu \nu \alpha \sigma }}
}

\dispSFinmath{
\Muserfunction{Contract}[\%]//\Muserfunction{Factor2}
}

\dispSFoutmath{
(1-D)\multsp (2-D)\multsp (3-D)\multsp {g^{\beta \sigma }}
}

\dispSFinmath{
\Mfunction{SetOptions}[\Mvariable{Eps},\Mvariable{Dimension}\rightarrow 4];
}

\Subsection*{Contract1}

\Subsubsection*{Description}

Contract1[exp] contracts Upper and Lower indices. Neither Upper and Upper nor Lower and Lower indices are contracted.

See also: { }LorentzIndex, Lower, Upper.

\Subsubsection*{Examples}

\dispSFinmath{
\Muserfunction{FV}[p,\Muserfunction{Lower}[\mu ]]\Muserfunction{FV}[q,\Muserfunction{Upper}[\mu ]]//\Muserfunction{FCI}//
   \Muserfunction{Contract1}
}

\dispSFoutmath{
p\cdot q
}

\dispSFinmath{
\Muserfunction{FV}[p,\Muserfunction{Upper}[\mu ]]\Muserfunction{FV}[q,\Muserfunction{Upper}[\mu ]]//\Muserfunction{FCI}//
   \Muserfunction{Contract1}
}

\dispSFoutmath{
{p^{\Muserfunction{Upper}(\mu )}}\multsp {q^{\Muserfunction{Upper}(\mu )}}
}

\dispSFinmath{
\Muserfunction{MT}[\Muserfunction{Lower}[\mu ],\Muserfunction{Upper}[\mu ]]//\Muserfunction{FCI}//\Muserfunction{Contract1}
}

\dispSFoutmath{
4
}

\dispSFinmath{
\Muserfunction{MT}[\Muserfunction{Upper}[\mu ],\Muserfunction{Upper}[\mu ]]//\Muserfunction{FCI}//\Muserfunction{Contract1}
}

\dispSFoutmath{
{g^{\Muserfunction{Upper}(\mu )\Muserfunction{Upper}(\mu )}}
}

\Subsection*{Convolute}

\Subsubsection*{Description}

Convolute[f, g, x] convolutes {\itshape f}({\itshape x}) and {\itshape g}({\itshape x}), i.e., \(\int _{0}^{1}{{dx}_1}\multsp \int _{0}^{1}{{dx}_2}\multsp
\delta (x\multsp -\multsp {x_1}\multsp {x_2})\multsp f({x_1})\multsp
      g({x_2})\multsp .\) Convolute[f, g] is equivalent to Convolute[f, g, x]. Convolute[exp, \{x1, x2\}] assumes that exp is polynomial in x1 and
x2. Convolute
  uses table-look-up and does not do any integral calculations, only linear algebra.

\dispSFinmath{
\Mfunction{Options}[\Mvariable{Convolute}]
}

\dispSFoutmath{
\{\Mvariable{Bracket}\rightarrow \{\varepsilon \},\Mvariable{FinalSubstitutions}\rightarrow
     \{\Mvariable{PlusDistribution}\rightarrow \Mvariable{Identity}\}\}
}

See also:  PlusDistribution, ConvoluteTable.

\Subsubsection*{Examples}

\dispSFinmath{
\Muserfunction{Convolute}[1,1]
}

\dispSFoutmath{
-\log(x)
}

\dispSFinmath{
\Muserfunction{Convolute}[x,x]
}

\dispSFoutmath{
-x\multsp \log(x)
}

\dispSFinmath{
\Muserfunction{Convolute}[1,x]
}

\dispSFoutmath{
1-x
}

\dispSFinmath{
\Muserfunction{Convolute}\big[1,\frac{1}{1-x}\big]
}

\dispSFoutmath{
\log(1-x)-\log(x)
}

\dispSFinmath{
\Muserfunction{Convolute}\big[1,\Muserfunction{PlusDistribution}\big[\frac{1}{1-x}\big]\big]
}

\dispSFoutmath{
\log(1-x)-\log(x)
}

\dispSFinmath{
\Muserfunction{Convolute}\big[\frac{1}{1-x},x\big]
}

\dispSFoutmath{
\log(1-x)\multsp x-\log(x)\multsp x-x+1
}

\dispSFinmath{
\Muserfunction{Convolute}\big[\frac{1}{1-x},\frac{1}{1-x}\big]
}

\dispSFoutmath{
-\zeta (2)\multsp \delta (1-x)+\frac{2\multsp \log(1-x)}{1-x}-\frac{\log(x)}{1-x}
}

\dispSFinmath{
\Muserfunction{Convolute}[1,\log [1-x]]
}

\dispSFoutmath{
-\log(1-x)\multsp \log(x)-{{\Mvariable{Li}}_2}(1-x)
}

\dispSFinmath{
\Muserfunction{Convolute}[1,x\multsp \log [1-x]]
}

\dispSFoutmath{
x+(1-x)\multsp \log(1-x)-1
}

\dispSFinmath{
\Muserfunction{Convolute}\big[\frac{1}{1-x},\log [1-x]\big]
}

\dispSFoutmath{
{{\log}^2}(1-x)-\log(x)\multsp \log(1-x)-\zeta (2)
}

\dispSFinmath{
\Muserfunction{Convolute}\big[\frac{1}{1-x},x\multsp \log [1-x]\big]
}

\dispSFoutmath{
x\multsp {{\log}^2}(1-x)+(1-x)\multsp \log(1-x)-x\multsp \log(x)\multsp \log(1-x)+x-x\multsp \zeta (2)-1
}

\dispSFinmath{
\Muserfunction{Convolute}\big[\frac{\log [1-x]}{1-x},x\big]
}

\dispSFoutmath{
\frac{1}{2}\multsp x\multsp {{\log}^2}(1-x)+(1-x)\multsp \log(1-x)-x\multsp \log(x)\multsp \log(1-x)+x\multsp \log(x)-
   x\multsp {{\Mvariable{Li}}_2}(1-x)
}

\dispSFinmath{
\Muserfunction{Convolute}[1,x\multsp \log [x]]
}

\dispSFoutmath{
-\log(x)\multsp x+x-1
}

\dispSFinmath{
\Muserfunction{Convolute}[\log [1-x],x]
}

\dispSFoutmath{
(1-x)\multsp \log(1-x)+x\multsp \log(x)
}

\dispSFinmath{
\Muserfunction{Convolute}\big[\frac{1}{1-x},\frac{\log [x]}{1-x}\big]
}

\dispSFoutmath{
\frac{\log(1-x)\multsp \log(x)}{1-x}-\frac{{{\log}^2}(x)}{2\multsp (1-x)}
}

\dispSFinmath{
\Muserfunction{Convolute}[1,\log [x]]
}

\dispSFoutmath{
-\frac{1}{2}\multsp {{\log}^2}(x)
}

\dispSFinmath{
\Muserfunction{Convolute}[x,\multsp x\multsp \log [x]]
}

\dispSFoutmath{
-\frac{1}{2}\multsp x\multsp {{\log}^2}(x)
}

\dispSFinmath{
\Muserfunction{Convolute}\big[\frac{1}{1-x},\log [x]\big]
}

\dispSFoutmath{
-\frac{1}{2}\multsp {{\log}^2}(x)+\log(1-x)\multsp \log(x)+{{\Mvariable{Li}}_2}(1-x)
}

\dispSFinmath{
\Muserfunction{Convolute}\big[1,\frac{\log [x]}{1-x}\big]
}

\dispSFoutmath{
-\frac{1}{2}\multsp {{\log}^2}(x)-{{\Mvariable{Li}}_2}(1-x)
}

\dispSFinmath{
\Muserfunction{Convolute}\big[\frac{1}{1-x},x\multsp \log [x]\big]
}

\dispSFoutmath{
-\frac{1}{2}\multsp x\multsp {{\log}^2}(x)-x\multsp \log(x)+x\multsp \log(1-x)\multsp \log(x)+x+x\multsp {{\Mvariable{Li}}_2}(1-x)-1
}

\dispSFinmath{
\Muserfunction{Convolute}\big[\frac{\log [x]}{1-x},x\big]
}

\dispSFoutmath{
-\frac{1}{2}\multsp x\multsp {{\log}^2}(x)+\log(x)-x-x\multsp {{\Mvariable{Li}}_2}(1-x)+1
}

\dispSFinmath{
\Muserfunction{Convolute}[1,x\multsp \log [x]]
}

\dispSFoutmath{
-\log(x)\multsp x+x-1
}

\dispSFinmath{
\Muserfunction{Convolute}[\log [x],x]
}

\dispSFoutmath{
-x+\log(x)+1
}

\dispSFinmath{
\Muserfunction{Convolute}\big[\frac{1}{1-x},\frac{\log [1-x]}{1-x}\big]
}

\dispSFoutmath{
\frac{3\multsp {{\log}^2}(1-x)}{2\multsp (1-x)}-\frac{\log(x)\multsp \log(1-x)}{1-x}-\frac{\zeta (2)}{1-x}+\delta (1-x)\multsp \zeta (3)
}

\Subsection*{ConvoluteTable ***unfinished***}

\Subsubsection*{Description}

ConvoluteTable[f, g, x] yields the convolution of f and g.

\dispSFinmath{
\Mfunction{Options}[\Mvariable{Convolute}]
}

\dispSFoutmath{
\{\Mvariable{Bracket}\rightarrow \{\varepsilon \},\Mvariable{FinalSubstitutions}\rightarrow
     \{\Mvariable{PlusDistribution}\rightarrow \Mvariable{Identity}\}\}
}

See also:  PlusDistribution, Convolute.

\Subsubsection*{Examples}

\dispSFinmath{
\Muserfunction{ConvoluteTable}[1,1]
}

\dispSFoutmath{
-\log(x)
}

\dispSFinmath{
\Muserfunction{ConvoluteTable}[x,x]
}

\dispSFoutmath{
-x\multsp \log(x)
}

\dispSFinmath{
\Muserfunction{ConvoluteTable}[1,x]
}

\dispSFoutmath{
1-x
}

\Subsection*{CovariantFieldDerivative}

\Subsubsection*{Description}

CovariantFieldDerivative[f[x],x,\{li1,li2,...\},opts] is a covariant derivative of f[x] with respect to space-time variables x and with
  Lorentz indices li1, li2,... CovariantFieldDerivative has only typesetting definitions by default. The user is must supply his/her own
  definition of the actual function.

See also:  CovariantD, ExpandPartialD, FieldDerivative.

\Subsubsection*{Examples}

\dispSFinmath{
\Muserfunction{CovariantFieldDerivative}[\Muserfunction{QuantumField}[A,\{\mu \}][x],x,\{\mu \}]
}

\dispSFoutmath{
\frac{1}{2}\multsp \ImaginaryI \multsp \Big(\Big(\overvar{{{{A^{}}}_{\mu }}}{\rightarrow }\cdot \overvar{\sigma }{\rightarrow }+
        \overvar{{{{V^{}}}_{\mu }}}{\rightarrow }\cdot \overvar{\sigma }{\rightarrow }\Big)\SixPointedStar {A_{\mu }}\Big)-
   \frac{1}{2}\multsp \ImaginaryI \multsp \Big({A_{\mu }}\SixPointedStar \Big(
      \overvar{{{{V^{}}}_{\mu }}}{\rightarrow }\cdot \overvar{\sigma }{\rightarrow }-
       \overvar{{{{A^{}}}_{\mu }}}{\rightarrow }\cdot \overvar{\sigma }{\rightarrow }\Big)\Big)+{{\partial }_{\mu }}A_{\mu }^{ }
}

\Subsection*{CounterT}

\Subsubsection*{Description}

CounterT is a factor used by GluonPropagator and QuarkPropagator when CounterTerms is set to All.

See also:  CounterTerm, GluonPropagator, QuarkPropagator.

\Subsubsection*{Examples}

\dispSFinmath{
\Muserfunction{GluonPropagator}[p,\mu ,a,\nu ,b,\Mvariable{Explicit}\rightarrow \Mvariable{True},
    \Mvariable{CounterTerm}\rightarrow \multsp \Mvariable{All}]
}

\dispSFoutmath{
\Mvariable{CounterT}\multsp \Bigg(\frac{\ImaginaryI \multsp {C_A}\multsp {S_n}\multsp
         \big(\frac{10\multsp {p^{\mu }}\multsp {p^{\nu }}}{3}-\frac{10}{3}\multsp {g^{\mu \nu }}\multsp {p^2}\big)\multsp
         {{\delta }_{ab}}\multsp g_{s}^{2}}{\varepsilon }+
      \frac{\ImaginaryI \multsp {S_n}\multsp {T_f}\multsp
         \big(\frac{4}{3}\multsp {g^{\mu \nu }}\multsp {p^2}-\frac{4\multsp {p^{\mu }}\multsp {p^{\nu }}}{3}\big)\multsp {{\delta }_{ab}}
         \multsp g_{s}^{2}}{\varepsilon }\Bigg)-\frac{\ImaginaryI \multsp {g^{\mu \nu }}\multsp {{\delta }_{ab}}}{{p^2}}
}

\Subsection*{CounterTerm}

\Subsubsection*{Description}

CounterTerm[name] is a database of counter terms. CounterTerm is also an option for the Feynman rule functions QuarkGluonVertex,
  GluonPropagator, QuarkPropagator.

See also:  CounterT, QuarkGluonVertex, GluonPropagator, QuarkPropagator.

\Subsubsection*{Examples}

\dispSFinmath{
\Muserfunction{CounterTerm}[\Mvariable{Zm}]
}

\dispSFoutmath{
\frac{{C_F}\multsp \bigg(\frac{4\multsp \big(\frac{11\multsp {C_A}}{2}+\frac{9\multsp {C_F}}{2}-2\multsp {N_f}\multsp {T_f}\big)}
         {{{\varepsilon }^2}}+\frac{2\multsp \Big(\frac{97\multsp {C_A}}{12}+\frac{3\multsp {C_F}}{4}-
             \frac{5\multsp {N_f}\multsp {T_f}}{3}\Big)}{\varepsilon }\bigg)\multsp g_{s}^{4}}{256\multsp {{\pi }^4}}+
   \frac{3\multsp {C_F}\multsp g_{s}^{2}}{8\multsp \varepsilon \multsp {{\pi }^2}}+1
}

\Subsection*{CouplingConstant}

\Subsubsection*{Description}

In general, CouplingConstant is an option for several Feynman rule functions and for CovariantD and FieldStrength.\\
In the convention of the subpackage PHI, CouplingConstant is also the head of coupling constants. { }CouplingConstant takes three extra
  optional arguments, with head RenormalizationState, RenormalizationScheme and ExpansionState respectively. { }E.g.
  CouplingConstant[QED[1]] is the unit charge, CouplingConstant[ChPT2[4],1] is the first of the coupling constants of the lagrangian
  ChPT2[4]. { }CouplingConstant[a\_{},b\_{},c\_{}\_{}\_{}][i\_{}] :\(=\) CouplingConstant[a,b,RenormalizationState[i],c].

See also:  CovariantD, FieldStrength.

\Subsection*{CovariantD}

\Subsubsection*{Description}

CovariantD[mu] is a generic covariant derivative with Lorentz index mu. With the option-setting Explicit \(\rightarrow \) True, an
  explicit expression for a fermionic field is returned, depending on the setting on the other options.\\
CovariantD[x, mu] is a generic covariant derivative with respect to x\(\RawWedge\)mu.\\
CovariantD[mu, a, b] is a covariant derivative for a bosonic field; acting on QuantumField[f,\{\},\{a,b\}], where f is some field name
  and a and b are two SU(N) indices. Again, with the option-setting Explicit \(\rightarrow \) True, an explicit expression is returned,
  depending on the setting on the other options.\\
CovariantD[OPEDelta, a, b] is a short form for { }CovariantD[mu,a,b]*FourVector[OPEDelta, mu]. CovariantD[\{OPEDelta, a, b\}, \{n\}]
  yields the product of n operators, where n is an integer. { }CovariantD[OPEDelta, a, b, \{m, n\}] { }gives the expanded form of
  CovariantD[OPEDelta, a, b]\(\RawWedge\)m up to order g\(\RawWedge\)n for the gluon, where n is an integer and g the coupling constant {
  }indicated by the setting of the option CouplingConstant. CovariantD[OPEDelta, \{m, n\}] gives the expanded form of {
  }CovariantD[OPEDelta]\(\RawWedge\)m up to order g\(\RawWedge\)n of the fermionic field.

\dispSFinmath{
\Mfunction{Options}[\Mvariable{CovariantD}]
}

\dispSFoutmath{
\MathBegin{MathArray}{l}
\{\Mvariable{CouplingConstant}\rightarrow {g_s},\Mvariable{DummyIndex}\rightarrow \Mvariable{Automatic},  \\
   \noalign{\vspace{0.666667ex}}
\hspace{1.em} \Mvariable{Explicit}\rightarrow \Mvariable{False},
    \Mvariable{PartialD}\rightarrow \Mvariable{RightPartialD},\Mvariable{QuantumField}\rightarrow A\}\\
\MathEnd{MathArray}
}

 Possible settings of PartialD are: LeftPartialD, LeftRigthPartialD, RightPartialD. The default setting of QuantumField is GaugeField.

See also:  LeftPartialD, LeftRightPartialD, RightPartialD.

\Subsubsection*{Examples}

\dispSFinmath{
\Muserfunction{CovariantD}[\mu ]
}

\dispSFoutmath{
{D_{\mu }}
}

\dispSFinmath{
\Muserfunction{CovariantD}[\mu ,a,b]
}

\dispSFoutmath{
D_{\mu }^{ab}
}

\dispSFinmath{
\Muserfunction{CovariantD}[\mu ,\Mvariable{Explicit}\rightarrow \Mvariable{True}]
}

\dispSFoutmath{
{{\left( \overvar{\partial }{\rightarrow } \right) }_{\mu }}-\ImaginaryI \multsp {g_s}\multsp {T_{{c_1}}}.A_{\mu }^{{c_1}}
}

The first argument of CovariantD is intepreted as type LorentzIndex, except for OPEDelta, which is type Momentum.

\dispSFinmath{
\Muserfunction{CovariantD}[\Mvariable{OPEDelta}]
}

\dispSFoutmath{
{D_{\Delta }}
}

\dispSFinmath{
\Muserfunction{CovariantD}[\Mvariable{OPEDelta},a,b]
}

\dispSFoutmath{
D_{\Delta }^{ab}
}

\dispSFinmath{
\Muserfunction{CovariantD}[\Mvariable{OPEDelta},a,b,\Mvariable{Explicit}\rightarrow \Mvariable{True}]
}

\dispSFoutmath{
{{\left( \overvar{\partial }{\rightarrow } \right) }_{\Delta }}\multsp {{\delta }_{ab}}-
   {g_s}\multsp A_{\Delta }^{{c_2}}\multsp {f_{ab{c_2}}}
}

\dispSFinmath{
\Muserfunction{CovariantD}[\Mvariable{OPEDelta},\Mvariable{Explicit}\rightarrow \Mvariable{True}]
}

\dispSFoutmath{
{{\left( \overvar{\partial }{\rightarrow } \right) }_{\Delta }}-\ImaginaryI \multsp {g_s}\multsp {T_{{c_3}}}.A_{\Delta }^{{c_3}}
}

\dispSFinmath{
\Muserfunction{CovariantD}[\Mvariable{OPEDelta},a,b,\{2\}]
}

\dispSFoutmath{
\big({{\left( \overvar{\partial }{\rightarrow } \right) }_{\Delta }}\multsp {{\delta }_{a\Mvariable{c9}}}-
     {g_s}\multsp A_{\Delta }^{\Mvariable{e1}}\multsp {f_{a\Mvariable{c9}\Mvariable{e1}}}\big).
   \big({{\left( \overvar{\partial }{\rightarrow } \right) }_{\Delta }}\multsp {{\delta }_{b\Mvariable{c9}}}-
     {g_s}\multsp A_{\Delta }^{\Mvariable{e2}}\multsp {f_{\Mvariable{c9}b\Mvariable{e2}}}\big)
}

This gives\(\multsp m\multsp \Mvariable{times}\multsp {{\left( \overvar{\partial }{\rightarrow } \right) }_{\Delta }},\multsp \)the partial derivative
\({{\left( \overvar{\partial }{\rightarrow } \right) }_{\mu \multsp }}\)contracted with \({{\Delta }^{\mu }}.\)

\dispSFinmath{
\Muserfunction{CovariantD}[\Mvariable{OPEDelta},a,b,\{\Mvariable{OPEm},0\}]
}

\dispSFoutmath{
{{({{\left( \overvar{\partial }{\rightarrow } \right) }_{\Delta }})}^m}\multsp {{\delta }_{ab}}
}

The expansion up to first order in the coupling constant \({g_s}\multsp (\Mvariable{the}\Mvariable{sum}\Mvariable{is}\Mvariable{the}\Mvariable{FeynCalc}\Mvariable{OPESum}).\)

\dispSFinmath{
\Muserfunction{CovariantD}[\Mvariable{OPEDelta},a,b,\{\Mvariable{OPEm},1\}]
}

\dispSFoutmath{
{{({{\left( \overvar{\partial }{\rightarrow } \right) }_{\Delta }})}^m}\multsp {{\delta }_{ab}}-
   {g_s}\multsp \sum _{i=0}^{m-1}{{({{\left( \overvar{\partial }{\rightarrow } \right) }_{\Delta }})}^i}.A_{\Delta }^{{c_1}}.
       {{({{\left( \overvar{\partial }{\rightarrow } \right) }_{\Delta }})}^{-i+m-1}}\multsp {f_{ab{c_1}}}
}

The expansion up to second order in the \({g_s}.\)

\dispSFinmath{
\Muserfunction{CovariantD}[\Mvariable{OPEDelta},a,b,\{\Mvariable{OPEm},2\}]
}

\dispSFoutmath{
{{\delta }_{ab}}\multsp {{({{\left( \overvar{\partial }{\rightarrow } \right) }_{\Delta }})}^m}-
   g_{s}^{2}\multsp \sum _{j=0}^{m-2}\sum _{i=0}^{j}{{({{\left( \overvar{\partial }{\rightarrow } \right) }_{\Delta }})}^i}.
        A_{\Delta }^{{c_1}}.{{({{\left( \overvar{\partial }{\rightarrow } \right) }_{\Delta }})}^{j-i}}.A_{\Delta }^{{c_2}}.
        {{({{\left( \overvar{\partial }{\rightarrow } \right) }_{\Delta }})}^{-j+m-2}}\multsp {f_{a{c_1}{e_1}}}\multsp {f_{b{c_2}{e_1}}}-
   {g_s}\multsp \sum _{i=0}^{m-1}{{({{\left( \overvar{\partial }{\rightarrow } \right) }_{\Delta }})}^i}.A_{\Delta }^{{c_1}}.
       {{({{\left( \overvar{\partial }{\rightarrow } \right) }_{\Delta }})}^{-i+m-1}}\multsp {f_{ab{c_1}}}
}

\dispSFinmath{
{{\Muserfunction{CovariantD}[\Mvariable{OPEDelta},a,b]}^{\Mvariable{OPEm}}}
}

\dispSFoutmath{
{{\big(D_{\Delta }^{ab}\big)}^m}
}

\dispSFinmath{
\Muserfunction{CovariantD}[\Mvariable{OPEDelta},\{\Mvariable{OPEm},2\}]
}

\dispSFoutmath{
{{({{\left( \overvar{\partial }{\rightarrow } \right) }_{\Delta }})}^m}-
   \ImaginaryI \multsp {g_s}\multsp \sum _{i=0}^{m-1}{T_{{c_1}}}.{{({{\left( \overvar{\partial }{\rightarrow } \right) }_{\Delta }})}^i}.
      A_{\Delta }^{{c_1}}.{{({{\left( \overvar{\partial }{\rightarrow } \right) }_{\Delta }})}^{-i+m-1}}-
   g_{s}^{2}\multsp \sum _{j=0}^{m-2}\sum _{i=0}^{j}{T_{{c_1}}}.{T_{{c_2}}}.
       {{({{\left( \overvar{\partial }{\rightarrow } \right) }_{\Delta }})}^i}.A_{\Delta }^{{c_1}}.
       {{({{\left( \overvar{\partial }{\rightarrow } \right) }_{\Delta }})}^{j-i}}.A_{\Delta }^{{c_2}}.
       {{({{\left( \overvar{\partial }{\rightarrow } \right) }_{\Delta }})}^{-j+m-2}}
}

\dispSFinmath{
\Muserfunction{CovariantD}[\Mvariable{OPEDelta},\Mvariable{Explicit}\rightarrow \Mvariable{True}]//\Mfunction{StandardForm}
}

\dispSFoutmath{
\MathBegin{MathArray}{l}
-\ImaginaryI \multsp \Mvariable{Gstrong}\multsp   \\
\noalign{\vspace{0.5ex}}
\hspace{2.em} \Muserfunction{SUNT}
      [\Muserfunction{SUNIndex}[{c_4}]].\Muserfunction{QuantumField}[
      \Mvariable{GaugeField},\Muserfunction{Momentum}[\Mvariable{OPEDelta}],\Muserfunction{SUNIndex}[{c_4}]]+  \\
   \noalign{\vspace{0.5ex}}
\hspace{1.em} \Muserfunction{RightPartialD}[\Muserfunction{Momentum}[\Mvariable{OPEDelta}]]\\
   \MathEnd{MathArray}
}

\dispSFinmath{
\Muserfunction{CovariantD}[\mu ,a,b,\Mvariable{Explicit}\rightarrow \Mvariable{True}]//\Mfunction{StandardForm}
}

\dispSFoutmath{
\MathBegin{MathArray}{l}
\Muserfunction{RightPartialD}[\Muserfunction{LorentzIndex}[\mu ]]\multsp \Muserfunction{SUNDelta}[a,b]-  \\
   \noalign{\vspace{0.5ex}}
\hspace{1.em} \Mvariable{Gstrong}\multsp
   \Muserfunction{QuantumField}[\Mvariable{GaugeField},\Muserfunction{LorentzIndex}[\mu ],\Muserfunction{SUNIndex}[{c_5}]]\multsp
   \Muserfunction{SUNF}[a,b,{c_5}]\\
\MathEnd{MathArray}
}

\Subsection*{CrossProduct}

\Subsubsection*{Description}

CrossProduct[a, b] denotes the three-dimensional cross-product of the three-vectors a and b.

See also:  DotProduct, ThreeVector.

\Subsubsection*{Examples}

\dispSFinmath{
\Muserfunction{CrossProduct}[\Muserfunction{ThreeVector}[a],
    \Muserfunction{CrossProduct}[\Muserfunction{ThreeVector}[b],\Muserfunction{ThreeVector}[c]]]
}

\dispSFoutmath{
\overvar{a}{\rightharpoonup }\cdot \overvar{c}{\rightharpoonup }\multsp \overvar{b}{\rightharpoonup }-
   \overvar{a}{\rightharpoonup }\cdot \overvar{b}{\rightharpoonup }\multsp \overvar{c}{\rightharpoonup }
}

\Subsection*{C0}

\Subsubsection*{Description}

C0[p10, p12, p20, m1\(\RawWedge\)2, m2\(\RawWedge\)2, m3\(\RawWedge\)2] is the scalar Passarino-Veltman \({C_0}\)function. The convention for the
arguments is that if the denominator of the integrand has the form ([q\(\RawWedge\)2-m1\(\RawWedge\)2]
  [(q\(+\)p1)\(\RawWedge\)2-m2\(\RawWedge\)2] [(q\(+\)p2)\(\RawWedge\)2-m3\(\RawWedge\)2]), the first three arguments of C0 are the
  scalar products p10 \(=\) p1\(\RawWedge\)2, p12 \(=\) (p1-p2).(p1-p2), p20 \(=\) p2\(\RawWedge\)2.

See also:  B0, D0, PaVe, PaVeOrder.

\Subsubsection*{Examples}

\dispSFinmath{
\Muserfunction{C0}[a,b,c,\multsp \Mvariable{m12},\Mvariable{m22},\Mvariable{m32}]
}

\dispSFoutmath{
{C_0}(a,b,c,\Mvariable{m12},\Mvariable{m22},\Mvariable{m32})
}

\dispSFinmath{
\Muserfunction{C0}[b,a,c,\Mvariable{m32},\Mvariable{m22},\Mvariable{m12}]//\Muserfunction{PaVeOrder}
}

\dispSFoutmath{
{C_0}(a,b,c,\Mvariable{m12},\Mvariable{m22},\Mvariable{m32})
}

\dispSFinmath{
\Muserfunction{PaVeOrder}[\Muserfunction{C0}[b,a,c,\Mvariable{m32},\Mvariable{m22},\Mvariable{m12}],
    \Mvariable{PaVeOrderList}\rightarrow \{c,a\}]
}

\dispSFoutmath{
{C_0}(b,c,a,\Mvariable{m22},\Mvariable{m32},\Mvariable{m12})
}

\Subsection*{DataType}

\Subsubsection*{Description}

DataType[exp, type] \(=\) True defines the object exp to have data-type type. DataType[exp1, exp2, ..., type] defines the objects exp1,
  exp2, ...to have data-type type. The default setting is DataType[\_{}\_{}, \_{}] :\(=\) False. To assign a certain data-type, do, e.g.,
  DataType[x, PositiveInteger] \(=\) True.

Currently used DataTypes: NonCommutative, PositiveInteger, NegativeInteger, PositiveNumber, FreeIndex, GrassmannParity

PHI adds the DataTypes: UMatrix, UScalar.

See also:  DeclareNonCommutative.

\Subsubsection*{Examples}

NonCommutative is just a data-type.

\dispSFinmath{
\Muserfunction{DataType}[f,g,\multsp \Mvariable{NonCommutative}]\multsp =\multsp \Mvariable{True};
}

\dispSFinmath{
t=f.g-g.(2a).f
}

\dispSFoutmath{
f.g-g.(2\multsp a).f
}

Since "f "and "g" have DataType NonCommutative the function DotSimplify extracts only "a" out of the noncommutative product.

\dispSFinmath{
\Muserfunction{DotSimplify}[t]
}

\dispSFoutmath{
f.g-2\multsp a\multsp g.f
}

\dispSFinmath{
\Muserfunction{DataType}[m,\Mvariable{odd}]=\Muserfunction{DataType}[a,\Mvariable{even}]=\Mvariable{True};
}

\dispSFinmath{
\Muserfunction{ptest1}[\Mvariable{x\_}]:=x/.{{(-1)}^{\Mvariable{n\_}}}/;\Muserfunction{DataType}[n,\Mvariable{odd}]\RuleDelayed -1;
}

\dispSFinmath{
\Muserfunction{ptest2}[\Mvariable{x\_}]:=x/.{{(-1)}^{\Mvariable{n\_}}}/;\Muserfunction{DataType}[n,\Mvariable{even}]\RuleDelayed 1;
}

\dispSFinmath{
t=(-1)\RawWedge m+(-1)\RawWedge a+(-1)\RawWedge z
}

\dispSFoutmath{
{{(-1)}^a}+{{(-1)}^m}+{{(-1)}^z}
}

\dispSFinmath{
\Muserfunction{ptest1}[t]
}

\dispSFoutmath{
-1+{{(-1)}^a}+{{(-1)}^z}
}

\dispSFinmath{
\Muserfunction{ptest2}[\%]
}

\dispSFoutmath{
{{(-1)}^z}
}

\dispSFinmath{
\Mfunction{Clear}[\Mvariable{ptest1},\Mvariable{ptest2},t,a,m];
}

\dispSFinmath{
\Muserfunction{DataType}[m,\Mvariable{ganzeZahl}]=\Mvariable{True};
}

\dispSFinmath{
f[\Mvariable{x\_}]:=x/.\{(-1)\RawWedge \Mvariable{p\_}/;\Muserfunction{DataType}[p,\Mvariable{ganzeZahl}]\RuleDelayed 1\};
}

\dispSFinmath{
\Mvariable{test}=(-1)\RawWedge m+(-1)\RawWedge n\multsp x
}

\dispSFoutmath{
{{(-1)}^n}\multsp x+{{(-1)}^m}
}

\dispSFinmath{
f[\Mvariable{test}]
}

\dispSFoutmath{
{{(-1)}^n}\multsp x+1
}

\dispSFinmath{
\Mfunction{Clear}[f,\Mvariable{test}];
}

\dispSFinmath{
\Muserfunction{DataType}[f,g,\multsp \Mvariable{NonCommutative}]\multsp =\multsp \Mvariable{False};
}

\dispSFinmath{
\Muserfunction{DataType}[m,\Mvariable{odd}]=\Muserfunction{DataType}[a,\Mvariable{even}]=\Mvariable{False};
}

Certain FeynCalc objects have DataType PositiveInteger set to True.

\dispSFinmath{
\Muserfunction{DataType}[\Mvariable{OPEm},\Mvariable{PositiveInteger}]
}

\dispSFoutmath{
\Mvariable{True}
}

PowerSimplify uses the DataType information.

\dispSFinmath{
\Muserfunction{PowerSimplify}[\multsp (-1)\RawWedge (2\Mvariable{OPEm})]
}

\dispSFoutmath{
1
}

\dispSFinmath{
\Muserfunction{PowerSimplify}[\multsp (-\multsp \Muserfunction{SO}[q])\RawWedge \Mvariable{OPEm}]
}

\dispSFoutmath{
{{(-1)}^m}\multsp {{(\Delta \cdot q)}^m}
}

\Subsection*{DB0}

\Subsubsection*{Description}

DB0[p2, m1\(\RawWedge\)2, m2\(\RawWedge\)2] is the derivative of the two-point function B0[p2, m1\(\RawWedge\)2, m2\(\RawWedge\)2] with
  respect to p2.

See also:  B0.

\Subsubsection*{Examples}

\dispSFinmath{
\Mfunction{D}\big[\Muserfunction{B0}\big[{p_2},{{{m_1}}^2},{{{m_2}}^2}\big],{p_2}\big]
}

\dispSFoutmath{
\Muserfunction{DB0}({p_2},m_{1}^{2},m_{2}^{2})
}

\Subsection*{DB1}

\Subsubsection*{Description}

DB1[p2,m1\(\RawWedge\)2,m2\(\RawWedge\)2] is the derivative of B1[p2,m1\(\RawWedge\)2,m2\(\RawWedge\)2] with respect to p2.

See also:  B1.

\Subsubsection*{Examples}

\dispSFinmath{
\Mfunction{D}\big[\Muserfunction{B1}\big[{p_2},{{{m_1}}^2},{{{m_2}}^2}\big],{p_2}\big]
}

\dispSFoutmath{
\frac{(m_{2}^{2}-m_{1}^{2})\multsp \Muserfunction{DB0}({p_2},m_{1}^{2},m_{2}^{2})}{2\multsp {p_2}}-
   \frac{1}{2}\multsp \Muserfunction{DB0}({p_2},m_{1}^{2},m_{2}^{2})-
   \frac{({B_0}({p_2},m_{1}^{2},m_{2}^{2})-{B_0}(0,m_{1}^{2},m_{2}^{2}))\multsp (m_{2}^{2}-m_{1}^{2})}{2\multsp p_{2}^{2}}
}

\Subsection*{DeclareNonCommutative}

\Subsubsection*{Description}

DeclareNonCommutative[a, b, ...] declares a,b, ... to be non-commutative, i.e., DataType[a,b, ..., NonCommutative] is set to True.

See also:  DataType, UnDeclareNonCommutative.

\Subsubsection*{Examples}

\dispSFinmath{
\Muserfunction{DeclareNonCommutative}[x]
}

As a side effect of DeclareNonCommutative x is declared to be of data type NonCommutative.

\dispSFinmath{
\Muserfunction{DataType}[x,\Mvariable{NonCommutative}]
}

\dispSFoutmath{
\Mvariable{True}
}

\dispSFinmath{
\Muserfunction{DeclareNonCommutative}[y,z]
}

\dispSFinmath{
\Muserfunction{DataType}[a,x,y,z,\Mvariable{NonCommutative}]
}

\dispSFoutmath{
\{\Mvariable{True},\Mvariable{True},\Mvariable{True},\Mvariable{True}\}
}

\dispSFinmath{
\Muserfunction{UnDeclareNonCommutative}[x,y,z]
}

\dispSFinmath{
\Muserfunction{DataType}[a,x,y,z,\Mvariable{NonCommutative}]
}

\dispSFoutmath{
\{\Mvariable{True},\Mvariable{True},\Mvariable{True},\Mvariable{True}\}
}

\Subsection*{DeltaFunction}

\Subsubsection*{Description}

 DeltaFunction[x] is the Dirac delta-function \(\delta (x)\).\\
*** After version 4 of {\itshape Mathematica}, there is a built-in function, DiracDelta, with comparable properties. ***

See also:  Convolute, DeltaFunctionPrime, Integrate2, SimplifyDeltaFunction.

\Subsubsection*{Examples}

\dispSFinmath{
\Muserfunction{DeltaFunction}[1-x]
}

\dispSFoutmath{
\delta (1-x)
}

\dispSFinmath{
\Muserfunction{Integrate2}[\Muserfunction{DeltaFunction}[1-x]\multsp f[x],\{x,0,1\}]
}

\dispSFoutmath{
f(1)
}

\dispSFinmath{
\Muserfunction{Integrate2}[\Muserfunction{DeltaFunction}[x]\multsp f[x],\{x,0,1\}]
}

\dispSFoutmath{
f(0)
}

\dispSFinmath{
\Muserfunction{Integrate2}[\Muserfunction{DeltaFunction}[1-x]\multsp f[x],\{x,0,1\}]
}

\dispSFoutmath{
f(1)
}

\dispSFinmath{
\Muserfunction{Convolute}[\Muserfunction{DeltaFunction}[1-x],x]
}

\dispSFoutmath{
x
}

\Subsection*{DeltaFunctionDoublePrime}

\Subsubsection*{Description}

 DeltaFunctionDoublePrime[1-x] is the second derivative of the Dirac delta-function \(\delta (x)\).

See also: DeltaFunctionPrime.

\Subsection*{DeltaFunctionPrime}

\Subsubsection*{Description}

 DeltaFunctionPrime[1-x] is the derivative of the Dirac delta-function \(\delta (x)\).

See also: Convolute, DeltaFunction, Integrate2, SimplifyDeltaFunction.

\Subsubsection*{Examples}

\dispSFinmath{
\Muserfunction{DeltaFunctionPrime}[1-x]
}

\dispSFoutmath{
{{\delta }^{\prime }}(1-x)
}

\dispSFinmath{
\Muserfunction{Integrate2}[\Muserfunction{DeltaFunctionPrime}[1-x]\multsp f[x],\{x,0,1\}]
}

\dispSFoutmath{
{f^{\prime }}(1)
}

\dispSFinmath{
\Muserfunction{Integrate2}\big[\Muserfunction{DeltaFunctionPrime}[1-x]\multsp {x^2},\{x,0,1\}\big]
}

\dispSFoutmath{
2
}

\Subsection*{DenominatorOrder}

\Subsubsection*{Description}

DenominatorOrder is an option for OneLoop, if set to True the PropagatorDenominator will be ordered in a standard way.

See also: OneLoop, PropagatorDenominator.

\Subsection*{Dimension}

\Subsubsection*{Description}

Dimension is an option of several functions and denotes the number of space-time dimensions. Possible settings are: 4, n, d, D, ... ,the
  variable does not matter, but it should have Head Symbol.

\dispSFinmath{
\Mfunction{Options}[\Mvariable{MetricTensor}]
}

\dispSFoutmath{
\{\Mvariable{Dimension}\rightarrow 4,\Mvariable{FeynCalcInternal}\rightarrow \Mvariable{False}\}
}

\dispSFinmath{
\Muserfunction{MetricTensor}[m,n,\Mvariable{Dimension}\rightarrow d]
    \Muserfunction{DiracMatrix}[\alpha ,\Mvariable{Dimension}\rightarrow d]//\Muserfunction{FCI}
}

\dispSFoutmath{
{{\gamma }^{\alpha }}\multsp {g^{mn}}
}

The dimension of the indices is not shown by default but can be inspected easily.

\dispSFinmath{
\Muserfunction{MetricTensor}[m,n,\Mvariable{Dimension}\rightarrow d]
    \Muserfunction{DiracMatrix}[\alpha ,\Mvariable{Dimension}\rightarrow d]//\Mfunction{StandardForm}
}

\dispSFoutmath{
\Muserfunction{DiracGamma}[\Muserfunction{LorentzIndex}[\alpha ,d],d]\multsp
   \Muserfunction{MetricTensor}[m,n,\Mvariable{Dimension}\rightarrow d]
}

Setting the global variable \${}LorentzIndices to True will display the dimension (if different from 4) as a subscript.

\dispSFinmath{
\$LorentzIndices=\Mvariable{True};
}

\dispSFinmath{
\Muserfunction{MetricTensor}[\alpha ,\beta ,\Mvariable{Dimension}\rightarrow n]
   \Muserfunction{DiracMatrix}[\alpha ,\Mvariable{Dimension}\rightarrow n]
}

\dispSFoutmath{
{{\gamma }^{{{\alpha }_n}}}\multsp \Muserfunction{MetricTensor}(\alpha ,\beta ,\Mvariable{Dimension}\rightarrow n)
}

\dispSFinmath{
\%//\Mfunction{StandardForm}
}

\dispSFoutmath{
\Muserfunction{DiracGamma}[\Muserfunction{LorentzIndex}[\alpha ,n],n]\multsp
   \Muserfunction{MetricTensor}[\alpha ,\beta ,\Mvariable{Dimension}\rightarrow n]
}

\dispSFinmath{
\$LorentzIndices=\Mvariable{False};
}

\Subsection*{DimensionalReduction}

\Subsubsection*{Description}

DimensionalReduction is an option for TID and OneLoopSimplify.

See also: TID, OneLoopSimplify.

\Subsection*{DiracBasis}

\Subsubsection*{Description}

DiracBasis[any] is a head which is wrapped around Dirac structures (and the 1) as a result of the function DiracReduce. Eventually you
  want to substitute DiracBasis by Identity (or set: DiracBasis[1] \(=\) S; DiracBasis[DiracMatrix[mu]] \(=\) P; etc.).

See also: DiracReduce.

\Subsection*{DiracCanonical}

\Subsubsection*{Description}

DiracCanonical is an option for DiracSimplify. If set to True DiracSimplify uses the function DiracOrder internally.

See also: DiracSimplify.

\Subsection*{DiracEquation}

\Subsubsection*{Description}

DiracEquation[exp] applies the Dirac equation without expanding exp. If that is needed, use DiracSimplify.

See also:  DiracSimplify.

\Subsubsection*{Examples}

\dispSFinmath{
\Muserfunction{Spinor}[\Muserfunction{Momentum}[p],m,1].\Muserfunction{DiracGamma}[\Muserfunction{Momentum}[p]]
}

\dispSFoutmath{
\varphi (p,m).(\gamma \cdot p)
}

\dispSFinmath{
\Muserfunction{DiracEquation}[\%]
}

\dispSFoutmath{
m\multsp \varphi (p,m)
}

\Subsection*{DiracGamma}

\Subsubsection*{Description}

DiracGamma[x, dim] is the head of all Dirac matrices and slashes (in the internal representation). Use DiracMatrix (or GA, GAD) and
  DiracSlash (or GS, GSD) for manual (short) input. DiracGamma[x, 4] simplifies to DiracGamma[x]. DiracGamma[5] is \({{\gamma }^5}\). DiracGamma[6]
is \((1+{{\gamma }^5})/2.\) DiracGamma[7] is \((1-{{\gamma }^5})/2.\)

See also:  DiracGammaExpand, DiracMatrix, DiracSimplify, DiracSlash, DiracTrick.

\Subsubsection*{Examples}

\dispSFinmath{
\Muserfunction{DiracGamma}[5]
}

\dispSFoutmath{
{{\gamma }^5}
}

\dispSFinmath{
\Muserfunction{DiracGamma}[\Muserfunction{LorentzIndex}[\alpha ]]
}

\dispSFoutmath{
{{\gamma }^{\alpha }}
}

A Dirac-slash, i.e., \({{\gamma }^{\mu }}{q_{\mu }}\), is displayed as \(\gamma \cdot q\).

\dispSFinmath{
\Muserfunction{DiracGamma}[\Muserfunction{Momentum}[q]]\multsp
}

\dispSFoutmath{
\gamma \cdot q
}

\dispSFinmath{
\Muserfunction{DiracGamma}[\Muserfunction{Momentum}[q]]\multsp .\multsp \Muserfunction{DiracGamma}[\Muserfunction{Momentum}[p-q]]
}

\dispSFoutmath{
(\gamma \cdot q).(\gamma \cdot (p-q))
}

\dispSFinmath{
\Muserfunction{DiracGamma}[\Muserfunction{Momentum}[q,D],D]\multsp
}

\dispSFoutmath{
\gamma \cdot q
}

\dispSFinmath{
\Mvariable{a1}=\Muserfunction{GS}[p-q].\Muserfunction{GS}[p]
}

\dispSFoutmath{
(\gamma \cdot (p-q)).(\gamma \cdot p)
}

\dispSFinmath{
\Mvariable{a2}=\Muserfunction{DiracGammaExpand}[\Mvariable{a1}]
}

\dispSFoutmath{
(\gamma \cdot p-\gamma \cdot q).(\gamma \cdot p)
}

\dispSFinmath{
\Mvariable{a3}=\Muserfunction{GAD}[\mu ].\Muserfunction{GSD}[p-q].\Muserfunction{GSD}[q].\Muserfunction{GAD}[\mu ]
}

\dispSFoutmath{
{{\gamma }^{\mu }}.(\gamma \cdot (p-q)).(\gamma \cdot q).{{\gamma }^{\mu }}
}

\dispSFinmath{
\Mvariable{a4}=\Muserfunction{DiracTrick}[\Mvariable{a3}]
}

\dispSFoutmath{
(D-4)\multsp (\gamma \cdot (p-q)).(\gamma \cdot q)+4\multsp (p-q)\cdot q
}

\dispSFinmath{
\Mvariable{a5}=\Muserfunction{DiracSimplify}[\Mvariable{a4}]
}

\dispSFoutmath{
D\multsp (\gamma \cdot p).(\gamma \cdot q)-4\multsp (\gamma \cdot p).(\gamma \cdot q)+4\multsp p\cdot q-D\multsp {q^2}
}

\dispSFinmath{
\Mfunction{Clear}[\Mvariable{a1},\Mvariable{a2},\Mvariable{a3},\Mvariable{a4},\Mvariable{a5}]
}

\Subsection*{DiracGammaCombine}

\Subsubsection*{Description}

DiracGammaCombine[exp] is (nearly) the inverse operation to DiracGammaExpand.

See also:  DiracGamma, DiracGammaExpand, DiracMatrix, DiracSimplify, DiracSlash, DiracTrick.

\Subsubsection*{Examples}

\dispSFinmath{
\Muserfunction{DiracGammaCombine}[\Muserfunction{GS}[p]\multsp +\multsp \Muserfunction{GS}[q]]
}

\dispSFoutmath{
\gamma \cdot (p+q)
}

\dispSFinmath{
\Mfunction{StandardForm}[\%]
}

\dispSFoutmath{
\Muserfunction{DiracGamma}[\Muserfunction{Momentum}[p+q]]
}

\dispSFinmath{
\Muserfunction{DiracGammaCombine}[2\multsp \Muserfunction{GS}[p]\multsp -\multsp 2\multsp \Muserfunction{GS}[q]]
}

\dispSFoutmath{
\gamma \cdot (2\multsp p-2\multsp q)
}

\dispSFinmath{
\Mfunction{StandardForm}[\%]
}

\dispSFoutmath{
\Muserfunction{DiracGamma}[\Muserfunction{Momentum}[2\multsp p-2\multsp q]]
}

\dispSFinmath{
\Muserfunction{DiracGammaExpand}[\%\%]
}

\dispSFoutmath{
2\multsp \gamma \cdot p-2\multsp \gamma \cdot q
}

\Subsection*{DiracGammaExpand}

\Subsubsection*{Description}

DiracGammaExpand[exp] expands all DiracGamma[Momentum[a\(+\)b\(+\)..]] in exp into (DiracGamma[Momentum[a]] \(+\) DiracGamma[Momentum[b]]
  \(+\) ...).

See also:  DiracGamma, DiracGammaCombine, DiracMatrix, DiracSimplify, DiracSlash, DiracTrick.

\Subsubsection*{Examples}

\dispSFinmath{
t=\Muserfunction{DiracGamma}[\Muserfunction{Momentum}[q]]\multsp .\multsp \Muserfunction{DiracGamma}[\Muserfunction{Momentum}[p-q]]
}

\dispSFoutmath{
(\gamma \cdot q).(\gamma \cdot (p-q))
}

Momentum is the head of p-q, i.e., it is treated as one four-momentum.

\dispSFinmath{
\Mfunction{StandardForm}[t]
}

\dispSFoutmath{
\Muserfunction{DiracGamma}[\Muserfunction{Momentum}[q]].\Muserfunction{DiracGamma}[\Muserfunction{Momentum}[p-q]]
}

With DiracGammaExpand the Momentum[p-q] gets expanded.

\dispSFinmath{
\Muserfunction{DiracGammaExpand}[t]
}

\dispSFoutmath{
(\gamma \cdot q).(\gamma \cdot p-\gamma \cdot q)
}

The inverse operation is DiracGammaCombine.

\dispSFinmath{
\Mfunction{StandardForm}[\Muserfunction{DiracGammaCombine}[\Muserfunction{DiracGammaExpand}[t]]]
}

\dispSFoutmath{
\Muserfunction{DiracGamma}[\Muserfunction{Momentum}[q]].\Muserfunction{DiracGamma}[\Muserfunction{Momentum}[p-q]]
}

\dispSFinmath{
\Mfunction{StandardForm}[\Muserfunction{DiracGammaExpand}[t]]
}

\dispSFoutmath{
\Muserfunction{DiracGamma}[\Muserfunction{Momentum}[q]].
   (\Muserfunction{DiracGamma}[\Muserfunction{Momentum}[p]]-\Muserfunction{DiracGamma}[\Muserfunction{Momentum}[q]])
}

In order to do non-commutative expansion use DiracSimplify.

\dispSFinmath{
\Muserfunction{DiracSimplify}[t]
}

\dispSFoutmath{
(\gamma \cdot q).(\gamma \cdot p)-{q^2}
}

\dispSFinmath{
\Mfunction{Clear}[t]
}

\Subsection*{DiracGammaT}

\Subsubsection*{Description}

DiracGammaT[x] denotes the transpose of DiracGamma[x]. Transpose[DiracGammaT[x]] gives DiracGamma[x]. Note that x must have Head
  LorentzIndex or Momentum.

See also:  DiracGamma.

\Subsubsection*{Examples}

\dispSFinmath{
\Muserfunction{DiracGammaT}[\Muserfunction{LorentzIndex}[\mu ]]
}

\dispSFoutmath{
\gamma _{\mu }^{T}
}

\dispSFinmath{
\Mfunction{Transpose}[\%]
}

\dispSFoutmath{
{{\gamma }^{\mu }}
}

\dispSFinmath{
\Muserfunction{GS}[p]//\Muserfunction{FCI}//\Mfunction{Transpose}
}

\dispSFoutmath{
{{(\gamma \cdot p)}^T}
}

\Subsection*{DiracMatrix}

\Subsubsection*{Description}

DiracMatrix[\(\mu \)] denotes a Dirac gamma matrix with Lorentz index \(\mu \). DiracMatrix[\(\mu ,\nu ,\multsp \)...] is a product of \(\gamma \)
matrices with Lorentz indices \(\mu ,\multsp \nu ,\multsp ...\) DiracMatrix[5] is \({{\gamma }^5}\). DiracMatrix[6] is \(1/2\)\(\multsp +\multsp
{{\gamma }^5}/2\). DiracMatrix[7] is\(\multsp 1/2\)\(\multsp -\multsp {{\gamma }^5}/2\).

See also:  DiracGammaExpand, DiracGamma, DiracSimplify, DiracSlash, DiracTrick, GA, GAD, GS, GSD.

\Subsubsection*{Examples}

\dispSFinmath{
\Muserfunction{DiracMatrix}[\mu ]
}

\dispSFoutmath{
{{\gamma }^{\mu }}
}

This is how to enter the non-commutative product of two \({{\gamma }^{\mu }}{{\gamma }^{\nu }}\). The {\itshape Mathematica} Dot "." is used as non-commutative
multiplication operator.

\dispSFinmath{
\Muserfunction{DiracMatrix}[\mu ].\Muserfunction{DiracMatrix}[\nu ]
}

\dispSFoutmath{
{{\gamma }^{\mu }}.{{\gamma }^{\nu }}
}

\dispSFinmath{
\Muserfunction{DiracMatrix}[\alpha ]//\Mfunction{StandardForm}
}

\dispSFoutmath{
\Muserfunction{DiracGamma}[\Muserfunction{LorentzIndex}[\alpha ]]
}

\dispSFinmath{
\Muserfunction{DiracMatrix}[\mu ]//\Muserfunction{FCE}
}

\dispSFoutmath{
{{\gamma }^{\mu }}
}

\dispSFinmath{
\%//\Mfunction{StandardForm}
}

\dispSFoutmath{
\Muserfunction{GA}[\mu ]
}

\dispSFinmath{
\Muserfunction{GAD}[\mu ]
}

\dispSFoutmath{
{{\gamma }^{\mu }}
}

\dispSFinmath{
\%//\Muserfunction{FCI}//\Mfunction{StandardForm}
}

\dispSFoutmath{
\Muserfunction{DiracGamma}[\Muserfunction{LorentzIndex}[\mu ,D],D]
}

\dispSFinmath{
\Muserfunction{GA}[\mu ,\nu ,\rho ]
}

\dispSFoutmath{
{{\gamma }^{\mu }}.{{\gamma }^{\nu }}.{{\gamma }^{\rho }}
}

\dispSFinmath{
\Muserfunction{GA}[a\multsp .\multsp b]//\Muserfunction{FCI}
}

\dispSFoutmath{
{{\gamma }^a}.{{\gamma }^b}
}

\dispSFinmath{
\%//\Mfunction{StandardForm}
}

\dispSFoutmath{
\Muserfunction{DiracGamma}[\Muserfunction{LorentzIndex}[a]].\Muserfunction{DiracGamma}[\Muserfunction{LorentzIndex}[b]]
}

\Subsection*{DiracOrder}

\Subsubsection*{Description}

DiracOrder[expr] orders the Dirac matrices in expr alphabetically. DiracOrder[expr,orderlist] orders the Dirac matrices in expr according
  to orderlist.

See also:  DiracSimplify, DiracTrick.

\Subsubsection*{Examples}

\dispSFinmath{
\Mvariable{t1}\multsp =\multsp \Muserfunction{GA}[\beta ,\alpha ]
}

\dispSFoutmath{
{{\gamma }^{\beta }}.{{\gamma }^{\alpha }}
}

\dispSFinmath{
\Muserfunction{DiracOrder}[\Mvariable{t1}]
}

\dispSFoutmath{
2\multsp {g^{\alpha \beta }}-{{\gamma }^{\alpha }}.{{\gamma }^{\beta }}
}

This is a string of Dirac matrices in D dimensions.

\dispSFinmath{
\Mvariable{t2}=\Muserfunction{GAD}[\mu ,\nu ,\mu ]
}

\dispSFoutmath{
{{\gamma }^{\mu }}.{{\gamma }^{\nu }}.{{\gamma }^{\mu }}
}

\dispSFinmath{
\Muserfunction{DiracOrder}[\Mvariable{t1}]
}

\dispSFoutmath{
2\multsp {g^{\alpha \beta }}-{{\gamma }^{\alpha }}.{{\gamma }^{\beta }}
}

\dispSFinmath{
\Mvariable{t3}=\Muserfunction{GA}[5,\mu ,\nu ]
}

\dispSFoutmath{
{{\gamma }^5}.{{\gamma }^{\mu }}.{{\gamma }^{\nu }}
}

By default \({{\gamma }^5}\multsp \)is moved to the right.

\dispSFinmath{
\Muserfunction{DiracOrder}[\Mvariable{t3}]
}

\dispSFoutmath{
{{\gamma }^{\mu }}.{{\gamma }^{\nu }}.{{\gamma }^5}
}

\dispSFinmath{
\Mvariable{t4}=\Muserfunction{GA}[6,\mu ,7]
}

\dispSFoutmath{
{{\gamma }^6}.{{\gamma }^{\mu }}.{{\gamma }^7}
}

\dispSFinmath{
\Muserfunction{DiracOrder}[\Mvariable{t4}]
}

\dispSFoutmath{
{{\gamma }^{\mu }}.{{\gamma }^7}
}

\dispSFinmath{
\Mvariable{t5}=\Muserfunction{GA}[\alpha ,\beta ,\delta ]
}

\dispSFoutmath{
{{\gamma }^{\alpha }}.{{\gamma }^{\beta }}.{{\gamma }^{\delta }}
}

This orders the \({{\gamma }^{\alpha }}{{\gamma }^{\beta }}{{\gamma }^{\delta }}\) in reverse order.

\dispSFinmath{
\Muserfunction{DiracOrder}[\Mvariable{t5},\{\delta ,\beta ,\alpha \}]
}

\dispSFoutmath{
-{{\gamma }^{\delta }}.{{\gamma }^{\beta }}.{{\gamma }^{\alpha }}+2\multsp {{\gamma }^{\delta }}\multsp {g^{\alpha \beta }}-
   2\multsp {{\gamma }^{\beta }}\multsp {g^{\alpha \delta }}+2\multsp {{\gamma }^{\alpha }}\multsp {g^{\beta \delta }}
}

\dispSFinmath{
\Muserfunction{DiracOrder}[\%]
}

\dispSFoutmath{
{{\gamma }^{\alpha }}.{{\gamma }^{\beta }}.{{\gamma }^{\delta }}
}

\dispSFinmath{
\Mfunction{Clear}[\Mvariable{t1},\Mvariable{t2},\Mvariable{t3},\Mvariable{t4},\Mvariable{t5}];
}

\Subsection*{DiracReduce}

\Subsubsection*{Description}

DiracReduce[exp] reduces all four-dimensional Dirac matrices in exp to the standard basis (S,P,V,A,T) using the Chisholm identity (see
  Chisholm). In the result the basic Dirac structures are wrapped with a head DiracBasis. I.e., S corresponds to DiracBasis[1], P :
  DiracBasis[DiracMatrix[5]], V: DiracBasis[DiracMatrix[mu]], A: DiracBasis[DiracMatrix[mu, 5]], T: DiracBasis[DiracSigma[DiracMatrix[mu,
  nu]]]. By default DiracBasis is substituted to Identity. Notice that the result of DiracReduce is given in the FeynCalcExternal-way,
  i.e.,evtl. you may have to use FeynCalcInternal on the result.

\dispSFinmath{
\Mfunction{Options}[\Mvariable{DiracReduce}]
}

\dispSFoutmath{
\{\Mvariable{Factoring}\rightarrow \Mvariable{False},\Mvariable{FinalSubstitutions}\rightarrow
     \{\Mvariable{DiracBasis}\rightarrow \Mvariable{Identity}\}\}
}

See also:  DiracSimplify.

\Subsubsection*{Examples}

\dispSFinmath{
\Mvariable{t1}\multsp =\multsp \Muserfunction{GA}[\mu ,\nu ]
}

\dispSFoutmath{
{{\gamma }^{\mu }}.{{\gamma }^{\nu }}
}

\dispSFinmath{
\Muserfunction{DiracReduce}[\Mvariable{t1}]
}

\dispSFoutmath{
{g^{\mu \nu }}-\ImaginaryI \multsp {{\sigma }^{\mu \nu }}
}

\dispSFinmath{
\Mvariable{t2}=\Muserfunction{DiracMatrix}[\mu ,\nu ,\rho ]
}

\dispSFoutmath{
{{\gamma }^{\mu }}{{\gamma }^{\nu }}{{\gamma }^{\rho }}
}

\dispSFinmath{
\Muserfunction{DiracReduce}[\Mvariable{t2}]
}

\dispSFoutmath{
\ImaginaryI \multsp {{\gamma }^{\$MU(1)}}.{{\gamma }^5}\multsp
    {{\epsilon }^{\mu \nu \rho \$MU(1)}}+{{\gamma }^{\rho }}\multsp {g^{\mu \nu }}-
   {{\gamma }^{\nu }}\multsp {g^{\mu \rho }}+{{\gamma }^{\mu }}\multsp {g^{\nu \rho }}
}

\dispSFinmath{
\Mvariable{t3}=\Muserfunction{DiracMatrix}[\mu ,\nu ,\rho ,\sigma ]
}

\dispSFoutmath{
{{\gamma }^{\mu }}{{\gamma }^{\nu }}{{\gamma }^{\rho }}{{\gamma }^{\sigma }}
}

\dispSFinmath{
\Mvariable{t4}=\Muserfunction{DiracReduce}[\Mvariable{t3}]
}

\dispSFoutmath{
\MathBegin{MathArray}{l}
-\ImaginaryI \multsp {{\gamma }^5}\multsp {{\epsilon }^{\mu \nu \rho \sigma }}-
   \ImaginaryI \multsp {{\sigma }^{\rho \sigma }}\multsp {g^{\mu \nu }}+
   \ImaginaryI \multsp {{\sigma }^{\nu \sigma }}\multsp {g^{\mu \rho }}-
   \ImaginaryI \multsp {{\sigma }^{\nu \rho }}\multsp {g^{\mu \sigma }}-  \\
\noalign{\vspace{0.666667ex}}
\hspace{1.em} \ImaginaryI
    \multsp {{\sigma }^{\mu \sigma }}\multsp {g^{\nu \rho }}+{g^{\mu \sigma }}\multsp {g^{\nu \rho }}+
   \ImaginaryI \multsp {{\sigma }^{\mu \rho }}\multsp {g^{\nu \sigma }}-{g^{\mu \rho }}\multsp {g^{\nu \sigma }}-
   \ImaginaryI \multsp {{\sigma }^{\mu \nu }}\multsp {g^{\rho \sigma }}+{g^{\mu \nu }}\multsp {g^{\rho \sigma }}\\
\MathEnd{MathArray}
}

\dispSFinmath{
\Mvariable{t5}\multsp =\Muserfunction{Calc}[\Muserfunction{DiracSimplify}[
     \Muserfunction{DiracSigmaExplicit}[\Mvariable{t4}.\Mvariable{t4}]]]
}

\dispSFoutmath{
-128
}

\dispSFinmath{
\Muserfunction{Calc}[\Mvariable{t4}.\Mvariable{t4}]
}

\dispSFoutmath{
-6\multsp {D^3}+17\multsp {D^2}-10\multsp D+24
}

\dispSFinmath{
\Mfunction{Clear}[\Mvariable{t1},\Mvariable{t2},\Mvariable{t3},\Mvariable{t4}]
}

\Subsection*{DiracSigma}

\Subsubsection*{Description}

DiracSigma[a, b] stands for i/2*(a . b - b . a) in 4 dimensions. a and b must have Head DiracGamma, DiracMatrix or DiracSlash. Only
  antisymmetry is implemented.

See also:  DiracSigmaExplicit.

\Subsubsection*{Examples}

\dispSFinmath{
\Mvariable{t1}=\Muserfunction{DiracSigma}[\Muserfunction{GA}[\alpha ],\Muserfunction{GA}[\beta ]]
}

\dispSFoutmath{
{{\sigma }^{\alpha \beta }}
}

\dispSFinmath{
\Muserfunction{DiracSigmaExplicit}[\Mvariable{t1}]
}

\dispSFoutmath{
\frac{1}{2}\multsp \ImaginaryI \multsp ({{\gamma }^{\alpha }}.{{\gamma }^{\beta }}-{{\gamma }^{\beta }}.{{\gamma }^{\alpha }})
}

\dispSFinmath{
\Mvariable{t2}=\Muserfunction{DiracSigma}[\Muserfunction{GA}[\beta ],\Muserfunction{GA}[\alpha ]]
}

\dispSFoutmath{
-{{\sigma }^{\alpha \beta }}
}

\dispSFinmath{
\Mvariable{t3}=\Muserfunction{DiracSigma}[\Muserfunction{GS}[p],\Muserfunction{GS}[q]]
}

\dispSFoutmath{
{{\sigma }^{pq}}
}

\dispSFinmath{
\Muserfunction{DiracSigmaExplicit}[\Mvariable{t3}]
}

\dispSFoutmath{
\frac{1}{2}\multsp \ImaginaryI \multsp ((\gamma \cdot p).(\gamma \cdot q)-(\gamma \cdot q).(\gamma \cdot p))
}

\dispSFinmath{
\Mfunction{Clear}[\Mvariable{t1},\Mvariable{t2},\Mvariable{t3}]
}

\Subsection*{DiracSigmaExplicit}

\Subsubsection*{Description}

DiracSigmaExplicit[exp] inserts in exp for all DiracSigma its definition. DiracSigmaExplict is also an option of DiracSimplify.

See also:  DiracSigma.

\Subsubsection*{Examples}

\dispSFinmath{
\Muserfunction{DiracSigma}[\Muserfunction{GA}[\alpha ],\Muserfunction{GA}[\beta ]]
}

\dispSFoutmath{
{{\sigma }^{\alpha \beta }}
}

\dispSFinmath{
\Muserfunction{DiracSigmaExplicit}[\%]
}

\dispSFoutmath{
\frac{1}{2}\multsp \ImaginaryI \multsp ({{\gamma }^{\alpha }}.{{\gamma }^{\beta }}-{{\gamma }^{\beta }}.{{\gamma }^{\alpha }})
}

\Subsection*{DiracSimpCombine}

\Subsubsection*{Description}

DiracSimpCombine is an option for DiracSimplify. If set to True, sums of DiracGamma's will be merged as much as possible in DiracGamma[
  .. \(+\) .. \(+\) ]'s.

See also: DiracSimplify.

\Subsection*{DiracSimplify}

\Subsubsection*{Description}

DiracSimplify[expr] simplifies products of Dirac matrices in expr and expands non-commutative products. Double Lorentz indices and four
  vectors are contracted. The Dirac equation is applied. All DiracMatrix[5], DiracMatrix[6] and DiracMatrix[7] are moved to the right.
  The order of the other Dirac matrices is not changed.

\dispSFinmath{
\Mfunction{Options}[\Mvariable{DiracSimplify}]
}

\dispSFoutmath{
\MathBegin{MathArray}{l}
\{\Mvariable{DiracCanonical}\rightarrow \Mvariable{False},
    \Mvariable{DiracSigmaExplicit}\rightarrow \Mvariable{True},\Mvariable{DiracSimpCombine}\rightarrow \Mvariable{False},
    \Mvariable{DiracSubstitute67}\rightarrow \Mvariable{False},  \\
\noalign{\vspace{0.666667ex}}
\hspace{1.em} \Mvariable{Expanding}
     \rightarrow \Mvariable{True},\Mvariable{Factoring}\rightarrow \Mvariable{False},
    \Mvariable{FeynCalcInternal}\rightarrow \Mvariable{False},\Mvariable{InsideDiracTrace}\rightarrow \Mvariable{False}\}\\
   \MathEnd{MathArray}
}

See also:  Calc, DiracGammaExpand, DiracTrick.

\Subsubsection*{Examples}

This is a string of Dirac matrices in four dimensions.

\dispSFinmath{
\Mvariable{t1}=\Muserfunction{GA}[\mu ,\nu ,\mu ]
}

\dispSFoutmath{
{{\gamma }^{\mu }}.{{\gamma }^{\nu }}.{{\gamma }^{\mu }}
}

\dispSFinmath{
\Muserfunction{DiracSimplify}[\Mvariable{t1}]
}

\dispSFoutmath{
-2\multsp {{\gamma }^{\nu }}
}

This is a string of Dirac matrices in D dimensions.

\dispSFinmath{
\Mvariable{t2}=\Muserfunction{GAD}[\mu ,\nu ,\mu ]
}

\dispSFoutmath{
{{\gamma }^{\mu }}.{{\gamma }^{\nu }}.{{\gamma }^{\mu }}
}

\dispSFinmath{
\Muserfunction{DiracSimplify}[\Mvariable{t1}]
}

\dispSFoutmath{
-2\multsp {{\gamma }^{\nu }}
}

\dispSFinmath{
\Mvariable{t3}=\Muserfunction{GA}[5,\mu ,\nu ]
}

\dispSFoutmath{
{{\gamma }^5}.{{\gamma }^{\mu }}.{{\gamma }^{\nu }}
}

By default \({{\gamma }^5}\)is moved to the right.

\dispSFinmath{
\Muserfunction{DiracSimplify}[\Mvariable{t3}]
}

\dispSFoutmath{
{{\gamma }^{\mu }}.{{\gamma }^{\nu }}.{{\gamma }^5}
}

\dispSFinmath{
\Mvariable{t4}=\Muserfunction{GA}[6,\mu ,7]
}

\dispSFoutmath{
{{\gamma }^6}.{{\gamma }^{\mu }}.{{\gamma }^7}
}

\dispSFinmath{
\Muserfunction{DiracSimplify}[\Mvariable{t4}]
}

\dispSFoutmath{
{{\gamma }^{\mu }}.{{\gamma }^7}
}

\dispSFinmath{
\Mvariable{t5}=\Muserfunction{GS}[a+b]\multsp .\multsp \Muserfunction{GS}[p].\Muserfunction{GS}[p].\Muserfunction{GS}[c+d]
}

\dispSFoutmath{
(\gamma \cdot (a+b)).(\gamma \cdot p).(\gamma \cdot p).(\gamma \cdot (c+d))
}

Contrary to DiracTrick DiracSimplify does non-commutative expansion.

\dispSFinmath{
\Muserfunction{DiracSimplify}[\Mvariable{t5}]
}

\dispSFoutmath{
(\gamma \cdot a).(\gamma \cdot c)\multsp {p^2}+(\gamma \cdot a).(\gamma \cdot d)\multsp {p^2}+
   (\gamma \cdot b).(\gamma \cdot c)\multsp {p^2}+(\gamma \cdot b).(\gamma \cdot d)\multsp {p^2}
}

\dispSFinmath{
\Muserfunction{DiracTrick}[\Mvariable{t5}]
}

\dispSFoutmath{
(\gamma \cdot (a+b)).(\gamma \cdot (c+d))\multsp {p^2}
}

\dispSFinmath{
\Mvariable{t6}\multsp =\multsp \Muserfunction{SpinorVBar}[p]\multsp .\multsp \Muserfunction{GS}[p]\multsp .\multsp
    \Muserfunction{SpinorUBar}[q]
}

\dispSFoutmath{
\overvar{v}{\_}(p).(\gamma \cdot p).\overvar{u}{\_}(q)
}

\dispSFinmath{
\Muserfunction{DiracSimplify}[\Mvariable{t6}]
}

\dispSFoutmath{
0
}

\dispSFinmath{
\Mvariable{GAD}@@\Mfunction{Join}[\{\mu \},\Mfunction{Table}[{{\nu }_i},\{i,6\}],\{\mu \}]
}

\dispSFoutmath{
{{\gamma }^{\mu }}.{{\gamma }^{{{\nu }_1}}}.{{\gamma }^{{{\nu }_2}}}.{{\gamma }^{{{\nu }_3}}}.{{\gamma }^{{{\nu }_4}}}.
   {{\gamma }^{{{\nu }_5}}}.{{\gamma }^{{{\nu }_6}}}.{{\gamma }^{\mu }}
}

\dispSFinmath{
\Muserfunction{DiracSimplify}[\%]
}

\dispSFoutmath{
\MathBegin{MathArray}{l}
D\multsp {{\gamma }^{{{\nu }_1}}}.{{\gamma }^{{{\nu }_2}}}.{{\gamma }^{{{\nu }_3}}}.{{\gamma }^{{{\nu }_4}}}.
     {{\gamma }^{{{\nu }_5}}}.{{\gamma }^{{{\nu }_6}}}-
   12\multsp {{\gamma }^{{{\nu }_1}}}.{{\gamma }^{{{\nu }_2}}}.{{\gamma }^{{{\nu }_3}}}.{{\gamma }^{{{\nu }_4}}}.{{\gamma }^{{{\nu }_5}}}
     .{{\gamma }^{{{\nu }_6}}}+4\multsp {{\gamma }^{{{\nu }_3}}}.{{\gamma }^{{{\nu }_4}}}.{{\gamma }^{{{\nu }_5}}}.
     {{\gamma }^{{{\nu }_6}}}\multsp {g^{{{\nu }_1}{{\nu }_2}}}-  \\
\noalign{\vspace{0.666667ex}}
\hspace{1.em} 4\multsp
    {{\gamma }^{{{\nu }_2}}}.{{\gamma }^{{{\nu }_4}}}.{{\gamma }^{{{\nu }_5}}}.{{\gamma }^{{{\nu }_6}}}\multsp {g^{{{\nu }_1}{{\nu }_3}}}
    +4\multsp {{\gamma }^{{{\nu }_2}}}.{{\gamma }^{{{\nu }_3}}}.{{\gamma }^{{{\nu }_5}}}.{{\gamma }^{{{\nu }_6}}}\multsp
    {g^{{{\nu }_1}{{\nu }_4}}}-4\multsp {{\gamma }^{{{\nu }_2}}}.{{\gamma }^{{{\nu }_3}}}.{{\gamma }^{{{\nu }_4}}}.
     {{\gamma }^{{{\nu }_6}}}\multsp {g^{{{\nu }_1}{{\nu }_5}}}+  \\
\noalign{\vspace{0.666667ex}}
\hspace{1.em} 4\multsp
    {{\gamma }^{{{\nu }_2}}}.{{\gamma }^{{{\nu }_3}}}.{{\gamma }^{{{\nu }_4}}}.{{\gamma }^{{{\nu }_5}}}\multsp {g^{{{\nu }_1}{{\nu }_6}}}
    +4\multsp {{\gamma }^{{{\nu }_1}}}.{{\gamma }^{{{\nu }_4}}}.{{\gamma }^{{{\nu }_5}}}.{{\gamma }^{{{\nu }_6}}}\multsp
    {g^{{{\nu }_2}{{\nu }_3}}}-4\multsp {{\gamma }^{{{\nu }_1}}}.{{\gamma }^{{{\nu }_3}}}.{{\gamma }^{{{\nu }_5}}}.
     {{\gamma }^{{{\nu }_6}}}\multsp {g^{{{\nu }_2}{{\nu }_4}}}+  \\
\noalign{\vspace{0.666667ex}}
\hspace{1.em} 4\multsp
    {{\gamma }^{{{\nu }_1}}}.{{\gamma }^{{{\nu }_3}}}.{{\gamma }^{{{\nu }_4}}}.{{\gamma }^{{{\nu }_6}}}\multsp {g^{{{\nu }_2}{{\nu }_5}}}
    -4\multsp {{\gamma }^{{{\nu }_1}}}.{{\gamma }^{{{\nu }_3}}}.{{\gamma }^{{{\nu }_4}}}.{{\gamma }^{{{\nu }_5}}}\multsp
    {g^{{{\nu }_2}{{\nu }_6}}}+4\multsp {{\gamma }^{{{\nu }_1}}}.{{\gamma }^{{{\nu }_2}}}.{{\gamma }^{{{\nu }_5}}}.
     {{\gamma }^{{{\nu }_6}}}\multsp {g^{{{\nu }_3}{{\nu }_4}}}-
   4\multsp {{\gamma }^{{{\nu }_1}}}.{{\gamma }^{{{\nu }_2}}}.{{\gamma }^{{{\nu }_4}}}.{{\gamma }^{{{\nu }_6}}}\multsp
    {g^{{{\nu }_3}{{\nu }_5}}}+  \\
\noalign{\vspace{0.666667ex}}
\hspace{1.em} 4\multsp
    {{\gamma }^{{{\nu }_1}}}.{{\gamma }^{{{\nu }_2}}}.{{\gamma }^{{{\nu }_4}}}.{{\gamma }^{{{\nu }_5}}}\multsp {g^{{{\nu }_3}{{\nu }_6}}}
    +4\multsp {{\gamma }^{{{\nu }_1}}}.{{\gamma }^{{{\nu }_2}}}.{{\gamma }^{{{\nu }_3}}}.{{\gamma }^{{{\nu }_6}}}\multsp
    {g^{{{\nu }_4}{{\nu }_5}}}-4\multsp {{\gamma }^{{{\nu }_1}}}.{{\gamma }^{{{\nu }_2}}}.{{\gamma }^{{{\nu }_3}}}.
     {{\gamma }^{{{\nu }_5}}}\multsp {g^{{{\nu }_4}{{\nu }_6}}}+
   4\multsp {{\gamma }^{{{\nu }_1}}}.{{\gamma }^{{{\nu }_2}}}.{{\gamma }^{{{\nu }_3}}}.{{\gamma }^{{{\nu }_4}}}\multsp
    {g^{{{\nu }_5}{{\nu }_6}}}\\
\MathEnd{MathArray}
}

 With the option DiracCanonical an alphabetic ordering is done.

\dispSFinmath{
\Muserfunction{DiracSimplify}[\Muserfunction{GA}[\nu ,\mu ],\Mvariable{DiracCanonical}\rightarrow \Mvariable{True}]
}

\dispSFoutmath{
2\multsp {g^{\mu \nu }}-{{\gamma }^{\mu }}.{{\gamma }^{\nu }}
}

Setting InsideDiracTrace\(\rightarrow \)True assumes that a trace is still to be taken later on.

\dispSFinmath{
\Muserfunction{DiracSimplify}[\Muserfunction{GA}[\mu ,\nu ,\rho ,\sigma ],\Mvariable{InsideDiracTrace}\rightarrow \Mvariable{True}]
}

\dispSFoutmath{
{g^{\mu \sigma }}\multsp {g^{\nu \rho }}-{g^{\mu \rho }}\multsp {g^{\nu \sigma }}+{g^{\mu \nu }}\multsp {g^{\rho \sigma }}
}

\dispSFinmath{
\Muserfunction{DiracSimplify}[\Muserfunction{GA}[\mu ,\nu ,\rho ],\Mvariable{InsideDiracTrace}\rightarrow \Mvariable{True}]
}

\dispSFoutmath{
0
}

\dispSFinmath{
\Mfunction{Clear}[\Mvariable{t1},\Mvariable{t2},\Mvariable{t3},\Mvariable{t4},\Mvariable{t5},\Mvariable{t6}]
}

\Subsection*{DiracSimplify2}

\Subsubsection*{Description}

DiracSimplify2[exp] simplifies the Dirac structure but leaves any \({{\gamma }^5}\) untouched.

See also:  DiracSimplify.

\Subsubsection*{Examples}

\dispSFinmath{
\Muserfunction{GAD}[\mu ,\nu ,\mu ,5,\alpha ,\beta ,\alpha ]
}

\dispSFoutmath{
{{\gamma }^{\mu }}.{{\gamma }^{\nu }}.{{\gamma }^{\mu }}.{{\gamma }^5}.{{\gamma }^{\alpha }}.{{\gamma }^{\beta }}.{{\gamma }^{\alpha }}
}

\dispSFinmath{
\Muserfunction{DiracSimplify2}[\%]
}

\dispSFoutmath{
{{\gamma }^{\mu }}.{{\gamma }^{\nu }}.{{\gamma }^{\mu }}.{{\gamma }^5}.{{\gamma }^{\alpha }}.{{\gamma }^{\beta }}.{{\gamma }^{\alpha }}
}

\Subsection*{DiracSlash}

\Subsubsection*{Description}

DiracSlash[p] is the contraction \({p^{\mu }}{{\gamma }_{\mu }}\multsp \)(FourVector[p, \(\mu \)] DiracMatrix[\(\mu \)]). Products of those can be
entered in the form DiracSlash[p1, p2, ...].

\dispSFinmath{
\Mfunction{Options}[\Mvariable{DiracSlash}]
}

\dispSFoutmath{
\{\Mvariable{Dimension}\rightarrow 4,\Mvariable{FeynCalcInternal}\rightarrow \Mvariable{False}\}
}

See also:  DiracGammaExpand, DiracGamma, DiracMatrix, DiracSimplify, DiracTrick, GS, GSD.

\Subsubsection*{Examples}

This is q-slash, i.e., \({{\gamma }^{\mu }}{q_{\mu }}.\)

\dispSFinmath{
\Muserfunction{DiracSlash}[q]
}

\dispSFoutmath{
\gamma \cdot q
}

\dispSFinmath{
\Muserfunction{DiracSlash}[p].\Muserfunction{DiracSlash}[q]
}

\dispSFoutmath{
(\gamma \cdot p).(\gamma \cdot q)
}

\dispSFinmath{
\Muserfunction{DiracSlash}[p,q]
}

\dispSFoutmath{
(\gamma \cdot p).(\gamma \cdot q)
}

\dispSFinmath{
\Muserfunction{GS}[p]
}

\dispSFoutmath{
\gamma \cdot p
}

\dispSFinmath{
\Muserfunction{DiracSlash}[q]//\Mfunction{StandardForm}
}

\dispSFoutmath{
\Muserfunction{DiracSlash}[q]
}

\dispSFinmath{
\Muserfunction{DiracSlash}[q,\Mvariable{Dimension}\rightarrow n]//\Mfunction{StandardForm}
}

\dispSFoutmath{
\Muserfunction{DiracSlash}[q,\Mvariable{Dimension}\rightarrow n]
}

\Subsection*{DiracSpinor}

\Subsubsection*{Description}

DiracSpinor is simply a quantity defined as noncommutative (with DeclareNonCommutative[DiracSpinor]). The convention intended is that
  DiracSpinor[p, m, ind] is a Dirac spinor for a fermion with momentum p and mass m and indices ind.

See also: Spinor.

\Subsection*{DiracSubstitute67}

\Subsubsection*{Description}

DiracSubstitute67 is an option for DiracSimplify. If set to True the chirality-projectors DiracGamma[6] and DiracGamma[7] are substituted
  by their definitions.

See also: DiracGamma, DiracSimplify.

\Subsection*{DiracTrace}

\Subsubsection*{Description}

DiracTrace[expr] is the head of Dirac traces. Whether the trace is evaluated depends on the option DiracTraceEvaluate. Direct trace
  evaluation should be performed with Tr. The argument expr may be a product of Dirac matrices or slashes separated by the Mathematica
  Dot (.).

\dispSFinmath{
\Mfunction{Options}[\Mvariable{DiracTrace}]
}

\dispSFoutmath{
\MathBegin{MathArray}{l}
\{\Mvariable{EpsContract}\rightarrow \Mvariable{False},\Mvariable{Factoring}\rightarrow \Mvariable{False},
    \Mvariable{FeynCalcExternal}\rightarrow \Mvariable{False},\Mvariable{Mandelstam}\rightarrow \{\},  \\
\noalign{\vspace{
   0.666667ex}}
\hspace{1.em} \Mvariable{PairCollect}\rightarrow \Mvariable{True},
    \Mvariable{DiracTraceEvaluate}\rightarrow \Mvariable{False},\Mvariable{Schouten}\rightarrow 0,
    \Mvariable{LeviCivitaSign}\rightarrow -1,\Mvariable{TraceOfOne}\rightarrow 4\}\\
\MathEnd{MathArray}
}

For comments regarding \({{\gamma }^5}\)schemes see the notes for Tr.

See also:  Tr.

\Subsubsection*{Examples}

\dispSFinmath{
\Muserfunction{DiracTrace}[\Muserfunction{GA}[\mu ,\nu ]]
}

\dispSFoutmath{
\Muserfunction{tr}({{\gamma }^{\mu }}.{{\gamma }^{\nu }})
}

\dispSFinmath{
\Muserfunction{DiracTrace}[\Muserfunction{GA}[\mu ,\nu ,\rho ,\sigma ]]
}

\dispSFoutmath{
\Muserfunction{tr}({{\gamma }^{\mu }}.{{\gamma }^{\nu }}.{{\gamma }^{\rho }}.{{\gamma }^{\sigma }})
}

\dispSFinmath{
\%\multsp /.\multsp \Mvariable{DiracTrace}\rightarrow \Mvariable{Tr}
}

\dispSFoutmath{
4\multsp ({g^{\mu \sigma }}\multsp {g^{\nu \rho }}-{g^{\mu \rho }}\multsp {g^{\nu \sigma }}+{g^{\mu \nu }}\multsp {g^{\rho \sigma }})
}

\dispSFinmath{
\Muserfunction{DiracTrace}[\Muserfunction{GA}[\mu ,\nu ,\rho ,\sigma ,5],\Mvariable{DiracTraceEvaluate}\rightarrow \Mvariable{True}]
}

\dispSFoutmath{
-4\multsp \ImaginaryI \multsp {{\epsilon }^{\mu \nu \rho \sigma }}
}

\dispSFinmath{
\Muserfunction{DiracTrace}[\Muserfunction{GA}[\mu ,\nu ,\rho ,\sigma ,\delta ,\tau ,5],
    \Mvariable{DiracTraceEvaluate}\rightarrow \Mvariable{True}]
}

\dispSFoutmath{
\MathBegin{MathArray}{l}
4\multsp (\ImaginaryI \multsp {{\epsilon }^{\nu \rho \sigma \tau }}\multsp {g^{\delta \mu }}-
     \ImaginaryI \multsp {{\epsilon }^{\mu \rho \sigma \tau }}\multsp {g^{\delta \nu }}+
     \ImaginaryI \multsp {{\epsilon }^{\mu \nu \sigma \tau }}\multsp {g^{\delta \rho }}-
     \ImaginaryI \multsp {{\epsilon }^{\mu \nu \rho \tau }}\multsp {g^{\delta \sigma }}-  \\
\noalign{\vspace{0.604167ex}}
   \hspace{3.em} \ImaginaryI \multsp {{\epsilon }^{\mu \nu \rho \sigma }}\multsp {g^{\delta \tau }}-
   \ImaginaryI \multsp {{\epsilon }^{\delta \rho \sigma \tau }}\multsp {g^{\mu \nu }}+
   \ImaginaryI \multsp {{\epsilon }^{\delta \nu \sigma \tau }}\multsp {g^{\mu \rho }}-
   \ImaginaryI \multsp {{\epsilon }^{\delta \nu \rho \tau }}\multsp {g^{\mu \sigma }}+
   \ImaginaryI \multsp {{\epsilon }^{\delta \nu \rho \sigma }}\multsp {g^{\mu \tau }}-  \\
\noalign{\vspace{0.604167ex}}
   \hspace{3.em} \ImaginaryI \multsp {{\epsilon }^{\delta \mu \sigma \tau }}\multsp {g^{\nu \rho }}+
    \ImaginaryI \multsp {{\epsilon }^{\delta \mu \rho \tau }}\multsp {g^{\nu \sigma }}-
    \ImaginaryI \multsp {{\epsilon }^{\delta \mu \rho \sigma }}\multsp {g^{\nu \tau }}-
    \ImaginaryI \multsp {{\epsilon }^{\delta \mu \nu \tau }}\multsp {g^{\rho \sigma }}+
    \ImaginaryI \multsp {{\epsilon }^{\delta \mu \nu \sigma }}\multsp {g^{\rho \tau }}-
    \ImaginaryI \multsp {{\epsilon }^{\delta \mu \nu \rho }}\multsp {g^{\sigma \tau }})\\
\MathEnd{MathArray}
}

\dispSFinmath{
\Muserfunction{DiracTrace}[\Muserfunction{GS}[p,q,r,s]]
}

\dispSFoutmath{
\Muserfunction{tr}((\gamma \cdot p).(\gamma \cdot q).(\gamma \cdot r).(\gamma \cdot s))
}

\dispSFinmath{
\Muserfunction{DiracTrace}[\Muserfunction{GA}[\mu ,\nu ],\Mvariable{DiracTraceEvaluate}\rightarrow \Mvariable{True},
    \Mvariable{FCE}\rightarrow \Mvariable{True}]
}

\dispSFoutmath{
4\multsp {g^{\mu \nu }}
}

\dispSFinmath{
\%//\Mfunction{StandardForm}
}

\dispSFoutmath{
4\multsp \Muserfunction{MT}[\mu ,\nu ]
}

\Subsection*{DiracTraceEvaluate}

\Subsubsection*{Description}

DiracTraceEvaluate is an option for DiracTrace and Tr. If set to False, DiracTrace remains unevaluated.

See also: DiracTrace, Tr.

\Subsection*{DiracTrick}

\Subsubsection*{Description}

DiracTrick[exp] contracts gamma matrices with each other and performs several simplifications, but no expansion, use Calc or
  DiracSimplify for non-commutative expansion.

\dispSFinmath{
\Mfunction{Options}[\Mvariable{DiracTrick}]
}

\dispSFoutmath{
\{\Mvariable{Expanding}\rightarrow \Mvariable{False}\}
}

See also:  Calc, DiracGammaExpand, DiracSimplify.

\Subsubsection*{Examples}

This is a string of Dirac matrices in four dimensions.

\dispSFinmath{
\Mvariable{t1}=\Muserfunction{GA}[\mu ,\nu ,\mu ]
}

\dispSFoutmath{
{{\gamma }^{\mu }}.{{\gamma }^{\nu }}.{{\gamma }^{\mu }}
}

\dispSFinmath{
\Muserfunction{DiracTrick}[\Mvariable{t1}]
}

\dispSFoutmath{
-2\multsp {{\gamma }^{\nu }}
}

This is a string of Dirac matrices in D dimensions.

\dispSFinmath{
\Mvariable{t2}=\Muserfunction{GAD}[\mu ,\nu ,\mu ]
}

\dispSFoutmath{
{{\gamma }^{\mu }}.{{\gamma }^{\nu }}.{{\gamma }^{\mu }}
}

\dispSFinmath{
\Muserfunction{DiracTrick}[\Mvariable{t2}]
}

\dispSFoutmath{
(2-D)\multsp {{\gamma }^{\nu }}
}

\dispSFinmath{
\Mvariable{t3}=\Muserfunction{GA}[5,\mu ,\nu ]
}

\dispSFoutmath{
{{\gamma }^5}.{{\gamma }^{\mu }}.{{\gamma }^{\nu }}
}

By default \({{\gamma }^5}\)is moved to the right.

\dispSFinmath{
\Muserfunction{DiracTrick}[\Mvariable{t3}]
}

\dispSFoutmath{
{{\gamma }^{\mu }}.{{\gamma }^{\nu }}.{{\gamma }^5}
}

\dispSFinmath{
\Mvariable{t4}=\Muserfunction{GA}[6,\mu ,7]
}

\dispSFoutmath{
{{\gamma }^6}.{{\gamma }^{\mu }}.{{\gamma }^7}
}

\dispSFinmath{
\Muserfunction{DiracTrick}[\Mvariable{t4}]
}

\dispSFoutmath{
{{\gamma }^{\mu }}.{{\gamma }^7}
}

\dispSFinmath{
\Mvariable{t5}=\Muserfunction{GS}[a+b]\multsp .\multsp \Muserfunction{GS}[p].\Muserfunction{GS}[p].\Muserfunction{GS}[c+d]
}

\dispSFoutmath{
(\gamma \cdot (a+b)).(\gamma \cdot p).(\gamma \cdot p).(\gamma \cdot (c+d))
}

\dispSFinmath{
\Muserfunction{DiracTrick}[\Mvariable{t5}]
}

\dispSFoutmath{
(\gamma \cdot (a+b)).(\gamma \cdot (c+d))\multsp {p^2}
}

\dispSFinmath{
\Muserfunction{Calc}[\Mvariable{t5}]
}

\dispSFoutmath{
(\gamma \cdot a).(\gamma \cdot c)\multsp {p^2}+(\gamma \cdot a).(\gamma \cdot d)\multsp {p^2}+
   (\gamma \cdot b).(\gamma \cdot c)\multsp {p^2}+(\gamma \cdot b).(\gamma \cdot d)\multsp {p^2}
}

\dispSFinmath{
\Mvariable{GAD}@@\Mfunction{Join}[\{\mu \},\Mfunction{Table}[{{\nu }_i},\{i,6\}],\{\mu \}]
}

\dispSFoutmath{
{{\gamma }^{\mu }}.{{\gamma }^{{{\nu }_1}}}.{{\gamma }^{{{\nu }_2}}}.{{\gamma }^{{{\nu }_3}}}.{{\gamma }^{{{\nu }_4}}}.
   {{\gamma }^{{{\nu }_5}}}.{{\gamma }^{{{\nu }_6}}}.{{\gamma }^{\mu }}
}

\dispSFinmath{
\Muserfunction{DiracTrick}[\%]
}

\dispSFoutmath{
\MathBegin{MathArray}{l}
(D-12)\multsp {{\gamma }^{{{\nu }_1}}}.{{\gamma }^{{{\nu }_2}}}.{{\gamma }^{{{\nu }_3}}}.
     {{\gamma }^{{{\nu }_4}}}.{{\gamma }^{{{\nu }_5}}}.{{\gamma }^{{{\nu }_6}}}-  \\
\noalign{\vspace{0.666667ex}}
\hspace{1.em} 4
   \multsp (-{{\gamma }^{{{\nu }_3}}}.{{\gamma }^{{{\nu }_4}}}.{{\gamma }^{{{\nu }_5}}}.{{\gamma }^{{{\nu }_6}}}\multsp
      {g^{{{\nu }_1}{{\nu }_2}}}+{{\gamma }^{{{\nu }_2}}}.{{\gamma }^{{{\nu }_4}}}.{{\gamma }^{{{\nu }_5}}}.{{\gamma }^{{{\nu }_6}}}
      \multsp {g^{{{\nu }_1}{{\nu }_3}}}-{{\gamma }^{{{\nu }_2}}}.{{\gamma }^{{{\nu }_3}}}.{{\gamma }^{{{\nu }_5}}}.
       {{\gamma }^{{{\nu }_6}}}\multsp {g^{{{\nu }_1}{{\nu }_4}}}+
     {{\gamma }^{{{\nu }_2}}}.{{\gamma }^{{{\nu }_3}}}.{{\gamma }^{{{\nu }_4}}}.{{\gamma }^{{{\nu }_6}}}\multsp
      {g^{{{\nu }_1}{{\nu }_5}}}-  \\
\noalign{\vspace{0.666667ex}}
\hspace{4.em} {{\gamma }^{{{\nu }_2}}}.{{\gamma }^{{{\nu }_3}}}.
     {{\gamma }^{{{\nu }_4}}}.{{\gamma }^{{{\nu }_5}}}\multsp {g^{{{\nu }_1}{{\nu }_6}}}-
   {{\gamma }^{{{\nu }_1}}}.{{\gamma }^{{{\nu }_4}}}.{{\gamma }^{{{\nu }_5}}}.{{\gamma }^{{{\nu }_6}}}\multsp {g^{{{\nu }_2}{{\nu }_3}}}+
   {{\gamma }^{{{\nu }_1}}}.{{\gamma }^{{{\nu }_3}}}.{{\gamma }^{{{\nu }_5}}}.{{\gamma }^{{{\nu }_6}}}\multsp {g^{{{\nu }_2}{{\nu }_4}}}-
    \\
\noalign{\vspace{0.666667ex}}
\hspace{4.em} {{\gamma }^{{{\nu }_1}}}.{{\gamma }^{{{\nu }_3}}}.{{\gamma }^{{{\nu }_4}}}.
     {{\gamma }^{{{\nu }_6}}}\multsp {g^{{{\nu }_2}{{\nu }_5}}}+
   {{\gamma }^{{{\nu }_1}}}.{{\gamma }^{{{\nu }_3}}}.{{\gamma }^{{{\nu }_4}}}.{{\gamma }^{{{\nu }_5}}}\multsp {g^{{{\nu }_2}{{\nu }_6}}}-
   {{\gamma }^{{{\nu }_1}}}.{{\gamma }^{{{\nu }_2}}}.{{\gamma }^{{{\nu }_5}}}.{{\gamma }^{{{\nu }_6}}}\multsp {g^{{{\nu }_3}{{\nu }_4}}}+
   {{\gamma }^{{{\nu }_1}}}.{{\gamma }^{{{\nu }_2}}}.{{\gamma }^{{{\nu }_4}}}.{{\gamma }^{{{\nu }_6}}}\multsp {g^{{{\nu }_3}{{\nu }_5}}}-
    \\
\noalign{\vspace{0.666667ex}}
\hspace{4.em} {{\gamma }^{{{\nu }_1}}}.{{\gamma }^{{{\nu }_2}}}.{{\gamma }^{{{\nu }_4}}}.
      {{\gamma }^{{{\nu }_5}}}\multsp {g^{{{\nu }_3}{{\nu }_6}}}-
    {{\gamma }^{{{\nu }_1}}}.{{\gamma }^{{{\nu }_2}}}.{{\gamma }^{{{\nu }_3}}}.{{\gamma }^{{{\nu }_6}}}\multsp {g^{{{\nu }_4}{{\nu }_5}}}
    +{{\gamma }^{{{\nu }_1}}}.{{\gamma }^{{{\nu }_2}}}.{{\gamma }^{{{\nu }_3}}}.{{\gamma }^{{{\nu }_5}}}\multsp
     {g^{{{\nu }_4}{{\nu }_6}}}-{{\gamma }^{{{\nu }_1}}}.{{\gamma }^{{{\nu }_2}}}.{{\gamma }^{{{\nu }_3}}}.{{\gamma }^{{{\nu }_4}}}
     \multsp {g^{{{\nu }_5}{{\nu }_6}}})\\
\MathEnd{MathArray}
}

\Subsection*{Divideout}

\Subsubsection*{Description}

Divideout is an option for OPEInt and OPEInsert. The setting is divided out at the end.

See also: OPEInt, OPEInsert.

\Subsection*{DOT}

\Subsubsection*{Description}

DOT[a, b, ...] is the FeynCalc function for non-commutative multiplication. By default it is set to the Mathematica Dot functions. By
  setting\\
DOT\(=\).\\
this can be disabled. Note that then non-commutative products should to be entered like DOT[ DiracMatrix[mu], m \(+\) DiracSlash[p],
  DiracMatrix[mu] ], etc.

See also: DotSimplify.

\Subsection*{DotExpand}

\Subsubsection*{Description}

DotExpand[expr] expands DOT products in expr.

See also: DOT, DotSimplify, DeclareNonCommutative, UnDeclareNonCommutative.

\Subsubsection*{Examples}

\dispSFinmath{
\Muserfunction{DotExpand}[\Muserfunction{DOT}[a\multsp x+b\multsp y+c\multsp z,d+e+f]]
}

\dispSFoutmath{
a\multsp d\multsp x+a\multsp e\multsp x+a\multsp f\multsp x+b\multsp d\multsp y+b\multsp e\multsp y+b\multsp f\multsp y+
   c\multsp d\multsp z+c\multsp e\multsp z+c\multsp f\multsp z
}

\dispSFinmath{
\Mvariable{DeclareNonCommutative}/@\{a,b,c,d,e,f\};
}

\dispSFinmath{
\Muserfunction{DotExpand}[\Muserfunction{DOT}[a\multsp x+b\multsp y+c\multsp z,d+e+f]]
}

\dispSFoutmath{
x\multsp a.d+x\multsp a.e+x\multsp a.f+y\multsp b.d+y\multsp b.e+y\multsp b.f+z\multsp c.d+z\multsp c.e+z\multsp c.f
}

\dispSFinmath{
\Mvariable{UnDeclareNonCommutative}/@\{a,b,c,d,e,f\};
}

\dispSFinmath{
\Muserfunction{DotExpand}[\Muserfunction{DOT}[a\multsp x+b\multsp y+c\multsp z,d+e+f]]
}

\dispSFoutmath{
a\multsp d\multsp x+a\multsp e\multsp x+a\multsp f\multsp x+b\multsp d\multsp y+b\multsp e\multsp y+b\multsp f\multsp y+
   c\multsp d\multsp z+c\multsp e\multsp z+c\multsp f\multsp z
}

\Subsection*{DotPower}

\Subsubsection*{Description}

DotPower is an option for DotSimplify. It determines whether non-commutative powers are represented by succesive multiplication or by
  Power.

See also: DotSimplify.

\Subsection*{DotProduct}

\Subsubsection*{Description}

DotProduct[x, y] denotes the three-dimensional dot-product. If x and y have Head List, DotProduct[x, a] (where a is a vector) performs
  Sum[ x[[k]] a[[k]], \{k, 0, 3\}].

See also:  CrossProduct, ThreeVector.

\Subsubsection*{Examples}

\dispSFinmath{
\Muserfunction{DotProduct}[\Muserfunction{ThreeVector}[a],3\Muserfunction{ThreeVector}[b]]
}

\dispSFoutmath{
3\multsp \overvar{a}{\rightharpoonup }\cdot \overvar{b}{\rightharpoonup }
}

\Subsection*{DotSimplifyRelations}

\Subsubsection*{Description}

DotSimplifyRelations is an option for DotSimplify. Its setting may be a list of substitution rules of the form DotSimplifyRelations
  \(\rightarrow \) \{a . b \(\rightarrow \) c, b\(\RawWedge\)2 \(\rightarrow \) 0, ...\}.

See also: DotSimplify.

\Subsection*{DotSimplify}

\Subsubsection*{Description}

DotSimplify[expr] expands and reorders noncommutative terms in expr. Simplifying relations may be specified by the option
  DotSimplifyRelations or by Commutator and AntiCommutator definitions. Whether expr is expanded noncommutatively depends on the option
  Expanding.

\dispSFinmath{
\Mfunction{Options}[\Mvariable{DotSimplify}]
}

\dispSFoutmath{
\{\Mvariable{Expanding}\rightarrow \Mvariable{True},\Mvariable{DotSimplifyRelations}\rightarrow \{\},
    \Mvariable{DotPower}\rightarrow \Mvariable{False}\}
}

See also:  AntiCommutator, Commutator, Calc.

\Subsubsection*{Examples}

\dispSFinmath{
\Mvariable{t1}=\Muserfunction{GA}[\mu ].(2\multsp \Muserfunction{GS}[p]-\Muserfunction{GS}[q]).\Muserfunction{GA}[\nu ]
}

\dispSFoutmath{
{{\gamma }^{\mu }}.(2\multsp \gamma \cdot p-\gamma \cdot q).{{\gamma }^{\nu }}
}

\dispSFinmath{
\Muserfunction{DotSimplify}[\Mvariable{t1}]
}

\dispSFoutmath{
2\multsp {{\gamma }^{\mu }}.(\gamma \cdot p).{{\gamma }^{\nu }}-{{\gamma }^{\mu }}.(\gamma \cdot q).{{\gamma }^{\nu }}
}

\dispSFinmath{
\Muserfunction{DeclareNonCommutative}[a,b,c]
}

\dispSFinmath{
\Mvariable{t2}=a.(b-z\multsp c).a
}

\dispSFoutmath{
a.(b-c\multsp z).a
}

\dispSFinmath{
\Muserfunction{DotSimplify}[\Mvariable{t2}]
}

\dispSFoutmath{
a.b.a-z\multsp a.c.a
}

\dispSFinmath{
\Muserfunction{Commutator}[a,c]=1
}

\dispSFoutmath{
1
}

\dispSFinmath{
\Muserfunction{DotSimplify}[\Mvariable{t2}]
}

\dispSFoutmath{
a.b.a-z\multsp (a+c.a.a)
}

\dispSFinmath{
\Muserfunction{Commutator}[a,c]=.
}

\dispSFinmath{
\Muserfunction{DotSimplify}[\Mvariable{t2}]
}

\dispSFoutmath{
a.b.a-z\multsp a.c.a
}

\dispSFinmath{
\Muserfunction{AntiCommutator}[b,a]=c
}

\dispSFoutmath{
c
}

\dispSFinmath{
\Muserfunction{DotSimplify}[\Mvariable{t2}]
}

\dispSFoutmath{
a.c-a.a.b-z\multsp a.c.a
}

\dispSFinmath{
\Muserfunction{AntiCommutator}[b,a]=.
}

\dispSFinmath{
\Muserfunction{DotSimplify}[\Mvariable{t2},\Mvariable{DotSimplifyRelations}\rightarrow \{a.c\rightarrow 1/z\}]
}

\dispSFoutmath{
a.b.a-a
}

\dispSFinmath{
\Muserfunction{DeclareNonCommutative}[x]
}

\dispSFinmath{
\Muserfunction{DotSimplify}[x.x.x]
}

\dispSFoutmath{
x.x.x
}

\dispSFinmath{
\Muserfunction{DotSimplify}[x.x.x,\Mvariable{DotPower}\rightarrow \Mvariable{False}]
}

\dispSFoutmath{
x.x.x
}

\dispSFinmath{
\Muserfunction{UnDeclareNonCommutative}[a,b,c,x]
}

\Subsection*{DummyIndex}

\Subsubsection*{Description}

DummyIndex is an option of CovariantD specifying an index to use as dummy summation index. If set to Automatic, unique indices are
  generated

See also: CovariantD.

\Subsection*{D0}

\Subsubsection*{Description}

D0[ p10, p12, p23, p30, p20, p13, m1\(\RawWedge\)2, m2\(\RawWedge\)2, m3\(\RawWedge\)2, m4\(\RawWedge\)2 ] is the Passarino-Veltman \({D_0}\) function.
The convention for the arguments is that if the denominator of the integrand has the form ([q\(\RawWedge\)2-m1\(\RawWedge\)2]
  [(q\(+\)p1)\(\RawWedge\)2-m2\(\RawWedge\)2] [(q\(+\)p2)\(\RawWedge\)2-m3\(\RawWedge\)2] [(q\(+\)p3)\(\RawWedge\)2-m4\(\RawWedge\)2] ),
  the first six arguments of D0 are the scalar products p10 \(=\) p1\(\RawWedge\)2, p12 \(=\) (p1-p2)\(\RawWedge\)2, p23 \(=\)
  (p2-p3)\(\RawWedge\)2, p30 \(=\) p3\(\RawWedge\)2, p20 \(=\) p2\(\RawWedge\)2, p13 \(=\) (p1-p3)\(\RawWedge\)2.

See also:  B0, C0, PaVe, PaVeOrder.

\Subsubsection*{Examples}

\dispSFinmath{
\Muserfunction{D0}[\Mvariable{p10},\Mvariable{p12},\Mvariable{p23},\Mvariable{p30},\Mvariable{p20},\Mvariable{p13},
    \Mvariable{m1}\RawWedge 2,\Mvariable{m2}\RawWedge 2,\Mvariable{m3}\RawWedge 2,\Mvariable{m4}\RawWedge 2]
}

\dispSFoutmath{
{D_0}\big({p_{10}},{p_{12}},\Mvariable{p23},\Mvariable{p30},\Mvariable{p20},\Mvariable{p13},{{\Mvariable{m1}}^2},{{\Mvariable{m2}}^2},
   {{\Mvariable{m3}}^2},{{\Mvariable{m4}}^2}\big)
}

\dispSFinmath{
\MathBegin{MathArray}{l}
\Muserfunction{PaVeOrder}[\Muserfunction{D0}[
     \Mvariable{p10},\Mvariable{p12},\Mvariable{p23},\Mvariable{p30},\Mvariable{p20},\Mvariable{p13},\Mvariable{m1}\RawWedge 2,
      \Mvariable{m2}\RawWedge 2,\Mvariable{m3}\RawWedge 2,\Mvariable{m4}\RawWedge 2],  \\
\noalign{\vspace{0.5ex}}
\hspace{1.em}
      \Mvariable{PaVeOrderList}\rightarrow \{\Mvariable{p13},\Mvariable{p20}\}]\\
\MathEnd{MathArray}
}

\dispSFoutmath{
{D_0}\big({p_{10}},\Mvariable{p30},\Mvariable{p23},{p_{12}},\Mvariable{p13},\Mvariable{p20},{{\Mvariable{m2}}^2},{{\Mvariable{m1}}^2},
   {{\Mvariable{m4}}^2},{{\Mvariable{m3}}^2}\big)
}

\dispSFinmath{
\Muserfunction{PaVeOrder}[\%]
}

\dispSFoutmath{
{D_0}\big({p_{10}},{p_{12}},\Mvariable{p23},\Mvariable{p30},\Mvariable{p20},\Mvariable{p13},{{\Mvariable{m1}}^2},{{\Mvariable{m2}}^2},
   {{\Mvariable{m3}}^2},{{\Mvariable{m4}}^2}\big)
}

\Subsection*{D0Convention}

\Subsubsection*{Description}

D0Convention is an option for Write2. If set to 1, the convention for the arguments of D0 is changed when writing a Fortran file with
  Write2: The fifth and sixth argument of D0 are interchanged and the square root is taken of the last four arguments.

See also: D0, Write2.

\Subsection*{Eps}

\Subsubsection*{Description}

Eps[a, b, c, d] is the head of the totally antisymmetric \(\multsp \epsilon \) (Levi-Civita) tensor. The a,b, ... may have head LorentzIndex, Momentum
or Integer. In case of integers the Levi-Civita tensor is
  evaluated immediately.

\dispSFinmath{
\Mfunction{Options}[\Mvariable{Eps}]
}

\dispSFoutmath{
\{\Mvariable{Dimension}\rightarrow 4\}
}

See also:  EpsEvaluate, LC, LCD, LeviCivita.

\Subsubsection*{Examples}

\dispSFinmath{
\Muserfunction{Eps}[\Muserfunction{LorentzIndex}[\mu ],\Muserfunction{LorentzIndex}[\nu ],\Muserfunction{LorentzIndex}[\rho ],
    \Muserfunction{LorentzIndex}[\sigma ]]
}

\dispSFoutmath{
{{\epsilon }^{\mu \nu \rho \sigma }}
}

\dispSFinmath{
\Muserfunction{Eps}[\Muserfunction{Momentum}[p],\Muserfunction{LorentzIndex}[\nu ],\Muserfunction{LorentzIndex}[\rho ],
    \Muserfunction{LorentzIndex}[\sigma ]]
}

\dispSFoutmath{
{{\epsilon }^{p\nu \rho \sigma }}
}

\dispSFinmath{
\Muserfunction{Eps}[b,a,c,d]//\Mfunction{StandardForm}
}

\dispSFoutmath{
-\Muserfunction{Eps}[a,b,c,d]
}

\dispSFinmath{
\Muserfunction{Eps}[0,1,2,3]
}

\dispSFoutmath{
1
}

\dispSFinmath{
\Muserfunction{Eps}[1,0,2,3]
}

\dispSFoutmath{
-1
}

\dispSFinmath{
\Mfunction{SetOptions}[\Mvariable{Eps},\Mvariable{Dimension}\rightarrow 4];
}

\dispSFinmath{
\MathBegin{MathArray}{l}
\Mvariable{a1}\multsp =  \\
\noalign{\vspace{0.5ex}}
\hspace{1.em} \Muserfunction{Eps}[
   \Muserfunction{LorentzIndex}[\mu ,D],\Muserfunction{LorentzIndex}[\nu ,D],\Muserfunction{LorentzIndex}[\rho ,D],
    \Muserfunction{LorentzIndex}[\sigma ,D]]\\
\MathEnd{MathArray}
}

\dispSFoutmath{
{{\epsilon }^{\mu \nu \rho \sigma }}
}

\dispSFinmath{
\Muserfunction{Contract}[\Mvariable{a1}\Mvariable{a1}]
}

\dispSFoutmath{
-24
}

\dispSFinmath{
\Mfunction{SetOptions}[\Mvariable{Eps},\Mvariable{Dimension}\rightarrow D];
}

\dispSFinmath{
\MathBegin{MathArray}{l}
\Mvariable{a2}\multsp =  \\
\noalign{\vspace{0.5ex}}
\hspace{1.em} \Muserfunction{Eps}[
   \Muserfunction{LorentzIndex}[\mu ,D],\Muserfunction{LorentzIndex}[\nu ,D],\Muserfunction{LorentzIndex}[\rho ,D],
    \Muserfunction{LorentzIndex}[\sigma ,D]]\\
\MathEnd{MathArray}
}

\dispSFoutmath{
{{\epsilon }^{\mu \nu \rho \sigma }}
}

\dispSFinmath{
\Muserfunction{Contract}[\Mvariable{a2}\Mvariable{a2}]//\Muserfunction{Factor2}
}

\dispSFoutmath{
(1-D)\multsp (2-D)\multsp (3-D)\multsp D
}

\dispSFinmath{
\Mvariable{g5}=-\frac{\ImaginaryI }{24}\multsp \Muserfunction{LCD}[\mu ,\nu ,\rho ,\alpha ].\Muserfunction{GAD}[\mu ,\nu ,\rho ,\alpha ]
    //\Muserfunction{FCI}
}

\dispSFoutmath{
-\frac{1}{24}\multsp \ImaginaryI \multsp {{\epsilon }^{\mu \nu \rho \alpha }}.{{\gamma }^{\mu }}.{{\gamma }^{\nu }}.{{\gamma }^{\rho }}.
    {{\gamma }^{\alpha }}
}

\dispSFinmath{
\Mvariable{g5p}=-\frac{\ImaginaryI }{24}\multsp \Muserfunction{LCD}[
       {{\mu }^{\prime }},{{\nu }^{\prime }},{{\rho }^{\prime }},{{\alpha }^{\prime }}].
      \Muserfunction{GAD}[{{\mu }^{\prime }},{{\nu }^{\prime }},{{\rho }^{\prime }},{{\alpha }^{\prime }}]//\Muserfunction{FCI}
}

\dispSFoutmath{
-\frac{1}{24}\multsp \ImaginaryI \multsp {{\epsilon }^{{{\mu }^{\prime }}{{\nu }^{\prime }}{{\rho }^{\prime }}{{\alpha }^{\prime }}}}.
    {{\gamma }^{{{\mu }^{\prime }}}}.{{\gamma }^{{{\nu }^{\prime }}}}.{{\gamma }^{{{\rho }^{\prime }}}}.
    {{\gamma }^{{{\alpha }^{\prime }}}}
}

\dispSFinmath{
\Mvariable{g52}=\Muserfunction{Factor2}[\Muserfunction{Calc}[\Mvariable{g5}.\Mvariable{g5p}]]
}

\dispSFoutmath{
-\frac{1}{24}\multsp (1-D)\multsp (2-D)\multsp (3-D)\multsp D
}

\dispSFinmath{
\Mvariable{g52}/.D\rightarrow 4
}

\dispSFoutmath{
1
}

\dispSFinmath{
\Mfunction{Clear}[\Mvariable{a1},\Mvariable{a2},\Mvariable{g5},\Mvariable{g5p},\Mvariable{g52}]
}

\Subsection*{EpsChisholm}

\Subsubsection*{Description}

EpsChisholm[expr] substitutes for a gamma matrix contracted with a Levi-Civita tensor (Eps) the Chisholm identity.

See also:  Chisholm.

\Subsubsection*{Examples}

\dispSFinmath{
\Muserfunction{Chisholm}[\Muserfunction{GA}[\mu ,\nu ,\rho ,\sigma ]]
}

\dispSFoutmath{
-\ImaginaryI \multsp {{\gamma }^{\$MU\$372}}.{{\gamma }^{\sigma }}.{{\gamma }^5}\multsp {{\epsilon }^{\mu \nu \rho \$MU\$372}}+
   {{\gamma }^{\rho }}.{{\gamma }^{\sigma }}\multsp {g^{\mu \nu }}-{{\gamma }^{\nu }}.{{\gamma }^{\sigma }}\multsp {g^{\mu \rho }}+
   {{\gamma }^{\mu }}.{{\gamma }^{\sigma }}\multsp {g^{\nu \rho }}
}

\dispSFinmath{
\Muserfunction{EpsChisholm}[\%]
}

\dispSFoutmath{
{{\gamma }^{\mu }}.{{\gamma }^{\nu }}.{{\gamma }^{\rho }}.{{\gamma }^{\sigma }}
}

\Subsection*{EpsContract}

\Subsubsection*{Description}

EpsContract is an option of Contract specifying whether Levi-Civita tensors Eps[...] will be contracted, i.e., products of two { }Eps are
  replaced via the determinant formula.

See also:  Eps, Contract.

\Subsubsection*{Examples}

\dispSFinmath{
\MathBegin{MathArray}{l}
\Mvariable{a1}\multsp =  \\
\noalign{\vspace{0.5ex}}
\hspace{1.em} \Muserfunction{Eps}[
   \Muserfunction{LorentzIndex}[\mu ,D],\Muserfunction{LorentzIndex}[\nu ,D],\Muserfunction{LorentzIndex}[\rho ,D],
    \Muserfunction{LorentzIndex}[\sigma ,D]]\\
\MathEnd{MathArray}
}

\dispSFoutmath{
{{\epsilon }^{\mu \nu \rho \sigma }}
}

\dispSFinmath{
\Muserfunction{Contract}[\Mvariable{a1}\multsp \Mvariable{a1},\Mvariable{EpsContract}\rightarrow \Mvariable{False}]
}

\dispSFoutmath{
{{({{\epsilon }^{\mu \nu \rho \sigma }})}^2}
}

\dispSFinmath{
\Muserfunction{Contract}[\Mvariable{a1}\multsp \Mvariable{a1},\Mvariable{EpsContract}\rightarrow \Mvariable{True}]
}

\dispSFoutmath{
-{D^4}+6\multsp {D^3}-11\multsp {D^2}+6\multsp D
}

\dispSFinmath{
\Mfunction{Clear}[\Mvariable{a1}]
}

\Subsection*{EpsDiscard}

\Subsubsection*{Description}

EpsDiscard is an option for FeynCalc2FORM. If set to True all Levi-Civita tensors are replaced by 0 after contraction.

See also:  FeynCalc2FORM.

\Subsection*{EpsEvaluate}

\Subsubsection*{Description}

EpsEvaluate[expr] applies total antisymmetry and linearity (w.r.t. Momentum's) to all Levi-Civita tensors (Eps') in expr.

See also:  Contract, Eps, LeviCivita, Trick.

\Subsubsection*{Examples}

\dispSFinmath{
\Muserfunction{Trick}[\Muserfunction{LeviCivita}[\mu ,\nu ,\rho ,\sigma ]\multsp \Muserfunction{FourVector}[p+q,\sigma ]]
}

\dispSFoutmath{
{{\epsilon }^{\mu \nu \rho p+q}}
}

\dispSFinmath{
\Muserfunction{EpsEvaluate}[\%]
}

\dispSFoutmath{
{{\epsilon }^{\mu \nu \rho p}}+{{\epsilon }^{\mu \nu \rho q}}
}

\dispSFinmath{
\Mfunction{StandardForm}[\%]
}

\dispSFoutmath{
\MathBegin{MathArray}{l}
\Muserfunction{Eps}[\Muserfunction{LorentzIndex}[\mu ,D],\Muserfunction{LorentzIndex}[\nu ,D],
     \Muserfunction{LorentzIndex}[\rho ,D],\Muserfunction{Momentum}[p]]+  \\
\noalign{\vspace{0.5ex}}
\hspace{1.em} \Muserfunction{Eps}[
   \Muserfunction{LorentzIndex}[\mu ,D],\Muserfunction{LorentzIndex}[\nu ,D],\Muserfunction{LorentzIndex}[\rho ,D],
    \Muserfunction{Momentum}[q]]\\
\MathEnd{MathArray}
}

\Subsection*{EpsUncontract}

\Subsubsection*{Description}

EpsUncontract does Uncontract on scalar products involving Eps.

See also:  Eps, Uncontract.

\Subsection*{Epsilon}

\Subsubsection*{Description}

Epsilon is ({\itshape n}-4), where {\itshape n} is the space-time dimension. Epsilon stands for a small positive number.

See also:  Series2.

\Subsubsection*{Examples}

\dispSFinmath{
\Mvariable{Epsilon}
}

\dispSFoutmath{
\varepsilon
}

Epsilon has no functional properties, but some upvalues are changed:

\dispSFinmath{
\MathBegin{MathArray}{l}
\{\Mfunction{Re}[\Mvariable{Epsilon}]\multsp >\multsp -4,\Mfunction{Re}[\Mvariable{Epsilon}]\multsp >\multsp -3
    ,  \\
\noalign{\vspace{0.5ex}}
\hspace{1.em} \Mfunction{Re}[\Mvariable{Epsilon}]\multsp >\multsp -2,
    \Mfunction{Re}[\Mvariable{Epsilon}]\multsp >\multsp -1,\Mfunction{Re}[\Mvariable{Epsilon}]\multsp >\multsp 0\}\\
\MathEnd{MathArray}
}

\dispSFoutmath{
\{\Mvariable{True},\Mvariable{True},\Mvariable{True},\Mvariable{True},\Mvariable{True}\}
}

\Subsection*{EpsilonOrder}

\Subsubsection*{Description}

EpsilonOrder is an option of OPEIntDelta and RHI. The setting determines the order n (\({{\Mvariable{Epsilon}}^n}\)) which should be kept.

See also:  OPEIntDelta, RHI.

\Subsection*{Expanding}

\Subsubsection*{Description}

Expanding is an option for Calc, Contract, DiracSimplify,DotSimplify, SUNSimplify, etc. As option for Contract it specifies whether
  expansion w.r.t. LorentzIndex is done BEFORE contraction. If set to False in DiracSimplify or SUNSimplify, only a limited set of
  simplifications (multiplicative linearity etc.) is performed.

See also:  Calc, Contract, DiracSimplify, DotSimplify, SUNSimplify.

\Subsection*{ExpandPartialD}

\Subsubsection*{Description}

ExpandPartialD[exp] expands all products of QuantumField's and partial differentiation operators in exp and applies the Leibniz rule.

\dispSFinmath{
\Mfunction{Options}[\Mvariable{ExpandPartialD}]
}

\dispSFoutmath{
\MathBegin{MathArray}{l}
\big\{\Mvariable{PartialDRelations}\rightarrow
    \big\{\Mvariable{a\$\_\_\_}.\partial \big/\partial {{\Mvariable{x\$\_}}^{\Mvariable{mu\$\_}}}.\Mvariable{b\$\_\_\_}\RuleDelayed
       \Mvariable{a\$}.{{\partial }_{\Mvariable{mu\$}}}(\Mfunction{Dot}[\Mvariable{b\$}]),  \\
\noalign{\vspace{0.666667ex}}
   \hspace{3.em} \Muserfunction{UMatrix}(\Mvariable{\DoubleStruckCapitalI \DoubleStruckD },\_\_\_)\SixPointedStar
     (\Mvariable{PartialD}|\Mvariable{RightPartialD}|\Mvariable{LeftPartialD})[\_\_]\RuleDelayed 0,
   \Mvariable{a\_\_\_}\SixPointedStar {D_{\Mvariable{mu\_}}}\SixPointedStar \Mvariable{b\_\_\_}\RuleDelayed   \\
\noalign{\vspace{
   0.666667ex}}
\hspace{4.em} a\SixPointedStar {{\ScriptCapitalD }_{\mu}}(b)/;
     (\Mvariable{go}=\Mvariable{False};\{b\}/.
        \_\_|\_[\_\_\_,\_\_,\_\_\_][\Mvariable{y\_}?\Mvariable{AtomQ}]\RuleDelayed (\Mvariable{go}=\Mvariable{True};z=y);\Mvariable{go}),
    \\
\noalign{\vspace{0.645833ex}}
\hspace{3.em} \Mvariable{a\_\_\_}\SixPointedStar \partial /D{{\Mvariable{x\_}}^{\Mvariable{mu\_}}}
     \SixPointedStar \Mvariable{b\_\_\_}\RuleDelayed a\SixPointedStar {{\ScriptCapitalD }_{\mu}}(b),
   \Mvariable{a\_\_\_}\SixPointedStar {{\left( \overvar{\partial }{\leftarrow } \right) }_{\Mvariable{mu\_}}}\SixPointedStar
     \Mvariable{b\_\_\_}\RuleDelayed   \\
\noalign{\vspace{0.666667ex}}
\hspace{4.em} {{\partial }_{\mu}}(a)\SixPointedStar b/;
     (\Mvariable{go}=\Mvariable{False};\{a\}/.
        \_\_|\_[\_\_\_,\_\_,\_\_\_][\Mvariable{y\_}?\Mvariable{AtomQ}]\RuleDelayed (\Mvariable{go}=\Mvariable{True};z=y);\Mvariable{go}),
    \\
\noalign{\vspace{0.666667ex}}
\hspace{3.em} \Mvariable{a\_\_\_}\SixPointedStar
      \Muserfunction{LeftPartialD}(\Mvariable{x\_},\Mvariable{mu\_})\SixPointedStar \Mvariable{b\_\_\_}\RuleDelayed
    {{\partial }_{\mu}}(a)\SixPointedStar b,\Mvariable{a\_\_\_}\SixPointedStar
      (\Mvariable{PartialD}|\Mvariable{RightPartialD})[\Mvariable{mu\_}]\SixPointedStar \Mvariable{b\_\_\_}\RuleDelayed   \\
   \noalign{\vspace{0.666667ex}}
\hspace{4.em} a\SixPointedStar {{\partial }_{\mu}}(b)/;
     (\Mvariable{go}=\Mvariable{False};\{b\}/.
        \_\_|\_[\_\_\_,\_\_,\_\_\_][\Mvariable{y\_}?\Mvariable{AtomQ}]\RuleDelayed (\Mvariable{go}=\Mvariable{True};z=y);\Mvariable{go}),
    \\
\noalign{\vspace{0.666667ex}}
\hspace{3.em} \Mvariable{a\_\_\_}\SixPointedStar
         (\Mvariable{PartialD}|\Mvariable{RightPartialD})[\Mvariable{x\_},\Mvariable{mu\_}]\SixPointedStar \Mvariable{b\_\_\_}
        \RuleDelayed a\SixPointedStar {{\partial }_{\mu}}(b)\big\}\big\}\\
\MathEnd{MathArray}
}

See also: ExplicitPartialD, LeftPartialD, LeftRightPartialD, PartialDRelations, RightPartialD.

\Subsubsection*{Examples}

\dispSFinmath{
\Muserfunction{RightPartialD}[\mu ].\Muserfunction{QuantumField}[A,\Muserfunction{LorentzIndex}[\mu ]].
   \Muserfunction{QuantumField}[A,\Muserfunction{LorentzIndex}[\nu ]]
}

\dispSFoutmath{
{{\left( \overvar{\partial }{\rightarrow } \right) }_{\mu }}.{A_{\mu }}.{A_{\nu }}
}

\dispSFinmath{
\Muserfunction{ExpandPartialD}[\%]
}

\dispSFoutmath{
{A_{\mu }}.{{\partial }_{\mu }}A_{\nu }^{ }+{{\partial }_{\mu }}A_{\mu }^{ }.{A_{\nu }}
}

\dispSFinmath{
\%//\Mfunction{StandardForm}
}

\dispSFoutmath{
\MathBegin{MathArray}{l}
\Muserfunction{QuantumField}[A,\Muserfunction{LorentzIndex}[\mu ]].  \\
\noalign{\vspace{0.5ex}}
   \hspace{2.em} \Muserfunction{QuantumField}[\Muserfunction{PartialD}[\Muserfunction{LorentzIndex}[\mu ]],A,
      \Muserfunction{LorentzIndex}[\nu ]]+  \\
\noalign{\vspace{0.5ex}}
\hspace{1.em} \Muserfunction{QuantumField}[
    \Muserfunction{PartialD}[\Muserfunction{LorentzIndex}[\mu ]],A,\Muserfunction{LorentzIndex}[\mu ]].  \\
\noalign{\vspace{0.5ex}}
   \hspace{2.em} \Muserfunction{QuantumField}[A,\Muserfunction{LorentzIndex}[\nu ]]\\
\MathEnd{MathArray}
}

\dispSFinmath{
\Muserfunction{LeftRightPartialD}[\mu ].\Muserfunction{QuantumField}[A,\Muserfunction{LorentzIndex}[\nu ]]
}

\dispSFoutmath{
{{\left( \overvar{\partial }{\leftrightarrow } \right) }_{\mu }}.{A_{\nu }}
}

\dispSFinmath{
\Muserfunction{ExpandPartialD}[\%]
}

\dispSFoutmath{
\frac{{{\partial }_{\mu }}A_{\nu }^{ }}{2}-\frac{1}{2}\multsp {{\left( \overvar{\partial }{\leftarrow } \right) }_{\mu }}.{A_{\nu }}
}

\dispSFinmath{
\MathBegin{MathArray}{l}
\Muserfunction{QuantumField}[A,\Muserfunction{LorentzIndex}[\mu ]].  \\
\noalign{\vspace{0.5ex}}
   \hspace{1.em} (\Muserfunction{LeftRightPartialD}[\Mvariable{OPEDelta}]\RawWedge 2).
   \Muserfunction{QuantumField}[A,\Muserfunction{LorentzIndex}[\rho ]]\\
\MathEnd{MathArray}
}

\dispSFoutmath{
{A_{\mu }}.\overvar{\partial }{\leftrightarrow }_{\Delta }^{2}.{A_{\rho }}
}

\dispSFinmath{
\Muserfunction{ExpandPartialD}[\%]
}

\dispSFoutmath{
\frac{1}{4}\multsp {A_{\mu }}.\big({{\partial }_{\Delta }}{{\partial }_{\Delta }}A_{\rho }^{ }\big)-
   \frac{1}{2}\multsp {{\partial }_{\Delta }}A_{\mu }^{ }.{{\partial }_{\Delta }}A_{\rho }^{ }+
   \frac{1}{4}\multsp \big({{\partial }_{\Delta }}{{\partial }_{\Delta }}A_{\mu }^{ }\big).{A_{\rho }}
}

\dispSFinmath{
8\multsp \Muserfunction{LeftRightPartialD}[\Mvariable{OPEDelta}]\RawWedge 3
}

\dispSFoutmath{
8\multsp \overvar{\partial }{\leftrightarrow }_{\Delta }^{3}
}

\dispSFinmath{
\Muserfunction{ExplicitPartialD}[\%]
}

\dispSFoutmath{
{{\big({{\left( \overvar{\partial }{\rightarrow } \right) }_{\Delta }}-{{\left( \overvar{\partial }{\leftarrow } \right) }_{\Delta }}
      \big)}^3}
}

\dispSFinmath{
\Muserfunction{ExpandPartialD}[\%]
}

\dispSFoutmath{
-{{\left( \overvar{\partial }{\leftarrow } \right) }_{\Delta }}.{{\left( \overvar{\partial }{\leftarrow } \right) }_{\Delta }}.
     {{\left( \overvar{\partial }{\leftarrow } \right) }_{\Delta }}+
   3\multsp {{\left( \overvar{\partial }{\leftarrow } \right) }_{\Delta }}.{{\left( \overvar{\partial }{\leftarrow } \right) }_{\Delta }}
     .{{\left( \overvar{\partial }{\rightarrow } \right) }_{\Delta }}-
   3\multsp {{\left( \overvar{\partial }{\leftarrow } \right) }_{\Delta }}.
     {{\left( \overvar{\partial }{\rightarrow } \right) }_{\Delta }}.{{\left( \overvar{\partial }{\rightarrow } \right) }_{\Delta }}+
   {{\left( \overvar{\partial }{\rightarrow } \right) }_{\Delta }}.{{\left( \overvar{\partial }{\rightarrow } \right) }_{\Delta }}.
    {{\left( \overvar{\partial }{\rightarrow } \right) }_{\Delta }}
}

\dispSFinmath{
\Muserfunction{LeviCivita}[\mu ,\nu ,\rho ,\tau ]\multsp \Muserfunction{RightPartialD}[\alpha ,\mu ,\beta ,\nu ]
}

\dispSFoutmath{
{{\left( \overvar{\partial }{\rightarrow } \right) }_{\alpha }}.{{\left( \overvar{\partial }{\rightarrow } \right) }_{\mu }}.
    {{\left( \overvar{\partial }{\rightarrow } \right) }_{\beta }}.{{\left( \overvar{\partial }{\rightarrow } \right) }_{\nu }}\multsp
   {{\epsilon }^{\mu \nu \rho \tau }}
}

\dispSFinmath{
\Muserfunction{ExpandPartialD}[\%]
}

\dispSFoutmath{
0
}

\Subsection*{ExpandScalarProduct, ScalarProductExpand}

\Subsubsection*{Description}

ExpandScalarProduct[expr] expands scalar products of sums of momenta in expr. ExpandScalarProduct does not use Expand on expr.\\
ScalarProductExpand\(=\)ExpandScalarProduct.

\dispSFinmath{
\Muserfunction{ExpandScalarProduct}//\Mfunction{Options}
}

\dispSFoutmath{
\{\Mvariable{FeynCalcInternal}\rightarrow \Mvariable{True}\}
}

See also:  Calc, MomentumExpand, MomentumCombine.

\Subsubsection*{Examples}

\dispSFinmath{
\multsp \Muserfunction{SP}[\Mvariable{p1}+\Mvariable{p2}+\Mvariable{p3},\Mvariable{p4}+\Mvariable{p5}+\Mvariable{p6}]
}

\dispSFoutmath{
({p_1}+{p_2}+{p_3})\cdot ({p_4}+{p_5}+{p_6})
}

\dispSFinmath{
\%//\Muserfunction{ScalarProductExpand}
}

\dispSFoutmath{
{p_1}\cdot {p_4}+{p_1}\cdot {p_5}+{p_1}\cdot {p_6}+{p_2}\cdot {p_4}+{p_2}\cdot {p_5}+{p_2}\cdot {p_6}+{p_3}\cdot {p_4}+{p_3}\cdot {p_5}+
   {p_3}\cdot {p_6}
}

\dispSFinmath{
\Muserfunction{SP}[p,p-q]
}

\dispSFoutmath{
p\cdot (p-q)
}

\dispSFinmath{
\Muserfunction{ExpandScalarProduct}[\%]
}

\dispSFoutmath{
{p^2}-p\cdot q
}

\dispSFinmath{
\Muserfunction{FV}[p-q,\mu ]
}

\dispSFoutmath{
{{p-q}^{\mu }}
}

\dispSFinmath{
\Muserfunction{ExpandScalarProduct}[\%]
}

\dispSFoutmath{
{p^{\mu }}-{q^{\mu }}
}

\dispSFinmath{
\Muserfunction{SP}[p-q,q-r]//\Muserfunction{FCI}
}

\dispSFoutmath{
(p-q)\cdot (q-r)
}

\dispSFinmath{
\%/.\Mvariable{Pair}\rightarrow \Mvariable{ExpandScalarProduct}
}

\dispSFoutmath{
p\cdot q-p\cdot r-{q^2}+q\cdot r
}

\Subsection*{Expand2}

\Subsubsection*{Description}

Expand2[exp, x] expands all sums containing x. Expand2[exp, \{x1, x2, ...\}] { }expands all sums containing x1, x2, ....

\Subsubsection*{Examples}

\dispSFinmath{
\Muserfunction{Expand2}[(\Mvariable{x1}+\Mvariable{x2}+\Mvariable{x3})(2\Mvariable{x1}+3\Mvariable{x2})+
     (\Mvariable{y1}+\Mvariable{y2}+\Mvariable{y3})(2\Mvariable{y1}+3\Mvariable{y2}),\{\Mvariable{x1},\Mvariable{x2}\}]
}

\dispSFoutmath{
2\multsp {{\Mvariable{x1}}^2}+5\multsp \Mvariable{x2}\multsp \Mvariable{x1}+2\multsp \Mvariable{x3}\multsp \Mvariable{x1}+
   3\multsp {{\Mvariable{x2}}^2}+3\multsp \Mvariable{x2}\multsp \Mvariable{x3}+
   (2\multsp \Mvariable{y1}+3\multsp \Mvariable{y2})\multsp (\Mvariable{y1}+\Mvariable{y2}+\Mvariable{y3})
}

\Subsection*{Explicit}

\Subsubsection*{Description}

Explicit is an option for FieldStrength, GluonVertex, SUNF, and Twist2GluonOperator. If set to True the full form of the operator is
  inserted. Explicit[exp] inserts explicit expressions of GluonVertex, Twist2GluonOperator, etc., in exp. SUNF's are replaced by SUNTrace objects.

See also:  GluonVertex, Twist2GluonOperator.

\dispSFinmath{
\Muserfunction{Explicit}//\Mfunction{Options}
}

\dispSFoutmath{
\{\Mvariable{CouplingConstant}\rightarrow {g_s},\Mvariable{Dimension}\rightarrow D,\Mvariable{Gauge}\rightarrow 1,
    \Omega \rightarrow \Mvariable{False}\}
}

\Subsubsection*{Examples}

\dispSFinmath{
\Muserfunction{GluonPropagator}[p,\mu ,\nu ]
}

\dispSFoutmath{
\Pi _{g}^{\mu \nu }(p)
}

\dispSFinmath{
\Muserfunction{Explicit}[\%]
}

\dispSFoutmath{
-\frac{\ImaginaryI \multsp {g^{\mu \nu }}}{{p^2}}
}

\dispSFinmath{
\Muserfunction{Explicit}[\Muserfunction{GluonPropagator}[p,\mu ,\nu ],\Mvariable{Gauge}\rightarrow \xi ]
}

\dispSFoutmath{
-\ImaginaryI \multsp \xi \multsp {p^{\mu }}\multsp {p^{\nu }}\multsp {{\bigg(\frac{1}{{p^2}}\bigg)}^2}+
   \ImaginaryI \multsp {p^{\mu }}\multsp {p^{\nu }}\multsp {{\bigg(\frac{1}{{p^2}}\bigg)}^2}-
   \frac{\ImaginaryI \multsp {g^{\mu \nu }}}{{p^2}}
}

\dispSFinmath{
\Muserfunction{GluonVertex}[p,\mu ,a,\multsp q,\nu ,b,r,\rho ,c]
}

\dispSFoutmath{
{V^{\mu \nu \rho }}(p,\multsp q,\multsp r)\multsp {f_{abc}}
}

\dispSFinmath{
\Muserfunction{Explicit}[\%]
}

\dispSFoutmath{
({g_s}\multsp {q^{\mu }}\multsp {g^{\nu \rho }}-{g_s}\multsp {r^{\mu }}\multsp {g^{\nu \rho }}-
     {g_s}\multsp {g^{\mu \rho }}\multsp {p^{\nu }}+{g_s}\multsp {g^{\mu \rho }}\multsp {r^{\nu }}+
     {g_s}\multsp {g^{\mu \nu }}\multsp {p^{\rho }}-{g_s}\multsp {g^{\mu \nu }}\multsp {q^{\rho }})\multsp {f_{abc}}
}

\dispSFinmath{
\Muserfunction{Twist2GluonOperator}[p,\mu ,a,\nu ,b]
}

\dispSFoutmath{
\frac{1}{2}\multsp ({{(-1)}^m}+1)\multsp {{\delta }_{ab}}\multsp
   \big(\Mfunction{O}_{\mu \VeryThinSpace \nu }^{\Mvariable{G2}}\Mfunction{(}p)\big)
}

\dispSFinmath{
\Muserfunction{Explicit}[\%]
}

\dispSFoutmath{
\frac{1}{2}\multsp ({g^{\mu \nu }}\multsp {{\Delta \cdot p}^2}-
     ({p^{\mu }}\multsp {{\Delta }^{\nu }}+{{\Delta }^{\mu }}\multsp {p^{\nu }})\multsp \Delta \cdot p+
     {{\Delta }^{\mu }}\multsp {{\Delta }^{\nu }}\multsp {p^2})\multsp ({{(-1)}^m}+1)\multsp {{(\Delta \cdot p)}^{m-2}}\multsp
   {{\delta }_{ab}}
}

\dispSFinmath{
\Muserfunction{FieldStrength}[\mu ,\nu ,a]
}

\dispSFoutmath{
F_{\mu \nu }^{a}
}

\dispSFinmath{
\Muserfunction{Explicit}[\%]
}

\dispSFoutmath{
{{\partial }_{\mu }}A_{\nu }^{a}-{{\partial }_{\nu }}A_{\mu }^{a}+
   {g_s}\multsp A_{\mu }^{\Mvariable{b1}}.A_{\nu }^{\Mvariable{c11}}\multsp {f_{a\Mvariable{b1}\Mvariable{c11}}}
}

\Subsection*{ExplicitLorentzIndex}

\Subsubsection*{Description}

ExplicitLorentzIndex[ind] is an explicit Lorentz index, i.e., ind is an integer.

See also: LorentzIndex, Pair.

\Subsubsection*{Examples}

\dispSFinmath{
\Muserfunction{Pair}[\Muserfunction{LorentzIndex}[1],\Muserfunction{LorentzIndex}[\mu ]]
}

\dispSFoutmath{
{g^{1\mu }}
}

\dispSFinmath{
\%//\Mfunction{StandardForm}
}

\dispSFoutmath{
\Muserfunction{Pair}[\Muserfunction{ExplicitLorentzIndex}[1],\Muserfunction{LorentzIndex}[\mu ]]
}

\Subsection*{ExplicitPartialD}

\Subsubsection*{Description}

ExplicitPartialD[exp] inserts in exp the definition for LeftRightPartialD[z] (and LeftRightPartialD2[z]).

See also: ExpandPartialD, LeftRightPartialD, LeftRightPartialD2.

\Subsubsection*{Examples}

\dispSFinmath{
\Muserfunction{ExplicitPartialD}[\multsp \Muserfunction{LeftRightPartialD}[\mu ]\multsp ]
}

\dispSFoutmath{
\frac{1}{2}\multsp \big({{\left( \overvar{\partial }{\rightarrow } \right) }_{\mu }}-
     {{\left( \overvar{\partial }{\leftarrow } \right) }_{\mu }}\big)
}

\dispSFinmath{
\Muserfunction{ExplicitPartialD}[\multsp \Muserfunction{LeftRightPartialD2}[\mu ]\multsp ]
}

\dispSFoutmath{
{{\left( \overvar{\partial }{\leftarrow } \right) }_{\mu }}+{{\left( \overvar{\partial }{\rightarrow } \right) }_{\mu }}
}

\dispSFinmath{
\Muserfunction{ExplicitPartialD}[\multsp \Muserfunction{LeftRightPartialD}[\Mvariable{OPEDelta}]\multsp ]
}

\dispSFoutmath{
\frac{1}{2}\multsp \big({{\left( \overvar{\partial }{\rightarrow } \right) }_{\Delta }}-
     {{\left( \overvar{\partial }{\leftarrow } \right) }_{\Delta }}\big)
}

\dispSFinmath{
16\multsp \Muserfunction{LeftRightPartialD}[\Mvariable{OPEDelta}]\RawWedge 4
}

\dispSFoutmath{
16\multsp \overvar{\partial }{\leftrightarrow }_{\Delta }^{4}
}

\dispSFinmath{
\Muserfunction{ExplicitPartialD}[\%]
}

\dispSFoutmath{
{{\big({{\left( \overvar{\partial }{\rightarrow } \right) }_{\Delta }}-{{\left( \overvar{\partial }{\leftarrow } \right) }_{\Delta }}
      \big)}^4}
}

\Subsection*{ExplicitSUNIndex}

\Subsubsection*{Description}

ExplicitSUNIndex[ind] is a specific SU({\itshape N}) index, i.e., ind is an integer.

See also: FCI, SUNDelta, SUNF, SUNIndex.

\Subsubsection*{Examples}

\dispSFinmath{
\Muserfunction{ExplicitSUNIndex}[1]
}

\dispSFoutmath{
1
}

\dispSFinmath{
\Muserfunction{SUNDelta}[1,a]//\Muserfunction{FCI}//\Mfunction{StandardForm}
}

\dispSFoutmath{
\Muserfunction{SUNDelta}[\Muserfunction{ExplicitSUNIndex}[1],\Muserfunction{SUNIndex}[a]]
}

\Subsection*{ExtraFactor}

\Subsubsection*{Description}

ExtraFactor is an option for FermionSpinSum. The setting ExtraFactor \(\rightarrow \) fa { }multiplies the whole amplitude with the
  factor fa before squaring.

See also: FermionSpinSum.

\Subsection*{ExtraVariables}

\Subsubsection*{Description}

ExtraVariables is an option for OneLoopSum; it may be set to a list of variables which are also bracketed out in the result, just like
  B0, C0, D0 and { }PaVe.

See also:  OneLoop, OneLoopSum.

\Subsection*{FactorFull}

\Subsubsection*{Description}

FactorFull is an option of Factor2 (default False). If set to False, products like ({\itshape a}-{\itshape b}) ({\itshape a}\(+\){\itshape b}) will
be replaced by (\({a^2}\)-\({b^2}\)).

See also:  Factor2.

\Subsection*{Factoring}

\Subsubsection*{Description}

Factoring is an option for Collect2, Contract, Tr and more functions. If set to True, the result will be factored, using Factor2. If set
  to any function f, this function will be used.

See also:  Collect2, Contract, Tr.

\Subsection*{Factorout}

\Subsubsection*{Description}

Factorout is an option for OPEInt.

See also:  OPEInt.

\Subsection*{FactorTime}

\Subsubsection*{Description}

FactorTime is an option for Factor2. It denotes the maximum time (in seconds) during which Factor2 tries to factor.

See also:  Factor2.

\Subsection*{Factor1}

\Subsubsection*{Description}

Factor1[poly] factorizes common terms { }in the summands of poly. It uses basically PolynomialGCD.

See also:  Factor2.

\Subsubsection*{Examples}

\dispSFinmath{
\Mvariable{t1}=(a-x)(b-x)
}

\dispSFoutmath{
(a-x)\multsp (b-x)
}

\dispSFinmath{
\Mvariable{t2}=\{\Muserfunction{Factor1}[\Mvariable{t1}],\multsp \Mfunction{Factor}[\Mvariable{t1}]\}
}

\dispSFoutmath{
\{(a-x)\multsp (b-x),-(a-x)\multsp (x-b)\}
}

\dispSFinmath{
\Mvariable{t3}=\Mfunction{Expand}[(a-b)(a+b)]
}

\dispSFoutmath{
{a^2}-{b^2}
}

\dispSFinmath{
\Mfunction{Factor}[\Mvariable{t3}]
}

\dispSFoutmath{
(a-b)\multsp (a+b)
}

\dispSFinmath{
\Muserfunction{Factor1}[\Mvariable{t3}]
}

\dispSFoutmath{
{a^2}-{b^2}
}

\dispSFinmath{
\Mfunction{Clear}[\Mvariable{t1},\Mvariable{t2},\Mvariable{t3}]
}

\Subsection*{Factor2}

\Subsubsection*{Description}

Factor2[poly] factors a polynomial in a standard way. Factor2 works sometimes better than Factor on polynomials involving rationals with
  sums in the denominator. Factor2 uses Factor internally and is in general slower than Factor. There are four possible settings of the
  option Method (0,1,2,3). In general Factor will work faster than Factor2.

\dispSFinmath{
\Mfunction{Options}[\Mvariable{Factor2}]
}

\dispSFoutmath{
\{\Mvariable{FactorFull}\rightarrow \Mvariable{False},\Mvariable{Method}\rightarrow 3\}
}

See also:  Collect2.

\Subsubsection*{Examples}

\dispSFinmath{
\Mvariable{t1}=(a-x)(b-x)
}

\dispSFoutmath{
(a-x)\multsp (b-x)
}

\dispSFinmath{
\Mvariable{t2}=\{\Muserfunction{Factor2}[\Mvariable{t1}],\multsp \Mfunction{Factor}[\Mvariable{t1}]\}
}

\dispSFoutmath{
\{(a-x)\multsp (b-x),-(a-x)\multsp (x-b)\}
}

\dispSFinmath{
\Mvariable{t3}=\Mfunction{Expand}[(a-b)(a+b)]
}

\dispSFoutmath{
{a^2}-{b^2}
}

\dispSFinmath{
\Mfunction{Factor}[\Mvariable{t3}]
}

\dispSFoutmath{
(a-b)\multsp (a+b)
}

\dispSFinmath{
\Muserfunction{Factor2}[\Mvariable{t3}]
}

\dispSFoutmath{
{a^2}-{b^2}
}

\dispSFinmath{
\Muserfunction{Factor2}[\Mvariable{t3},\Mvariable{FactorFull}\rightarrow \Mvariable{True}]
}

\dispSFoutmath{
(a-b)\multsp (a+b)
}

\dispSFinmath{
\Mfunction{Clear}[\Mvariable{t1},\Mvariable{t2},\Mvariable{t3}]
}

\Subsection*{FAD}

\Subsubsection*{Description}

FAD is the FeynCalc external form of FeynAmpDenominator and denotes an inverse propagator. FAD[q, q-p, ...] is 1/(q\(\RawWedge\)2
  (q-p)\(\RawWedge\)2 ...). FAD[\{q1,m\}, \{q1-p,m\}, q2, ...] is 1/( (q1\(\RawWedge\)2 - m\(\RawWedge\)2) ( (q1-p)\(\RawWedge\)2 -
  m\(\RawWedge\)2 ) q2\(\RawWedge\)2 ... ). Translation into FeynCalc internal form is performed by FeynCalcInternal.

See also:  FAD, FCE, FCI, FeynAmpDenominator, FeynAmpDenominatorSimplify, PropagatorDenominator.

\Subsubsection*{Examples}

\dispSFinmath{
\Muserfunction{FAD}[q,p-q]
}

\dispSFoutmath{
\frac{1}{([ {{(p-q)}^2} ])\multsp ([ {q^2} ])}
}

\dispSFinmath{
\Muserfunction{FAD}[p,\{p-q,m\}]
}

\dispSFoutmath{
\frac{1}{([ {p^2} ])\multsp
     ([ {{(p-q)}^2} - {m^2} ])}
}

\dispSFinmath{
\Muserfunction{FAD}[q,p-q]//\Muserfunction{FCI}//\Muserfunction{FCE}//\Mfunction{StandardForm}
}

\dispSFoutmath{
\Muserfunction{FAD}[q,p-q]
}

\dispSFinmath{
\Muserfunction{FAD}[q,p-q]//\Muserfunction{FCI}//\Mfunction{StandardForm}
}

\dispSFoutmath{
\MathBegin{MathArray}{l}
\Muserfunction{FeynAmpDenominator}[\Muserfunction{PropagatorDenominator}[\Muserfunction{Momentum}[q,D],0],  \\
   \noalign{\vspace{0.5ex}}
\hspace{1.em} \Muserfunction{PropagatorDenominator}[
     \Muserfunction{Momentum}[p,D]-\Muserfunction{Momentum}[q,D],0]]\\
\MathEnd{MathArray}
}

\dispSFinmath{
\Muserfunction{FAD}[p]\multsp \Muserfunction{FAD}[p-q]\multsp //\multsp \Mvariable{FeynAmpDenominatorCombine}//\Mfunction{StandardForm}
}

\dispSFoutmath{
\MathBegin{MathArray}{l}
\Muserfunction{FeynAmpDenominator}[\Muserfunction{PropagatorDenominator}[\Muserfunction{Momentum}[p,D],0],  \\
   \noalign{\vspace{0.5ex}}
\hspace{1.em} \Muserfunction{PropagatorDenominator}[
     \Muserfunction{Momentum}[p,D]-\Muserfunction{Momentum}[q,D],0]]\\
\MathEnd{MathArray}
}

\Subsection*{FC}

\Subsubsection*{Description}

FC changes the output format to FeynCalcForm. To change to InputForm use FI.

See also:  FeynCalcForm, FI, FeynCalcExternal, FeynCalcInternal.

\Subsubsection*{Examples}

\dispSFinmath{
\Mvariable{FI}
}

\dispSFinmath{
\{\Muserfunction{DiracGamma}[5],\Muserfunction{DiracGamma}[\Muserfunction{Momentum}[p]]\}
}

\dispSFoutmath{
\{\Muserfunction{DiracGamma}[5],\multsp \Muserfunction{DiracGamma}[\Muserfunction{Momentum}[p]]\}
}

\dispSFinmath{
\Mvariable{FC}
}

\dispSFinmath{
\{\Muserfunction{DiracGamma}[5],\Muserfunction{DiracGamma}[\Muserfunction{Momentum}[p]]\}
}

\dispSFoutmath{
\big\{{{\gamma }^5},\gamma \cdot p\big\}
}

\Subsection*{FCE}

\Subsubsection*{Description}

FCE[exp] translates exp from the internal FeynCalc representation to a short form.

FCE is equivalent to FeynCalcExternal.

See also:  FeynCalcExternal, FCI, FeynCalcInternal.

\Subsubsection*{Examples}

\dispSFinmath{
\Muserfunction{FCE}[\{\Muserfunction{DiracGamma}[5],\Muserfunction{DiracGamma}[\Muserfunction{Momentum}[p]]\}]
}

\dispSFoutmath{
\big\{{{\gamma }^5},\gamma \cdot p\big\}
}

\dispSFinmath{
\%//\Mfunction{StandardForm}
}

\dispSFoutmath{
\{\Muserfunction{GA}[5],\Muserfunction{GS}[p]\}
}

\dispSFinmath{
\{\Muserfunction{GA}[\mu ],\Muserfunction{GAD}[\rho ],\Muserfunction{GS}[p],\Muserfunction{SP}[p,q],\Muserfunction{MT}[\alpha ,\beta ],
    \Muserfunction{FV}[p,\mu ]\}
}

\dispSFoutmath{
\{{{\gamma }^{\mu }},{{\gamma }^{\rho }},\gamma \cdot p,p\cdot q,{g^{\alpha \beta }},{p^{\mu }}\}
}

\dispSFinmath{
\%//\Mfunction{StandardForm}
}

\dispSFoutmath{
\{\Muserfunction{GA}[\mu ],\Muserfunction{GAD}[\rho ],\Muserfunction{GS}[p],\Muserfunction{SP}[p,q],\Muserfunction{MT}[\alpha ,\beta ],
    \Muserfunction{FV}[p,\mu ]\}
}

\dispSFinmath{
\%//\Muserfunction{FCI}
}

\dispSFoutmath{
\{{{\gamma }^{\mu }},{{\gamma }^{\rho }},\gamma \cdot p,p\cdot q,{g^{\alpha \beta }},{p^{\mu }}\}
}

\dispSFinmath{
\%//\Mfunction{StandardForm}
}

\dispSFoutmath{
\MathBegin{MathArray}{l}
\{\Muserfunction{DiracGamma}[\Muserfunction{LorentzIndex}[\mu ]],
    \Muserfunction{DiracGamma}[\Muserfunction{LorentzIndex}[\rho ,D],D],  \\
\noalign{\vspace{0.5ex}}
\hspace{1.em} \Muserfunction{DiracG
     amma}[\Muserfunction{Momentum}[p]],\Muserfunction{Pair}[\Muserfunction{Momentum}[p],\Muserfunction{Momentum}[q]],  \\
   \noalign{\vspace{0.5ex}}
\hspace{1.em} \Muserfunction{Pair}[\Muserfunction{LorentzIndex}[\alpha ],\Muserfunction{LorentzIndex}[\beta ]
     ],\Muserfunction{Pair}[\Muserfunction{LorentzIndex}[\mu ],\Muserfunction{Momentum}[p]]\}\\
\MathEnd{MathArray}
}

\dispSFinmath{
\Muserfunction{FCE}[\%]//\Mfunction{StandardForm}
}

\dispSFoutmath{
\{\Muserfunction{GA}[\mu ],\Muserfunction{GAD}[\rho ],\Muserfunction{GS}[p],\Muserfunction{SP}[p,q],\Muserfunction{MT}[\alpha ,\beta ],
    \Muserfunction{FV}[p,\mu ]\}
}

\Subsection*{FCI}

\Subsubsection*{Description}

FCI[exp] translates exp into the internal FeynCalc (datatype-)representation.FCI is equivalent to FeynCalcInternal.

See also:  FeynCalcExternal, FeynCalcInternal, FCE.

\Subsubsection*{Examples}

\dispSFinmath{
\{\Muserfunction{GA}[\mu ],\Muserfunction{GAD}[\rho ],\Muserfunction{GS}[p],\Muserfunction{SP}[p,q],\Muserfunction{MT}[\alpha ,\beta ],
    \Muserfunction{FV}[p,\mu ]\}
}

\dispSFoutmath{
\{{{\gamma }^{\mu }},{{\gamma }^{\rho }},\gamma \cdot p,p\cdot q,{g^{\alpha \beta }},{p^{\mu }}\}
}

\dispSFinmath{
\%//\Mfunction{StandardForm}
}

\dispSFoutmath{
\{\Muserfunction{GA}[\mu ],\Muserfunction{GAD}[\rho ],\Muserfunction{GS}[p],\Muserfunction{SP}[p,q],\Muserfunction{MT}[\alpha ,\beta ],
    \Muserfunction{FV}[p,\mu ]\}
}

\dispSFinmath{
\%//\Muserfunction{FCI}
}

\dispSFoutmath{
\{{{\gamma }^{\mu }},{{\gamma }^{\rho }},\gamma \cdot p,p\cdot q,{g^{\alpha \beta }},{p^{\mu }}\}
}

\dispSFinmath{
\%//\Mfunction{StandardForm}
}

\dispSFoutmath{
\MathBegin{MathArray}{l}
\{\Muserfunction{DiracGamma}[\Muserfunction{LorentzIndex}[\mu ]],
    \Muserfunction{DiracGamma}[\Muserfunction{LorentzIndex}[\rho ,D],D],  \\
\noalign{\vspace{0.5ex}}
\hspace{1.em} \Muserfunction{DiracG
     amma}[\Muserfunction{Momentum}[p]],\Muserfunction{Pair}[\Muserfunction{Momentum}[p],\Muserfunction{Momentum}[q]],  \\
   \noalign{\vspace{0.5ex}}
\hspace{1.em} \Muserfunction{Pair}[\Muserfunction{LorentzIndex}[\alpha ],\Muserfunction{LorentzIndex}[\beta ]
     ],\Muserfunction{Pair}[\Muserfunction{LorentzIndex}[\mu ],\Muserfunction{Momentum}[p]]\}\\
\MathEnd{MathArray}
}

\dispSFinmath{
\Muserfunction{FCE}[\%]//\Mfunction{StandardForm}
}

\dispSFoutmath{
\{\Muserfunction{GA}[\mu ],\Muserfunction{GAD}[\rho ],\Muserfunction{GS}[p],\Muserfunction{SP}[p,q],\Muserfunction{MT}[\alpha ,\beta ],
    \Muserfunction{FV}[p,\mu ]\}
}

\Subsection*{FCIntegral}

\Subsubsection*{Description}

FCIntegral is the head of integrals in a setting of the option IntegralTable of FeynAmpDenominatorSimplify. Currently implemented only
  for 2-loop integrals.

See also:  IntegralTable, FeynAmpDenominatorSimplify.

\Subsection*{FCIntegrate}

\Subsubsection*{Description}

FCIntegrate is an option of certain Feynman integral related functions. It determines which integration function is used to evaluate
  analytic integrals. Possible settings include Integrate, NIntegrate, (Dot[Integratedx@@\#{}2, \#{}1] \&{}).

See also:  FCNIntegrate.

\Subsection*{FCNIntegrate}

\Subsubsection*{Description}

FCNIntegrate is an option of certain Feynman integral related functions which may return output containing both integrals that can be
  evaluated and integrals that can only be evaluated numerically. It then determines which integration function is used to evaluate
  numeric integrals. Possible settings include NIntegrate, (0*\#{}1)\&{}, (Dot[Integratedx@@\#{}2, \#{}1] \&{}).

See also:  FCIntegrate.

\Subsection*{FC2RHI }

\Subsubsection*{Description}

FC2RHI[exp, k1, k2] transforms all 2-loop OPE-integrals in FeynAmpDenominator form to the RHI-integrals. { }FC2RHI[exp] is equivalent to
  FC2RHI[exp,q1,q2]. The option IncludePair governs the inclusion { }of scalar products p.k1, p.k2 and k1.k2 (setting True).

\dispSFinmath{
\Mfunction{Options}[\Mvariable{FC2RHI}]
}

\dispSFoutmath{
\{\Mvariable{Dimension}\rightarrow D,\Mvariable{IncludePair}\rightarrow \Mvariable{True},\Mvariable{Do}\rightarrow \Mvariable{True}\}
}

\dispSFinmath{
t=\Muserfunction{FAD}[\Mvariable{q1},\Mvariable{q1}-\Mvariable{q2},\Mvariable{q2}-p]\multsp
     \Muserfunction{SP}[\Mvariable{q1},\Mvariable{OPEDelta}]\RawWedge \Mvariable{OPEm}//\Muserfunction{FCI}
}

\dispSFoutmath{
\frac{{{(\Delta \cdot {q_1})}^m}}{q_{1}^{2}.{{({q_1}-{q_2})}^2}.{{({q_2}-p)}^2}}
}

\dispSFinmath{
\Muserfunction{FC2RHI}[t,\Mvariable{q1},\Mvariable{q2}]
}

\dispSFoutmath{
T_{10011}^{m0000}
}

\dispSFinmath{
\%//\Mfunction{InputForm}
}

\dispSFoutmath{
\Muserfunction{RHI}[\{0,\multsp 0,\multsp \Mvariable{OPEm},\multsp 0,\multsp 0\},\multsp \{0,\multsp 1,\multsp 1,\multsp 0,\multsp 1\}]
}

See also:  RHI.

\Subsubsection*{Examples}

\dispSFinmath{
\{ \}
}

\Subsection*{FC2TLI }

\Subsubsection*{Description}

FC2TLI[exp, k1, k2] transforms all 2-loop OPE-integrals in FeynAmpDenominator form to the TLI-integrals. { }The option IncludePair
  governs the inclusion { }of scalar products p.k1, p.k2 and k1.k2 (setting True).


\dispSFoutmath{
\Mfunction{Options}[\Mvariable{FC2TLI}]
}

\dispSFinmath{
\{\Mvariable{Dimension}\rightarrow D,\Mvariable{IncludePair}\rightarrow \Mvariable{True},\Mvariable{Do}\rightarrow \Mvariable{True}\}
}

\Print{\(?\Mvariable{FAD}\)}

\dispSFinmath{
\MathBegin{MathArray}{l}
\Mvariable{FAD[q,\multsp q-p,\multsp ...]\multsp denotes\multsp 1/(q\RawWedge 2\multsp (q-p)\RawWedge 2\multsp
   ...).\multsp FAD[\{q1,m\},\multsp \{q1-p,m\},\multsp }  \\
\noalign{\vspace{0.5ex}}
\hspace{2.em} \Mvariable{q2,\multsp ...]\multsp
   is\multsp 1/(\multsp (q1\RawWedge 2\multsp -\multsp m\RawWedge 2)\multsp (\multsp (q1-p)\RawWedge 2\multsp -\multsp m\RawWedge
   2\multsp )\multsp q2\RawWedge 2\multsp ...\multsp ).\multsp (Translation\multsp }  \\
\noalign{\vspace{0.5ex}}
\hspace{2.em}
   \Mvariable{into\multsp FeynCalc\multsp internal\multsp form\multsp is\multsp performed\multsp by\multsp FeynCalcInternal.)}\\
   \MathEnd{MathArray}
}

\dispSFoutmath{
t=\Muserfunction{FAD}[\{\Mvariable{q1},\Mvariable{m1}\},\Mvariable{q2}-\Mvariable{q1},\Mvariable{q2}-p]\multsp
    \Muserfunction{SP}[\Mvariable{q1},\Mvariable{OPEDelta}]\RawWedge \Mvariable{OPEm}
}

\dispSFinmath{
\frac{1}{([ {{({q_2}-p)}^2} ])\multsp ([ {{({q_2}-{q_1})}^2} ])\multsp
      \big([ q_{1}^{2} - {{\Mvariable{m1}}^2} ]\big)}\multsp
   {{(\Delta \cdot {q_1})}^m}
}

\dispSFoutmath{
\Muserfunction{FC2TLI}[t,\Mvariable{q1},\Mvariable{q2}]
}

\dispSFinmath{
T_{10011}^{m0000}
}

\dispSFoutmath{
\%//\Mfunction{InputForm}
}

\dispSFinmath{
\Muserfunction{TLI}[\{\Mvariable{OPEm},\multsp 0,\multsp 0,\multsp 0,\multsp 0\},\multsp
    \{\{1,\multsp \Mvariable{m1}\},\multsp 0,\multsp 0,\multsp 1,\multsp 1\}]
}

\dispSFoutmath{
\Muserfunction{TLI2FC}[\%]
}

See also:  TLI.

\Subsubsection*{Examples}

\dispSFinmath{
\frac{{{(\Delta \cdot {q_1})}^m}}{{{({q_2}-p)}^2}.\big(q_{1}^{2}-{{\Mvariable{m1}}^2}\big).{{({q_1}-{q_2})}^2}}
}

\Subsection*{FDS}

\Subsubsection*{Description}

FDS is shorthand for FeynAmpDenominatorSimplify.

See also:  FeynAmpDenominatorSimplify.

\Subsection*{FermionSpinSum}

\Subsubsection*{Description}

FermionSpinSum[x] constructs the Traces out of squared ampliudes.

\dispSFinmath{
\{ \}
}


See also:  Spinor, ComplexConjugate, DiracTrace, Tr.

\Subsubsection*{Examples}

Spinors of fermions of mass {\itshape m} are normalized to have square 2 {\itshape m} or -2 {\itshape m}.

\dispSFinmath{
\Mfunction{Options}[\Mvariable{FermionSpinSum}]
}

\dispSFoutmath{
\{\Mvariable{SpinPolarizationSum}\rightarrow \Mvariable{Identity},\Mvariable{SpinorCollect}\rightarrow \Mvariable{False},
    \Mvariable{ExtraFactor}\rightarrow 1\}
}

\dispSFinmath{
\Muserfunction{SpinorUBar}[\Muserfunction{Momentum}[p],m].\Muserfunction{SpinorU}[\Muserfunction{Momentum}[p],m]
}

\dispSFoutmath{
\overvar{u}{\_}(p,m).u(p,m)
}

\dispSFinmath{
\%//\Muserfunction{FCI}//\Muserfunction{FermionSpinSum}
}

\dispSFoutmath{
\Muserfunction{tr}(m+\gamma \cdot p)
}

\dispSFinmath{
\%/.\Mvariable{DiracTrace}\rightarrow \Mvariable{Tr}
}

\dispSFoutmath{
4\multsp m
}

\dispSFinmath{
\Muserfunction{SpinorVBar}[\Muserfunction{Momentum}[p],m].\Muserfunction{SpinorV}[\Muserfunction{Momentum}[p],m]
}

\dispSFoutmath{
\overvar{v}{\_}(p,m).v(p,m)
}

\dispSFinmath{
\%//\Muserfunction{FCI}//\Muserfunction{FermionSpinSum}
}

\dispSFoutmath{
\Muserfunction{tr}(\gamma \cdot p-m)
}

\dispSFinmath{
\%/.\Mvariable{DiracTrace}\rightarrow \Mvariable{Tr}
}

\dispSFoutmath{
-4\multsp m
}

Notice that SpinorUBar and SpinorU are only input functions. Internally they are converted to Spinor objects.

\dispSFinmath{
t=\Muserfunction{Spinor}[\Mvariable{k1},m].\Muserfunction{DiracSlash}[p].\Muserfunction{GA}[5].\Muserfunction{Spinor}[\Mvariable{p1},m]
}

\dispSFoutmath{
\varphi (\Mvariable{k1},m).(\gamma \cdot p).{{\gamma }^5}.\varphi ({p_1},m)
}

\dispSFinmath{
\Mvariable{ct}=\Muserfunction{ComplexConjugate}[t]
}

\dispSFoutmath{
-\varphi ({p_1},m).{{\gamma }^5}.(\gamma \cdot p).\varphi (\Mvariable{k1},m)
}

\dispSFinmath{
\Muserfunction{FermionSpinSum}[t\multsp \Mvariable{ct}]
}

\dispSFoutmath{
-\Muserfunction{tr}\big((m+\gamma \cdot \Mvariable{k1}).(\gamma \cdot p).{{\gamma }^5}.(m+\gamma \cdot {p_1}).{{\gamma }^5}.
     (\gamma \cdot p)\big)
}

\dispSFinmath{
\%\multsp /.\Mvariable{DiracTrace}\rightarrow \Mvariable{Tr}
}

\Subsection*{FeynAmp}

\Subsubsection*{Description}

FeynAmp[q, amp] is the head of a Feynman amplitude. amp denotes the analytical expression for the amplitude and q is the integration
  variable. FeynAmp[q1, q2, amp] denotes a two-loop amplitude.

FeynAmp has no functional properties and serves just as a head. There are however special typesetting rules attached.

See also:  Amplitude.

\Subsubsection*{Examples}

This is a 1-loop gluon self-energy amplitude (ignoring factors of (2 \(\pi \))).

\dispSFinmath{
-4\multsp ({p^2}\multsp {m^2}+\Mvariable{k1}\cdot {p_1}\multsp {p^2}-2\multsp \Mvariable{k1}\cdot p\multsp p\cdot {p_1})
}

\dispSFoutmath{
\Mfunction{Clear}[t,\multsp \Mvariable{ct}]
}

This is a generic 2-loop amplitude.

\dispSFinmath{
\MathBegin{MathArray}{l}
\Muserfunction{FeynAmp}[q,\Muserfunction{GV}[p,\mu ,a,\multsp q-p,\alpha ,c,\multsp -q,\beta ,e]\multsp   \\
   \noalign{\vspace{0.5ex}}
\hspace{2.em} \Muserfunction{GP}[p-q,\multsp \alpha ,c,\multsp \rho ,d]
     \Muserfunction{GV}[-p,\nu ,b,\multsp p-q,\rho ,d,\multsp q,\sigma ,f]\multsp
     \Muserfunction{GP}[q,\multsp \beta ,e,\multsp \sigma ,f]]\\
\MathEnd{MathArray}
}

\dispSFoutmath{
\int {{\DifferentialD }^D}q\big(\Pi _{cd}^{\alpha \rho }(p-q)\multsp \Pi _{ef}^{\beta \sigma }(q)\multsp
    {V^{\nu \rho \sigma }}(-p,\multsp p-q,\multsp q)\multsp {V^{\mu \alpha \beta }}(p,\multsp q-p,\multsp -q)\multsp {f_{ace}}\multsp
    {f_{bdf}}\big)
}

\Subsection*{FeynAmpDenominator}

\Subsubsection*{Description}

FeynAmpDenominator[ PropagatorDenominator[ ... ], PropagatorDenominator[ ... ], ...] is the head of the denominators of the propagators,
  i.e., FeynAmpDenominator[x] is the representation of 1/x .

See also:  FAD, FeynAmpDenominatorSimplify.

\Subsubsection*{Examples}

\dispSFinmath{
\Muserfunction{FeynAmp}[{q_1},{q_2},\Mvariable{anyexpression}]
}

\dispSFoutmath{
\int {{\DifferentialD }^D}{q_1}\int {{\DifferentialD }^D}{q_2}(\Mvariable{anyexpression})
}

\dispSFinmath{
\Muserfunction{FeynAmpDenominator}[\Muserfunction{PropagatorDenominator}[p,m]]
}

\dispSFoutmath{
\frac{1}{{p^2}-{m^2}}
}

\dispSFinmath{
\Muserfunction{FeynAmpDenominator}[\Muserfunction{PropagatorDenominator}[p,m],\Muserfunction{PropagatorDenominator}[p-q,m]]
}

\dispSFinmath{
\frac{1}{({p^2}-{m^2}).({{(p-q)}^2}-{m^2})}
}

\dispSFoutmath{
t=\Muserfunction{FeynAmpDenominator}[\Muserfunction{PropagatorDenominator}[p,m]];
}

\dispSFinmath{
\Mfunction{StandardForm}[t//\Muserfunction{FCI}]
}

\dispSFoutmath{
\Muserfunction{FeynAmpDenominator}[\Muserfunction{PropagatorDenominator}[\Muserfunction{Momentum}[p,D],m]]
}

\dispSFinmath{
\Mfunction{StandardForm}[t//\Muserfunction{FCE}]
}

\Subsection*{FeynAmpDenominatorCombine}

\Subsubsection*{Description}

FeynAmpDenominatorCombine[expr] expands expr with respect to FeynAmpDenominator and combines products of FeynAmpDenominator in expr into
  one FeynAmpDenominator.

See also:  FeynAmpDenominatorSplit.

\Subsubsection*{Examples}

\dispSFinmath{
\Muserfunction{FAD}[\{p,m\}]
}

\dispSFoutmath{
\Mfunction{Clear}[t];
}

\dispSFinmath{
t\multsp =\multsp \Muserfunction{FAD}[q]\multsp \Muserfunction{FAD}[q-p]
}

\dispSFoutmath{
\frac{1}{[ {q^2} ]}\multsp \frac{1}{[ {{(q-p)}^2} ]}
}

\dispSFinmath{
\Muserfunction{FeynAmpDenominatorCombine}[\%]//\Muserfunction{FCE}//\Mfunction{StandardForm}
}

\dispSFoutmath{
\Muserfunction{FAD}[q,-p+q]
}

\Subsection*{FeynAmpDenominatorSimplify}

\Subsubsection*{Description}

FeynAmpDenominatorSimplify[exp] tries to simplify each PropagatorDenominator in a canonical way. FeynAmpDenominatorSimplify[exp, q1]
  simplifies all FeynAmpDenominator's in exp in a canonical way, including some translation of momenta. FeynAmpDenominatorSimplify[exp,
  q1, q2] additionally removes integrals with no mass scale.

FDS can be used as an alias.

\dispSFinmath{
\Muserfunction{FeynAmpDenominatorSplit}[\%]//\Muserfunction{FCE}//\Mfunction{StandardForm}
}

\dispSFoutmath{
\Muserfunction{FAD}[q]\multsp \Muserfunction{FAD}[-p+q]
}

See also:  OneLoopSimplify.

\Subsubsection*{Examples}

The cornerstone of dimensional regularization is that \(\Mvariable{FDS}\)

\dispSFinmath{
\Mvariable{FeynAmpDenominatorSimplify}
}

\dispSFoutmath{
\int {d^n}k\multsp f(k)/{k^{2m}}=\multsp 0\multsp .
}

This brings \(\Muserfunction{FeynAmpDenominatorSimplify}[f[k]\multsp \Muserfunction{FAD}[k,k],k]\) into a standard form.

\dispSFinmath{
0
}

\dispSFoutmath{
1/({{(k-{p_1})}^2}\multsp {{(k-{p_2})}^2})
}

\dispSFinmath{
\Muserfunction{FeynAmpDenominatorSimplify}[\Muserfunction{FAD}[k-{p_1},k-{p_2}],k]
}

\dispSFoutmath{
\frac{1}{{k^2}.{{(k-{p_1}+{p_2})}^2}}
}

\dispSFinmath{
t=\Muserfunction{FeynAmpDenominatorSimplify}[\Muserfunction{FAD}[k-{p_1},k-{p_2}]\multsp \Muserfunction{SPD}[k,k],k]
}

\dispSFoutmath{
\frac{{k^2}}{{k^2}.{{(k-{p_1}+{p_2})}^2}}+\frac{2\multsp k\cdot {p_2}}{{k^2}.{{(k-{p_1}+{p_2})}^2}}+
   \frac{p_{2}^{2}}{{k^2}.{{(k-{p_1}+{p_2})}^2}}
}

\dispSFinmath{
r=\Muserfunction{SPC}[t,k,\Mvariable{FDS}\rightarrow \Mvariable{True}]
}

\dispSFoutmath{
\frac{p_{2}^{2}}{{k^2}.{{(k-{p_1}+{p_2})}^2}}-\frac{2\multsp k\cdot {p_2}}{{k^2}.{{(k+{p_1}-{p_2})}^2}}
}

\dispSFinmath{
\Muserfunction{OneLoopSimplify}[r,k]
}

\dispSFoutmath{
\frac{{p_1}\cdot {p_2}}{{k^2}.{{(k-{p_1}+{p_2})}^2}}
}

\dispSFinmath{
\Muserfunction{FDS}[\Muserfunction{FAD}[k-\Mvariable{p1},k-\Mvariable{p2}]\Muserfunction{SPD}[k,\Mvariable{OPEDelta}]\RawWedge 2,k]
}

\Subsection*{FeynAmpDenominatorSplit}

\Subsubsection*{Description}

FeynAmpDenominatorSplit[expr] splits all FeynAmpDenominator[a,b, ...] in expr into FeynAmpDenominator[a]*FeynAmpDenominator[b] ... .
  FeynAmpDenominatorSplit[expr, q1] splits all FeynAmpDenominator in expr into a product of two, one containing q1 and other momenta, the
  second without q1.

See also:  FeynAmpDenominatorCombine.

\Subsubsection*{Examples}

\dispSFinmath{
\frac{{{k\cdot \Delta }^2}}{{k^2}.{{(k-{p_1}+{p_2})}^2}}+
   \frac{2\multsp k\cdot \Delta \multsp \Delta \cdot {p_2}}{{k^2}.{{(k-{p_1}+{p_2})}^2}}+
   \frac{{{\Delta \cdot {p_2}}^2}}{{k^2}.{{(k-{p_1}+{p_2})}^2}}
}

\dispSFoutmath{
\Mfunction{Clear}[t,r]
}

\dispSFinmath{
t=\Muserfunction{FAD}[\Mvariable{q1},\Mvariable{q1}-p,\Mvariable{q1}-\Mvariable{q2},\Mvariable{q2},\Mvariable{q2}-p]//\Muserfunction{FCI}
}

\dispSFoutmath{
\frac{1}{q_{1}^{2}.{{({q_1}-p)}^2}.{{({q_1}-{q_2})}^2}.q_{2}^{2}.{{({q_2}-p)}^2}}
}

\dispSFinmath{
t//\Mfunction{Head}
}

\dispSFoutmath{
\Mvariable{FeynAmpDenominator}
}

\dispSFinmath{
\Muserfunction{FeynAmpDenominatorSplit}[t]
}

\dispSFoutmath{
\frac{1}{q_{1}^{2}\multsp {{({q_1}-p)}^2}\multsp {{({q_1}-{q_2})}^2}\multsp q_{2}^{2}\multsp {{({q_2}-p)}^2}}
}

\dispSFinmath{
\%//\Muserfunction{FCE}//\Mfunction{StandardForm}
}

\dispSFoutmath{
\Muserfunction{FAD}[\Mvariable{q1}]\multsp \Muserfunction{FAD}[-p+\Mvariable{q1}]\multsp
   \Muserfunction{FAD}[\Mvariable{q1}-\Mvariable{q2}]\multsp \Muserfunction{FAD}[\Mvariable{q2}]\multsp
   \Muserfunction{FAD}[-p+\Mvariable{q2}]
}

\dispSFinmath{
\Muserfunction{FeynAmpDenominatorSplit}[t,\Mvariable{q1}]//\Muserfunction{FCE}//\Mfunction{StandardForm}
}

\dispSFoutmath{
\Muserfunction{FAD}[\Mvariable{q2},-p+\Mvariable{q2}]\multsp
   \Muserfunction{FAD}[\Mvariable{q1},-p+\Mvariable{q1},\Mvariable{q1}-\Mvariable{q2}]
}

\dispSFinmath{
\Muserfunction{FeynAmpDenominatorCombine}[\%]//\Muserfunction{FCE}//\Mfunction{StandardForm}
}

\Subsection*{FeynAmpList}

\Subsubsection*{Description}

FeynAmpList[info][FeynAmp[...], FeynAmp[...], ...] is a head of a list of Feynman amplitudes.

FeynAmpList has no functional properties and serves just as a head.

See also:  FeynAmp.

\Subsection*{FeynCalc}

\Subsubsection*{Description}

FeynCalc is simply a symbol with a usage definition.

\dispSFinmath{
\Muserfunction{FAD}[\Mvariable{q1},\Mvariable{q2},\Mvariable{q1}-\Mvariable{q2},-p+\Mvariable{q1},-p+\Mvariable{q2}]
}

\Print{\(\Mfunction{Clear}[t]\)}

\Subsection*{FeynCalcExternal}

\Subsubsection*{Description}

FeynCalcExternal[exp] translates exp from the internal FeynCalc representation to a shorthand form.

See also:  FeynCalcInternal.

\Subsubsection*{Examples}

\dispSFinmath{
?\Mvariable{FeynCalc}
}

\dispSFoutmath{
\MathBegin{MathArray}{l}
\Mvariable{For\multsp installation\multsp notes\multsp visit\multsp www.feyncalc.org}  \\
   \noalign{\vspace{0.5ex}}\multsp For\multsp a\multsp list\multsp of\multsp availabe\multsp objects\multsp type\multsp
   \$FeynCalcStuff,\multsp   \\
\noalign{\vspace{0.5ex}}
\hspace{2.em} \Mvariable{which\multsp contains\multsp a\multsp list\multsp
   of\multsp all\multsp functions\multsp and\multsp options\multsp in\multsp StringForm.\multsp }  \\
\noalign{\vspace{0.5ex}}
   \hspace{2.em} \Mvariable{You\multsp can\multsp get\multsp on-line\multsp information\multsp by\multsp ?function,\multsp e.g.,\multsp
   ?Contract.}  \\
\noalign{\vspace{0.5ex}}\multsp There\multsp are\multsp several\multsp useful\multsp functions\multsp for\multsp
   short\multsp input,\multsp type\multsp   \\
\noalign{\vspace{0.5ex}}
\hspace{2.em} \$FCS\multsp for\multsp a\multsp list\multsp
   of\multsp short\multsp commands.\multsp Then\multsp type,\multsp e.g.,\multsp ?GA.  \\
\noalign{\vspace{0.5ex}}\multsp   \\
   \noalign{\vspace{0.5ex}}\multsp To\multsp get\multsp rid\multsp of\multsp the\multsp start-up\multsp messages\multsp put\multsp
   the\multsp line\multsp   \\
\noalign{\vspace{0.5ex}}\multsp \$FeynCalcStartupMessages\multsp =\multsp False;\multsp   \\
   \noalign{\vspace{0.5ex}}\multsp \multsp into\multsp your\multsp init.m\multsp or\multsp the\multsp
   HighEnergyPhysics/FeynCalcConfig.m\multsp file.\\
\MathEnd{MathArray}
}

\dispSFinmath{
\Muserfunction{FeynCalcExternal}[\Muserfunction{DiracGamma}[5]]
}

\dispSFoutmath{
{{\gamma }^5}
}

\dispSFinmath{
\%//\Mfunction{StandardForm}
}

\dispSFoutmath{
\Muserfunction{GA}[5]
}

\dispSFinmath{
\{\Muserfunction{GA}[\mu ],\Muserfunction{GAD}[\rho ],\Muserfunction{GS}[p],\Muserfunction{SP}[p,q],\Muserfunction{MT}[\alpha ,\beta ],
    \Muserfunction{FV}[p,\mu ]\}
}

\dispSFoutmath{
\{{{\gamma }^{\mu }},{{\gamma }^{\rho }},\gamma \cdot p,p\cdot q,{g^{\alpha \beta }},{p^{\mu }}\}
}

\dispSFinmath{
\%//\Mfunction{StandardForm}
}

\dispSFoutmath{
\{\Muserfunction{GA}[\mu ],\Muserfunction{GAD}[\rho ],\Muserfunction{GS}[p],\Muserfunction{SP}[p,q],\Muserfunction{MT}[\alpha ,\beta ],
    \Muserfunction{FV}[p,\mu ]\}
}

\dispSFinmath{
\%//\Muserfunction{FeynCalcInternal}
}

\dispSFoutmath{
\{{{\gamma }^{\mu }},{{\gamma }^{\rho }},\gamma \cdot p,p\cdot q,{g^{\alpha \beta }},{p^{\mu }}\}
}

\dispSFinmath{
\%//\Mfunction{StandardForm}
}

\dispSFoutmath{
\MathBegin{MathArray}{l}
\{\Muserfunction{DiracGamma}[\Muserfunction{LorentzIndex}[\mu ]],
    \Muserfunction{DiracGamma}[\Muserfunction{LorentzIndex}[\rho ,D],D],  \\
\noalign{\vspace{0.5ex}}
\hspace{1.em} \Muserfunction{DiracG
     amma}[\Muserfunction{Momentum}[p]],\Muserfunction{Pair}[\Muserfunction{Momentum}[p],\Muserfunction{Momentum}[q]],  \\
   \noalign{\vspace{0.5ex}}
\hspace{1.em} \Muserfunction{Pair}[\Muserfunction{LorentzIndex}[\alpha ],\Muserfunction{LorentzIndex}[\beta ]
     ],\Muserfunction{Pair}[\Muserfunction{LorentzIndex}[\mu ],\Muserfunction{Momentum}[p]]\}\\
\MathEnd{MathArray}
}

\Subsection*{FeynCalcForm}

\Subsubsection*{Description}

FeynCalcForm[expr] changes the printed output to a an easy-to-read form. It allows a readable output also when running a terminal based {\itshape
Mathematica} session. Whether the result of FeynCalcForm[expr] is displayed or not, depends on the setting of \${}PrePrint. \${}PrePrint \(=\)
  FeynCalcForm forces displaying everything after applying FeynCalcForm. In order to change to the normal (internal) Mathematica
  OutputForm, do: (\${}PrePrint\(=\).).

See also:  FC, FeynCalcExternal, FeynCalcInternal.

\Subsubsection*{Examples}

This is the normal notebook display:

\dispSFinmath{
\Muserfunction{FeynCalcExternal}[\%]//\Mfunction{StandardForm}
}

\dispSFoutmath{
\{\Muserfunction{GA}[\mu ],\Muserfunction{GAD}[\rho ],\Muserfunction{GS}[p],\Muserfunction{SP}[p,q],\Muserfunction{MT}[\alpha ,\beta ],
    \Muserfunction{FV}[p,\mu ]\}
}

This is the shorthand (terminal) display (easy-to-read form):

\dispSFinmath{
\Muserfunction{SUNTrace}[\Muserfunction{SUNT}[a].\Muserfunction{SUNT}[b].\Muserfunction{SUNT}[c]]
}

\dispSFinmath{
\Muserfunction{tr}({T_a}.{T_b}.{T_c})
}

\dispSFinmath{
\$PrePrint\multsp =\multsp \Mvariable{FeynCalcForm};
}

\mathout
tr[T[a] T[b] T[c]]\endmathout
Reset to normal notebook display:

\dispSFinmath{
\MathBegin{MathArray}{l}
\Mfunction{SetOptions}[\$FrontEnd,
    \Mfunction{Evaluate}[(\Mfunction{Options}[\$FrontEnd,"CommonDefaultFormatTypes"]/.  \\
\noalign{\vspace{0.5ex}}
\hspace{5.em} (
           "Output"\rightarrow \_)\rightarrow ("Output"\rightarrow \Mvariable{OutputForm}))[[1]]]];\\
\MathEnd{MathArray}
}

\dispSFinmath{
\Muserfunction{SUNTrace}[\Muserfunction{SUNT}[a].\Muserfunction{SUNT}[b].\Muserfunction{SUNT}[c]]
}

\Subsection*{FeynCalcInternal}

\Subsubsection*{Description}

FeynCalcInternal[exp] translates exp into the internal FeynCalc (abstract data-type) representation.

See also:  FeynCalcExternal, FCI, FCE.

\Subsubsection*{Examples}

\dispSFinmath{
\$PrePrint=.;
}

\dispSFoutmath{
\MathBegin{MathArray}{l}
\Mfunction{SetOptions}[\$FrontEnd,
    \Mfunction{Evaluate}[(\Mfunction{Options}[\$FrontEnd,"CommonDefaultFormatTypes"]/.  \\
\noalign{\vspace{0.5ex}}
\hspace{5.em} (
           "Output"\rightarrow \_)\rightarrow ("Output"\rightarrow \Mvariable{TraditionalForm}))[[1]]]];\\
\MathEnd{MathArray}
}

\dispSFinmath{
\{\Muserfunction{GA}[\mu ],\Muserfunction{GAD}[\rho ],\Muserfunction{GS}[p],\Muserfunction{SP}[p,q],\Muserfunction{MT}[\alpha ,\beta ],
    \Muserfunction{FV}[p,\mu ]\}
}

\dispSFoutmath{
\{{{\gamma }^{\mu }},{{\gamma }^{\rho }},\gamma \cdot p,p\cdot q,{g^{\alpha \beta }},{p^{\mu }}\}
}

\dispSFinmath{
\%//\Mfunction{StandardForm}
}

\dispSFoutmath{
\{\Muserfunction{GA}[\mu ],\Muserfunction{GAD}[\rho ],\Muserfunction{GS}[p],\Muserfunction{SP}[p,q],\Muserfunction{MT}[\alpha ,\beta ],
    \Muserfunction{FV}[p,\mu ]\}
}

\dispSFinmath{
\%//\Muserfunction{FeynCalcInternal}
}

\dispSFoutmath{
\{{{\gamma }^{\mu }},{{\gamma }^{\rho }},\gamma \cdot p,p\cdot q,{g^{\alpha \beta }},{p^{\mu }}\}
}

\dispSFinmath{
\%//\Mfunction{StandardForm}
}

\dispSFoutmath{
\MathBegin{MathArray}{l}
\{\Muserfunction{DiracGamma}[\Muserfunction{LorentzIndex}[\mu ]],
    \Muserfunction{DiracGamma}[\Muserfunction{LorentzIndex}[\rho ,D],D],  \\
\noalign{\vspace{0.5ex}}
\hspace{1.em} \Muserfunction{DiracG
     amma}[\Muserfunction{Momentum}[p]],\Muserfunction{Pair}[\Muserfunction{Momentum}[p],\Muserfunction{Momentum}[q]],  \\
   \noalign{\vspace{0.5ex}}
\hspace{1.em} \Muserfunction{Pair}[\Muserfunction{LorentzIndex}[\alpha ],\Muserfunction{LorentzIndex}[\beta ]
     ],\Muserfunction{Pair}[\Muserfunction{LorentzIndex}[\mu ],\Muserfunction{Momentum}[p]]\}\\
\MathEnd{MathArray}
}

\dispSFinmath{
\Muserfunction{FeynCalcExternal}[\%]//\Mfunction{StandardForm}
}

\dispSFoutmath{
\{\Muserfunction{GA}[\mu ],\Muserfunction{GAD}[\rho ],\Muserfunction{GS}[p],\Muserfunction{SP}[p,q],\Muserfunction{MT}[\alpha ,\beta ],
    \Muserfunction{FV}[p,\mu ]\}
}

\dispSFinmath{
\Muserfunction{FCI}[\{\Muserfunction{SD}[a,b],\Muserfunction{SUND}[a,b,c],\Muserfunction{SUNF}[a,b,c],\Muserfunction{FAD}[q],
     \Muserfunction{LC}[\mu ,\nu ,\rho ,\sigma ]\}]
}

\dispSFoutmath{
\big\{{{\delta }_{ab}},{d_{abc}},{f_{abc}},\frac{1}{{q^2}},{{\epsilon }^{\mu \nu \rho \sigma }}\big\}
}

\Subsection*{FeynCalc2FORM}

\Subsubsection*{Description}

FeynCalc2FORM[expr] displays expr in FORM syntax. FeynCalc2FORM[file, x] writes x in FORM syntax to a file. FeynCalc2FORM[file,
  x\(=\)\(=\)y] writes x\(=\)y to a file in FORM syntax.

See also:  FORM2FeynCalc.

\dispSFinmath{
\%//\Mfunction{StandardForm}
}

\dispSFoutmath{
\MathBegin{MathArray}{l}
\{\Muserfunction{SUNDelta}[\Muserfunction{SUNIndex}[a],\Muserfunction{SUNIndex}[b]],
    \Muserfunction{SUND}[\Muserfunction{SUNIndex}[a],\Muserfunction{SUNIndex}[b],\Muserfunction{SUNIndex}[c]],  \\
   \noalign{\vspace{0.5ex}}
\hspace{1.em} \Muserfunction{SUNF}[
    \Muserfunction{SUNIndex}[a],\Muserfunction{SUNIndex}[b],\Muserfunction{SUNIndex}[c]],  \\
\noalign{\vspace{0.5ex}}
   \hspace{1.em} \Muserfunction{FeynAmpDenominator}[\Muserfunction{PropagatorDenominator}[\Muserfunction{Momentum}[q,D],0]],  \\
   \noalign{\vspace{0.5ex}}
\hspace{1.em} \Muserfunction{Eps}[
     \Muserfunction{LorentzIndex}[\mu ],\Muserfunction{LorentzIndex}[\nu ],\Muserfunction{LorentzIndex}[\rho ],
      \Muserfunction{LorentzIndex}[\sigma ]]\}\\
\MathEnd{MathArray}
}

\Subsubsection*{Examples}

\dispSFinmath{
\Mfunction{Options}[\Mvariable{FeynCalc2FORM}]
}

\dispSFoutmath{
\MathBegin{MathArray}{l}
\{\Mvariable{EpsDiscard}\rightarrow \Mvariable{False},\Mvariable{FORMEpilog}\rightarrow ,
    \Mvariable{FORMProlog}\rightarrow \Mvariable{write\multsp statistics;},  \\
\noalign{\vspace{0.666667ex}}
\hspace{1.em} \Mvariable{Re
    place}\rightarrow \{\backslash [Alpha]\rightarrow \Mvariable{al},\backslash [Beta]\rightarrow \Mvariable{be},
     \backslash [Gamma]\rightarrow \Mvariable{ga},\backslash [Delta]\rightarrow \Mvariable{de},  \\
\noalign{\vspace{0.666667ex}}
   \hspace{3.em} \backslash [Mu]\rightarrow \mu,\backslash [Nu]\rightarrow \nu,\backslash [Rho]\rightarrow \Mvariable{ro},
       \backslash [Sigma]\rightarrow \Mvariable{si}\},\Mvariable{TraceDimension}\rightarrow 4\}\\
\MathEnd{MathArray}
}

\dispSFinmath{
\Muserfunction{MT}[\mu ,\nu ]\Muserfunction{FV}[p,\rho ]\multsp y\RawWedge 2/d
}

\Print{\(\frac{{y^2}\multsp {p^{\rho }}\multsp {g^{\mu \nu }}}{d}\)}

\dispSFinmath{
\Muserfunction{FeynCalc2FORM}[\%];
}

\dispSFoutmath{
(y\RawWedge 2*d\_(mu,nu)*p(ro))/d
}

\dispSFinmath{
\multsp \Muserfunction{LC}[\alpha ,\beta ,\delta ,\rho ]
}

\Print{\({{\epsilon }^{\alpha \beta \delta \rho }}\)}

\dispSFinmath{
\Muserfunction{FeynCalc2FORM}[\%];
}

\dispSFoutmath{
(-i\_)*e\_(al,be,de,ro)
}

\dispSFinmath{
\Muserfunction{DiracTrace}[\Muserfunction{GA}[\mu ,\nu ,\rho ,\sigma ]]
}

\Print{\(\Muserfunction{tr}({{\gamma }^{\mu }}.{{\gamma }^{\nu }}.{{\gamma }^{\rho }}.{{\gamma }^{\sigma }})\)}

\dispSFinmath{
\Muserfunction{FeynCalc2FORM}[\%];
}

\dispSFoutmath{
\Mvariable{g\_(0,mu)*g\_(0,nu)*g\_(0,ro)*g\_(0,si)}
}

\dispSFinmath{
\Muserfunction{DiracTrace}[\Muserfunction{GA}[\mu ,\nu ]]\Muserfunction{DiracTrace}[\Muserfunction{GA}[\mu ,\rho ]]
}

\Print{\(\Muserfunction{tr}({{\gamma }^{\mu }}.{{\gamma }^{\nu }})\multsp \Muserfunction{tr}({{\gamma }^{\mu }}.{{\gamma }^{\rho }})\)}

\dispSFinmath{
\Muserfunction{FeynCalc2FORM}[\%];
}

\dispSFinmath{
\Mvariable{g\_(0,mu)*g\_(0,nu)*g\_(1,mu)*g\_(1,ro)}
}

\dispSFoutmath{
\Muserfunction{FeynCalc2FORM}["fc2ftest.f",\Muserfunction{MT}[\mu ,\nu ]\Muserfunction{FV}[p,\mu ]];
}

\dispSFinmath{
\Mfunction{ReadList}[\Mfunction{If}[\$OperatingSystem==="MacOS",":",""]<>"fc2ftest.f",\Mvariable{String}]
}

\dispSFoutmath{
\{\Mvariable{d\_(mu,nu)*p(mu)}\}
}

\dispSFinmath{
t=\Muserfunction{Tr}[\Muserfunction{GA}[\mu ,\nu ,\rho ,\sigma ].\Muserfunction{GS}[p,q]]
}

\dispSFinmath{
\MathBegin{MathArray}{l}
4\multsp ({q^{\mu }}\multsp {p^{\nu }}\multsp {g^{\rho \sigma }}-
     {p^{\mu }}\multsp {q^{\nu }}\multsp {g^{\rho \sigma }}+{g^{\mu \nu }}\multsp p\cdot q\multsp {g^{\rho \sigma }}-
     {q^{\mu }}\multsp {g^{\nu \sigma }}\multsp {p^{\rho }}+{g^{\mu \sigma }}\multsp {q^{\nu }}\multsp {p^{\rho }}+
     {p^{\mu }}\multsp {g^{\nu \sigma }}\multsp {q^{\rho }}-{g^{\mu \sigma }}\multsp {p^{\nu }}\multsp {q^{\rho }}+
     {q^{\mu }}\multsp {g^{\nu \rho }}\multsp {p^{\sigma }}-  \\
\noalign{\vspace{0.666667ex}}
\hspace{3.em} {g^{\mu \rho }}\multsp
     {q^{\nu }}\multsp {p^{\sigma }}+{g^{\mu \nu }}\multsp {q^{\rho }}\multsp {p^{\sigma }}-
    {p^{\mu }}\multsp {g^{\nu \rho }}\multsp {q^{\sigma }}+{g^{\mu \rho }}\multsp {p^{\nu }}\multsp {q^{\sigma }}-
    {g^{\mu \nu }}\multsp {p^{\rho }}\multsp {q^{\sigma }}+{g^{\mu \sigma }}\multsp {g^{\nu \rho }}\multsp p\cdot q-
    {g^{\mu \rho }}\multsp {g^{\nu \sigma }}\multsp p\cdot q)\\
\MathEnd{MathArray}
}

\dispSFoutmath{
\Muserfunction{FeynCalc2FORM}["fc2ftest.f",L\multsp ==t];
}

\dispSFinmath{
\MathBegin{MathArray}{l}
\Mvariable{TableForm}[  \\
\noalign{\vspace{0.5ex}}
\hspace{1.em} \Mfunction{ReadList}[
    \Mfunction{If}[\$OperatingSystem==="MacOS",":",""]<>"fc2ftest.f",\Mvariable{String}]]\\
\MathEnd{MathArray}
}

\dispSFinmath{
\MathBegin{MathArray}[c]{l}
  \Mvariable{Indices\multsp \backslash [Mu],\backslash [Nu],\backslash [Rho],\backslash [Sigma];} \\

    \Mvariable{Vectors\multsp p,q;} \\
   \\
  \Mvariable{write\multsp statistics;} \\
   \\
  \Mvariable{Local\multsp L\multsp =\multsp
    (\multsp } \\
  4*(d\_(mu,si)*d\_(nu,ro)*q.p-d\_(mu,ro)*d\_(nu,si)*q.p+d\_(mu,nu)*d\_(ro,si)*q.p+ \\
  \Mvariable{d\_(ro,si)*p(nu)*q(mu
    )-d\_(nu,si)*p(ro)*q(mu)+d\_(nu,ro)*p(si)*q(mu)-} \\
  \Mvariable{d\_(ro,si)*p(mu)*q(nu)+d\_(mu,si)*p(ro)*q(nu)-d\_(mu,ro)*p(si)*q(nu)
    +} \\
  \Mvariable{d\_(nu,si)*p(mu)*q(ro)-d\_(mu,si)*p(nu)*q(ro)+d\_(mu,nu)*p(si)*q(ro)-} \\
  \Mvariable{d\_(nu,ro)*p(mu)*q(si)+d\_(mu
    ,ro)*p(nu)*q(si)-d\_(mu,nu)*p(ro)*q(si))\multsp );\multsp } \\
  \multsp \multsp \multsp  \\
  \Mvariable{Print;\multsp } \\
  .end

    \MathEnd{MathArray}
}

\dispSFinmath{
\Mfunction{If}[\Mfunction{FileNames}["fc2ftest.f"]=!=\{\},\Mfunction{DeleteFile}["fc2ftest.f"]];
}

\Print{\(\Mfunction{Clear}[t];\)}

\Subsection*{FeynCalcToLaTeX}

\Subsubsection*{Description}

FeynCalcToLaTeX[expr] generates LaTeX with line-breaking { }for expr. \\
FeynCalcToLaTeX[expr, 500] generates LaTeX for expr with 500 being the Window width { }setting for the Mathematica frontend. Increasing
  its value will generate less line breaks.

\dispSFinmath{
?\Mvariable{FeynCalcToLaTeX}
}

\dispSFoutmath{
\MathBegin{MathArray}{l}
\Mvariable{FeynCalcToLaTeX[expr]\multsp generates\multsp LaTeX\multsp with\multsp line-breaking\multsp \multsp
   }  \\
\noalign{\vspace{0.5ex}}
\hspace{2.em} \Mvariable{for\multsp expr.\multsp FeynCalcToLaTeX[expr,\multsp 500]\multsp
   generates\multsp LaTeX\multsp for\multsp expr\multsp }  \\
\noalign{\vspace{0.5ex}}
\hspace{2.em} \Mvariable{with\multsp 500\multsp
   being\multsp the\multsp Window\multsp width\multsp \multsp setting\multsp for\multsp the\multsp Mathematica\multsp }  \\
   \noalign{\vspace{0.5ex}}
\hspace{2.em} \Mvariable{frontend.\multsp Increasing\multsp its\multsp value\multsp will\multsp
   generate\multsp less\multsp line\multsp breaks.}\\
\MathEnd{MathArray}
}

\dispSFinmath{
\Muserfunction{GluonPropagator}[p,1,2]//\Muserfunction{Explicit}
}

\dispSFoutmath{
-\frac{\ImaginaryI \multsp {g^{\Mvariable{li1}\Mvariable{li2}}}\multsp {{\delta }_{\Mvariable{ci1}\Mvariable{ci2}}}}{{p^2}}
}

\Subsection*{FeynmanParameterNames}

\Subsubsection*{Description}

FeynmanParameterNames is an option for FeynmanParametrize and FeynmanParametrize.

See also:  FeynmanParametrize, FeynmanParametrize.

\Subsection*{FeynmanParametrize ***unfinished***}

\Subsubsection*{Description}

FeynmanParametrize[exp,k] introduces feynman parameters for all one-loop integrals in exp (k \(=\) integration momentum).

\dispSFinmath{
\Muserfunction{FeynCalcToLaTeX}[\%]
}

\dispSFoutmath{
\MathBegin{MathArray}{l}
-\backslash frac\{\backslash ImaginaryI\multsp \backslash multsp\multsp \{g\RawWedge \{\backslash
   Mvariable\{li1\}\backslash   \\
\noalign{\vspace{0.666667ex}}
\hspace{2.em} \Mvariable{Mvariable\{li2\}\}\}\backslash multsp\multsp
   \{\{\backslash delta\multsp \}\_\{\backslash Mvariable\{ci1\}\backslash Mvariable\{ci2\}\}\}\}\{\{p\RawWedge 2\}\}}\\
   \MathEnd{MathArray}
}

\Subsubsection*{Examples}

\dispSFinmath{
\Mfunction{Options}[\Mvariable{FeynmanParametrize}]
}

\Subsection*{FeynRule}

\Subsubsection*{Description}

FeynRule[lag, \{fields\}] derives the Feynman rule corresponding to the field configuration fields of the lagrangian lag.

\dispSFinmath{
\{\Mvariable{FeynmanParameterNames}\rightarrow \{x,y,z\}\}
}

\dispSFoutmath{
\{ \}
}

FeynRule does not calculate propagator Feynman rules.

The option ZeroMomentumInsertion can be used for twist-2 and higher twist operators.

See also:  Lagrangian.

\Subsubsection*{Examples}


\dispSFoutmath{
\Mfunction{Options}[\Mvariable{FeynRule}]
}

\dispSFinmath{
\MathBegin{MathArray}{l}
\{\Mvariable{Anti5}\rightarrow -\infty ,\Mvariable{Contract}\rightarrow \Mvariable{False},
    \Mvariable{Factor1}\rightarrow \Mvariable{False},\Mvariable{FinalSubstitutions}\rightarrow \{\},
    \Mvariable{PartialD}\rightarrow \Mvariable{RightPartialD},  \\
\noalign{\vspace{0.666667ex}}
\hspace{1.em} \Mvariable{Schouten}
     \rightarrow \Mvariable{False},\Mvariable{ZeroMomentumInsertion}\rightarrow \Mvariable{True},
    \Mvariable{InitialFunction}\rightarrow \Mvariable{PhiToFC}\}\\
\MathEnd{MathArray}
}

\dispSFoutmath{
\Mvariable{gou}=\Muserfunction{Lagrangian}["ogu"]
}

\dispSFinmath{
\frac{1}{2}\multsp {{\ImaginaryI }^{m-1}}\multsp F_{\alpha \Delta }^{a}.{{\big(D_{\Delta }^{ab}\big)}^{m-2}}.F_{\alpha \Delta }^{b}
}

\dispSFoutmath{
\Mvariable{gop}=\Muserfunction{Lagrangian}["ogp"]
}

\dispSFinmath{
\frac{1}{2}\multsp {{\ImaginaryI }^m}\multsp {{\epsilon }^{\alpha \beta \gamma \Delta }}.F_{\beta \gamma }^{a}.
    {{\big(D_{\Delta }^{ab}\big)}^{m-2}}.F_{\alpha \Delta }^{b}
}

\dispSFoutmath{
\Muserfunction{Explicit}[\Mvariable{gop}]
}

2-gluon Feynman rules (unpolarized)

\dispSFinmath{
\MathBegin{MathArray}{l}
\frac{1}{2}\multsp {{\ImaginaryI }^m}\multsp
   {{\epsilon }^{\alpha \beta \gamma \Delta }}.\big({{\partial }_{\beta }}A_{\gamma }^{a}-{{\partial }_{\gamma }}A_{\beta }^{a}+
      {g_s}\multsp A_{\beta }^{\Mvariable{b2}}.A_{\gamma }^{\Mvariable{c12}}\multsp {f_{a\Mvariable{b2}\Mvariable{c12}}}\big).  \\
   \noalign{\vspace{1.03125ex}}
\hspace{2.em} {{\big(D_{\Delta }^{ab}\big)}^{m-2}}.
   \big(\Muserfunction{QuantumField}({{\partial }_{\alpha }},A,\Delta ,b)-{{\partial }_{\Delta }}A_{\alpha }^{b}+
     {g_s}\multsp A_{\alpha }^{\Mvariable{b3}}.A_{\Delta }^{\Mvariable{c13}}\multsp {f_{b\Mvariable{b3}\Mvariable{c13}}}\big)\\
   \MathEnd{MathArray}
}

\dispSFoutmath{
\Muserfunction{Cases2}[\%,\Mvariable{QuantumField}]
}

\dispSFinmath{
\big\{A_{\alpha }^{\Mvariable{b3}},A_{\beta }^{\Mvariable{b2}},A_{\gamma }^{\Mvariable{c12}},A_{\Delta }^{\Mvariable{c13}},
    \Muserfunction{QuantumField}({{\partial }_{\alpha }},A,\Delta ,b),{{\partial }_{\beta }}A_{\gamma }^{a},
    {{\partial }_{\gamma }}A_{\beta }^{a},{{\partial }_{\Delta }}A_{\alpha }^{b}\big\}
}

\dispSFoutmath{
\Mvariable{fi}=\{\Muserfunction{QuantumField}[\Mvariable{GaugeField},\{\mu \},\{a\}][p],
     \Muserfunction{QuantumField}[\Mvariable{GaugeField},\{\nu \},\{b\}][q]\}
}

2-gluon Feynman rules (polarized)

\dispSFinmath{
\big\{A_{\mu }^{a},A_{\nu }^{b}\big\}
}

\dispSFoutmath{
\Mvariable{f2u}=\Mvariable{FullSimplify}/@\Muserfunction{Factor2}[
     \Muserfunction{FeynRule}[\Mvariable{gou},\Mvariable{fi},\Mvariable{ZeroMomentumInsertion}\rightarrow \Mvariable{False}]]
}

\dispSFinmath{
\frac{({{\Delta \cdot q}^2}\multsp {{(\Delta \cdot p)}^m}+{{(\Delta \cdot q)}^m}\multsp {{\Delta \cdot p}^2})\multsp
     ({q^{\mu }}\multsp {{\Delta }^{\nu }}\multsp \Delta \cdot p-{g^{\mu \nu }}\multsp \Delta \cdot q\multsp \Delta \cdot p+
       {{\Delta }^{\mu }}\multsp ({p^{\nu }}\multsp \Delta \cdot q-{{\Delta }^{\nu }}\multsp p\cdot q))\multsp {{\delta }_{ab}}}{2
     \multsp {{\Delta \cdot p}^2}\multsp {{\Delta \cdot q}^2}}
}

\dispSFoutmath{
\Mvariable{fi}=\{\Muserfunction{QuantumField}[\Mvariable{GaugeField},\{\mu \},\{a\}][p],
     \Muserfunction{QuantumField}[\Mvariable{GaugeField},\{\nu \},\{b\}][q]\}
}

\dispSFinmath{
\big\{A_{\mu }^{a},A_{\nu }^{b}\big\}
}

\dispSFoutmath{
\Mvariable{f2p}=\Mvariable{FullSimplify}/@\Muserfunction{Factor2}[
     \Muserfunction{FeynRule}[\Mvariable{gop},\Mvariable{fi},\Mvariable{ZeroMomentumInsertion}\rightarrow \Mvariable{False}]]
}

Compare with the Feynman rule tabulated in Twist2GluonOperator.

\dispSFinmath{
-\frac{\ImaginaryI \multsp ({{(\Delta \cdot p)}^m}\multsp
         ({{\epsilon }^{\nu pq\Delta }}\multsp {{\Delta }^{\mu }}+{{\epsilon }^{\mu \nu q\Delta }}\multsp \Delta \cdot p)\multsp
         {{\Delta \cdot q}^2}-{{\Delta \cdot p}^2}\multsp {{(\Delta \cdot q)}^m}\multsp
         ({{\epsilon }^{\mu pq\Delta }}\multsp {{\Delta }^{\nu }}+{{\epsilon }^{\mu \nu p\Delta }}\multsp \Delta \cdot q))\multsp
      {{\delta }_{ab}}}{{{\Delta \cdot p}^2}\multsp {{\Delta \cdot q}^2}}
}

\dispSFoutmath{
\Muserfunction{Factor2}[\Muserfunction{Calc}[\Mvariable{f2p}/.p\rightarrow -q]]
}

quark-quark Feynman rule (unpolarized)

\dispSFinmath{
\ImaginaryI \multsp (1-{{(-1)}^m})\multsp {{\epsilon }^{\mu \nu \Delta q}}\multsp {{(\Delta \cdot q)}^{m-1}}\multsp {{\delta }_{ab}}
}

\dispSFoutmath{
\Muserfunction{Twist2GluonOperator}[q,\{\mu ,a\},\{\nu ,b\},\Mvariable{Polarization}\rightarrow 1,
    \Mvariable{Explicit}\rightarrow \Mvariable{True}]
}

quark-quark -gluon-gluon Feynman rule (unpolarized)

\dispSFinmath{
\ImaginaryI \multsp {{\epsilon }^{\mu \nu \Delta q}}\multsp (1-{{(-1)}^m})\multsp {{(\Delta \cdot q)}^{m-1}}\multsp {{\delta }_{ab}}
}

\dispSFoutmath{
\Mvariable{qo}=\Muserfunction{Lagrangian}["oqu"]
}

\dispSFinmath{
{{\ImaginaryI }^m}\multsp \overvar{\psi }{\_}.(\gamma \cdot \Delta ).{{{D_{\Delta }}}^{m-1}}.\psi
}

\dispSFoutmath{
\Mvariable{qo}=\Muserfunction{Lagrangian}["oqu"]
}

\dispSFinmath{
{{\ImaginaryI }^m}\multsp \overvar{\psi }{\_}.(\gamma \cdot \Delta ).{{{D_{\Delta }}}^{m-1}}.\psi
}

\dispSFoutmath{
\MathBegin{MathArray}{l}
\Mvariable{qggf}=\{\Muserfunction{QuantumField}[\Mvariable{QuarkField}][p],
     \Muserfunction{QuantumField}[\Mvariable{AntiQuarkField}][q],  \\
\noalign{\vspace{0.5ex}}
\hspace{2.em} \Muserfunction{QuantumField}
      [\Mvariable{GaugeField},\{\mu \},\{a\}][r],\Muserfunction{QuantumField}[\Mvariable{GaugeField},\{\nu \},\{b\}][s]\}\\
   \MathEnd{MathArray}
}

\dispSFinmath{
\big\{\psi ,\overvar{\psi }{\_},A_{\mu }^{a},A_{\nu }^{b}\big\}
}

\dispSFoutmath{
\Mvariable{n4}=\Muserfunction{FeynRule}[\Mvariable{qo},\Mvariable{qggf},\Mvariable{ZeroMomentumInsertion}\rightarrow \Mvariable{True}]
}

\dispSFinmath{
\MathBegin{MathArray}{l}
g_{s}^{2}\multsp {T_b}.{T_a}.(\gamma \cdot \Delta )\multsp
    \Bigg(\sum _{j=0}^{m-3}\multsp (j+1){{(-1)}^j}\multsp {{(\Delta \cdot p)}^{-j+m-3}}\multsp {{(\Delta \cdot q)}^i}\multsp
       {{(\Delta \cdot q+\Delta \cdot s)}^{j-i}}\Bigg)\multsp {{\Delta }^{\mu }}\multsp {{\Delta }^{\nu }}-  \\
\noalign{\vspace{
   1.67708ex}}
\hspace{1.em} {{(-1)}^m}\multsp g_{s}^{2}\multsp {T_a}.{T_b}.(\gamma \cdot \Delta )\multsp
   \Bigg(\sum _{j=0}^{m-3}\multsp (j+1){{(-1)}^j}\multsp {{(\Delta \cdot p)}^i}\multsp {{(\Delta \cdot q)}^{-j+m-3}}\multsp
      {{(\Delta \cdot p+\Delta \cdot s)}^{j-i}}\Bigg)\multsp {{\Delta }^{\mu }}\multsp {{\Delta }^{\nu }}\\
\MathEnd{MathArray}
}

\dispSFoutmath{
\Mvariable{t4}=\Muserfunction{Twist2QuarkOperator}[\{p\},\{q\},\{r,\mu ,a\},\{s,\nu ,b\},\Mvariable{Polarization}\rightarrow 0]
}

 In general equality can be shown by Timing[Factor2[ FCE[Calc[ChangeDimension[FCE[OPESumExplicit[n4-t4]],4]/.s\(\rightarrow \)-p-q-r]]]]
  but it is a little bit slow ...

\dispSFinmath{
\MathBegin{MathArray}{l}
-{{(-1)}^m}\multsp g_{s}^{2}\multsp
   (\gamma \cdot \Delta ).\bigg({T_a}.{T_b}\multsp \bigg(
        \sum _{i=0}^{m-3}\multsp (i+1){{(-(\Delta \cdot p))}^{-i+m-3}}\multsp {{(\Delta \cdot q)}^j}\multsp
          {{(\Delta \cdot q+\Delta \cdot r)}^{i-j}}\bigg)+  \\
\noalign{\vspace{1.5625ex}}
\hspace{4.em} {T_b}.{T_a}\multsp
       \bigg(\sum _{i=0}^{m-3}\multsp (i+1){{(-(\Delta \cdot p))}^{-i+m-3}}\multsp {{(\Delta \cdot q)}^j}\multsp
          {{(\Delta \cdot q+\Delta \cdot s)}^{i-j}}\bigg)\bigg)\multsp {{\Delta }^{\mu }}\multsp {{\Delta }^{\nu }}\\
\MathEnd{MathArray}
}

QCD vertices

\dispSFinmath{
\Muserfunction{Calc}[\Mvariable{n4}-\Mvariable{t4}/.\Mvariable{OPEm}\rightarrow 5/.s\rightarrow -p-q-r/.D\rightarrow 4]
}

\dispSFoutmath{
0
}

\dispSFinmath{
\Mfunction{Clear}[\Mvariable{qggf},\Mvariable{n2},\Mvariable{n4}]
}

\dispSFoutmath{
\MathBegin{MathArray}{l}
\Mvariable{fii}=\{\Muserfunction{QuantumField}[\Mvariable{GaugeField},\{\mu \},\{a\}][p],  \\
   \noalign{\vspace{0.5ex}}
\hspace{2.em} \Muserfunction{QuantumField}[\Mvariable{GaugeField},\{\nu \},\{b\}][q],
    \Muserfunction{QuantumField}[\Mvariable{GaugeField},\{\rho \},\{c\}][r]\}\\
\MathEnd{MathArray}
}

\dispSFinmath{
\big\{A_{\mu }^{a},A_{\nu }^{b},A_{\rho }^{c}\big\}
}

\dispSFoutmath{
\Mvariable{g3}=\Muserfunction{FeynRule}[\Muserfunction{Lagrangian}["QCD"],\Mvariable{fii}]
}

\dispSFinmath{
{g_s}\multsp (({q^{\mu }}-{r^{\mu }})\multsp {g^{\nu \rho }}-{g^{\mu \rho }}\multsp ({p^{\nu }}-{r^{\nu }})+
     {g^{\mu \nu }}\multsp ({p^{\rho }}-{q^{\rho }}))\multsp {f_{abc}}
}

\dispSFoutmath{
\Muserfunction{GluonVertex}[\{p,\mu ,a\},\{q,\nu ,b\},\{r,\rho ,c\},\Mvariable{Explicit}\rightarrow \Mvariable{True}]
}

\dispSFinmath{
{g_s}\multsp ({{(q-r)}^{\mu }}\multsp {g^{\nu \rho }}+{g^{\mu \rho }}\multsp {{(r-p)}^{\nu }}+{g^{\mu \nu }}\multsp {{(p-q)}^{\rho }})
   \multsp {f_{abc}}
}

\dispSFoutmath{
\Muserfunction{Calc}[\Mvariable{g3}-\Muserfunction{ChangeDimension}[\%,4]]
}

\dispSFinmath{
0
}

\dispSFoutmath{
\MathBegin{MathArray}{l}
\Mvariable{fi4}=\{\Muserfunction{QuantumField}[\Mvariable{GaugeField},\{\mu \},\{a\}][p],
     \Muserfunction{QuantumField}[\Mvariable{GaugeField},\{\nu \},\{b\}][q],  \\
\noalign{\vspace{0.5ex}}
\hspace{2.em} \Muserfunction{Qu
       antumField}[\Mvariable{GaugeField},\{\rho \},\{c\}][r],\Muserfunction{QuantumField}[\Mvariable{GaugeField},\{\sigma \},\{d\}][s]\}
   \\
\MathEnd{MathArray}
}

\dispSFinmath{
\big\{A_{\mu }^{a},A_{\nu }^{b},A_{\rho }^{c},A_{\sigma }^{d}\big\}
}

\dispSFoutmath{
\Mvariable{g4}=\Muserfunction{FeynRule}[\Muserfunction{Lagrangian}["QCD"],\Mvariable{fi4}]
}

\dispSFinmath{
\MathBegin{MathArray}{l}
\ImaginaryI \multsp ({g^{\mu \rho }}\multsp {g^{\nu \sigma }}-{g^{\mu \nu }}\multsp {g^{\rho \sigma }})\multsp
    {f_{ad\Mvariable{si1}}}\multsp {f_{bc\Mvariable{si1}}}\multsp g_{s}^{2}+  \\
\noalign{\vspace{0.604167ex}}
\hspace{1.em} \ImaginaryI
    \multsp ({g^{\mu \sigma }}\multsp {g^{\nu \rho }}-{g^{\mu \nu }}\multsp {g^{\rho \sigma }})\multsp {f_{ac\Mvariable{si1}}}\multsp
    {f_{bd\Mvariable{si1}}}\multsp g_{s}^{2}+\ImaginaryI \multsp
    ({g^{\mu \sigma }}\multsp {g^{\nu \rho }}-{g^{\mu \rho }}\multsp {g^{\nu \sigma }})\multsp {f_{ab\Mvariable{si1}}}\multsp
    {f_{cd\Mvariable{si1}}}\multsp g_{s}^{2}\\
\MathEnd{MathArray}
}

\dispSFoutmath{
\Muserfunction{GluonVertex}[\{p,\mu ,a\},\{q,\nu ,b\},\{r,\rho ,c\},\{s,\sigma ,d\},\Mvariable{Explicit}\rightarrow \Mvariable{True}]
}

\dispSFinmath{
\MathBegin{MathArray}{l}
-\ImaginaryI \multsp g_{s}^{2}\multsp   \\
\noalign{\vspace{0.666667ex}}
\hspace{1.em} (
   ({g^{\mu \nu }}\multsp {g^{\rho \sigma }}-{g^{\mu \rho }}\multsp {g^{\nu \sigma }})\multsp {f_{ad\Mvariable{u8}}}\multsp
     {f_{bc\Mvariable{u8}}}+({g^{\mu \nu }}\multsp {g^{\rho \sigma }}-{g^{\mu \sigma }}\multsp {g^{\nu \rho }})\multsp
     {f_{ac\Mvariable{u8}}}\multsp {f_{bd\Mvariable{u8}}}+
    ({g^{\mu \rho }}\multsp {g^{\nu \sigma }}-{g^{\mu \sigma }}\multsp {g^{\nu \rho }})\multsp {f_{ab\Mvariable{u8}}}\multsp
     {f_{cd\Mvariable{u8}}})\\
\MathEnd{MathArray}
}

\Subsection*{FI}

\Subsubsection*{Description}

FI changes the output format to InputForm. This is useful to see the internal representation of FeynCalc objects. To change back to
  FeynCalcForm use FC.

See also:  FeynCalcForm, FC, FeynCalcExternal, FeynCalcInternal.

\Subsection*{FieldDerivative}

\Subsubsection*{Description}

FieldDerivative[f[x],x,li1,li2,...] is the derivative of f[x] with respect to space-time variables x and with Lorentz indices li1, li2,
  ..., where li1, li2, ... have head LorentzIndex. FieldDerivative[f[x],x,li1,li2,...] can be given as
  FieldDerivative[f[x],x,\{l1,l2,...\}], where l1 is li1 without the head, ... FieldDerivative is defined only for objects with head
  QuantumField[...]. If the space-time derivative of other objects is wanted, the corresponding rule must be specified.

See also:  PartialD, ExpandPartialD.

\Subsubsection*{Examples}

\dispSFinmath{
\Muserfunction{Calc}[\Mvariable{g4}-\Muserfunction{ChangeDimension}[\%,4]]
}

\dispSFoutmath{
0
}

\dispSFinmath{
\MathBegin{MathArray}{l}
\Mfunction{Clear}[\Mvariable{f2p},\Mvariable{f2u},\Mvariable{f3},\Mvariable{f32},\Mvariable{fi},\Mvariable{fi4},
    \Mvariable{fii},\Mvariable{g3},\Mvariable{g4},\Mvariable{gop},\Mvariable{gou},  \\
\noalign{\vspace{0.5ex}}
\hspace{1.em} \Mvariable{
     n3},\Mvariable{nf3},\Mvariable{n4},\Mvariable{np2},\Mvariable{npf3},\Mvariable{p33},\Mvariable{pf3},\Mvariable{pn3},\Mvariable{pqo},
    \Mvariable{qf},\Mvariable{qp},\Mvariable{qgf},\Mvariable{qo},\Mvariable{t4}]\\
\MathEnd{MathArray}
}

\dispSFoutmath{
\MathBegin{MathArray}{l}
\Muserfunction{QuantumField}[A,\{\mu \}][x].\Muserfunction{QuantumField}[B,\{\nu \}][y].  \\
   \noalign{\vspace{0.5ex}}
\hspace{1.em} \Muserfunction{QuantumField}[C,\{\rho \}][x].
   \Muserfunction{QuantumField}[D,\{\sigma \}][y]\\
\MathEnd{MathArray}
}

\dispSFinmath{
{A_{\mu }}.{B_{\nu }}.{C_{\rho }}.{D_{\sigma }}
}

\dispSFoutmath{
\Muserfunction{FieldDerivative}[\%,x,\{\mu \}]//\Muserfunction{DotExpand}
}

\Subsection*{FieldStrength}

\Subsubsection*{Description}

FieldStrength[\(\mu \), \(\nu \), a] is the field strength tensor \({A_{\mu }}.{B_{\nu }}.{{\partial }_{\mu }}C_{\rho }^{ }.{D_{\sigma }}+
   {{\partial }_{\mu }}A_{\mu }^{ }.{B_{\nu }}.{C_{\rho }}.{D_{\sigma }}\)\(\Muserfunction{FieldDerivative}[\%,y,\{\nu \}]//\Muserfunction{DotExpand}\).
FieldStrength[\(\mu \), \(\nu \)] is the field strength tensor \({A_{\mu }}.{B_{\nu }}.{{\partial }_{\mu }}C_{\rho }^{ }.{{\partial }_{\nu }}D_{\sigma
}^{ }+
   {A_{\mu }}.{{\partial }_{\nu }}B_{\nu }^{ }.{{\partial }_{\mu }}C_{\rho }^{ }.{D_{\sigma }}+
   {{\partial }_{\mu }}A_{\mu }^{ }.{B_{\nu }}.{C_{\rho }}.{{\partial }_{\nu }}D_{\sigma }^{ }+
   {{\partial }_{\mu }}A_{\mu }^{ }.{{\partial }_{\nu }}B_{\nu }^{ }.{C_{\rho }}.{D_{\sigma }}\) The name of the field (A) and the coupling
constant (g) can be set through the options or by additional arguments. The first two indices
  are interpreted as type LorentzIndex, except OPEDelta, which is converted to Momentum[OPEDelta].

\dispSFinmath{
{{\partial }_{\mu }}A_{\nu }^{a}-{{\partial }_{\nu }}A_{\mu }^{a}+{g_s}\multsp A_{\mu }^{\Mvariable{b1}}A_{\nu }^{\Mvariable{c1}}
}

\dispSFoutmath{
\multsp {f^{a\multsp \Mvariable{b1}\multsp \Mvariable{c1}}}
}

See also:  QuantumField.

\Subsubsection*{Examples}

\dispSFinmath{
{{\partial }_{\mu }}A_{\nu }^{ }-{{\partial }_{\nu }}A_{\mu }^{ }.
}

\dispSFoutmath{
\Mfunction{Options}[\Mvariable{FieldStrength}]
}

\dispSFinmath{
\MathBegin{MathArray}{l}
\{\Mvariable{CouplingConstant}\rightarrow {g_s},\Mvariable{Explicit}\rightarrow \Mvariable{False},  \\
   \noalign{\vspace{0.666667ex}}
\hspace{1.em} \Mvariable{HighEnergyPhysics`fctools`IndexPosition`IndexPosition}\rightarrow \{0,0\},
    \Mvariable{Symbol}\rightarrow F,\Mvariable{QuantumField}\rightarrow A\}\\
\MathEnd{MathArray}
}

\dispSFoutmath{
\Muserfunction{FieldStrength}[\mu ,\nu ]
}

\dispSFinmath{
F_{\mu \nu }^{ }
}

\dispSFoutmath{
\Muserfunction{FieldStrength}[\mu ,\nu ,a]
}

\dispSFinmath{
F_{\mu \nu }^{a}
}

\dispSFoutmath{
\Muserfunction{FieldStrength}[\mu ,\nu ,\Mvariable{Explicit}\rightarrow \Mvariable{True}]
}

\dispSFinmath{
{{\partial }_{\mu }}A_{\nu }^{ }-{{\partial }_{\nu }}A_{\mu }^{ }
}

\dispSFoutmath{
\Muserfunction{FieldStrength}[\mu ,\nu ,a,\Mvariable{Explicit}\rightarrow \Mvariable{True}]
}

\dispSFinmath{
{{\partial }_{\mu }}A_{\nu }^{a}-{{\partial }_{\nu }}A_{\mu }^{a}+
   {g_s}\multsp A_{\mu }^{\Mvariable{b12}}.A_{\nu }^{\Mvariable{c50}}\multsp {f_{a\Mvariable{b12}\Mvariable{c50}}}
}

\dispSFoutmath{
\Mfunction{StandardForm}[\Muserfunction{FieldStrength}[\mu ,\nu ,\Mvariable{Explicit}\rightarrow \Mvariable{True}]]
}

\dispSFinmath{
\MathBegin{MathArray}{l}
\Muserfunction{QuantumField}[\Muserfunction{PartialD}[\Muserfunction{LorentzIndex}[\mu ]],\Mvariable{GaugeField
      },\Muserfunction{LorentzIndex}[\nu ]]-  \\
\noalign{\vspace{0.5ex}}
\hspace{1.em} \Muserfunction{QuantumField}[
   \Muserfunction{PartialD}[\Muserfunction{LorentzIndex}[\nu ]],\Mvariable{GaugeField},\Muserfunction{LorentzIndex}[\mu ]]\\
   \MathEnd{MathArray}
}

\dispSFoutmath{
\Mfunction{StandardForm}[\Muserfunction{FieldStrength}[\mu ,\Mvariable{OPEDelta},\Mvariable{Explicit}\rightarrow \Mvariable{True}]]
}

\Subsection*{FinalFunction}

\Subsubsection*{Description}

FinalFunction is an option for OneLoopSum.

See also:  OneLoopSum.

\Subsection*{FinalSubstitutions}

\Subsubsection*{Description}

FinalSubstitutions is an option for OneLoop and OneLoopSum and Write2. All substitutions indicated hereby are done at the end of the
  calculation.

See also:  OneLoop, OneLoopSum, Write2.

\Subsection*{FORM}

\Subsubsection*{Description}

FORM is an option for RHI. If set to True a FORM file is generated and run from Mathematica (provided R. Hamberg's FORM-program is
  installed correctly ... ).

See also:  RHI, FeynCalc2FORM.

\Subsection*{FORMEpilog}

\Subsubsection*{Description}

FORMEpilog is an option for FeynCalc2FORM. It may be set to a string which is put at the end of the FORM-file.

See also:  FeynCalc2FORM, FORMProlog.

\Subsection*{FORMProlog}

\Subsubsection*{Description}

FORMProlog is an option for FeynCalc2FORM. It may be set to a string which is put after the type declarations of the FORM-file.

See also:  FeynCalc2FORM, FORMEpilog.

\Subsection*{FORM2FeynCalc}

\Subsubsection*{Description}

FORM2FeynCalc[expr] translates the FORM expr into FeynCalc notation. FORM2FeynCalc[file] translates the FORM expresssions in file into
  FeynCalcnotation. FORM2FeynCalc[file, x1, x2, ...] reads in a file in FORM-format and translates the assignments for the variables a,
  b, ... into FeynCalc syntax. If the option Set is True, the variables x1, x2 are assigned to the right hand sides defined in the
  FORM-file.

See also:  FeynCalc2FORM.

\dispSFinmath{
\MathBegin{MathArray}{l}
\Muserfunction{QuantumField}[\Muserfunction{PartialD}[\Muserfunction{LorentzIndex}[\mu ]],\Mvariable{GaugeField
      },\Muserfunction{Momentum}[\Mvariable{OPEDelta}]]-  \\
\noalign{\vspace{0.5ex}}
\hspace{1.em} \Muserfunction{QuantumField}[
   \Muserfunction{PartialD}[\Muserfunction{Momentum}[\Mvariable{OPEDelta}]],\Mvariable{GaugeField},\Muserfunction{LorentzIndex}[\mu ]]\\
   \MathEnd{MathArray}
}

\dispSFoutmath{
\Muserfunction{FieldStrength}[\mu ,\nu ,a,\Mvariable{CouplingConstant}\rightarrow -\Mvariable{Gstrong},
    \Mvariable{Explicit}\rightarrow \Mvariable{True}]
}

\Subsubsection*{Examples}

\dispSFinmath{
{{\partial }_{\mu }}A_{\nu }^{a}-{{\partial }_{\nu }}A_{\mu }^{a}-
   {g_s}\multsp A_{\mu }^{\Mvariable{b13}}.A_{\nu }^{\Mvariable{c51}}\multsp {f_{a\Mvariable{b13}\Mvariable{c51}}}
}

\dispSFoutmath{
\Mfunction{Options}[\Mvariable{FORM2FeynCalc}]
}

\dispSFinmath{
\MathBegin{MathArray}{l}
\{\Mvariable{Dimension}\rightarrow 4,\Mvariable{FinalSubstitutions}\rightarrow \{\},
    \Mvariable{Dot}\rightarrow \Mvariable{Dot},\Mvariable{HoldForm}\rightarrow \Mvariable{True},  \\
\noalign{\vspace{0.666667ex}}
   \hspace{1.em} \Mvariable{LorentzIndex}\rightarrow \{\mu,\nu,\Mvariable{al},\Mvariable{be}\},
    \Mvariable{Set}\rightarrow \Mvariable{False},\Mvariable{Replace}\rightarrow \{\},\Mvariable{Vectors}\rightarrow \Mvariable{Automatic}
    \}\\
\MathEnd{MathArray}
}

\dispSFoutmath{
\Muserfunction{FORM2FeynCalc}["p.q\multsp +\multsp 2*x\multsp m\RawWedge 2"]
}

Functions are automatically converted right, but bracketed expressions need to be substituted explicitly.

\dispSFinmath{
2\multsp x.{m^2}+p\cdot q
}

\dispSFoutmath{
\%//\Mfunction{StandardForm}
}

\dispSFinmath{
2\multsp x.{m^2}+\Muserfunction{SP}[p,q]
}

\dispSFoutmath{
\MathBegin{MathArray}{l}
\Muserfunction{FORM2FeynCalc}["x\multsp +f(z)+\multsp log(x)\RawWedge 2+[li2(1-x)]",  \\
   \noalign{\vspace{0.5ex}}
\hspace{1.em} \Mvariable{Replace}\rightarrow \{"[li2(1-x)]"\rightarrow "PolyLog[2,1-x]"\}]\\
   \MathEnd{MathArray}
}

\dispSFinmath{
{{\log}^2}(x)+x+f(z)+{{\Mvariable{Li}}_2}(1-x)
}

\dispSFoutmath{
\%//\Mfunction{StandardForm}
}

\dispSFinmath{
x+f[z]+{{\log [x]}^2}+\Mfunction{PolyLog}[2,1-x]
}

\dispSFoutmath{
\Muserfunction{FORM2FeynCalc}["x\multsp +\multsp [(1)]*y\multsp -[(-1)\RawWedge m]"]
}

\dispSFinmath{
x+\Mfunction{Hold}[1].y-\Mfunction{Hold}[{{(-1)}^m}]
}

\dispSFoutmath{
\Mfunction{ReleaseHold}[\%]
}

\dispSFinmath{
x-{{(-1)}^m}+1.y
}

\dispSFoutmath{
\Muserfunction{FORM2FeynCalc}["p(mu)*q(nu)+d\_(mu,nu)"]
}

\dispSFinmath{
{p^{\mu}}.{q^{\nu}}+{g^{\mu\nu}}
}

\dispSFoutmath{
\%//\Mfunction{StandardForm}
}

\dispSFinmath{
\Muserfunction{FV}[p,\mu].\Muserfunction{FV}[q,\nu]+\Muserfunction{MT}[\mu,\nu]
}

\dispSFoutmath{
\Muserfunction{FORM2FeynCalc}["p(mu)*q(nu)+d\_(mu,nu)",\Mvariable{Replace}\rightarrow \{\mu\rightarrow \mu ,\nu\rightarrow \nu \}]
}

\Subsection*{FourDivergence}

\Subsubsection*{Description}

FourDivergence[exp, FourVector[p, mu]] calculates the partial derivative of exp w.r.t. p(mu). FourDivergence[exp, FourVector[p, mu],
  FourVector[p,nu], ...]gives the multiple derivative.

See also:  RussianTrick.

\Subsubsection*{Examples}

\dispSFinmath{
{p^{\mu }}.{q^{\nu }}+{g^{\mu \nu }}
}

\dispSFoutmath{
\Muserfunction{FORM2FeynCalc}["i\_*az*bz*aM\RawWedge 2*D1*[(1)]*b\_G1\multsp *\multsp (\multsp 4*eperp(mu,nu)*avec.bvec*blam\multsp )"]
}

\dispSFinmath{
(4\multsp \ImaginaryI ).\Mvariable{az}.\Mvariable{bz}.{{\Mvariable{aM}}^2}.\Mvariable{D1}.\Mfunction{Hold}[1].\Mvariable{b\$G1}.
   \Muserfunction{eperp}(\mu,\nu).(\Mvariable{avec}\cdot \Mvariable{bvec}).\Mvariable{blam}
}

\dispSFoutmath{
t=\Muserfunction{ScalarProduct}[p,q]
}

\dispSFinmath{
p\cdot q
}

\dispSFoutmath{
\Muserfunction{FourDivergence}[t,\Muserfunction{FourVector}[q,\mu ]]
}

\dispSFinmath{
{p^{\mu }}
}

\dispSFoutmath{
t=\Muserfunction{ScalarProduct}[p-k,q]
}

\dispSFinmath{
(p-k)\cdot q
}

\Subsection*{FourLaplacian}

\Subsubsection*{Description}

FourLaplacian[exp, p, q] is \(\Muserfunction{FourDivergence}[t,\Muserfunction{FourVector}[k-p,\mu ]]\)\(0\)exp.

\dispSFinmath{
\Mfunction{Clear}[t]
}

\dispSFoutmath{
\partial /\partial {p_{\mu }}\multsp
}

See also:  FourDivergence, RussianTrick.

\Subsubsection*{Examples}

\dispSFinmath{
\partial /\partial {q_{\mu }}\multsp
}

\dispSFoutmath{
\Mfunction{Options}[\Mvariable{FourLaplacian}]
}

\dispSFinmath{
\{\Mvariable{Dimension}\rightarrow D\}
}

\dispSFoutmath{
\Muserfunction{SP}[q,q]
}

\dispSFinmath{
{q^2}
}

\dispSFoutmath{
\Muserfunction{FourLaplacian}[\%,q,q]
}

\dispSFinmath{
2\multsp D
}

\dispSFoutmath{
\Muserfunction{SOD}[q]\RawWedge \Mvariable{OPEm}\Muserfunction{FAD}[q,q-p]//\Muserfunction{FCI}
}

\Subsection*{FourVector}

\Subsubsection*{Description}

FourVector[p, \(\frac{{{(\Delta \cdot q)}^m}}{{q^2}.{{(q-p)}^2}}\)] is the four-dimensional vector p with Lorentz index \(\Muserfunction{FourLaplacian}[\%,q,\multsp
q]\). A vector with space-time Dimension D is obtained by supplying the option Dimension \(\rightarrow \) D.

See also:  FV, FVD, Pair.

\Subsubsection*{Examples}

\dispSFinmath{
\MathBegin{MathArray}{l}
\frac{4\multsp m\multsp \Delta \cdot p\multsp {{(\Delta \cdot q)}^{m-1}}}{{q^2}.{{(q-p)}^2}.{{(q-p)}^2}}-
   \frac{2\multsp D\multsp {{(\Delta \cdot q)}^m}}{{q^2}.{q^2}.{{(q-p)}^2}}-
   \frac{4\multsp m\multsp {{(\Delta \cdot q)}^m}}{{q^2}.{q^2}.{{(q-p)}^2}}+
   \frac{12\multsp {{(\Delta \cdot q)}^m}}{{q^2}.{q^2}.{{(q-p)}^2}}-  \\
\noalign{\vspace{1.52083ex}}
\hspace{1.em} \frac{2\multsp D
      \multsp {{(\Delta \cdot q)}^m}}{{q^2}.{{(q-p)}^2}.{{(q-p)}^2}}-
   \frac{4\multsp m\multsp {{(\Delta \cdot q)}^m}}{{q^2}.{{(q-p)}^2}.{{(q-p)}^2}}+
   \frac{12\multsp {{(\Delta \cdot q)}^m}}{{q^2}.{{(q-p)}^2}.{{(q-p)}^2}}-
   \frac{4\multsp {{(\Delta \cdot q)}^m}\multsp {p^2}}{{q^2}.{q^2}.{{(q-p)}^2}.{{(q-p)}^2}}\\
\MathEnd{MathArray}
}

\dispSFoutmath{
\mu
}

\dispSFinmath{
\mu
}

\dispSFoutmath{
\Muserfunction{FourVector}[p,\mu ]
}

\dispSFinmath{
{p_{\mu }}
}

\dispSFoutmath{
\Muserfunction{FourVector}[p-q,\mu ]
}

\dispSFinmath{
{{(p-q)}_{\mu }}
}

\dispSFoutmath{
\Mfunction{StandardForm}[\Muserfunction{FourVector}[p,\mu ]]
}

\dispSFinmath{
\Muserfunction{FourVector}[p,\mu ]
}

\dispSFoutmath{
\Mfunction{StandardForm}[\Muserfunction{FourVector}[p,\mu ,\Mvariable{Dimension}\rightarrow D]]
}

There is no special function to expand momenta in FourVector. Since FourVector is turned into Pair internally ExpandScalarProduct may be
  used.

\dispSFinmath{
\Muserfunction{FourVector}[p,\mu ,\Mvariable{Dimension}\rightarrow D]
}

\dispSFoutmath{
\Mfunction{StandardForm}[\Muserfunction{FCE}[\Muserfunction{FourVector}[p,\mu ]]]
}

\Subsection*{FreeIndex}

\Subsubsection*{Description}

FreeIndex is a datatype which is recognized by Contract. Possible use: DataType[mu, FreeIndex] \(=\) True.

See also:  Contract, DataType.

\Subsection*{FreeQ2}

\Subsubsection*{Description}

FreeQ2[expr, \{form1, form2, ...\}] yields True if expr does not contain any occurence of form1, form2, ... and False otherwise.
  FreeQ2[expr, form] is the same as FreeQ[expr, form].

See also:  SelectFree, SelectNotFree.

\Subsubsection*{Examples}

\dispSFinmath{
\Muserfunction{FourVector}[p,\mu ]
}

\dispSFoutmath{
\Muserfunction{ExpandScalarProduct}[\Muserfunction{FourVector}[p-q,\mu ]]
}

\dispSFinmath{
{p^{\mu }}-{q^{\mu }}
}

\dispSFoutmath{
\Muserfunction{FreeQ2}[x+f[x]+y,\multsp \{a,x\}]
}

\dispSFinmath{
\Mvariable{False}
}

\dispSFoutmath{
\Muserfunction{FreeQ2}[x+f[x]+y,\{a,b\}]
}

\dispSFinmath{
\Mvariable{True}
}

\dispSFoutmath{
\Muserfunction{FreeQ2}[x,\multsp y]
}

\Subsection*{FRH}

\Subsubsection*{Description}

FRH[exp\_{}] :\(=\) FixedPoint[ReleaseHold, exp], i.e., FRH removes all HoldForm and Hold in exp.

See also:  Isolate.

\Subsubsection*{Examples}

\dispSFinmath{
\Mvariable{True}
}

\dispSFoutmath{
\Muserfunction{FreeQ2}[f[x],\multsp f]
}

\dispSFinmath{
\Mvariable{False}
}

\dispSFoutmath{
\Mfunction{Hold}[1-1\multsp -\multsp \Mfunction{Hold}[2-2]]
}

\dispSFinmath{
\Mfunction{Hold}[-\Mfunction{Hold}[2-2]+1-1]
}

\dispSFoutmath{
\Muserfunction{FRH}[\%]
}

\dispSFinmath{
0
}

\dispSFoutmath{
\Muserfunction{Isolate}[\Mfunction{Solve}[x\RawWedge 3-x-1==0],x,\Mvariable{IsolateNames}\rightarrow \Mvariable{HH}]
}

\Subsection*{FromTFi}

\Subsubsection*{Description}

FromTFi[expr, q1, q2, p] translates the TFi notatation from the TARCER package to the usual FeynCalc notation. See TFi for details on the
  conventions.

See also: TFi, ToTFi.

\Subsubsection*{Examples}

\dispSFinmath{
\{\{x\rightarrow \Muserfunction{HH}(3)\},\{x\rightarrow \Muserfunction{HH}(6)\},\{x\rightarrow \Muserfunction{HH}(7)\}\}
}

\dispSFoutmath{
\Muserfunction{FRH}[\Muserfunction{HH}[3]]
}

\dispSFinmath{
\frac{1}{3}\multsp {\root{3}\of{\frac{27}{2}-\frac{3\multsp {\sqrt{69}}}{2}}}+
   \frac{{\root{3}\of{\frac{1}{2}\multsp \big(9+{\sqrt{69}}\big)}}}{{3^{2/3}}}
}

\dispSFoutmath{
\Muserfunction{FAD}[\Mvariable{q1},\Mvariable{q1}-p,\{\Mvariable{q2},M\},\{\Mvariable{q2}-p,m\},\Mvariable{q1}-\Mvariable{q2}]//
   \Muserfunction{ToTFi}
}

\dispSFinmath{
F_{\{1,0\}\{1,M\}\{1,0\}\{1,m\}\{1,0\}}^{(D)}
}

\dispSFoutmath{
\Muserfunction{FromTFi}[\Muserfunction{TFi}[D,\Muserfunction{SPD}[p,p],\Muserfunction{SOD}[p],\{\{1,0\},\{1,M\},\{1,0\},\{1,m\},\{1,0\}\}
     ],\Mvariable{q1},\Mvariable{q2},p]
}

\Subsection*{FUNCTION}

\Subsubsection*{Description}

FUNCTION[exp, string] is a head of an expression to be declared a function (of type string), if used in Write2.

See also:  Write2.

\Subsection*{FunctionalD}

\Subsubsection*{Description}

FunctionalD[expr, \{QuantumField[name, \{mu\}, \{a\}][p], ...\}] calculates the functional derivative of expr with respect to the field
  list (with incoming momenta p, etc.) and does the fourier transform. FunctionalD[expr, \{QuantumField[name, \{mu\},\{a\}], ...\}]
  calculates the functional derivative and does partial integration but omits the x-space delta functions.

FunctionalD is a low level function used in FeynRule.

See also:  FeynRule, QuantumField.

\Subsubsection*{Examples}

Instead of the usual \(\frac{1}{q_{2}^{2}.({{({q_1}-p)}^2}-{M^2}).{{({q_2}-p)}^2}.{{({q_1}-{q_2})}^2}.(q_{1}^{2}-{m^2})}\) the arguments and the
\(\delta \) function are omitted, i.e., for the program for simplicity: \(\MathBegin{MathArray}{l}
\Mvariable{FromTFi}[  \\
\noalign{\vspace{0.5ex}}
\hspace{1.em} \Muserfunction{TFi}[
     D,\Muserfunction{SPD}[p,p],\Muserfunction{SOD}[p],\{0,1\},\{\{1,m\},\{1,M\},\{1,0\},\{1,m\},\{1,0\}\}],\Mvariable{q1},\Mvariable{q2
     },p]\\
\MathEnd{MathArray}\)

\dispSFinmath{
\frac{\Delta \cdot {q_2}}{{{({q_1}-{q_2})}^2}.(q_{2}^{2}-{M^2}).({{({q_2}-p)}^2}-{m^2}).(q_{1}^{2}-{m^2}).{{({q_1}-p)}^2}}
}

\dispSFoutmath{
\delta \phi (x)/\delta \phi (y)={{\delta }^{(D)}}(x-y)
}

\dispSFinmath{
\Mvariable{\delta \phi }/\Mvariable{\delta \phi }=1
}

\dispSFoutmath{
\Muserfunction{FunctionalD}[\Muserfunction{QuantumField}[\phi ],\Muserfunction{QuantumField}[\phi ]]
}

Instead of the usual \(1\) the arguments are omitted, and the \(\Muserfunction{FunctionalD}[\Muserfunction{QuantumField}[\phi ]\RawWedge 2,\Muserfunction{QuantumField}[\phi
]]\)operator is specified by default to be an integration by parts operator, i.e., the right hand side will be just \(2\multsp \phi \) or, more precisely
(by default) \((\delta \multsp {{\partial }_{\mu }}\phi (x))/\delta \phi (y)={{\partial }_{\mu }}{{\delta }^{(D)}}(x-y)\).

\dispSFinmath{
{{\partial }_{\mu }}
}

\dispSFoutmath{
-{{\partial }_{\mu }},
}

\(-{{\left( \overvar{\partial }{\rightarrow } \right) }_{\mu }}\)

\dispSFinmath{
\Muserfunction{FunctionalD}[\Muserfunction{QuantumField}[\Muserfunction{PartialD}[\mu ],\phi ],\Muserfunction{QuantumField}[\phi ]]
}

\dispSFoutmath{
-{{\left( \overvar{\partial }{\rightarrow } \right) }_{\mu }}
}

\dispSFinmath{
S[\phi ]\multsp =1/2\int \multsp {d^D}x\multsp [\multsp
       {{\partial }_{\mu }}\phi (x)\multsp {{\partial }^{\mu }}\phi (x)-{m^2}\phi (x)\multsp \phi (y)]
}

\dispSFoutmath{
\MathBegin{MathArray}{l}
s[\phi ]=(\Muserfunction{QuantumField}[\Muserfunction{PartialD}[\mu ],\phi ].
      \Muserfunction{QuantumField}[\Muserfunction{PartialD}[\mu ],\phi ]-  \\
\noalign{\vspace{0.5ex}}
\hspace{4.em} m\RawWedge 2\multsp
      \Muserfunction{QuantumField}[\phi ].\Muserfunction{QuantumField}[\phi ])/2\\
\MathEnd{MathArray}
}

\(\frac{1}{2}\multsp ({{\partial }_{\mu }}\phi _{ }^{ }.{{\partial }_{\mu }}\phi _{ }^{ }-{m^2}\multsp \phi .\phi )\)

{\bfseries First approach}

\dispSFinmath{
\Muserfunction{FunctionalD}[s[\phi ],\Muserfunction{QuantumField}[\phi ]]
}

\dispSFoutmath{
-\phi \multsp {m^2}-{{\partial }_{\mu }}{{\partial }_{\mu }}\phi _{ }^{ }
}

\dispSFinmath{
S[A]\multsp =\multsp -\int \multsp {d^D}x\multsp \frac{1}{4}\multsp F_{a}^{\Mvariable{\mu \nu }}(x)\multsp {F_{\Mvariable{\mu \nu a}}}(x)
}

\dispSFoutmath{
\Mvariable{F1}=\Muserfunction{FieldStrength}[\mu ,\nu ,a,\{A,b,c\},1,\Mvariable{Explicit}\rightarrow \Mvariable{True}]
}

\dispSFinmath{
{{\partial }_{\mu }}A_{\nu }^{a}-{{\partial }_{\nu }}A_{\mu }^{a}+A_{\mu }^{b}.A_{\nu }^{c}\multsp {f_{abc}}
}

\dispSFoutmath{
\Mvariable{F2}=\Muserfunction{FieldStrength}[\mu ,\nu ,a,\{A,d,e\},1,\Mvariable{Explicit}\rightarrow \Mvariable{True}]
}

In order to derive the equation of motion the functional derivative of \({{\partial }_{\mu }}A_{\nu }^{a}-{{\partial }_{\nu }}A_{\mu }^{a}+A_{\mu
}^{d}.A_{\nu }^{e}\multsp {f_{ade}}\) with respect to \(S[A]=-1/4\Mvariable{F1}.\Mvariable{F2}\)has to be set to zero. Bearing in mind that for FeynCalc
we have to be precise as to where which operators (coming from the substitution
  of the derivative of the \(\delta \) function) act:, act with the functional derivative operator on the first field strength:

\(-\frac{1}{4}\multsp \big({{\partial }_{\mu }}A_{\nu }^{a}-{{\partial }_{\nu }}A_{\mu }^{a}+A_{\mu }^{b}.A_{\nu }^{c}\multsp {f_{abc}}
     \big).\big({{\partial }_{\mu }}A_{\nu }^{a}-{{\partial }_{\nu }}A_{\mu }^{a}+A_{\mu }^{d}.A_{\nu }^{e}\multsp {f_{ade}}\big)\)\(S\)

\(A_{\sigma }^{g}\)

\dispSFinmath{
\MathBegin{MathArray}{l}
0\multsp =\multsp (\delta S)/(\delta A_{\sigma }^{g}(y))=-2/4\int {d^D}x\multsp
    (\delta /(\delta A_{\sigma }^{g}(y))\multsp   \\
\noalign{\vspace{1.ex}}
\hspace{2.em} {F_{\Mvariable{\mu \nu a}}}(x)\big)\\
   \MathEnd{MathArray}
}

\dispSFoutmath{
\multsp F_{a}^{,\Mvariable{\mu \nu }}(x)
}

\dispSFinmath{
\Mvariable{See}\multsp \Mvariable{what}\multsp \Mvariable{happens}\multsp \Mvariable{with}\multsp \Mvariable{just}
   (\delta S[A])/(\delta A_{\sigma }^{g}).
}

\dispSFoutmath{
\Mvariable{Ag}=\Muserfunction{QuantumField}[A,\{\sigma \},\{g\}]
}

In order to minimize the number of dummy indices, replace b \(\rightarrow \) c.

\dispSFinmath{
A_{\sigma }^{g}
}

\dispSFoutmath{
\Mvariable{t1}=\Muserfunction{FunctionalD}[\Mvariable{F1},\Mvariable{Ag}]
}

Instead of inserting the definition for the second \(-{g^{\nu \sigma }}\multsp {{\left( \overvar{\partial }{\rightarrow } \right) }_{\mu }}\multsp
{{\delta }_{ag}}+
   {g^{\mu \sigma }}\multsp {{\left( \overvar{\partial }{\rightarrow } \right) }_{\nu }}\multsp {{\delta }_{ag}}+
   {g^{\nu \sigma }}\multsp A_{\mu }^{b}\multsp {f_{abg}}-{g^{\mu \sigma }}\multsp A_{\nu }^{c}\multsp {f_{acg}}\), introduce a QuantumField object
with antisymmetry built into the Lorentz indices:

\dispSFinmath{
\Mvariable{t1}\multsp =\multsp \Mvariable{t1}\multsp /.\multsp b\rightarrow c
}

\dispSFinmath{
-{g^{\nu \sigma }}\multsp {{\left( \overvar{\partial }{\rightarrow } \right) }_{\mu }}\multsp {{\delta }_{ag}}+
   {g^{\mu \sigma }}\multsp {{\left( \overvar{\partial }{\rightarrow } \right) }_{\nu }}\multsp {{\delta }_{ag}}+
   {g^{\nu \sigma }}\multsp A_{\mu }^{c}\multsp {f_{acg}}-{g^{\mu \sigma }}\multsp A_{\nu }^{c}\multsp {f_{acg}}
}

\dispSFoutmath{
F_{a}^{\Mvariable{\mu \nu }}
}

\dispSFinmath{
\MathBegin{MathArray}{l}
F/:\multsp \Muserfunction{QuantumField}[
    \Mvariable{pard\_\_\_},F,\Mvariable{\beta \_},\Mvariable{\alpha \_},\Mvariable{s\_}]:=  \\
\noalign{\vspace{0.5ex}}
   \hspace{1.em} -\Muserfunction{QuantumField}[\Mvariable{pard},F,\alpha ,\beta ,s]/;!\Mfunction{OrderedQ}[\{\beta ,\alpha \}]\\
   \MathEnd{MathArray}
}

\dispSFoutmath{
\Muserfunction{QuantumField}[F,\{\mu ,\nu \},\{a\}]
}

\dispSFinmath{
F_{\mu \nu }^{a}
}

\dispSFoutmath{
\%/.\{\mu \RuleDelayed \nu ,\nu \RuleDelayed \mu \}
}

\dispSFinmath{
-F_{\mu \nu }^{a}
}

\dispSFoutmath{
\MathBegin{MathArray}{l}
\Mvariable{t2}=\Muserfunction{Contract}[\Muserfunction{ExpandPartialD}[-1/2\multsp \Mvariable{t1}.  \\
   \noalign{\vspace{0.5ex}}
\hspace{6.em} \Muserfunction{QuantumField}[
        F,\Muserfunction{LorentzIndex}[\mu ],\Muserfunction{LorentzIndex}[\nu ],\Muserfunction{SUNIndex}[a]]]]/.
   \Mvariable{Dot}\rightarrow \Mvariable{Times}\\
\MathEnd{MathArray}
}

\dispSFinmath{
\frac{1}{2}\multsp {{\partial }_{\mu }}F_{\mu \sigma }^{g}+\frac{1}{2}\multsp {{\partial }_{\nu }}F_{\nu \sigma }^{g}-
   \frac{1}{2}\multsp A_{\mu }^{c}\multsp F_{\mu \sigma }^{a}\multsp {f_{acg}}-
   \frac{1}{2}\multsp A_{\nu }^{c}\multsp F_{\nu \sigma }^{a}\multsp {f_{acg}}
}

\dispSFoutmath{
\Mvariable{t3}\multsp =\multsp \Mvariable{t2}\multsp /.\multsp \nu \rightarrow \mu
}

Since the variational derivative vanishes t4 implies that 0 \(=\) \({{\partial }_{\mu }}F_{\mu \sigma }^{g}-A_{\mu }^{c}\multsp F_{\mu \sigma }^{a}\multsp
{f_{acg}}\) .

{\bfseries Second approach}

It is of course also possible to do the functional deriviate on the S[A] with both field strength tensors inserted.

\dispSFinmath{
\Mvariable{t4}=\Muserfunction{FCE}[\Mvariable{t3}]/.\multsp \Muserfunction{SUNF}[a,c,g]\rightarrow -\Muserfunction{SUNF}[g,c,a]
}

\dispSFoutmath{
{{\partial }_{\mu }}F_{\mu \sigma }^{g}+A_{\mu }^{c}\multsp F_{\mu \sigma }^{a}\multsp {f_{gca}}
}

\dispSFinmath{
{D_{\mu }}F_{g}^{\Mvariable{\mu \sigma }}
}

\dispSFoutmath{
S[A]
}

This is just funcional derivatves and partial integration and simple contraction of indices. No attempt is made to rename dummy indices
  (since this is difficult in general ...).

With a general replacement rule only valid for commuting fields the color indices can be canonicalized a bit more. The idea is to use the
  commutative properties of the vector fields, and canonicalize the color indices by a trick. This function will work on any commuting
  product of fields.

\dispSFinmath{
-\frac{1}{4}\multsp \big({{\partial }_{\mu }}A_{\nu }^{a}-{{\partial }_{\nu }}A_{\mu }^{a}+A_{\mu }^{b}.A_{\nu }^{c}\multsp {f_{abc}}
     \big).\big({{\partial }_{\mu }}A_{\nu }^{a}-{{\partial }_{\nu }}A_{\mu }^{a}+A_{\mu }^{d}.A_{\nu }^{e}\multsp {f_{ade}}\big)
}

\dispSFinmath{
\Mvariable{r1}=\Muserfunction{FunctionalD}[S[A],\Mvariable{Ag}]
}

\dispSFinmath{
\MathBegin{MathArray}{l}
\frac{1}{2}\multsp ({{\partial }_{\mu }}{{\partial }_{\mu }}A_{\sigma }^{g})-
   \frac{1}{2}\multsp ({{\partial }_{\mu }}{{\partial }_{\sigma }}A_{\mu }^{g})+
   \frac{1}{2}\multsp ({{\partial }_{\nu }}{{\partial }_{\nu }}A_{\sigma }^{g})-
   \frac{1}{2}\multsp ({{\partial }_{\nu }}{{\partial }_{\sigma }}A_{\nu }^{g})-
   \frac{1}{4}\multsp A_{\mu }^{b}.{{\partial }_{\mu }}A_{\sigma }^{a}\multsp {f_{abg}}+
   \frac{1}{4}\multsp A_{\mu }^{b}.{{\partial }_{\sigma }}A_{\mu }^{a}\multsp {f_{abg}}-  \\
\noalign{\vspace{1.19792ex}}
   \hspace{1.em} \frac{1}{4}\multsp A_{\nu }^{c}.{{\partial }_{\nu }}A_{\sigma }^{a}\multsp {f_{acg}}+
   \frac{1}{4}\multsp A_{\nu }^{c}.{{\partial }_{\sigma }}A_{\nu }^{a}\multsp {f_{acg}}-
   \frac{1}{4}\multsp A_{\mu }^{b}.A_{\mu }^{d}.A_{\sigma }^{e}\multsp {f_{abg}}\multsp {f_{ade}}+
   \frac{1}{4}\multsp A_{\nu }^{c}.A_{\sigma }^{d}.A_{\nu }^{e}\multsp {f_{acg}}\multsp {f_{ade}}-  \\
\noalign{\vspace{1.19792ex}}
   \hspace{1.em} \frac{1}{4}\multsp {{\partial }_{\mu }}A_{\sigma }^{a}.A_{\mu }^{d}\multsp {f_{adg}}+
   \frac{1}{4}\multsp {{\partial }_{\sigma }}A_{\mu }^{a}.A_{\mu }^{d}\multsp {f_{adg}}-
   \frac{1}{4}\multsp A_{\mu }^{b}.A_{\sigma }^{c}.A_{\mu }^{d}\multsp {f_{abc}}\multsp {f_{adg}}-
   \frac{1}{4}\multsp {{\partial }_{\nu }}A_{\sigma }^{a}.A_{\nu }^{e}\multsp {f_{aeg}}+
   \frac{1}{4}\multsp {{\partial }_{\sigma }}A_{\nu }^{a}.A_{\nu }^{e}\multsp {f_{aeg}}+  \\
\noalign{\vspace{1.19792ex}}
   \hspace{1.em} \frac{1}{4}\multsp A_{\sigma }^{b}.A_{\nu }^{c}.A_{\nu }^{e}\multsp {f_{abc}}\multsp {f_{aeg}}+
   \frac{1}{4}\multsp A_{\mu }^{b}.{{\partial }_{\mu }}A_{\sigma }^{c}\multsp {f_{bcg}}-
   \frac{1}{4}\multsp A_{\sigma }^{b}.{{\partial }_{\nu }}A_{\nu }^{c}\multsp {f_{bcg}}+
   \frac{1}{4}\multsp {{\partial }_{\mu }}A_{\mu }^{b}.A_{\sigma }^{c}\multsp {f_{bcg}}-  \\
\noalign{\vspace{1.19792ex}}
   \hspace{1.em} \frac{1}{4}\multsp {{\partial }_{\nu }}A_{\sigma }^{b}.A_{\nu }^{c}\multsp {f_{bcg}}+
   \frac{1}{4}\multsp A_{\mu }^{d}.{{\partial }_{\mu }}A_{\sigma }^{e}\multsp {f_{deg}}-
   \frac{1}{4}\multsp A_{\sigma }^{d}.{{\partial }_{\nu }}A_{\nu }^{e}\multsp {f_{deg}}+
   \frac{1}{4}\multsp {{\partial }_{\mu }}A_{\mu }^{d}.A_{\sigma }^{e}\multsp {f_{deg}}-
   \frac{1}{4}\multsp {{\partial }_{\nu }}A_{\sigma }^{d}.A_{\nu }^{e}\multsp {f_{deg}}\\
\MathEnd{MathArray}
}

\dispSFoutmath{
\Mfunction{Clear}[\Mvariable{symfun}];
}

\dispSFinmath{
\MathBegin{MathArray}{l}
\Muserfunction{symfun}[\Mvariable{z\_},\Mvariable{fieldname\_Symbol}]:=  \\
\noalign{\vspace{0.5ex}}
   \hspace{1.em} \Mfunction{Expand}[\Muserfunction{SUNSimplify}[
    \Mfunction{FixedPoint}[\Muserfunction{Collect2}[\Muserfunction{DotSimplify}[\#1/.\Mvariable{Times}\rightarrow \Mvariable{Dot}]/.  \\
   \noalign{\vspace{0.5ex}}
\hspace{8.em} (((\Mvariable{qi\_\_\_}).
     \Muserfunction{QuantumField}[\Mvariable{par1\_\_\_},\Mvariable{fieldname},\Mvariable{li1\_},\Mvariable{sui1\_}].  \\
   \noalign{\vspace{0.5ex}}
\hspace{13.em} \Muserfunction{QuantumField}[
        \Mvariable{par2\_\_\_},\Mvariable{fieldname},\Mvariable{li2\_},\Mvariable{sui2\_}].\Mvariable{qf\_\_\_}\multsp )\Mvariable{any\_}
     )\RuleDelayed   \\
\noalign{\vspace{0.5ex}}
\hspace{9.em} (
   (\Mvariable{qi}.\Muserfunction{QuantumField}[\Mvariable{par1},A,\Mvariable{li1},\Mvariable{sui2}].
     \Muserfunction{QuantumField}[\Mvariable{par2},\Mvariable{fieldname},  \\
\noalign{\vspace{0.5ex}}
\hspace{15.em} \Mvariable{li2},
         \Mvariable{sui1}].\Mvariable{qf}\multsp )(\Mvariable{any}/.
       \{\Mvariable{sui1}\RuleDelayed \Mvariable{sui2},\Mvariable{sui2}\RuleDelayed \Mvariable{sui1}\}))/;  \\
   \noalign{\vspace{0.5ex}}
\hspace{10.em} (!(\Muserfunction{FreeQ2}[\Mvariable{any},\{\Mvariable{sui1},\Mvariable{su2}\}]))\&\&
       !(\Mfunction{OrderedQ}[\{\Mvariable{sui1},\Mvariable{sui2}\}])/.  \\
\noalign{\vspace{0.5ex}}
\hspace{7.em} \Mvariable{Dot}
           \rightarrow \Mvariable{Times},\Mvariable{QuantumField}]\&,z,42]]]\\
\MathEnd{MathArray}
}

\dispSFoutmath{
\Mvariable{r2}\multsp =\multsp \Muserfunction{symfun}[\Mvariable{r1},A]
}

Inspection reveals that still terms are the same. Gather the terms with two \(\MathBegin{MathArray}{l}
\frac{1}{2}\multsp ({{\partial }_{\mu }}{{\partial }_{\mu }}A_{\sigma }^{g})-
   \frac{1}{2}\multsp ({{\partial }_{\mu }}{{\partial }_{\sigma }}A_{\mu }^{g})+
   \frac{1}{2}\multsp ({{\partial }_{\nu }}{{\partial }_{\nu }}A_{\sigma }^{g})-
   \frac{1}{2}\multsp ({{\partial }_{\nu }}{{\partial }_{\sigma }}A_{\nu }^{g})-
   \frac{1}{2}\multsp A_{\sigma }^{a}\multsp {{\partial }_{\mu }}A_{\mu }^{b}\multsp {f_{abg}}+  \\
\noalign{\vspace{1.19792ex}}
   \hspace{1.em} A_{\mu }^{a}\multsp {{\partial }_{\mu }}A_{\sigma }^{b}\multsp {f_{abg}}-
   \frac{1}{2}\multsp A_{\sigma }^{a}\multsp {{\partial }_{\nu }}A_{\nu }^{b}\multsp {f_{abg}}+
   A_{\nu }^{a}\multsp {{\partial }_{\nu }}A_{\sigma }^{b}\multsp {f_{abg}}-
   \frac{1}{2}\multsp A_{\mu }^{a}\multsp {{\partial }_{\sigma }}A_{\mu }^{b}\multsp {f_{abg}}-
   \frac{1}{2}\multsp A_{\nu }^{a}\multsp {{\partial }_{\sigma }}A_{\nu }^{b}\multsp {f_{abg}}-  \\
\noalign{\vspace{1.19792ex}}
   \hspace{1.em} \frac{1}{4}\multsp A_{\mu }^{b}\multsp A_{\mu }^{c}\multsp A_{\sigma }^{e}\multsp {f_{abg}}\multsp {f_{ace}}-
   \frac{1}{2}\multsp A_{\nu }^{b}\multsp A_{\nu }^{c}\multsp A_{\sigma }^{e}\multsp {f_{abg}}\multsp {f_{ace}}-
   \frac{1}{4}\multsp A_{\mu }^{b}\multsp A_{\mu }^{c}\multsp A_{\sigma }^{e}\multsp {f_{abe}}\multsp {f_{acg}}\\
\MathEnd{MathArray}\)'s:

\dispSFinmath{
\Mvariable{r3}=\Mvariable{r2}/.\nu \rightarrow \mu
}

\dispSFoutmath{
\MathBegin{MathArray}{l}
{{\partial }_{\mu }}{{\partial }_{\mu }}A_{\sigma }^{g}-{{\partial }_{\mu }}{{\partial }_{\sigma }}A_{\mu }^{g}-
   A_{\sigma }^{a}\multsp {{\partial }_{\mu }}A_{\mu }^{b}\multsp {f_{abg}}+
   2\multsp A_{\mu }^{a}\multsp {{\partial }_{\mu }}A_{\sigma }^{b}\multsp {f_{abg}}-  \\
\noalign{\vspace{0.958333ex}}
   \hspace{1.em} A_{\mu }^{a}\multsp {{\partial }_{\sigma }}A_{\mu }^{b}\multsp {f_{abg}}-
   \frac{3}{4}\multsp A_{\mu }^{b}\multsp A_{\mu }^{c}\multsp A_{\sigma }^{e}\multsp {f_{abg}}\multsp {f_{ace}}-
   \frac{1}{4}\multsp A_{\mu }^{b}\multsp A_{\mu }^{c}\multsp A_{\sigma }^{e}\multsp {f_{abe}}\multsp {f_{acg}}\\
\MathEnd{MathArray}
}

\dispSFinmath{
f\multsp
}

\dispSFoutmath{
\Mvariable{twof}=\Mfunction{Select}[\Mvariable{r3},\Mfunction{Count}[\#,\Muserfunction{SUNF}[\_\_]]===2\&]
}

\dispSFinmath{
-\frac{3}{4}\multsp A_{\mu }^{b}\multsp A_{\mu }^{c}\multsp A_{\sigma }^{e}\multsp {f_{abg}}\multsp {f_{ace}}-
   \frac{1}{4}\multsp A_{\mu }^{b}\multsp A_{\mu }^{c}\multsp A_{\sigma }^{e}\multsp {f_{abe}}\multsp {f_{acg}}
}

\dispSFoutmath{
\Mvariable{twofnew}=(\Mvariable{twof}\LeftDoubleBracket 1\RightDoubleBracket +
      (\Muserfunction{twof}[[2]]/.\{b\RuleDelayed c,c\RuleDelayed b\}))/.\{a\RuleDelayed c,c\RuleDelayed a\}
}

Check that this is now indeed the same as the t4 result from the first attempt.

\dispSFinmath{
-A_{\mu }^{a}\multsp A_{\mu }^{b}\multsp A_{\sigma }^{e}\multsp {f_{ace}}\multsp {f_{bcg}}
}

\dispSFoutmath{
\Mvariable{r4}\multsp =\multsp \Mvariable{r3}-\Mvariable{twof}+\Mvariable{twofnew}
}

\dispSFinmath{
{{\partial }_{\mu }}{{\partial }_{\mu }}A_{\sigma }^{g}-{{\partial }_{\mu }}{{\partial }_{\sigma }}A_{\mu }^{g}-
   A_{\sigma }^{a}\multsp {{\partial }_{\mu }}A_{\mu }^{b}\multsp {f_{abg}}+
   2\multsp A_{\mu }^{a}\multsp {{\partial }_{\mu }}A_{\sigma }^{b}\multsp {f_{abg}}-
   A_{\mu }^{a}\multsp {{\partial }_{\sigma }}A_{\mu }^{b}\multsp {f_{abg}}-
   A_{\mu }^{a}\multsp A_{\mu }^{b}\multsp A_{\sigma }^{e}\multsp {f_{ace}}\multsp {f_{bcg}}
}

\dispSFoutmath{
\Mvariable{t4}
}

\dispSFinmath{
{{\partial }_{\mu }}F_{\mu \sigma }^{g}+A_{\mu }^{c}\multsp F_{\mu \sigma }^{a}\multsp {f_{gca}}
}

\dispSFoutmath{
\MathBegin{MathArray}{l}
\Mvariable{w0}=\Muserfunction{RightPartialD}[\mu ].\Muserfunction{FieldStrength}[\mu ,\sigma ,g,\{A,a,b\},1]
    +  \\
\noalign{\vspace{0.5ex}}
\hspace{2.em} \Muserfunction{QuantumField}[
    A,\Muserfunction{LorentzIndex}[\mu ],\Muserfunction{SUNIndex}[c]]  \\
\noalign{\vspace{0.5ex}}
\hspace{3.em} \Muserfunction{FieldStre
     ngth}[\mu ,\sigma ,a,\{A,b,d\},1]\Muserfunction{SUNF}[g,c,a]\\
\MathEnd{MathArray}
}

\dispSFinmath{
{{\left( \overvar{\partial }{\rightarrow } \right) }_{\mu }}.F_{\mu \sigma }^{g\{A,a,b\}1}+
   F_{\mu \sigma }^{a\{A,b,d\}1}\multsp A_{\mu }^{c}\multsp {f_{gca}}
}

\dispSFoutmath{
\Mvariable{w1}=\Muserfunction{Explicit}[\Mvariable{w0}]
}

\dispSFinmath{
{{\left( \overvar{\partial }{\rightarrow } \right) }_{\mu }}.
    \big({{\partial }_{\mu }}A_{\sigma }^{g}-{{\partial }_{\sigma }}A_{\mu }^{g}+A_{\mu }^{a}.A_{\sigma }^{b}\multsp {f_{abg}}\big)-
   A_{\mu }^{c}\multsp \big({{\partial }_{\mu }}A_{\sigma }^{a}-{{\partial }_{\sigma }}A_{\mu }^{a}+
      A_{\mu }^{b}.A_{\sigma }^{d}\multsp {f_{abd}}\big)\multsp {f_{acg}}
}

\dispSFoutmath{
\Mvariable{w2}=\Muserfunction{ExpandPartialD}[\Mvariable{w1}]/.\Mvariable{Dot}\rightarrow \Mvariable{Times}
}

\dispSFinmath{
\MathBegin{MathArray}{l}
{{\partial }_{\mu }}{{\partial }_{\mu }}A_{\sigma }^{g}-{{\partial }_{\mu }}{{\partial }_{\sigma }}A_{\mu }^{g}+
   {{A_{\sigma }^{b}}^{\GothicC }}\multsp {{{{\partial }_{\mu }}A_{\mu }^{a}}^{\GothicC }}\multsp {f_{abg}}+
   {{A_{\mu }^{a}}^{\GothicC }}\multsp {{{{\partial }_{\mu }}A_{\sigma }^{b}}^{\GothicC }}\multsp {f_{abg}}+  \\
\noalign{\vspace{
   0.770833ex}}
\hspace{1.em} {{A_{\mu }^{c}}^{\GothicC }}\multsp {{{{\partial }_{\sigma }}A_{\mu }^{a}}^{\GothicC }}\multsp {f_{acg}}-
   A_{\mu }^{c}\multsp {{\partial }_{\mu }}A_{\sigma }^{a}\multsp {f_{acg}}-
   A_{\mu }^{b}\multsp A_{\mu }^{c}\multsp A_{\sigma }^{d}\multsp {f_{abd}}\multsp {f_{acg}}\\
\MathEnd{MathArray}
}

\dispSFoutmath{
\Mvariable{dif1}\multsp =\multsp \Mvariable{w2}-\Mvariable{r4}
}

quod erat demonstrandum.

\dispSFinmath{
\MathBegin{MathArray}{l}
-({{\partial }_{\mu }}{{\partial }_{\mu }}A_{\sigma }^{g})+
   {{\partial }_{\mu }}{{\partial }_{\sigma }}A_{\mu }^{g}+{{\partial }_{\mu }}{{\partial }_{\mu }}A_{\sigma }^{g}-
   {{\partial }_{\mu }}{{\partial }_{\sigma }}A_{\mu }^{g}+
   {{A_{\sigma }^{b}}^{\GothicC }}\multsp {{{{\partial }_{\mu }}A_{\mu }^{a}}^{\GothicC }}\multsp {f_{abg}}+  \\
\noalign{\vspace{
   0.791667ex}}
\hspace{1.em} {{A_{\mu }^{a}}^{\GothicC }}\multsp {{{{\partial }_{\mu }}A_{\sigma }^{b}}^{\GothicC }}\multsp {f_{abg}}+
   A_{\sigma }^{a}\multsp {{\partial }_{\mu }}A_{\mu }^{b}\multsp {f_{abg}}-
   2\multsp A_{\mu }^{a}\multsp {{\partial }_{\mu }}A_{\sigma }^{b}\multsp {f_{abg}}+
   A_{\mu }^{a}\multsp {{\partial }_{\sigma }}A_{\mu }^{b}\multsp {f_{abg}}+  \\
\noalign{\vspace{0.770833ex}}
\hspace{1.em} {{A_{\mu }
        ^{c}}^{\GothicC }}\multsp {{{{\partial }_{\sigma }}A_{\mu }^{a}}^{\GothicC }}\multsp {f_{acg}}-
   A_{\mu }^{c}\multsp {{\partial }_{\mu }}A_{\sigma }^{a}\multsp {f_{acg}}-
   A_{\mu }^{b}\multsp A_{\mu }^{c}\multsp A_{\sigma }^{d}\multsp {f_{abd}}\multsp {f_{acg}}+
   A_{\mu }^{a}\multsp A_{\mu }^{b}\multsp A_{\sigma }^{e}\multsp {f_{ace}}\multsp {f_{bcg}}\\
\MathEnd{MathArray}
}

\Subsubsection*{Examples of funcional differentiation as used in FeynRule}

This is a part of the QCD Lagrangian.

\dispSFinmath{
\Mvariable{dif2}=\Muserfunction{symfun}[\Mvariable{dif1},A]
}

\dispSFoutmath{
\MathBegin{MathArray}{l}
-({{\partial }_{\mu }}{{\partial }_{\mu }}A_{\sigma }^{g})+
   {{\partial }_{\mu }}{{\partial }_{\sigma }}A_{\mu }^{g}+{{\partial }_{\mu }}{{\partial }_{\mu }}A_{\sigma }^{g}-
   {{\partial }_{\mu }}{{\partial }_{\sigma }}A_{\mu }^{g}+
   {{A_{\sigma }^{b}}^{\GothicC }}\multsp {{{{\partial }_{\mu }}A_{\mu }^{a}}^{\GothicC }}\multsp {f_{abg}}+  \\
\noalign{\vspace{
   0.791667ex}}
\hspace{1.em} {{A_{\mu }^{a}}^{\GothicC }}\multsp {{{{\partial }_{\mu }}A_{\sigma }^{b}}^{\GothicC }}\multsp {f_{abg}}+
   {{A_{\mu }^{b}}^{\GothicC }}\multsp {{{{\partial }_{\sigma }}A_{\mu }^{a}}^{\GothicC }}\multsp {f_{abg}}+
   A_{\sigma }^{a}\multsp {{\partial }_{\mu }}A_{\mu }^{b}\multsp {f_{abg}}-
   2\multsp A_{\mu }^{a}\multsp {{\partial }_{\mu }}A_{\sigma }^{b}\multsp {f_{abg}}+  \\
\noalign{\vspace{0.770833ex}}
   \hspace{1.em} A_{\mu }^{a}\multsp {{\partial }_{\sigma }}A_{\mu }^{b}\multsp {f_{abg}}+
   A_{\mu }^{a}\multsp {{\partial }_{\mu }}A_{\sigma }^{b}\multsp {f_{abg}}+
   A_{\mu }^{b}\multsp A_{\mu }^{c}\multsp A_{\sigma }^{d}\multsp {f_{abd}}\multsp {f_{acg}}-
   A_{\mu }^{b}\multsp A_{\mu }^{c}\multsp A_{\sigma }^{d}\multsp {f_{abd}}\multsp {f_{acg}}\\
\MathEnd{MathArray}
}

\dispSFinmath{
\MathBegin{MathArray}{l}
\Mfunction{Unset}[s[\phi ]];\Mfunction{Unset}[S[A]];
   \Mfunction{Clear}[\Mvariable{Ag},\Mvariable{F1},\Mvariable{F2},\Mvariable{t1},\Mvariable{t2},\Mvariable{t3},\Mvariable{t4},F,  \\
   \noalign{\vspace{0.5ex}}
\hspace{1.em} \Mvariable{r1},\Mvariable{r2},\Mvariable{r3},\Mvariable{r4},\Mvariable{symfun},\Mvariable{twof
     },\Mvariable{twofnew},\Mvariable{w0},\Mvariable{w1},\Mvariable{w2},\Mvariable{dif1},\Mvariable{dif2},\Mvariable{dif3}]\\
   \MathEnd{MathArray}
}

\dispSFoutmath{
\MathBegin{MathArray}{l}
(\Mvariable{Gstrong}*\Muserfunction{QuantumField}[
      \Mvariable{GaugeField},\multsp \Muserfunction{LorentzIndex}[\Mvariable{li1}],\multsp \multsp
       \Muserfunction{SUNIndex}[\Mvariable{si2}]].\multsp \multsp   \\
\noalign{\vspace{0.5ex}}
\hspace{4.em} \Muserfunction{QuantumField
     }[\Mvariable{GaugeField},\multsp \Muserfunction{LorentzIndex}[\Mvariable{li2}],\multsp \multsp \multsp
     \Muserfunction{SUNIndex}[\Mvariable{si4}]]\multsp .\multsp \multsp \Mvariable{QuantumField}[  \\
\noalign{\vspace{0.5ex}}
   \hspace{5.em} \Muserfunction{PartialD}[\Muserfunction{LorentzIndex}[\Mvariable{li1}]],\multsp \Mvariable{GaugeField},\multsp \multsp
      \multsp \Muserfunction{LorentzIndex}[\Mvariable{li2}],\multsp \Muserfunction{SUNIndex}[\Mvariable{si1}]]*\multsp \multsp   \\
   \noalign{\vspace{0.5ex}}
\hspace{3.em} \Muserfunction{SUNF}[
      \Muserfunction{SUNIndex}[\Mvariable{si1}],\multsp \Muserfunction{SUNIndex}[\Mvariable{si2}],\multsp
       \Muserfunction{SUNIndex}[\Mvariable{si4}]])/4\\
\MathEnd{MathArray}
}

\dispSFinmath{
\frac{1}{4}\multsp {g_s}\multsp A_{\Mvariable{li1}}^{\Mvariable{si2}}.A_{\Mvariable{li2}}^{\Mvariable{si4}}.
    {{\partial }_{\Mvariable{li1}}}A_{\Mvariable{li2}}^{\Mvariable{si1}}\multsp {f_{\Mvariable{si1}\Mvariable{si2}\Mvariable{si4}}}
}

\dispSFoutmath{
\MathBegin{MathArray}{l}
\Muserfunction{FunctionalD}[\%,\multsp
    \{\Muserfunction{QuantumField}[\Mvariable{GaugeField},\{\mu \},\{a\}][p],\multsp   \\
\noalign{\vspace{0.5ex}}
\hspace{2.em}
         \Muserfunction{QuantumField}[\Mvariable{GaugeField},\{\nu \},\{b\}][q],\multsp
      \Muserfunction{QuantumField}[\Mvariable{GaugeField},\{\rho \},\{c\}][r]\}]\\
\MathEnd{MathArray}
}

\Subsection*{FV}

\Subsubsection*{Description}

FV[{\itshape p}, \(\MathBegin{MathArray}{l}
\frac{1}{4}\multsp {g_s}\multsp
   \big(\big({g^{\Mvariable{li1}\mu }}\multsp {{\delta }_{a\Mvariable{si2}}}\big).
      \big({g^{\Mvariable{li2}\nu }}\multsp {{\delta }_{b\Mvariable{si4}}}\big).
      \big(-\ImaginaryI \multsp {r^{\Mvariable{li1}}}\multsp {g^{\Mvariable{li2}\rho }}\multsp {{\delta }_{c\Mvariable{si1}}}\big)+
     \big({g^{\Mvariable{li1}\mu }}\multsp {{\delta }_{a\Mvariable{si2}}}\big).
      \big({g^{\Mvariable{li2}\rho }}\multsp {{\delta }_{c\Mvariable{si4}}}\big).
      \big(-\ImaginaryI \multsp {q^{\Mvariable{li1}}}\multsp {g^{\Mvariable{li2}\nu }}\multsp {{\delta }_{b\Mvariable{si1}}}\big)+  \\
   \noalign{\vspace{1.01042ex}}
\hspace{3.em} \big({g^{\Mvariable{li1}\nu }}\multsp {{\delta }_{b\Mvariable{si2}}}\big).
    \big({g^{\Mvariable{li2}\mu }}\multsp {{\delta }_{a\Mvariable{si4}}}\big).
    \big(-\ImaginaryI \multsp {r^{\Mvariable{li1}}}\multsp {g^{\Mvariable{li2}\rho }}\multsp {{\delta }_{c\Mvariable{si1}}}\big)+
   \big({g^{\Mvariable{li1}\nu }}\multsp {{\delta }_{b\Mvariable{si2}}}\big).
    \big({g^{\Mvariable{li2}\rho }}\multsp {{\delta }_{c\Mvariable{si4}}}\big).
    \big(-\ImaginaryI \multsp {p^{\Mvariable{li1}}}\multsp {g^{\Mvariable{li2}\mu }}\multsp {{\delta }_{a\Mvariable{si1}}}\big)+  \\
   \noalign{\vspace{0.770833ex}}
\hspace{3.em} \big({g^{\Mvariable{li1}\rho }}\multsp {{\delta }_{c\Mvariable{si2}}}\big).
      \big({g^{\Mvariable{li2}\mu }}\multsp {{\delta }_{a\Mvariable{si4}}}\big).
      \big(-\ImaginaryI \multsp {q^{\Mvariable{li1}}}\multsp {g^{\Mvariable{li2}\nu }}\multsp {{\delta }_{b\Mvariable{si1}}}\big)+
     \big({g^{\Mvariable{li1}\rho }}\multsp {{\delta }_{c\Mvariable{si2}}}\big).
      \big({g^{\Mvariable{li2}\nu }}\multsp {{\delta }_{b\Mvariable{si4}}}\big).
      \big(-\ImaginaryI \multsp {p^{\Mvariable{li1}}}\multsp {g^{\Mvariable{li2}\mu }}\multsp {{\delta }_{a\Mvariable{si1}}}\big)\big)
    \multsp {f_{\Mvariable{si1}\Mvariable{si2}\Mvariable{si4}}}\\
\MathEnd{MathArray}\)] is the four-dimensional vector \(\Muserfunction{Calc}[\%//\Muserfunction{Calc}]//\Mfunction{Factor}\).

See also:  FCE, FCI, FVD, FourVector, Pair.

\Subsubsection*{Examples}

\dispSFinmath{
\frac{1}{4}\multsp \ImaginaryI \multsp {g_s}\multsp ({q^{\mu }}\multsp {g^{\nu \rho }}-{r^{\mu }}\multsp {g^{\nu \rho }}-
     {g^{\mu \rho }}\multsp {p^{\nu }}+{g^{\mu \rho }}\multsp {r^{\nu }}+{g^{\mu \nu }}\multsp {p^{\rho }}-
     {g^{\mu \nu }}\multsp {q^{\rho }})\multsp {f_{abc}}
}

\dispSFoutmath{
\mu
}

\dispSFinmath{
{p^{\mu }}
}

\dispSFoutmath{
\Muserfunction{FV}[p,\mu ]
}

\dispSFinmath{
{p^{\mu }}
}

\dispSFoutmath{
\Muserfunction{FV}[p-q,\mu ]
}

\dispSFinmath{
{{p-q}^{\mu }}
}

\dispSFoutmath{
\Muserfunction{FV}[p,\mu ]//\Mfunction{StandardForm}
}

There is no special function to expand momenta in FV; ExpandScalarProduct does the job.

\dispSFinmath{
\Muserfunction{FV}[p,\mu ]
}

\dispSFoutmath{
\Muserfunction{FCI}[\Muserfunction{FV}[p,\mu ]]//\Mfunction{StandardForm}
}

\dispSFinmath{
\Muserfunction{Pair}[\Muserfunction{LorentzIndex}[\mu ],\Muserfunction{Momentum}[p]]
}

\dispSFoutmath{
\Muserfunction{ExpandScalarProduct}[\Muserfunction{FV}[p-q,\mu ]]
}

\Subsection*{FVD}

\Subsubsection*{Description}

FVD[{\itshape p}, \({p^{\mu }}-{q^{\mu }}\)] is the D-dimensional vector {\itshape p} with Lorentz index \(\Mfunction{StandardForm}[\%]\).

See also:  FCE, FCI, FV, FourVector, Pair.

\Subsubsection*{Examples}

\dispSFinmath{
\Muserfunction{Pair}[\Muserfunction{LorentzIndex}[\mu ],\Muserfunction{Momentum}[p]]-
   \Muserfunction{Pair}[\Muserfunction{LorentzIndex}[\mu ],\Muserfunction{Momentum}[q]]
}

\dispSFoutmath{
\mu
}

\dispSFinmath{
\mu
}

\dispSFoutmath{
\Muserfunction{FVD}[p,\mu ]
}

\dispSFinmath{
{p^{\mu }}
}

\dispSFoutmath{
\Muserfunction{FVD}[p-q,\mu ]
}

\dispSFinmath{
{{p-q}^{\mu }}
}

\dispSFoutmath{
\Muserfunction{FVD}[p,\mu ]//\Mfunction{StandardForm}
}

There is no special function to expand momenta in FVD.

\dispSFinmath{
\Muserfunction{FVD}[p,\mu ]
}

\dispSFoutmath{
\Muserfunction{FCI}[\Muserfunction{FVD}[p,\mu ]]//\Mfunction{StandardForm}
}

\dispSFinmath{
\Muserfunction{Pair}[\Muserfunction{LorentzIndex}[\mu ,D],\Muserfunction{Momentum}[p,D]]
}

\dispSFoutmath{
\Muserfunction{ExpandScalarProduct}[\Muserfunction{FVD}[p-q,\mu ]]
}

\Subsection*{GA}

\Subsubsection*{Description}

GA[\({p^{\mu }}-{q^{\mu }}\)] can be used as input for a 4-dimensional \(\Mfunction{StandardForm}[\%]\)and is transformed into DiracGamma[LorentzIndex[\(\Muserfunction{Pair}[\Muserfunction{LorentzIndex}[\mu
,D],\Muserfunction{Momentum}[p,D]]-
   \Muserfunction{Pair}[\Muserfunction{LorentzIndex}[\mu ,D],\Muserfunction{Momentum}[q,D]]\)]] by FeynCalcInternal (\(=\)FCI).GA[\(\mu \) ...] is
a short form for GA[\({{\gamma }_{\mu }}\multsp \)].GA[\(\mu \)]. ... .

See also:  DiracMatrix, GAD, GS.

\Subsubsection*{Examples}

\dispSFinmath{
\mu ,\nu ,
}

\dispSFoutmath{
\mu
}

\dispSFinmath{
\nu
}

\dispSFoutmath{
\Muserfunction{GA}[\mu ]
}

\dispSFinmath{
{{\gamma }^{\mu }}
}

\dispSFoutmath{
\Muserfunction{GA}[\mu ,\nu ]-\Muserfunction{GA}[\nu ,\mu ]
}

\dispSFinmath{
{{\gamma }^{\mu }}.{{\gamma }^{\nu }}-{{\gamma }^{\nu }}.{{\gamma }^{\mu }}
}

\dispSFoutmath{
\Mfunction{StandardForm}[\Muserfunction{FCI}[\Muserfunction{GA}[\mu ]]]
}

\dispSFinmath{
\Muserfunction{DiracGamma}[\Muserfunction{LorentzIndex}[\mu ]]
}

\dispSFoutmath{
\Muserfunction{GA}[\mu ,\nu ,\rho ,\sigma ]
}

\dispSFinmath{
{{\gamma }^{\mu }}.{{\gamma }^{\nu }}.{{\gamma }^{\rho }}.{{\gamma }^{\sigma }}
}

\dispSFoutmath{
\Mfunction{StandardForm}[\Muserfunction{GA}[\mu ,\nu ,\rho ,\sigma ]]
}

\Subsection*{GAD}

\Subsubsection*{Description}

GAD[\(\Muserfunction{GA}[\mu ].\Muserfunction{GA}[\nu ].\Muserfunction{GA}[\rho ].\Muserfunction{GA}[\sigma ]\)] can be used as input for a D-dimensional
\(\Muserfunction{GA}[\alpha ].(\Muserfunction{GS}[p]+m).\Muserfunction{GA}[\beta ]\)and is transformed into DiracGamma[LorentzIndex[\({{\gamma }^{\alpha
}}.(m+\gamma \cdot p).{{\gamma }^{\beta }}\),D],D] by FeynCalcInternal (\(=\)FCI).GAD[\(\mu \)] is a short form for GAD[\({{\gamma }_{\mu }}\multsp
\)].GAD[\(\mu \)]. ... .

See also:  DiracMatrix, GA, GS.

\Subsubsection*{Examples}

\dispSFinmath{
\mu ,\nu ,\multsp ...
}

\dispSFoutmath{
\mu
}

\dispSFinmath{
\nu
}

\dispSFoutmath{
\Muserfunction{GAD}[\mu ]
}

\dispSFinmath{
{{\gamma }^{\mu }}
}

\dispSFoutmath{
\Muserfunction{GAD}[\mu ,\nu ]-\Muserfunction{GAD}[\nu ,\mu ]
}

\dispSFinmath{
{{\gamma }^{\mu }}.{{\gamma }^{\nu }}-{{\gamma }^{\nu }}.{{\gamma }^{\mu }}
}

\dispSFoutmath{
\Mfunction{StandardForm}[\Muserfunction{FCI}[\Muserfunction{GAD}[\mu ]]]
}

\dispSFinmath{
\Muserfunction{DiracGamma}[\Muserfunction{LorentzIndex}[\mu ,D],D]
}

\dispSFoutmath{
\Muserfunction{GAD}[\mu ,\nu ,\rho ,\sigma ]
}

\dispSFinmath{
{{\gamma }^{\mu }}.{{\gamma }^{\nu }}.{{\gamma }^{\rho }}.{{\gamma }^{\sigma }}
}

\dispSFoutmath{
\Mfunction{StandardForm}[\Muserfunction{GAD}[\mu ,\nu ,\rho ,\sigma ]]
}

\Subsection*{GammaEpsilon}

\Subsubsection*{Description}

GammaEpsilon[exp] gives a series expansion of Gamma[exp] in Epsilon up to order 6 (where EulerGamma is neglected).

See also:  GammaExpand, Series2.

\Subsubsection*{Examples}

If the argument is of the form (1\(+\)a Epsilon) the result is not calculated but tabulated.

\dispSFinmath{
\Muserfunction{GAD}[\mu ].\Muserfunction{GAD}[\nu ].\Muserfunction{GAD}[\rho ].\Muserfunction{GAD}[\sigma ]
}

\dispSFoutmath{
\Muserfunction{GAD}[\alpha ].(\Muserfunction{GSD}[p]+m).\Muserfunction{GAD}[\beta ]
}

\dispSFinmath{
{{\gamma }^{\alpha }}.(m+\gamma \cdot p).{{\gamma }^{\beta }}
}

\dispSFoutmath{
\Muserfunction{GammaEpsilon}[1+a\multsp \Mvariable{Epsilon}]
}

For other arguments the expansion is calculated.

\dispSFinmath{
\Mvariable{C\$1075}\multsp {{\varepsilon }^6}+\Big(-\frac{1}{5}\multsp \zeta (5)\multsp {a^5}-
      \frac{1}{36}\multsp {{\pi }^2}\multsp \zeta (3)\multsp {a^5}\Big)\multsp {{\varepsilon }^5}+
   \frac{1}{160}\multsp {a^4}\multsp {{\pi }^4}\multsp {{\varepsilon }^4}-
   \frac{1}{3}\multsp {a^3}\multsp \zeta (3)\multsp {{\varepsilon }^3}+
   \frac{1}{12}\multsp {a^2}\multsp {{\pi }^2}\multsp {{\varepsilon }^2}+1
}

\dispSFoutmath{
\Muserfunction{GammaEpsilon}[1-\Mvariable{Epsilon}/2]
}

\dispSFinmath{
\Mvariable{C\$1076}\multsp {{\varepsilon }^6}+\bigg(\frac{{{\pi }^2}\multsp \zeta (3)}{1152}+\frac{\zeta (5)}{160}\bigg)\multsp
    {{\varepsilon }^5}+\frac{{{\pi }^4}\multsp {{\varepsilon }^4}}{2560}+\frac{1}{24}\multsp \zeta (3)\multsp {{\varepsilon }^3}+
   \frac{{{\pi }^2}\multsp {{\varepsilon }^2}}{48}+1
}

\dispSFoutmath{
\Muserfunction{GammaEpsilon}[\Mvariable{Epsilon}]
}

\Subsection*{GammaExpand}

\Subsubsection*{Description}

GammaExpand[exp] rewrites Gamma[n \(+\) m] in exp (where n has Head Integer).

See also:  GammaEpsilon.

\Subsubsection*{Examples}

\dispSFinmath{
\MathBegin{MathArray}{l}
\Mvariable{C\$1080}\multsp {{\varepsilon }^6}+
   \frac{\big(\frac{21}{4}\multsp {{\pi }^4}\multsp {{\psi }^{(2)}}(1)+{{\psi }^{(6)}}(1)-84\multsp {{\pi }^2}\multsp \zeta (5)\big)
      \multsp {{\varepsilon }^6}}{5040}+\frac{1}{720}\multsp
    \bigg(\frac{61\multsp {{\pi }^6}}{168}+10\multsp {{{{\psi }^{(2)}}(1)}^2}\bigg)\multsp {{\varepsilon }^5}+  \\
   \noalign{\vspace{1.5ex}}
\hspace{1.em} \frac{1}{120}\multsp
    \Big(\frac{5}{3}\multsp {{\pi }^2}\multsp {{\psi }^{(2)}}(1)-24\multsp \zeta (5)\Big)\multsp {{\varepsilon }^4}+
   \frac{{{\pi }^4}\multsp {{\varepsilon }^3}}{160}+\frac{1}{6}\multsp {{\psi }^{(2)}}(1)\multsp {{\varepsilon }^2}+
   \frac{{{\pi }^2}\multsp \varepsilon }{12}+\frac{1}{\varepsilon }\\
\MathEnd{MathArray}
}

\dispSFoutmath{
\Muserfunction{GammaEpsilon}[x]
}

\dispSFinmath{
\Gamma (x)
}

\dispSFoutmath{
\Muserfunction{GammaExpand}[\Mfunction{Gamma}[2\multsp +\multsp \Mvariable{Epsilon}]]
}

\dispSFinmath{
(\varepsilon +1)\multsp \Gamma (\varepsilon +1)
}

\dispSFoutmath{
\Muserfunction{GammaExpand}[\Mfunction{Gamma}[-3+\Mvariable{Epsilon}]]
}

\Subsection*{Gamma1}

\Subsubsection*{Description}

Gamma1[al,ga, be,de] is a special product of Gamma functions expanded up to order Epsilon\(\RawWedge\)2.

See also:  Gamma2, Gamma3.

\Subsection*{Gamma2}

\Subsubsection*{Description}

Gamma2[x,y] is a special product of Gamma functions expanded up to order Epsilon\(\RawWedge\)3 when positive integer arguments are given.

See also:  Gamma1, Gamma3.

\Subsection*{Gamma3}

\Subsubsection*{Description}

Gamma3[al,be,ga,ep] is a special product of Gamma functions expanded up to order Epsilon\(\RawWedge\)n when positive integer arguments
  are given (the order n is determined by the option EpsilonOrder).

See also:  Gamma1, Gamma2.

\Subsection*{Gauge}

\Subsubsection*{Description}

Gauge is an option for GluonProgagator. If set to 1 the 't Hooft Feynman gauge is used.

See also:  \${}Gauge, GluonProgagator.

\Subsection*{GaugeField}

\Subsubsection*{Description}

GaugeField is just a name. No functional properties are associated with it. GaugeField is used as default setting for the option
  QuantumField of FieldStrength.

See also:  FieldStrength, QuantumField.

\Subsubsection*{Examples}

\dispSFinmath{
\frac{\Gamma (\varepsilon +1)}{(\varepsilon -3)\multsp (\varepsilon -2)\multsp (\varepsilon -1)\multsp \varepsilon }
}

\dispSFoutmath{
\Muserfunction{GammaExpand}[\Mfunction{Gamma}[1\multsp +\multsp \Mvariable{Epsilon}]]
}

\dispSFinmath{
\Gamma (\varepsilon +1)
}

\dispSFoutmath{
\Mvariable{GaugeField}
}

\Subsection*{GaugeXi}

\Subsubsection*{Description}

GaugeXi is a head for gauge parameters.

See also:  Gauge, GaugeField.

\Subsection*{GA5}

\Subsubsection*{Description}

GA5 is equivalent to DiracGamma[5] and denotes gamma5.

See also:  DiracGamma, GA, GS.

\Subsubsection*{Examples}

\dispSFinmath{
A
}

\dispSFoutmath{
\Muserfunction{QuantumField}[\Mvariable{GaugeField},\Muserfunction{LorentzIndex}[\mu ],\Muserfunction{SUNIndex}[a]]
}

\dispSFinmath{
A_{\mu }^{a}
}

\dispSFoutmath{
\Mvariable{GA5}
}

\Subsection*{GGV}

\Subsubsection*{Description}

GGV is equivalent to GluonGhostVertex.

See also:  GluonGhostVertex.

\Subsection*{GhostPropagator}

\Subsubsection*{Description}

GhostPropagator[p, a, b] gives the ghost propagator where "a" and "b" are the color indices. GhostPropagator[p] omits the \({{\gamma }^5}\)

See also:  GHP, GluonPropagator, GluonGhostVertex.

\dispSFinmath{
\%//\Muserfunction{Tr}
}

\dispSFoutmath{
0
}

\Subsubsection*{Examples}

\dispSFinmath{
{{\delta }_{a\NoBreak b}}.
}

\dispSFoutmath{
\Mfunction{Options}[\Mvariable{GhostPropagator}]
}

\dispSFinmath{
\{\Mvariable{Explicit}\rightarrow \Mvariable{False}\}
}

\dispSFoutmath{
\Muserfunction{GhostPropagator}[p,a,b]
}

\dispSFinmath{
{{\Pi }_{ab}}(p)
}

\dispSFoutmath{
\Muserfunction{GhostPropagator}[p,a,b,\Mvariable{Explicit}\rightarrow \Mvariable{True}]
}

\dispSFinmath{
\frac{\ImaginaryI \multsp {{\delta }_{ab}}}{{p^2}}
}

\dispSFoutmath{
\Muserfunction{GhostPropagator}[p]
}

\dispSFinmath{
{{\Pi }_u}(p)
}

\dispSFoutmath{
\Muserfunction{GhostPropagator}[p,1,2]//\Muserfunction{Explicit}
}

\Subsection*{GHP}

\Subsubsection*{Description}

GHP[p, a, b] gives the ghost propagator where "a" and "b" are the color indices. GHP[p] omits the \(\frac{\ImaginaryI \multsp {{\delta }_{\Mvariable{ci1}\Mvariable{ci2}}}}{{p^2}}\)

See also:  GhostPropagator, GluonPropagator, GluonGhostVertex.

\Subsubsection*{Examples}

\dispSFinmath{
\Muserfunction{GhostPropagator}[-k,3,4]//\Muserfunction{Explicit}//\multsp \Mvariable{FCE}//\Mfunction{StandardForm}
}

\dispSFoutmath{
\ImaginaryI \multsp \Muserfunction{FAD}[-k]\multsp \Muserfunction{SD}[\Mvariable{ci3},\Mvariable{ci4}]
}

\dispSFinmath{
{{\delta }_{a\NoBreak b}}.
}

\dispSFoutmath{
\Muserfunction{GHP}[p,a,b]
}

\dispSFinmath{
{{\Pi }_{ab}}(p)
}

\dispSFoutmath{
\Muserfunction{GHP}[p]//\Muserfunction{Explicit}
}

\dispSFinmath{
\frac{\ImaginaryI }{{p^2}}
}

\dispSFoutmath{
\Muserfunction{GHP}[p,1,2]
}

\Subsection*{GluonField}

\Subsubsection*{Description}

GluonField is a name of a gauge field.

See also:  GaugeField.

\Subsubsection*{Examples}

\dispSFinmath{
{{\Pi }_{\Mvariable{ci1}\Mvariable{ci2}}}(p)
}

\dispSFoutmath{
\Mfunction{StandardForm}[\Muserfunction{FCE}[\Muserfunction{GHP}[-k,3,4]//\Muserfunction{Explicit}]]
}

\dispSFinmath{
\ImaginaryI \multsp \Muserfunction{FAD}[-k]\multsp \Muserfunction{SD}[\Mvariable{ci3},\Mvariable{ci4}]
}

\dispSFoutmath{
\Mvariable{GluonField}
}

\Subsection*{GluonGhostVertex}

\Subsubsection*{Description}

GluonGhostVertex[\{p,mu,a\}, \{q,nu,b\}, \{k,rho,c\}] or GluonGhostVertex[ p,mu,a , q,nu,b , k,rho,c ] yields the Gluon-Ghost vertex. The
  first argument represents the gluon and the third argument the outgoing ghost field (but incoming four-momentum). The dimension and the
  name of the coupling constant are determined by the options Dimension and CouplingConstant.

See also: GluonPropagator, GluonSelfEnergy, GhostPropagator, GluonVertex.

\dispSFinmath{
A
}

\dispSFoutmath{
\Muserfunction{QuantumField}[\Mvariable{GluonField},\Muserfunction{LorentzIndex}[\mu ],\Muserfunction{SUNIndex}[a]]
}

\Subsubsection*{Examples}

\dispSFinmath{
A_{\mu }^{a}
}

\dispSFoutmath{
\Mfunction{Options}[\Mvariable{GluonGhostVertex}]
}

\dispSFinmath{
\{\Mvariable{CouplingConstant}\rightarrow {g_s},\Mvariable{Dimension}\rightarrow D,\Mvariable{Explicit}\rightarrow \Mvariable{False}\}
}

\dispSFoutmath{
\Muserfunction{GluonGhostVertex}[\{p,\mu ,a\},\{q,\nu ,b\},\{k,\rho ,c\}]
}

\dispSFinmath{
{{\left( \overvar{\Lambda }{\RawTilde } \right) }^{\mu }}(k)\multsp {f_{abc}}
}

\dispSFoutmath{
\Muserfunction{Explicit}[\%]
}

\dispSFinmath{
-{g_s}\multsp {k^{\mu }}\multsp {f_{abc}}
}

\dispSFoutmath{
\Muserfunction{GluonGhostVertex}[k,\mu ]
}

\dispSFinmath{
{{\left( \overvar{\Lambda }{\RawTilde } \right) }^{\mu }}(k)
}

\dispSFoutmath{
\Muserfunction{Explicit}[\%]
}

\dispSFinmath{
-{g_s}\multsp {k^{\mu }}
}

\dispSFoutmath{
\%//\Muserfunction{Explicit}//\Mfunction{StandardForm}
}

\Subsection*{GluonPropagator}

\Subsubsection*{Description}

GluonPropagator[p, \{mu, a\}, \{nu, b\}] or GluonPropagator[p, mu, a , nu, b ] yields the gluon propagator. GluonPropagator[p, \{mu\},
  \{nu\}] or GluonPropagator[p, mu, nu] omits the SUNDelta. The gauge and the dimension is determined by the option Gauge and Dimension.
  The following settings of Gauge are possible: 1 for the Feynman gauge; \(-\Mvariable{Gstrong}\multsp \Muserfunction{Pair}[\Muserfunction{LorentzIndex}[\mu
,D],\Muserfunction{Momentum}[k,D]]\) for the general covariant gauge; \{Momentum[n] ,1\} for the axial gauge.

GP can be used as an abbreviation of GluonPropagator.

\dispSFinmath{
\Muserfunction{GluonGhostVertex}[p,1,q,2,k,3]
}

\dispSFoutmath{
\Muserfunction{GluonGhostVertex}(p,\Mvariable{li1},\Mvariable{ci1},q,\Mvariable{li2},\Mvariable{ci2},k,\Mvariable{li3},\Mvariable{ci3})
}

\dispSFinmath{
\alpha
}

\dispSFoutmath{
\Mvariable{GP}
}

See also:  GluonSelfEnergy, GluonVertex, GhostPropagator, GluonGhostVertex.

\Subsubsection*{\(\Mvariable{GluonPropagator}\)}

\dispSFinmath{
\Mfunction{Options}[\Mvariable{GluonPropagator}]
}

\dispSFoutmath{
\{\Mvariable{CounterTerm}\rightarrow \Mvariable{False},\Mvariable{CouplingConstant}\rightarrow {g_s},\Mvariable{Dimension}\rightarrow D,
    \Mvariable{Explicit}\rightarrow \Mvariable{False},\Mvariable{Gauge}\rightarrow 1,\Omega \rightarrow \Mvariable{False}\}
}

\dispSFinmath{
\Mvariable{Examples}
}

\dispSFoutmath{
\Muserfunction{GluonPropagator}[p,\mu ,a,\nu ,b]
}

\dispSFinmath{
\Pi _{ab}^{\mu \nu }(p)
}

\dispSFoutmath{
\Muserfunction{GluonPropagator}[p,\mu ,a,\nu ,b]//\Muserfunction{Explicit}
}

\dispSFinmath{
-\frac{\ImaginaryI \multsp {g^{\mu \nu }}\multsp {{\delta }_{ab}}}{{p^2}}
}

\dispSFoutmath{
\Muserfunction{GluonPropagator}[p,\mu ,a,\nu ,b,\Mvariable{Gauge}\rightarrow \alpha ]//\Muserfunction{Explicit}
}

\dispSFinmath{
-\ImaginaryI \multsp \alpha \multsp {p^{\mu }}\multsp {p^{\nu }}\multsp {{\delta }_{ab}}\multsp {{\bigg(\frac{1}{{p^2}}\bigg)}^2}+
   \ImaginaryI \multsp {p^{\mu }}\multsp {p^{\nu }}\multsp {{\delta }_{ab}}\multsp {{\bigg(\frac{1}{{p^2}}\bigg)}^2}-
   \frac{\ImaginaryI \multsp {g^{\mu \nu }}\multsp {{\delta }_{ab}}}{{p^2}}
}

\dispSFoutmath{
\Muserfunction{GluonPropagator}[p,\mu ,a,\nu ,b,\Mvariable{Gauge}\rightarrow \{\Muserfunction{Momentum}[n],1\}]//\Muserfunction{Explicit}
}

This is a convenient way to enter amplitudes by hand (GP is an abbreviation GluonPropagator).

\dispSFinmath{
-\frac{\ImaginaryI \multsp {g^{\mu \nu }}\multsp {{\delta }_{ab}}}{{p^2}}+
   \frac{\ImaginaryI \multsp {n^{\mu }}\multsp {n^{\nu }}\multsp {p^2}\multsp {{\delta }_{ab}}}{{p^2}\multsp {{n\cdot p}^2}}+
   \frac{\ImaginaryI \multsp {p^{\mu }}\multsp {n^{\nu }}\multsp {{\delta }_{ab}}}{{p^2}\multsp n\cdot p}+
   \frac{\ImaginaryI \multsp {n^{\mu }}\multsp {p^{\nu }}\multsp {{\delta }_{ab}}}{{p^2}\multsp n\cdot p}-
   \frac{\ImaginaryI \multsp {p^{\mu }}\multsp {p^{\nu }}\multsp {n^2}\multsp {{\delta }_{ab}}}{{p^2}\multsp {{n\cdot p}^2}}
}

\dispSFoutmath{
\Muserfunction{GluonPropagator}[p,\mu ,\nu ]//\Muserfunction{Explicit}
}

\dispSFinmath{
-\frac{\ImaginaryI \multsp {g^{\mu \nu }}}{{p^2}}
}

\dispSFoutmath{
\Muserfunction{Explicit}[\Muserfunction{GP}[p,1,2],\Mvariable{Gauge}\rightarrow \xi ]
}

\dispSFinmath{
-\ImaginaryI \multsp \xi \multsp {p^{\Mvariable{li1}}}\multsp {p^{\Mvariable{li2}}}\multsp {{\delta }_{\Mvariable{ci1}\Mvariable{ci2}}}
    \multsp {{\bigg(\frac{1}{{p^2}}\bigg)}^2}+\ImaginaryI \multsp {p^{\Mvariable{li1}}}\multsp {p^{\Mvariable{li2}}}\multsp
    {{\delta }_{\Mvariable{ci1}\Mvariable{ci2}}}\multsp {{\bigg(\frac{1}{{p^2}}\bigg)}^2}-
   \frac{\ImaginaryI \multsp {g^{\Mvariable{li1}\Mvariable{li2}}}\multsp {{\delta }_{\Mvariable{ci1}\Mvariable{ci2}}}}{{p^2}}
}

\dispSFoutmath{
\Muserfunction{GP}[-k,3,4]//\Muserfunction{Explicit}//\Muserfunction{FCE}//\Mfunction{StandardForm}
}

\dispSFinmath{
-\ImaginaryI \multsp \Muserfunction{FAD}[-k]\multsp \Muserfunction{MTD}[\Mvariable{li3},\Mvariable{li4}]\multsp
   \Muserfunction{SD}[\Mvariable{ci3},\Mvariable{ci4}]
}

\dispSFoutmath{
\Muserfunction{GluonPropagator}[p,\mu ,a,\nu ,b,\Mvariable{CounterTerm}\rightarrow \multsp 1]//\Muserfunction{Explicit}
}

\dispSFinmath{
\frac{19\multsp \ImaginaryI \multsp {C_A}\multsp g_{s}^{2}\multsp {S_n}\multsp {g^{\mu \nu }}\multsp {p^2}\multsp {{\delta }_{ab}}}
    {6\multsp \varepsilon }-\frac{11\multsp \ImaginaryI \multsp {C_A}\multsp g_{s}^{2}\multsp {S_n}\multsp {p^{\mu }}\multsp {p^{\nu }}
      \multsp {{\delta }_{ab}}}{3\multsp \varepsilon }
}

\dispSFoutmath{
\Muserfunction{GluonPropagator}[p,\mu ,a,\nu ,b,\Mvariable{CounterTerm}\rightarrow \multsp 2]//\Muserfunction{Explicit}
}

\dispSFinmath{
\frac{\ImaginaryI \multsp {C_A}\multsp {S_n}\multsp {p^{\mu }}\multsp {p^{\nu }}\multsp {{\delta }_{ab}}\multsp g_{s}^{2}}
    {3\multsp \varepsilon }+\frac{\ImaginaryI \multsp {C_A}\multsp {S_n}\multsp {g^{\mu \nu }}\multsp {p^2}\multsp {{\delta }_{ab}}
      \multsp g_{s}^{2}}{6\multsp \varepsilon }
}

\dispSFoutmath{
\Muserfunction{GluonPropagator}[p,\mu ,a,\nu ,b,\Mvariable{CounterTerm}\rightarrow \multsp 3]//\Muserfunction{Explicit}
}

\dispSFinmath{
\frac{8\multsp \ImaginaryI \multsp g_{s}^{2}\multsp {S_n}\multsp {T_f}\multsp {p^{\mu }}\multsp {p^{\nu }}\multsp {{\delta }_{ab}}}
    {3\multsp \varepsilon }-\frac{8\multsp \ImaginaryI \multsp g_{s}^{2}\multsp {S_n}\multsp {T_f}\multsp {g^{\mu \nu }}\multsp {p^2}
      \multsp {{\delta }_{ab}}}{3\multsp \varepsilon }
}

\dispSFoutmath{
\Muserfunction{GluonPropagator}[p,\mu ,a,\nu ,b,\Mvariable{CounterTerm}\rightarrow \multsp 4]//\Muserfunction{Explicit}
}

\Subsection*{GluonSelfEnergy}

\Subsubsection*{Description}

GluonSelfEnergy[\{mu, a\}, \{nu,b\}] yields the 1-loop Gluon selfenergy.

\dispSFinmath{
\frac{10\multsp \ImaginaryI \multsp {C_A}\multsp g_{s}^{2}\multsp {S_n}\multsp {g^{\mu \nu }}\multsp {p^2}\multsp {{\delta }_{ab}}}
    {3\multsp \varepsilon }-\frac{10\multsp \ImaginaryI \multsp {C_A}\multsp g_{s}^{2}\multsp {S_n}\multsp {p^{\mu }}\multsp {p^{\nu }}
      \multsp {{\delta }_{ab}}}{3\multsp \varepsilon }
}

\dispSFoutmath{
\Muserfunction{GluonPropagator}[p,\mu ,a,\nu ,b,\Mvariable{CounterTerm}\rightarrow \multsp 5]//\Muserfunction{Explicit}
}

See also:  GluonPropagator, GluonVertex, GhostPropagator, GluonGhostVertex.

\Subsubsection*{\(\frac{10\multsp \ImaginaryI \multsp {C_A}\multsp {S_n}\multsp {p^{\mu }}\multsp {p^{\nu }}\multsp {{\delta }_{ab}}\multsp g_{s}^{2}}
    {3\multsp \varepsilon }-\frac{4\multsp \ImaginaryI \multsp {S_n}\multsp {T_f}\multsp {p^{\mu }}\multsp {p^{\nu }}\multsp
      {{\delta }_{ab}}\multsp g_{s}^{2}}{3\multsp \varepsilon }-
   \frac{10\multsp \ImaginaryI \multsp {C_A}\multsp {S_n}\multsp {g^{\mu \nu }}\multsp {p^2}\multsp {{\delta }_{ab}}\multsp g_{s}^{2}}
    {3\multsp \varepsilon }+\frac{4\multsp \ImaginaryI \multsp {S_n}\multsp {T_f}\multsp {g^{\mu \nu }}\multsp {p^2}\multsp
      {{\delta }_{ab}}\multsp g_{s}^{2}}{3\multsp \varepsilon }\)}

\dispSFinmath{
\Mfunction{Options}[\Mvariable{GluonSelfEnergy}]
}

\dispSFoutmath{
\MathBegin{MathArray}{l}
\{\Mvariable{Dimension}\rightarrow D,\Mvariable{CouplingConstant}\rightarrow {g_s},  \\
\noalign{\vspace{
   0.604167ex}}
\hspace{1.em} \Mvariable{FinalSubstitutions}\rightarrow
     \{\log({{\mu }^2}\multsp \_)\RuleDelayed 0,\Gamma_E \RuleDelayed \log(4\multsp \pi )\},\Mvariable{Gauge}\rightarrow 1,
    \Mvariable{Momentum}\rightarrow p\}\\
\MathEnd{MathArray}
}

\Subsection*{GluonVertex}

\Subsubsection*{Description}

Note: All momenta are flowing into the vertex.

GluonVertex[\{p,mu,a\}, \{q,nu,b\}, \{k,la,c\}] or GluonVertex[p,mu,a , q,nu,b , k,la,c ] yields the 3-gluon vertex.
  GluonVertex[\{p,mu\}, \{q,nu\}, \{k,la\}] yields the 3-gluon vertex without color structure and the coupling constant.
  GluonVertex[\{p,mu,a\}, \{q,nu,b\}, \{k,la,c\}, \{s,si,d\}] or GluonVertex[\{mu,a\}, \{nu,b\}, \{la,c\}, \{si,d\}] or
  GluonVertex[p,mu,a , q,nu,b , k,la,c , s,si,d] or GluonVertex[ mu,a , nu,b , la,c , si,d ] yields the 4-gluon vertex. The dimension and
  the name of the coupling constant are determined by the options Dimension and CouplingConstant.

GV can be used as an abbreviation of GluonVertex.

\dispSFinmath{
\Mvariable{Examples}
}

\dispSFoutmath{
\Muserfunction{GluonSelfEnergy}[\{\mu ,a\},\{\nu ,b\}]
}

\dispSFinmath{
\frac{1}{2}\multsp \ImaginaryI \multsp {C_A}\multsp \Big(\frac{20}{3\multsp \varepsilon }-\frac{62}{9}\Big)\multsp
    ({p^{\mu }}\multsp {p^{\nu }}-{g^{\mu \nu }}\multsp {p^2})\multsp {{\delta }_{ab}}\multsp g_{s}^{2}+
   \ImaginaryI \multsp \Big(\frac{20}{9}-\frac{8}{3\multsp \varepsilon }\Big)\multsp {T_f}\multsp
    ({p^{\mu }}\multsp {p^{\nu }}-{g^{\mu \nu }}\multsp {p^2})\multsp {{\delta }_{ab}}\multsp g_{s}^{2}
}

\dispSFoutmath{
\Mvariable{GV}
}

See also:  GluonGhostVertex, GluonPropagator.

\Subsubsection*{Examples}

\dispSFinmath{
\Mvariable{GluonVertex}
}

\dispSFoutmath{
\Mfunction{Options}[\Mvariable{GluonVertex}]
}

\dispSFinmath{
\{\Mvariable{CouplingConstant}\rightarrow {g_s},\Mvariable{Dimension}\rightarrow D,\Mvariable{Explicit}\rightarrow \Mvariable{False},
    \Omega \rightarrow \Mvariable{False}\}
}

\dispSFoutmath{
\Muserfunction{GluonVertex}[\{p,\mu ,a\},\{q,\nu ,b\},\{r,\rho ,c\}]
}

\dispSFinmath{
{V^{\mu \nu \rho }}(p,\multsp q,\multsp r)\multsp {f_{abc}}
}

\dispSFoutmath{
\Muserfunction{GluonVertex}[\{p,\mu ,a\},\{q,\nu ,b\},\{r,\rho ,c\}]//\Muserfunction{Explicit}
}

\dispSFinmath{
({g_s}\multsp {q^{\mu }}\multsp {g^{\nu \rho }}-{g_s}\multsp {r^{\mu }}\multsp {g^{\nu \rho }}-
     {g_s}\multsp {g^{\mu \rho }}\multsp {p^{\nu }}+{g_s}\multsp {g^{\mu \rho }}\multsp {r^{\nu }}+
     {g_s}\multsp {g^{\mu \nu }}\multsp {p^{\rho }}-{g_s}\multsp {g^{\mu \nu }}\multsp {q^{\rho }})\multsp {f_{abc}}
}

\dispSFoutmath{
\Muserfunction{GluonVertex}[\{p,\mu \},\{q,\nu \},\{r,\rho \}]
}

\dispSFinmath{
{V^{\mu \nu \rho }}(p,\multsp q,\multsp r)
}

\dispSFoutmath{
\Muserfunction{GluonVertex}[\{p,\mu \},\{q,\nu \},\{r,\rho \}]//\Muserfunction{Explicit}
}

\dispSFinmath{
{g_s}\multsp {q^{\mu }}\multsp {g^{\nu \rho }}-{g_s}\multsp {r^{\mu }}\multsp {g^{\nu \rho }}-
   {g_s}\multsp {g^{\mu \rho }}\multsp {p^{\nu }}+{g_s}\multsp {g^{\mu \rho }}\multsp {r^{\nu }}+
   {g_s}\multsp {g^{\mu \nu }}\multsp {p^{\rho }}-{g_s}\multsp {g^{\mu \nu }}\multsp {q^{\rho }}
}

\dispSFoutmath{
\Muserfunction{GluonVertex}[\{p,\mu ,a\},\{q,\nu ,b\},\{r,\rho ,c\},\{s,\sigma ,d\}]
}

\dispSFinmath{
V_{abcd}^{\mu \nu \rho \sigma }(p,\multsp q,\multsp r,\multsp s)
}

\dispSFoutmath{
\Muserfunction{GluonVertex}[\{p,\mu ,a\},\{q,\nu ,b\},\{r,\rho ,c\},\{s,\sigma ,d\}]//\Muserfunction{Explicit}
}

\dispSFinmath{
\MathBegin{MathArray}{l}
\ImaginaryI \multsp {g^{\mu \rho }}\multsp {g^{\nu \sigma }}\multsp {f_{ad\Mvariable{u9}}}\multsp
    {f_{bc\Mvariable{u9}}}\multsp g_{s}^{2}-\ImaginaryI \multsp {g^{\mu \nu }}\multsp {g^{\rho \sigma }}\multsp {f_{ad\Mvariable{u9}}}
    \multsp {f_{bc\Mvariable{u9}}}\multsp g_{s}^{2}+\ImaginaryI \multsp {g^{\mu \sigma }}\multsp {g^{\nu \rho }}\multsp
    {f_{ac\Mvariable{u9}}}\multsp {f_{bd\Mvariable{u9}}}\multsp g_{s}^{2}-  \\
\noalign{\vspace{0.5625ex}}
\hspace{1.em} \ImaginaryI
    \multsp {g^{\mu \nu }}\multsp {g^{\rho \sigma }}\multsp {f_{ac\Mvariable{u9}}}\multsp {f_{bd\Mvariable{u9}}}\multsp g_{s}^{2}+
   \ImaginaryI \multsp {g^{\mu \sigma }}\multsp {g^{\nu \rho }}\multsp {f_{ab\Mvariable{u9}}}\multsp {f_{cd\Mvariable{u9}}}\multsp
    g_{s}^{2}-\ImaginaryI \multsp {g^{\mu \rho }}\multsp {g^{\nu \sigma }}\multsp {f_{ab\Mvariable{u9}}}\multsp {f_{cd\Mvariable{u9}}}
    \multsp g_{s}^{2}\\
\MathEnd{MathArray}
}

\dispSFoutmath{
\Muserfunction{GluonVertex}[\{\mu ,a\},\{\nu ,b\},\{\rho ,c\},\{\sigma ,d\}]
}

A very convenient way to enter diagrams by hand is to label each line hitting a vertex by a number and put this number after the
  inflowing momentum.

\dispSFinmath{
{W^{abcd}}
}

\dispSFoutmath{
\Muserfunction{GluonVertex}[\mu ,a,\nu ,b,\rho ,c,\sigma ,d]//\Muserfunction{Explicit}
}

\dispSFinmath{
\MathBegin{MathArray}{l}
\ImaginaryI \multsp {g^{\mu \rho }}\multsp {g^{\nu \sigma }}\multsp {f_{ad\Mvariable{u10}}}\multsp
    {f_{bc\Mvariable{u10}}}\multsp g_{s}^{2}-\ImaginaryI \multsp {g^{\mu \nu }}\multsp {g^{\rho \sigma }}\multsp {f_{ad\Mvariable{u10}}}
    \multsp {f_{bc\Mvariable{u10}}}\multsp g_{s}^{2}+\ImaginaryI \multsp {g^{\mu \sigma }}\multsp {g^{\nu \rho }}\multsp
    {f_{ac\Mvariable{u10}}}\multsp {f_{bd\Mvariable{u10}}}\multsp g_{s}^{2}-  \\
\noalign{\vspace{0.5625ex}}
\hspace{1.em} \ImaginaryI
    \multsp {g^{\mu \nu }}\multsp {g^{\rho \sigma }}\multsp {f_{ac\Mvariable{u10}}}\multsp {f_{bd\Mvariable{u10}}}\multsp g_{s}^{2}+
   \ImaginaryI \multsp {g^{\mu \sigma }}\multsp {g^{\nu \rho }}\multsp {f_{ab\Mvariable{u10}}}\multsp {f_{cd\Mvariable{u10}}}\multsp
    g_{s}^{2}-\ImaginaryI \multsp {g^{\mu \rho }}\multsp {g^{\nu \sigma }}\multsp {f_{ab\Mvariable{u10}}}\multsp {f_{cd\Mvariable{u10}}}
    \multsp g_{s}^{2}\\
\MathEnd{MathArray}
}

\dispSFoutmath{
\Muserfunction{GV}[p,1,q,2,r,3]
}

\Subsection*{GO}

\Subsubsection*{Description}

GO is equivalent to Twist2GluonOperator.

See also:  Twist2GluonOperator.

\Subsection*{GP}

\Subsubsection*{Description}

GP is equivalent to GluonPropagator.

See also:  GluonPropagator.

\Subsection*{GrassmannParity}

\Subsubsection*{Description}

GrassmannParity is a data type. E.g. DataType[F, GrassmannParity] \(=\) 1 declares F to be of bosonic type and DataType[F,
  GrassmannParity] \(=\) -1 of fermionic one.

See also:  DataType.

\Subsection*{GS}

\Subsubsection*{Description}

GS[p] can be used as input for a 4-dimensional \({V^{\Mvariable{li1}\Mvariable{li2}\Mvariable{li3}}}(p,\multsp q,\multsp r)\multsp {f_{\Mvariable{ci1}\Mvariable{ci2}\Mvariable{ci3}}}\)
and is transformed into DiracGamma[Momentum[p]] by FeynCalcInternal (\(=\)FCI). GS[p,q, ...] is a short form for GS[p].GS[q]. ... .

See also:  DiracGamma, DiracSlash, GA, GAD.

\Subsubsection*{Examples}

\dispSFinmath{
\Muserfunction{GV}[p,1,q,2,r,3,s,4]//\Muserfunction{Explicit}
}

\dispSFoutmath{
\MathBegin{MathArray}{l}
\ImaginaryI \multsp {g^{\Mvariable{li1}\Mvariable{li3}}}\multsp {g^{\Mvariable{li2}\Mvariable{li4}}}\multsp
    {f_{\Mvariable{ci1}\Mvariable{ci4}\Mvariable{u11}}}\multsp {f_{\Mvariable{ci2}\Mvariable{ci3}\Mvariable{u11}}}\multsp g_{s}^{2}-
   \ImaginaryI \multsp {g^{\Mvariable{li1}\Mvariable{li2}}}\multsp {g^{\Mvariable{li3}\Mvariable{li4}}}\multsp
    {f_{\Mvariable{ci1}\Mvariable{ci4}\Mvariable{u11}}}\multsp {f_{\Mvariable{ci2}\Mvariable{ci3}\Mvariable{u11}}}\multsp g_{s}^{2}+  \\
   \noalign{\vspace{0.729167ex}}
\hspace{1.em} \ImaginaryI \multsp {g^{\Mvariable{li1}\Mvariable{li4}}}\multsp
    {g^{\Mvariable{li2}\Mvariable{li3}}}\multsp {f_{\Mvariable{ci1}\Mvariable{ci3}\Mvariable{u11}}}\multsp
    {f_{\Mvariable{ci2}\Mvariable{ci4}\Mvariable{u11}}}\multsp g_{s}^{2}-
   \ImaginaryI \multsp {g^{\Mvariable{li1}\Mvariable{li2}}}\multsp {g^{\Mvariable{li3}\Mvariable{li4}}}\multsp
    {f_{\Mvariable{ci1}\Mvariable{ci3}\Mvariable{u11}}}\multsp {f_{\Mvariable{ci2}\Mvariable{ci4}\Mvariable{u11}}}\multsp g_{s}^{2}+  \\
   \noalign{\vspace{0.729167ex}}
\hspace{1.em} \ImaginaryI \multsp {g^{\Mvariable{li1}\Mvariable{li4}}}\multsp
    {g^{\Mvariable{li2}\Mvariable{li3}}}\multsp {f_{\Mvariable{ci1}\Mvariable{ci2}\Mvariable{u11}}}\multsp
    {f_{\Mvariable{ci3}\Mvariable{ci4}\Mvariable{u11}}}\multsp g_{s}^{2}-
   \ImaginaryI \multsp {g^{\Mvariable{li1}\Mvariable{li3}}}\multsp {g^{\Mvariable{li2}\Mvariable{li4}}}\multsp
    {f_{\Mvariable{ci1}\Mvariable{ci2}\Mvariable{u11}}}\multsp {f_{\Mvariable{ci3}\Mvariable{ci4}\Mvariable{u11}}}\multsp g_{s}^{2}\\
   \MathEnd{MathArray}
}

\dispSFinmath{
p\multsp     /\multsp (=\gamma .p\multsp =\multsp {{\gamma }_{\mu }}{p^{\mu }})
}

\dispSFoutmath{
\Muserfunction{GS}[p]
}

\dispSFinmath{
\gamma \cdot p
}

\dispSFoutmath{
\Muserfunction{GS}[p]//\Muserfunction{FCI}//\Mfunction{StandardForm}
}

\dispSFinmath{
\Muserfunction{DiracGamma}[\Muserfunction{Momentum}[p]]
}

\dispSFoutmath{
\Muserfunction{GS}[p,q,r,s]
}

\dispSFinmath{
(\gamma \cdot p).(\gamma \cdot q).(\gamma \cdot r).(\gamma \cdot s)
}

\dispSFoutmath{
\Muserfunction{GS}[p,q,r,s]//\Mfunction{StandardForm}
}

\Subsection*{GSD}

\Subsubsection*{Description}

GSD[p] can be used as input for a D-dimensional \(\Muserfunction{GS}[p].\Muserfunction{GS}[q].\Muserfunction{GS}[r].\Muserfunction{GS}[s]\) and is
transformed into DiracGamma[Momentum[p,D],D] by FeynCalcInternal (\(=\)FCI). GSD[p,q, ...] is a short form for GSD[p].GSD[q]. ...
  .

See also:  DiracGamma, DiracSlash, GA, GAD.

\Subsubsection*{Examples}

\dispSFinmath{
\Muserfunction{GS}[q].(\Muserfunction{GS}[p]+m).\Muserfunction{GS}[q]
}

\dispSFoutmath{
(\gamma \cdot q).(m+\gamma \cdot p).(\gamma \cdot q)
}

\dispSFinmath{
p\multsp     /\multsp (=\gamma .p\multsp =\multsp {{\gamma }_{\mu }}{p^{\mu }})
}

\dispSFoutmath{
\Muserfunction{GSD}[p]
}

\dispSFinmath{
\gamma \cdot p
}

\dispSFoutmath{
\Muserfunction{GSD}[p]//\Muserfunction{FCI}//\Mfunction{StandardForm}
}

\dispSFinmath{
\Muserfunction{DiracGamma}[\Muserfunction{Momentum}[p,D],D]
}

\dispSFoutmath{
\Muserfunction{GSD}[p,q,r,s]
}

\dispSFinmath{
(\gamma \cdot p).(\gamma \cdot q).(\gamma \cdot r).(\gamma \cdot s)
}

\dispSFoutmath{
\Muserfunction{GSD}[p,q,r,s]//\Mfunction{StandardForm}
}

\Subsection*{Gstrong}

\Subsubsection*{Description}

Gstrong denotes the strong coupling constant.

See also: CovariantD, FieldStrength, GluonVertex.

\Subsubsection*{Examples}

Gstrong has no functional properties. Only a typesetting rule is defined.

\dispSFinmath{
\Muserfunction{GSD}[p].\Muserfunction{GSD}[q].\Muserfunction{GSD}[r].\Muserfunction{GSD}[s]
}

\dispSFoutmath{
\Muserfunction{GSD}[q].(\Muserfunction{GSD}[p]+m).\Muserfunction{GSD}[q]
}

\Subsection*{GTI}

\Subsubsection*{Description}

GTI is like RHI, but with no functional properties.

See also:  RHI.

\Subsection*{GV}

\Subsubsection*{Description}

GV is equivalent to GluonVertex.

See also:  GluonVertex.

\Subsection*{Hill}

\Subsubsection*{Description}

Hill[x, y] gives the Hill identity with arguments x and y. The returned object is 0.

See also:  SimplifyPolyLog.

\Subsubsection*{Examples}

\dispSFinmath{
(\gamma \cdot q).(m+\gamma \cdot p).(\gamma \cdot q)
}

\dispSFoutmath{
\Mvariable{Gstrong}
}

\dispSFinmath{
{g_s}
}

\dispSFoutmath{
\Muserfunction{Hill}[a,b]
}

\dispSFinmath{
\MathBegin{MathArray}{l}
\log(a)\multsp (\log(1-a)-\log(1-b))+
   \log\Big(\frac{1-a}{1-b}\Big)\multsp \Big(-\log(a)+\log(1-b)-\log\Big(\frac{a-b}{a}\Big)+\log\Big(\frac{a-b}{1-b}\Big)\Big)-  \\
   \noalign{\vspace{1.33333ex}}
\hspace{1.em} \Big(\log(1-b)-\log\Big(\frac{a-b}{a}\Big)+\log\Big(\frac{a-b}{a\multsp (1-b)}\Big)\Big)
    \multsp \log\Big(\frac{(1-a)\multsp b}{a\multsp (1-b)}\Big)+  \\
\noalign{\vspace{1.40625ex}}
\hspace{1.em} {{\Mvariable{Li}}_2}(a)+
   {{\Mvariable{Li}}_2}\Big(\frac{1-a}{1-b}\Big)-{{\Mvariable{Li}}_2}(b)+{{\Mvariable{Li}}_2}\Big(\frac{b}{a}\Big)-
   {{\Mvariable{Li}}_2}\Big(\frac{(1-a)\multsp b}{a\multsp (1-b)}\Big)-\frac{{{\pi }^2}}{6}\\
\MathEnd{MathArray}
}

\dispSFoutmath{
\%\multsp /.\multsp a\RuleDelayed \multsp 0.123\multsp /.\multsp b\RuleDelayed \multsp 0.656\multsp //\Mfunction{\multsp }
   \Mvariable{Chop}
}

\dispSFinmath{
0
}

\dispSFoutmath{
\Muserfunction{Hill}[x,x\multsp y]//\Mfunction{PowerExpand}//\Muserfunction{SimplifyPolyLog}//\Mfunction{Expand}
}

\Subsection*{HypergeometricAC}

\Subsubsection*{Description}

HypergeometricAC[n][exp] analytically continues Hypergeometric2F1 functions in exp. The second argument n refers to the equation number
  (n) in chapter 2.10 of "Higher Transcendental Functions" by Ergelyi, Magnus, Oberhettinger, Tricomi. In case of eq. (6) (p.109) the
  last line is returned for HypergeometricAC[6][exp], while the first equality is given by HypergeometricAC[61][exp]. ((2.10.1) is
  identical to eq. (9.5.7) of "Special Functions \&{} their Applications" by N.N.Lebedev).

\dispSFinmath{
\zeta (2)-\log(1-y)\multsp \log(y)-\log(x)\multsp \log(1-x\multsp y)-{{\Mvariable{Li}}_2}(1-x)-{{\Mvariable{Li}}_2}(1-y)-
   {{\Mvariable{Li}}_2}(x\multsp y)+{{\Mvariable{Li}}_2}\bigg(\frac{1-x}{1-x\multsp y}\bigg)-
   {{\Mvariable{Li}}_2}\bigg(\frac{(1-x)\multsp y}{1-x\multsp y}\bigg)
}

\dispSFoutmath{
\%\multsp /.\multsp x\RuleDelayed \multsp 0.34/.\multsp y\rightarrow \multsp 0.6//\Mfunction{N}//\Mfunction{Chop}
}

See also:  HypExplicit, HypergeometricIR, HypergeometricSE, ToHypergeometric.

\Subsubsection*{Examples}

These are all transformation rules currently built in.

\dispSFinmath{
0
}

\dispSFoutmath{
\Mfunction{Options}[\Mvariable{HypergeometricAC}]
}

\dispSFinmath{
\{\Mvariable{Collect2}\rightarrow \Mvariable{True}\}
}

\dispSFoutmath{
\Muserfunction{HypergeometricAC}[1][\Mfunction{Hypergeometric2F1}[\alpha ,\beta ,\gamma ,z]]
}

\dispSFinmath{
\MathBegin{MathArray}{l}
\frac{\Gamma (\alpha +\beta -\gamma )\multsp \Gamma (\gamma )\multsp
      {{\InvisiblePrefixScriptBase }_2}{F_1}(\gamma -\alpha ,\gamma -\beta ;-\alpha -\beta +\gamma +1;1-z)\multsp
      {{(1-z)}^{-\alpha -\beta +\gamma }}}{\Gamma (\alpha )\multsp \Gamma (\beta )}+  \\
\noalign{\vspace{1.36458ex}}
\hspace{1.em}
    \frac{\Gamma (\gamma )\multsp \Gamma (-\alpha -\beta +\gamma )\multsp
     {{\InvisiblePrefixScriptBase }_2}{F_1}(\alpha ,\beta ;\alpha +\beta -\gamma +1;1-z)}{\Gamma (\gamma -\alpha )
     \multsp \Gamma (\gamma -\beta )}\\
\MathEnd{MathArray}
}

\dispSFoutmath{
\Muserfunction{HypergeometricAC}[2][\Mfunction{Hypergeometric2F1}[\alpha ,\beta ,\gamma ,z]]
}

\dispSFinmath{
\frac{\Gamma (\beta -\alpha )\multsp \Gamma (\gamma )\multsp
      {{\InvisiblePrefixScriptBase }_2}{F_1}\big(\alpha ,\alpha -\gamma +1;\alpha -\beta +1;\frac{1}{z}\big)\multsp
      {{(-z)}^{-\alpha }}}{\Gamma (\beta )\multsp \Gamma (\gamma -\alpha )}+
   \frac{\Gamma (\alpha -\beta )\multsp \Gamma (\gamma )\multsp
      {{\InvisiblePrefixScriptBase }_2}{F_1}\big(\beta ,\beta -\gamma +1;-\alpha +\beta +1;\frac{1}{z}\big)\multsp
      {{(-z)}^{-\beta }}}{\Gamma (\alpha )\multsp \Gamma (\gamma -\beta )}
}

\dispSFoutmath{
\Muserfunction{HypergeometricAC}[3][\Mfunction{Hypergeometric2F1}[\alpha ,\beta ,\gamma ,z]]
}

\dispSFinmath{
\frac{\Gamma (\beta -\alpha )\multsp \Gamma (\gamma )\multsp
      {{\InvisiblePrefixScriptBase }_2}{F_1}\big(\alpha ,\gamma -\beta ;\alpha -\beta +1;\frac{1}{1-z}\big)\multsp
      {{(1-z)}^{-\alpha }}}{\Gamma (\beta )\multsp \Gamma (\gamma -\alpha )}+
   \frac{\Gamma (\alpha -\beta )\multsp \Gamma (\gamma )\multsp
      {{\InvisiblePrefixScriptBase }_2}{F_1}\big(\beta ,\gamma -\alpha ;-\alpha +\beta +1;\frac{1}{1-z}\big)\multsp
      {{(1-z)}^{-\beta }}}{\Gamma (\alpha )\multsp \Gamma (\gamma -\beta )}
}

\dispSFoutmath{
\Muserfunction{HypergeometricAC}[4][\Mfunction{Hypergeometric2F1}[\alpha ,\beta ,\gamma ,z]]
}

\dispSFinmath{
\MathBegin{MathArray}{l}
\frac{\Gamma (\gamma )\multsp \Gamma (-\alpha -\beta +\gamma )\multsp
      {{\InvisiblePrefixScriptBase }_2}{F_1}
       \big(\alpha ,\alpha -\gamma +1;\alpha +\beta -\gamma +1;-\frac{1-z}{z}\big)\multsp {z^{-\alpha }}}{\Gamma (\gamma -\alpha )
      \multsp \Gamma (\gamma -\beta )}+  \\
\noalign{\vspace{1.65625ex}}
\hspace{1.em} \frac{{{(1-z)}^{-\alpha -\beta +\gamma }}\multsp
     \Gamma (\alpha +\beta -\gamma )\multsp \Gamma (\gamma )\multsp
     {{\InvisiblePrefixScriptBase }_2}{F_1}
      \big(\gamma -\alpha ,1-\alpha ;-\alpha -\beta +\gamma +1;-\frac{1-z}{z}\big)\multsp {z^{\alpha -\gamma }}}{\Gamma (\alpha )\multsp
     \Gamma (\beta )}\\
\MathEnd{MathArray}
}

\dispSFoutmath{
\Muserfunction{HypergeometricAC}[6][\Mfunction{Hypergeometric2F1}[\alpha ,\beta ,\gamma ,z]]
}

\Subsection*{HypergeometricIR}

\Subsubsection*{Description}

HypergeometricIR[exp, t] substitutes for all Hypergeometric2F1[a,b,c,z] in exp by its Euler integral reprentation. The factor
  Integratedx[t, 0, 1] can be omitted by setting the option Integratedx \(\rightarrow \) False.

See also:  HypergeometricAC, HypergeometricSE, ToHypergeometric.

\dispSFinmath{
{{(1-z)}^{-\beta }}\multsp {{\InvisiblePrefixScriptBase }_2}{F_1}
    \big(\beta ,\gamma -\alpha ;\gamma ;-\frac{z}{1-z}\big)
}

\dispSFoutmath{
\Muserfunction{HypergeometricAC}[61][\Mfunction{Hypergeometric2F1}[\alpha ,\beta ,\gamma ,z]]
}

\Subsubsection*{Examples}

\dispSFinmath{
{{(1-z)}^{-\alpha }}\multsp {{\InvisiblePrefixScriptBase }_2}{F_1}
    \big(\alpha ,\gamma -\beta ;\gamma ;-\frac{z}{1-z}\big)
}

\dispSFoutmath{
\Mfunction{Options}[\Mvariable{HypergeometricIR}]
}

\dispSFinmath{
\{\Mvariable{Integratedx}\rightarrow \Mvariable{False}\}
}

\dispSFoutmath{
\Muserfunction{HypergeometricIR}[\Mfunction{Hypergeometric2F1}[a,b,c,z],t]
}

\dispSFinmath{
\frac{{{(1-t)}^{-b+c-1}}\multsp {t^{b-1}}\multsp {{(1-t\multsp z)}^{-a}}\multsp \Gamma (c)}{\Gamma (b)\multsp \Gamma (c-b)}
}

\dispSFoutmath{
\Muserfunction{ToHypergeometric}[t\RawWedge b\multsp (1-t)\RawWedge c\multsp (1+t\multsp z)\RawWedge a,t]
}

\Subsection*{HypergeometricSE}

\Subsubsection*{Description}

HypergeometricSE[exp, \(\frac{\Gamma (b+1)\multsp \Gamma (c+1)\multsp {{\InvisiblePrefixScriptBase }_2}{F_1}(-a,b+1;b+c+2;-z)}
   {\Gamma (b+c+2)}\)] expresses Hypergeometric functions by their series expansion in terms of a sum (the Sum is omitted and \(\Muserfunction{HypergeometricIR}[\%,t]\),
running from 0 to \({{(1-t)}^c}\multsp {t^b}\multsp {{(t\multsp z+1)}^a}\), is the summation index).

\dispSFinmath{
\nu
}

\dispSFoutmath{
\nu
}

See also:  HypergeometricIR.

\Subsubsection*{Examples}

\dispSFinmath{
\infty
}

\dispSFoutmath{
\Mfunction{Options}[\Mvariable{HypergeometricSE}]
}

\dispSFinmath{
\{\Mvariable{Simplify}\rightarrow \Mvariable{FullSimplify}\}
}

\dispSFoutmath{
\Muserfunction{HypergeometricSE}[\Mfunction{Hypergeometric2F1}[a,b,c,z],\nu ]
}

\Subsection*{HypExplicit}

\Subsubsection*{Description}

HypExplicit[exp, \(\frac{{z^{\nu }}\multsp \Gamma (c)\multsp \Gamma (a+\nu )\multsp \Gamma (b+\nu )}
   {\Gamma (a)\multsp \Gamma (b)\multsp \Gamma (\nu +1)\multsp \Gamma (c+\nu )}\)] expresses Hypergeometric functions in exp by their definition
in terms of a sum (the Sum is omitted and \(\Muserfunction{HypergeometricSE}[\Mfunction{HypergeometricPFQ}[\{a,b,c\},\{d,e\},z],\nu ]\) is the summation
index).

See also:  HypergeometricIR.

\Subsubsection*{Examples}

\dispSFinmath{
\frac{{z^{\nu }}\multsp \Gamma (d)\multsp \Gamma (e)\multsp \Gamma (a+\nu )\multsp \Gamma (b+\nu )\multsp \Gamma (c+\nu )}
   {\Gamma (a)\multsp \Gamma (b)\multsp \Gamma (c)\multsp \Gamma (\nu +1)\multsp \Gamma (d+\nu )\multsp \Gamma (e+\nu )}
}

\dispSFoutmath{
\nu
}

\dispSFinmath{
\nu
}

\dispSFoutmath{
\Mfunction{Hypergeometric2F1}[a,b,c,z]
}

\dispSFinmath{
{{\InvisiblePrefixScriptBase }_2}{F_1}(a,b;c;z)
}

\dispSFoutmath{
\Muserfunction{HypExplicit}[\%,\nu ]
}

\dispSFinmath{
\frac{{z^{\nu }}\multsp \Gamma (c)\multsp \Gamma (a+\nu )\multsp \Gamma (b+\nu )}
   {\Gamma (a)\multsp \Gamma (b)\multsp \Gamma (\nu +1)\multsp \Gamma (c+\nu )}
}

\dispSFoutmath{
\Mfunction{HypergeometricPFQ}[\{a,b,c\},\{d,e\},z]
}

\Subsection*{HypInt}

\Subsubsection*{Description}

HypInt[exp, t] { }substitutes for all { }Hypergeometric2F1[a,b,c,z] in exp Gamma[c]/(Gamma[b] Gamma[c-b]) Integratedx[t,0,1] {
  }t\(\RawWedge\)(b-1) (1-t)\(\RawWedge\)(c-b-1) (1-t z)\(\RawWedge\)(-a).

\Subsubsection*{Examples}

\dispSFinmath{
{{\InvisiblePrefixScriptBase }_3}{F_2}(a,b,c;d,e;z)
}

\dispSFoutmath{
\Muserfunction{HypExplicit}[\%,\nu ]
}

\dispSFinmath{
\frac{{z^{\nu }}\multsp \Gamma (d)\multsp \Gamma (e)\multsp \Gamma (a+\nu )\multsp \Gamma (b+\nu )\multsp \Gamma (c+\nu )}
   {\Gamma (a)\multsp \Gamma (b)\multsp \Gamma (c)\multsp \Gamma (\nu +1)\multsp \Gamma (d+\nu )\multsp \Gamma (e+\nu )}
}

\dispSFoutmath{
\Mfunction{Hypergeometric2F1}[a,b,c,z]
}

\Subsection*{IFPD}

\Subsubsection*{Description}

IFPD[p, m] denotes (p\(\RawWedge\)2 - m\(\RawWedge\)2).

See also:  PropagatorDenominator.

\Subsection*{IFPDOff}

\Subsubsection*{Description}

IFPDOff[exp\_{},q1\_{}, q2\_{}, ...] changes from IFPD representation to FeynAmpDenominator[ ...]. The q1, q2, ... are the integration
  momenta.

See also:  IFPD, IFPDOn.

\Subsubsection*{Examples}

\dispSFinmath{
{{\InvisiblePrefixScriptBase }_2}{F_1}(a,b;c;z)
}

\dispSFoutmath{
\Muserfunction{HypInt}[\%,t]
}

\dispSFinmath{
\frac{{{(1-t)}^{-b+c-1}}\multsp {t^{b-1}}\multsp {{(1-t\multsp z)}^{-a}}\multsp \Gamma (c)\multsp
     \int _{0}^{1}\DifferentialD t\VeryThinSpace }{\Gamma (b)\multsp \Gamma (c-b)}
}

\dispSFoutmath{
\Muserfunction{IFPD}[\Muserfunction{Momentum}[p],m]
}

\dispSFinmath{
({p^2}\multsp -\multsp {m^2})
}

\dispSFoutmath{
\%//\Mfunction{StandardForm}
}

\dispSFinmath{
\Muserfunction{IFPD}[\Muserfunction{Momentum}[p],m]
}

\dispSFoutmath{
\Muserfunction{IFPDOff}[\%,p]
}

\Subsection*{IFPDOn}

\Subsubsection*{Description}

IFPDOn[exp\_{},q1\_{}, q2\_{}, ...] changes from FeynAmpDenominator[ ...] representation to the IFPD one (Inverse Feynman Propagator
  Denominator). I.e., FeynAmpDenominator[PropagatorDenominator[a,b]] is replaced by 1/IFPD[a,b] and The q1, q2, ... are the integration
  momenta.

See also:  IFPD, IFPDOff.

\Subsection*{IncludePair}

\Subsubsection*{Description}

IncludePair is an option for FC2RHI. Possible settings are True and False.

See also:  FC2RHI.

\Subsection*{IndexPosition}

\Subsubsection*{Description}

IndexPosition is an option for FieldStrength.

See also:  FieldStrength.

\Subsection*{Indices}

\Subsubsection*{Description}

Indices is an option for FORM2FeynCalc. Its default setting is Automatic. It may be set to a list, if the FORM-file does not contain a
  I(ndices) statement.

See also:  FORM2FeynCalc.

\Subsection*{InitialFunction}

\Subsubsection*{Description}

InitialFunction is an option of FeynRule the setting of which is applied to the first argument of FeynRule before anything else

See also:  FeynRule.

\Subsection*{InitialSubstitutions}

\Subsubsection*{Description}

InitialSubstitutions is an option for OneLoop and OneLoopSum and Write2. All substitutions indicated hereby are done at the end of the
  calculation.

See also:  OneLoop, OneLoopSum, Write2.

\Subsection*{InsideDiracTrace}

\Subsubsection*{Description}

InsideDiracTrace is and option of DiracSimplify. If set to True, DiracSimplify assumes to operate inside a DiracTrace, i.e., products of
  an odd number of Dirac matrices are discarded. Furthermore simple traces are calculated (but divided by a factor 4, i.e. : {
  }DiracSimplify[DiracMatrix[a,b], InsideDiracTrace\(\rightarrow \)True] { }yields ScalarProduct[a,b]) Traces involving more than four
  DiracGamma's and DiracGamma[5] are not performed.

See also:  DiracSimplify, DiracTrace.

\Subsection*{IntegralTable}

\Subsubsection*{Description}

IntegralTable is an option of OneLoopSimplify, TwoLoopSimplify and FeynAmpDenominatorSimplify. It may be set to a list of the form :
  \{FCIntegral[ ... ] \(\RuleDelayed \) bla, ...\}.

See also:  OneLoopSimplify, TwoLoopSimplify, FeynAmpDenominatorSimplify.

\Subsection*{IntegrateByParts}

\Subsubsection*{Description}

IntegrateByParts[(1-t)\(\RawWedge\)(a Epsilon -1) g[t], deriv, t] does an integration by parts of the definite integral over t from 0 to
  1.

\dispSFinmath{
{p^2}-{m^2}
}

\dispSFoutmath{
\%//\Mfunction{StandardForm}
}

\Subsubsection*{Examples}

\dispSFinmath{
-{m^2}+\Muserfunction{Pair}[\Muserfunction{Momentum}[p],\Muserfunction{Momentum}[p]]
}

\dispSFoutmath{
\Mfunction{Options}[\Mvariable{IntegrateByParts}]
}

\Subsection*{Integratedx}

\Subsubsection*{Description}

Integratedx[x, low, up] is a variable representing the integration operator Integrate[\#{}, \{x,low,up\}]\&{}.

See also:  FeynmanParametrize, CalculateCounterTerm, TimedIntegrate, HypergeometricIR, HypInt, TimedIntegrate.

\Subsection*{Integrate2}

\Subsubsection*{Description}

Integrate2 is like Integrate, but : Integrate2[a\_{}Plus, b\_{}\_{}] :\(=\) Map[Integrate2[\#{}, b]\&{}, a] ( more linear algebra and
  partial fraction decomposition is done) Integrate2[f[x] DeltaFunction[x], x] \(\rightarrow \) f[0] Integrate2[f[x] DeltaFunction[x0-x],
  x] \(\rightarrow \) f[x0]. Integrate2[f[x] DeltaFunction[a \(+\) b x], x] \(\rightarrow \) Integrate[f[x] (1/Abs[b]) DeltaFunction[a/b
  \(+\) x], x], where abs[b] \(\rightarrow \) b, if b is a Symbol, and if b \(=\) -c, then abs[-c] \(\rightarrow \) c, i.e., the variable
  contained in b is supposed to be positive. \(\{\Mvariable{Hold}\rightarrow \Mvariable{False}\}\) is replaced by 6 Zeta2. Integrate2[1/(1-y),\{y,x,1\}]
is intepreted as distribution, i.e. as Integrate2[-1/(1-y)],\{y, 0, x\}]
  \(\rightarrow \) Log[1-y]. Integrate2[1/(1-x),\{x,0,1\}] \(\rightarrow \) 0.

See also:  DeltaFunction, Integrate3, SumS, SumT.

NOTE: Since Integrate2 does do a reordering and partial fraction decomposition before calling the integral table of Integrate3 it will in
  general be slower compared to Integrate3 for sums of integrals. I.e., if the integrand has already an expanded form and if partial
  fraction decomposition is not necessary it is more effective to use Integrate3.

\Subsubsection*{Examples}

\dispSFinmath{
\Muserfunction{IntegrateByParts}[(1-t)\RawWedge (a\multsp \Mvariable{Epsilon}-1)\multsp g[t],
    (1-t)\RawWedge (a\multsp \Mvariable{Epsilon}-1),t]
}

\dispSFoutmath{
\frac{{g^{\prime }}(t)\multsp {{(1-t)}^{a\multsp \varepsilon }}}{a\multsp \varepsilon }+\frac{g(0)}{a\multsp \varepsilon }
}

Since Integrate2 uses table-look-up methods it is much faster than {\itshape Mathematica}'s Integrate.

\dispSFinmath{
\multsp {{\pi }^2}
}

\dispSFoutmath{
\Muserfunction{Integrate2}[\log [1+x]\log [x]/(1-x),\{x,0,1\}]//\Mfunction{Timing}
}

\dispSFinmath{
\big\{0.05\multsp \Mvariable{Second},\zeta (3)-\frac{3}{2}\multsp \zeta (2)\multsp \log(2)\big\}
}

\dispSFoutmath{
\Muserfunction{Integrate2}[\Mfunction{PolyLog}[2,x\RawWedge 2],\{x,0,1\}]
}

\dispSFinmath{
\zeta (2)+4\multsp \log(2)-4
}

\dispSFoutmath{
\Muserfunction{Integrate2}[\Mfunction{PolyLog}[3,-x],\{x,0,1\}]
}

\dispSFinmath{
\frac{\zeta (2)}{2}-\frac{3\multsp \zeta (3)}{4}-2\multsp \log(2)+1
}

\dispSFoutmath{
\Muserfunction{Integrate2}[\Mfunction{PolyLog}[3,1/(1+x)],\{x,0,1\}]
}

Integrate2 does integration in a Hadamard sense, i.e., \(-\log(2)\multsp \zeta (2)+\frac{3\multsp \zeta (3)}{4}+\frac{{{\log}^3}(2)}{3}-{{\log}^2}(2)+2\multsp
\log(2)\) means acutally expanding the result of \(\Muserfunction{Integrate2}[\Muserfunction{DeltaFunction}[1-x]\multsp f[x],\{x,0,1\}]\)up do \(f(1)\)
and neglecting all \(\int _{0}^{1}f(x)\DifferentialD x\multsp \)-dependent terms. E.g. \(\int _{\delta }^{1-\delta }f(x)\DifferentialD x\multsp \)

\dispSFinmath{
\Mfunction{O}(\delta )
}

\dispSFoutmath{
\delta
}

In the physics literature sometimes the "\(+\)" notation is used. In FeynCalc the \(\MathBegin{MathArray}{l}
\int _{\delta }^{1-\delta }1/(1-x)\DifferentialD x\multsp =\multsp   \\
\noalign{\vspace{0.927083ex}}
   \hspace{1.em} -\Mvariable{log}(1-x)\VerticalSeparator _{\delta }^{1-\delta }=-\log(\delta )+\log(1)\Rightarrow 0.\multsp \\
   \MathEnd{MathArray}\) is represented by \(\Muserfunction{Integrate2}[1/(1-x),\{x,0,1\}]\) or just \(0\)

\dispSFinmath{
{{\Big(\frac{1}{1-x}\Big)}_+}
}

\dispSFoutmath{
\Muserfunction{PlusDistribution}[1/(1-x)]
}

\dispSFinmath{
1/(1-x)\multsp .
}

\dispSFoutmath{
\Muserfunction{Integrate2}[\Muserfunction{PlusDistribution}[1/(1-x)],\{x,0,1\}]
}

\dispSFinmath{
0
}

\dispSFoutmath{
\Muserfunction{Integrate2}[\Mfunction{PolyLog}[2,1-x]/(1-x)\RawWedge 2,\{x,0,1\}]
}

\dispSFinmath{
2-\zeta (2)
}

\dispSFoutmath{
\Muserfunction{Integrate2}[(\log [x]\multsp \log [1+x])/(1+x),\{x,0,1\}]
}

\dispSFinmath{
-\frac{\zeta (3)}{8}
}

\dispSFoutmath{
\Muserfunction{Integrate2}[\log [x]\RawWedge 2/(1-x),\{x,0,1\}]
}

\dispSFinmath{
2\multsp \zeta (3)
}

\dispSFoutmath{
\Muserfunction{Integrate2}[\Mfunction{PolyLog}[2,-x]/(1+x),\{x,0,1\}]
}

\dispSFinmath{
\frac{\zeta (3)}{4}-\frac{1}{2}\multsp \zeta (2)\multsp \log(2)
}

\dispSFoutmath{
\Muserfunction{Integrate2}[\log [x]\multsp \Mfunction{PolyLog}[2,x],\{x,0,1\}]
}

\dispSFinmath{
3-2\multsp \zeta (2)
}

\dispSFoutmath{
\Muserfunction{Integrate2}[x\multsp \Mfunction{PolyLog}[3,x],\{x,0,1\}]
}

\dispSFinmath{
-\frac{\zeta (2)}{4}+\frac{\zeta (3)}{2}+\frac{3}{16}
}

\dispSFoutmath{
\Muserfunction{Integrate2}[(\log [x]\RawWedge 2\multsp \log [1-x])/(1+x),\{x,0,1\}]
}

\dispSFinmath{
{{\log}^2}(2)\multsp \zeta (2)-4\multsp {{\Mvariable{Li}}_4}\Big(\frac{1}{2}\Big)-\frac{{{\log}^4}(2)}{6}+\frac{{{\pi }^4}}{90}
}

\dispSFoutmath{
\Muserfunction{Integrate2}[\Mfunction{PolyLog}[2,((x\multsp (1-z)+z)\multsp (1-x+x\multsp z))/z]/(1-x+x\multsp z),\{x,0,1\}]
}

\dispSFinmath{
\MathBegin{MathArray}{l}
\frac{{{\log}^3}(z)}{6\multsp (1-z)}-\frac{\log(1-z)\multsp {{\log}^2}(z)}{1-z}-
   \frac{\log(z+1)\multsp {{\log}^2}(z)}{1-z}-\frac{\ImaginaryI \multsp \pi \multsp {{\log}^2}(z)}{2\multsp (1-z)}-
   \frac{\zeta (2)\multsp \log(z)}{1-z}+  \\
\noalign{\vspace{1.36458ex}}
\hspace{1.em} \frac{4\multsp \log(1-z)\multsp \log(z+1)\multsp
      \log(z)}{1-z}+\frac{2\multsp \ImaginaryI \multsp \pi \multsp \log(z+1)\multsp \log(z)}{1-z}-
   \frac{2\multsp {{\Mvariable{Li}}_2}(1-z)\multsp \log(z)}{1-z}-\frac{2\multsp {{\Mvariable{Li}}_2}(-z)\multsp \log(z)}{1-z}-  \\
   \noalign{\vspace{1.38542ex}}
\hspace{1.em} \frac{2\multsp {{\log}^3}(z+1)}{3\multsp (1-z)}+
   \frac{\ImaginaryI \multsp \pi \multsp \zeta (2)}{1-z}+\frac{2\multsp \zeta (2)\multsp \log(1-z)}{1-z}+
   \frac{2\multsp \zeta (2)\multsp \log(z+1)}{1-z}-\frac{2\multsp {S_{12}}(1-z)}{1-z}+
   \frac{4\multsp \log(1-z)\multsp {{\Mvariable{Li}}_2}(-z)}{1-z}+  \\
\noalign{\vspace{1.48958ex}}
\hspace{1.em} \frac{2\multsp
      \ImaginaryI \multsp \pi \multsp {{\Mvariable{Li}}_2}(-z)}{1-z}+\frac{4\multsp {{\Mvariable{Li}}_3}(1-z)}{1-z}+
   \frac{2\multsp {{\Mvariable{Li}}_3}(-z)}{1-z}+\frac{4\multsp {{\Mvariable{Li}}_3}\big(\frac{1}{z+1}\big)}{1-z}+
   \frac{4\multsp {{\Mvariable{Li}}_3}\big(-\frac{1-z}{z+1}\big)}{1-z}-\frac{4\multsp {{\Mvariable{Li}}_3}\big(\frac{1-z}{z+1}\big)}{1-z}
   -\frac{2\multsp \zeta (3)}{1-z}\\
\MathEnd{MathArray}
}

\dispSFoutmath{
\Mfunction{Apart}[\Muserfunction{Integrate2}[x\RawWedge (\Mvariable{OPEm}-1)\multsp \Mfunction{PolyLog}[3,1-x],\{x,0,1\}],
    \Mvariable{OPEm}]
}

\dispSFinmath{
-\frac{\zeta (2)}{m-1}-\frac{\zeta (2)}{{m^2}}+\frac{-{S_1}(m-2)\multsp \zeta (2)+\zeta (2)+{S_{12}}(m)+\zeta (3)}{m}
}

\dispSFoutmath{
\Muserfunction{Integrate2}\big[x\RawWedge (\Mvariable{OPEm}-1)\multsp \log [1-x]\multsp \log [x]\multsp \frac{\log [1+x]}{1+x},\{x,0,1\}
   \big]
}

\dispSFinmath{
\MathBegin{MathArray}{l}
\frac{1}{4}\multsp {{(-1)}^m}\multsp \zeta (2)\multsp S_{-1}^{2}(m-1)+
   {{(-1)}^m}\multsp \log(2)\multsp {S_{-2}}(m-1)\multsp {S_{-1}}(m-1)-{{(-1)}^m}\multsp \zeta (3)\multsp {S_{-1}}(m-1)+  \\
   \noalign{\vspace{1.19792ex}}
\hspace{1.em} \frac{3}{2}\multsp {{(-1)}^m}\multsp \zeta (2)\multsp \log(2)\multsp {S_{-1}}(m-1)-
   \frac{1}{2}\multsp {{(-1)}^m}\multsp \zeta (2)\multsp {S_{-2}}(m-1)+
   \frac{1}{2}\multsp {{(-1)}^m}\multsp {{\log}^2}(2)\multsp {S_{-2}}(m-1)-  \\
\noalign{\vspace{1.19792ex}}
\hspace{1.em} \frac{3}{2}
    \multsp {{(-1)}^m}\multsp \zeta (2)\multsp \log(2)\multsp {S_1}(m-1)+\frac{3}{4}\multsp {{(-1)}^m}\multsp \zeta (2)\multsp {S_2}(m-1)
   -\frac{1}{2}\multsp {{(-1)}^m}\multsp {{\log}^2}(2)\multsp {S_2}(m-1)+  \\
\noalign{\vspace{0.90625ex}}
\hspace{1.em} {{(-1)}^m}
    \multsp \log(2)\multsp {S_3}(m-1)-{{(-1)}^m}\multsp \log(2)\multsp {S_{-21}}(m-1)-{{(-1)}^m}\multsp \log(2)\multsp {S_{-12}}(m-1)-
   \\
\noalign{\vspace{0.666667ex}}
\hspace{1.em} {{(-1)}^m}\multsp \zeta (2)\multsp {S_{1-1}}(m-1)+{{(-1)}^m}\multsp {S_{-2-1-1}}(m-1)+
   {{(-1)}^m}\multsp {S_{-1-2-1}}(m-1)+{{(-1)}^m}\multsp {S_{-1-1-2}}(m-1)+  \\
\noalign{\vspace{0.958333ex}}
\hspace{1.em} {{(-1)}^m}
    \multsp {S_{1-21}}(m-1)+{{(-1)}^m}\multsp {S_{1-12}}(m-1)+{{(-1)}^m}\multsp {S_{2-11}}(m-1)+
   \frac{13}{8}\multsp {{(-1)}^m}\multsp {S_1}(m-1)\multsp \zeta (3)-  \\
\noalign{\vspace{1.19792ex}}
\hspace{1.em} \frac{21}{8}\multsp
    {{(-1)}^m}\multsp \log(2)\multsp \zeta (3)-2\multsp {{(-1)}^m}\multsp {{\Mvariable{Li}}_4}\Big(\frac{1}{2}\Big)-
   \frac{1}{12}\multsp {{(-1)}^m}\multsp {{\log}^4}(2)+\frac{5}{4}\multsp {{(-1)}^m}\multsp \zeta (2)\multsp {{\log}^2}(2)+
   \frac{1}{45}\multsp {{(-1)}^m}\multsp {{\pi }^4}\\
\MathEnd{MathArray}
}

\dispSFoutmath{
\%\multsp /.\multsp \Mvariable{OPEm}\rightarrow 2
}

\dispSFinmath{
\MathBegin{MathArray}{l}
\frac{5}{4}\multsp {{\log}^2}(2)\multsp \zeta (2)-3\multsp \log(2)\multsp \zeta (2)+\frac{5\multsp \zeta (2)}{2}
   -\frac{21}{8}\multsp \log(2)\multsp \zeta (3)+  \\
\noalign{\vspace{1.38542ex}}
\hspace{1.em} \frac{21\multsp \zeta (3)}{8}-
   2\multsp {{\Mvariable{Li}}_4}\Big(\frac{1}{2}\Big)-\frac{{{\log}^4}(2)}{12}-{{\log}^2}(2)+4\multsp \log(2)+\frac{{{\pi }^4}}{45}-6\\
   \MathEnd{MathArray}
}

\dispSFoutmath{
\Mfunction{N}[\%]
}

\dispSFinmath{
0.0505138
}

\dispSFoutmath{
\Mfunction{NIntegrate}\big[x\multsp \log [1-x]\multsp \log [x]\multsp \frac{\log [1+x]}{1+x},\{x,0,1\}\big]
}

\dispSFinmath{
0.0505138
}

\dispSFoutmath{
\Muserfunction{Integrate2}\big[x\RawWedge (\Mvariable{OPEm}-1)\multsp
     \Big(\Mfunction{PolyLog}\big[3,\frac{1-x}{1+x}\big]-\Mfunction{PolyLog}\big[3,-\frac{1-x}{1+x}\big]\Big),\{x,0,1\}\big]
}

\dispSFinmath{
\MathBegin{MathArray}{l}
\frac{{{(-1)}^m}\multsp {S_{-1}}(m)\multsp \zeta (2)}{m}-\frac{{S_{-1}}(m)\multsp \zeta (2)}{2\multsp m}+
   \frac{{{(-1)}^m}\multsp {S_1}(m)\multsp \zeta (2)}{2\multsp m}-\frac{{S_1}(m)\multsp \zeta (2)}{m}+
   \frac{3\multsp {{(-1)}^m}\multsp \log(2)\multsp \zeta (2)}{2\multsp m}-  \\
\noalign{\vspace{1.17708ex}}
\hspace{1.em} \frac{3\multsp
      \log(2)\multsp \zeta (2)}{2\multsp m}+\frac{{{(-1)}^m}\multsp {S_{-3}}(m)}{m}+
   \frac{{{(-1)}^m}\multsp {S_{-2}}(m)\multsp {S_1}(m)}{m}+\frac{{S_1}(m)\multsp {S_2}(m)}{m}+\frac{{S_3}(m)}{m}-  \\
   \noalign{\vspace{1.17708ex}}
\hspace{1.em} \frac{{{(-1)}^m}\multsp {S_{-21}}(m)}{m}-\frac{{S_{-1-2}}(m)}{m}-
   \frac{{{(-1)}^m}\multsp {S_{-12}}(m)}{m}-\frac{{S_{21}}(m)}{m}-\frac{7\multsp {{(-1)}^m}\multsp \zeta (3)}{8\multsp m}+
   \frac{21\multsp \zeta (3)}{8\multsp m}\\
\MathEnd{MathArray}
}

\dispSFoutmath{
\Muserfunction{DataType}[\Mvariable{OPEm},\Mvariable{PositiveInteger}]
}

This is the polarized non-singlet spin splitting function whose first moment vanishes.

\dispSFinmath{
\Mvariable{True}
}

\dispSFoutmath{
\Muserfunction{Integrate2}[x\RawWedge (\Mvariable{OPEm}-1)\multsp \Muserfunction{DeltaFunction}[1-x],\{x,0,1\}]
}

\dispSFinmath{
1
}

\dispSFoutmath{
t=\Muserfunction{SplittingFunction}[\Mvariable{PQQNS}]
}

Expanding t with respect to x yields a form already suitable for Integrate3 and therefore the following is faster:

\dispSFinmath{
\MathBegin{MathArray}{l}
\bigg(-4\multsp (x+1)\multsp {{\log}^2}(x)-8\multsp \Big(2\multsp x+\frac{3}{1-x}\Big)\multsp \log(x)-  \\
   \noalign{\vspace{1.33333ex}}
\hspace{4.em} \frac{16\multsp ({x^2}+1)\multsp \log(1-x)\multsp \log(x)}{1-x}-40\multsp (1-x)+
      \delta (1-x)\multsp (-24\multsp \zeta (2)+48\multsp \zeta (3)+3)\bigg)\multsp C_{F}^{2}+  \\
\noalign{\vspace{1.33333ex}}
   \hspace{1.em} {N_f}\multsp \bigg(\frac{88\multsp x}{9}+\Big(-\frac{16\multsp \zeta (2)}{3}-\frac{2}{3}\Big)\multsp \delta (1-x)-
      \frac{8\multsp ({x^2}+1)\multsp \log(x)}{3\multsp (1-x)}-\frac{80}{9}\multsp {{\Big(\frac{1}{1-x}\Big)}_+}-\frac{8}{9}\bigg)\multsp
     {C_F}-  \\
\noalign{\vspace{1.42708ex}}
\hspace{1.em} 8\multsp \Big({C_F}-\frac{{C_A}}{2}\Big)\multsp
    \bigg(4\multsp (1-x)+2\multsp (x+1)\multsp \log(x)+
      \frac{({x^2}+1)\multsp \big({{\log}^2}(x)-4\multsp \log(x+1)\multsp \log(x)-2\multsp \zeta (2)-4\multsp {{\Mvariable{Li}}_2}(-x)
          \big)}{x+1}\bigg)\multsp {C_F}+  \\
\noalign{\vspace{1.64583ex}}
\hspace{1.em} {C_A}\multsp
   \bigg(\frac{4\multsp ({x^2}+1)\multsp {{\log}^2}(x)}{1-x}-\frac{4}{3}\multsp \Big(5\multsp x-\frac{22}{1-x}+5\Big)\multsp \log(x)+
     \frac{4}{9}\multsp (53-187\multsp x)+  \\
\noalign{\vspace{1.5625ex}}
\hspace{4.em} 8\multsp (x+1)\multsp \zeta (2)+
     \Big(\frac{536}{9}-16\multsp \zeta (2)\Big)\multsp {{\Big(\frac{1}{1-x}\Big)}_+}+
     \delta (1-x)\multsp \Big(\frac{88\multsp \zeta (2)}{3}-24\multsp \zeta (3)+\frac{17}{3}\Big)\bigg)\multsp {C_F}\\
   \MathEnd{MathArray}
}

\dispSFoutmath{
\Muserfunction{Integrate2}[t,\{x,0,1\}]//\Mfunction{Timing}
}

\dispSFinmath{
\{0.52\multsp \Mvariable{Second},0\}
}

\dispSFinmath{
\Muserfunction{Integrate3}[\Mfunction{Expand}[t,x],\{x,0,1\}]//\Mfunction{Expand}//\Mfunction{Timing}
}

\dispSFoutmath{
\{0.11\multsp \Mvariable{Second},0\}
}

\dispSFinmath{
\Mfunction{Clear}[t];
}

\dispSFoutmath{
\Muserfunction{Integrate2}[\Muserfunction{DeltaFunction}[1-x]\multsp f[x],\{x,0,1\}]
}

\dispSFinmath{
f(1)
}

\dispSFoutmath{
\Muserfunction{Integrate2}\big[{x^5}\log [1+x]\RawWedge 2,\{x,0,1\}\big]
}

\dispSFinmath{
-\frac{6959}{10800}+\frac{46\multsp \log(2)}{45}
}

\dispSFoutmath{
N@\%
}

\dispSFinmath{
0.0641986
}

\dispSFoutmath{
\Mfunction{NIntegrate}\big[{x^5}\log [1+x]\RawWedge 2,\{x,0,1\}\big]
}

\Subsection*{Integrate3}

\Subsubsection*{Description}

Integrate3 contains the integral table used by Integrate2. Integration is performed in a distributional sense. Integrate3 works more
  effectively on a sum of expressions if they are expanded or collected with respect to the integration variable. See the examples in
  Integrate2.

See also:  Integrate2.

\Subsubsection*{Examples}

\dispSFinmath{
0.0641986
}

\dispSFoutmath{
\Muserfunction{Integrate2}\big[x\RawWedge (\Mvariable{OPEm}-1){{\log [1+x]}^2},\{x,0,1\}\big]
}

\dispSFinmath{
\MathBegin{MathArray}{l}
-\frac{2\multsp {{(-1)}^m}\multsp S_{1}^{2}(m)}{m}+
   \frac{{{(-1)}^m}\multsp {S_1}\big(\frac{m-1}{2}\big)\multsp {S_1}(m)}{m}-\frac{{S_1}\big(\frac{m-1}{2}\big)\multsp {S_1}(m)}{m}+  \\
   \noalign{\vspace{1.48958ex}}
\hspace{1.em} \frac{{{(-1)}^m}\multsp {S_1}\big(\frac{m}{2}\big)\multsp {S_1}(m)}{m}+
   \frac{{S_1}\big(\frac{m}{2}\big)\multsp {S_1}(m)}{m}+\frac{4\multsp {{(-1)}^m}\multsp \log(2)\multsp {S_1}(m)}{m}-
   \frac{{{(-1)}^m}\multsp \log(2)\multsp {S_1}\big(\frac{m-1}{2}\big)}{m}+  \\
\noalign{\vspace{1.48958ex}}
\hspace{1.em} \frac{\log(2)
      \multsp {S_1}\big(\frac{m-1}{2}\big)}{m}-\frac{{{(-1)}^m}\multsp \log(2)\multsp {S_1}\big(\frac{m}{2}\big)}{m}-
   \frac{\log(2)\multsp {S_1}\big(\frac{m}{2}\big)}{m}+\frac{{{(-1)}^m}\multsp {S_2}\big(\frac{m-1}{2}\big)}{2\multsp m}-
   \frac{{S_2}\big(\frac{m-1}{2}\big)}{2\multsp m}+  \\
\noalign{\vspace{1.51042ex}}
\hspace{1.em} \frac{{{(-1)}^m}\multsp
      {S_2}\big(\frac{m}{2}\big)}{2\multsp m}+\frac{{S_2}\big(\frac{m}{2}\big)}{2\multsp m}-\frac{2\multsp {{(-1)}^m}\multsp {S_2}(m)}{m}
   -\frac{2\multsp {{(-1)}^m}\multsp {S_{-11}}(m)}{m}-\frac{{{(-1)}^m}\multsp {{\log}^2}(2)}{m}+\frac{{{\log}^2}(2)}{m}\\
   \MathEnd{MathArray}
}

\dispSFoutmath{
\Muserfunction{Integrate3}[{x^{\Mvariable{OPEm}}}\multsp \log [x],\{x,0,1\}]
}

\dispSFinmath{
-\frac{1}{{{(m+1)}^2}}
}

\dispSFoutmath{
\Muserfunction{Integrate3}\big[\frac{{x^{\Mvariable{OPEm}}}\multsp \log [x]\multsp \log [1-x]}{1-x},\{x,0,1\}\big]
}

\dispSFinmath{
\zeta (2)\multsp {S_1}(m)-{S_{12}}(m)-{S_{21}}(m)+\zeta (3)
}

\dispSFoutmath{
\Muserfunction{Integrate3}\big[a\frac{{x^{\Mvariable{OPEm}}}\multsp \log [x]\multsp \log [1-x]}{1-x}+
     b\frac{{x^{\Mvariable{OPEm}}}\Mfunction{PolyLog}[3,-x]}{1+x},\{x,0,1\}\big]
}

\dispSFinmath{
\MathBegin{MathArray}{l}
a\multsp (\zeta (2)\multsp {S_1}(m)-{S_{12}}(m)-{S_{21}}(m)+\zeta (3))+  \\
\noalign{\vspace{1.14583ex}}
   \hspace{1.em} {{(-1)}^m}\multsp b\multsp \bigg(\frac{{{\zeta (2)}^2}}{8}+\frac{1}{2}\multsp {S_{-2}}(m)\multsp \zeta (2)+
     \log(2)\multsp ({S_3}(m)-{S_{-3}}(m))+{S_{3-1}}(m)-\frac{3}{4}\multsp {S_{-1}}(m)\multsp \zeta (3)-
     \frac{3}{4}\multsp \log(2)\multsp \zeta (3)\bigg)\\
\MathEnd{MathArray}
}

\dispSFoutmath{
\Muserfunction{Integrate3}[\Muserfunction{DeltaFunctionPrime}[1-x],\{x,0,1\}]
}

\dispSFinmath{
0
}

\dispSFoutmath{
\Muserfunction{Integrate3}[f[x]\multsp \Muserfunction{DeltaFunctionPrime}[1-x],\{x,0,1\}]
}

\Subsection*{IntermediateSubstitutions}

\Subsubsection*{Description}

IntermediateSubstitutions is an option for OneLoop. All substitutions indicated hereby are done at an intermediate stage of the
  calculation.

See also:  OneLoop.

\Subsection*{InverseMellin}

\Subsubsection*{Description}

InverseMellin[exp, y] performs the inverse Mellin transform of polynomials in OPEm. The inverse transforms are not calculated but a
  table-lookup is done. WARNING: do not "trust" the results for the inverse Mellin transform involving SumT's; there is an unresolved
  inconsistency here (related to \({f^{\prime }}(1)\)

See also:  DeltaFunction, Integrate2, OPEm, SumS, SumT.

\Subsubsection*{Examples}

\dispSFinmath{
\Muserfunction{Integrate3}[1/(1-x),\{x,0,1\}]
}

\dispSFoutmath{
0
}

\dispSFinmath{
(-1)\RawWedge m).
}

\dispSFoutmath{
\Muserfunction{InverseMellin}[1/\Mvariable{OPEm},y]
}

\dispSFinmath{
{y^{m-1}}
}

\dispSFoutmath{
\Muserfunction{InverseMellin}[1/(\Mvariable{OPEm}+3),y]
}

\dispSFinmath{
{y^{m+2}}
}

\dispSFoutmath{
\Muserfunction{InverseMellin}[1,y]
}

\dispSFinmath{
{y^{m-1}}\multsp \delta (1-y)
}

\dispSFoutmath{
\Muserfunction{InverseMellin}[1/\Mvariable{OPEm}\RawWedge 4,y]
}

\dispSFinmath{
-\frac{1}{6}\multsp {y^{m-1}}\multsp {{\log}^3}(y)
}

\dispSFoutmath{
\Muserfunction{InverseMellin}[1/\Mvariable{OPEm}+1,y]
}

The inverse operation to InverseMellin is done by Integrate2.

\dispSFinmath{
\delta (1-y)\multsp {y^{m-1}}+{y^{m-1}}
}

\dispSFoutmath{
\Muserfunction{InverseMellin}[1/i+1,y,i]
}

Below is a list of all built-in basic inverse Mellin transforms .

\dispSFinmath{
\delta (1-y)\multsp {y^{i-1}}+{y^{i-1}}
}

\dispSFinmath{
\Muserfunction{Integrate2}[\Muserfunction{InverseMellin}[1/\Mvariable{OPEm},y],\{y,0,1\}]
}

\dispSFinmath{
\frac{1}{m}
}

\dispSFoutmath{
\MathBegin{MathArray}{l}
\Mvariable{list}=\big\{1,\frac{1}{\Mvariable{OPEm}+n},\frac{1}{-\Mvariable{OPEm}+n},
     \Mfunction{PolyGamma}[0,\Mvariable{OPEm}],\Muserfunction{SumS}[1,-1+\Mvariable{OPEm}],  \\
\noalign{\vspace{1.19792ex}}
   \hspace{2.em} \frac{\Muserfunction{SumS}[1,-1+\Mvariable{OPEm}]}{\Mvariable{OPEm}-1},
   \frac{\Muserfunction{SumS}[1,-1+\Mvariable{OPEm}]}{1-\Mvariable{OPEm}},
   \frac{\Muserfunction{SumS}[1,-1+\Mvariable{OPEm}]}{\Mvariable{OPEm}+1},  \\
\noalign{\vspace{1.21875ex}}
\hspace{2.em} \frac{
       \Muserfunction{SumS}[1,-1+\Mvariable{OPEm}]}{{{\Mvariable{OPEm}}^2}},
   \frac{\Muserfunction{SumS}[1,-1+\Mvariable{OPEm}]}{\Mvariable{OPEm}},
   \frac{{{\Muserfunction{SumS}[1,-1+\Mvariable{OPEm}]}^2}}{\Mvariable{OPEm}},\Muserfunction{SumS}[2,-1+\Mvariable{OPEm}],  \\
   \noalign{\vspace{1.3125ex}}
\hspace{2.em} \frac{\Muserfunction{SumS}[2,-1+\Mvariable{OPEm}]}{\Mvariable{OPEm}},
   \Muserfunction{SumS}[3,-1+\Mvariable{OPEm}],\Muserfunction{SumS}[1,1,-1+\Mvariable{OPEm}],
   {{\Muserfunction{SumS}[1,\Mvariable{OPEm}-1]}^2},  \\
\noalign{\vspace{0.927083ex}}
\hspace{2.em} \Muserfunction{SumS}[
    1,2,-1+\Mvariable{OPEm}],\Muserfunction{SumS}[2,1,-1+\Mvariable{OPEm}],{{\Muserfunction{SumS}[1,-1+\Mvariable{OPEm}]}^3},  \\
   \noalign{\vspace{0.666667ex}}
\hspace{2.em} \Muserfunction{SumS}[1,-1+\Mvariable{OPEm}]\multsp
       \Muserfunction{SumS}[2,-1+\Mvariable{OPEm}],\Muserfunction{SumS}[1,1,1,-1+\Mvariable{OPEm}]\big\};\\
\MathEnd{MathArray}
}

\dispSFinmath{
\Muserfunction{im}[\Mvariable{z\_}]:=z\multsp \longrightarrow \Muserfunction{InverseMellin}[z,y]
}

\dispSFoutmath{
\Muserfunction{im}[\Mvariable{OPEm}\RawWedge (-3)]
}

\dispSFinmath{
\frac{1}{{m^3}}\longrightarrow \Big(\frac{1}{2}\multsp {y^{m-1}}\multsp {{\log}^2}(y)\Big)
}

\dispSFoutmath{
\Muserfunction{im}[\Mvariable{OPEm}\RawWedge (-2)]
}

\dispSFinmath{
\frac{1}{{m^2}}\longrightarrow \big(-{y^{m-1}}\multsp \log(y)\big)
}

\dispSFinmath{
\Muserfunction{im}[\Mfunction{PolyGamma}[0,\Mvariable{OPEm}]]
}

\dispSFoutmath{
{{\psi }^{(0)}}(m)\longrightarrow \bigg(-\Gamma_E \multsp \delta (1-y)\multsp {y^{m-1}}-
     {{\bigg(\frac{1}{1-y}\bigg)}_+}\multsp {y^{m-1}}\bigg)
}

\dispSFinmath{
\multsp
}

\dispSFoutmath{
\Muserfunction{im}[\Muserfunction{SumS}[1,\Mvariable{OPEm}-1]]
}

\dispSFinmath{
{S_1}(m-1)\longrightarrow \bigg(-{y^{m-1}}\multsp {{\bigg(\frac{1}{1-y}\bigg)}_+}\bigg)
}

\dispSFoutmath{
\Muserfunction{im}[\Muserfunction{SumS}[1,\Mvariable{OPEm}-1]/(\Mvariable{OPEm}-1)]
}

\dispSFinmath{
\frac{{S_1}(m-1)}{m-1}\longrightarrow (-{y^{m-2}}\multsp \log(1-y))
}

\dispSFoutmath{
\Muserfunction{im}[\Muserfunction{SumS}[1,\Mvariable{OPEm}-1]/(\Mvariable{OPEm}+1)]
}

\dispSFinmath{
\frac{{S_1}(m-1)}{m+1}\longrightarrow \big(-{y^{m-1}}-\log(1-y)\multsp {y^m}+\log(y)\multsp {y^m}+{y^m}\big)
}

\dispSFoutmath{
\Muserfunction{im}[\Muserfunction{SumS}[1,-1+\Mvariable{OPEm}]/\Mvariable{OPEm}\RawWedge 2]
}

\dispSFinmath{
\frac{{S_1}(m-1)}{{m^2}}\longrightarrow \Big({y^{m-1}}\multsp
     \Big(-\frac{1}{2}\multsp {{\log}^2}(y)+\zeta (2)-{{\Mvariable{Li}}_2}(y)\Big)\Big)
}

\dispSFoutmath{
\Muserfunction{im}[\Muserfunction{SumS}[1,-1+\Mvariable{OPEm}]/\Mvariable{OPEm}]
}

\dispSFinmath{
\frac{{S_1}(m-1)}{m}\longrightarrow \big({y^{m-1}}\multsp (\log(y)-\log(1-y))\big)
}

\dispSFoutmath{
\Muserfunction{im}[\Muserfunction{SumS}[1,-1+\Mvariable{OPEm}]\RawWedge 2/\Mvariable{OPEm}]
}

\dispSFinmath{
\frac{S_{1}^{2}(m-1)}{m}\longrightarrow \bigg({y^{m-1}}\multsp
     \bigg({{\log}^2}(1-y)+\frac{{{\log}^2}(y)}{2}-3\multsp \zeta (2)+{{\Mvariable{Li}}_2}(1-y)+2\multsp {{\Mvariable{Li}}_2}(y)\bigg)
      \bigg)
}

\dispSFoutmath{
\Muserfunction{im}[\Muserfunction{SumS}[2,\Mvariable{OPEm}-1]]
}

\dispSFinmath{
{S_2}(m-1)\longrightarrow \bigg({y^{m-1}}\multsp \bigg(\zeta (2)\multsp \delta (1-y)+\frac{\log(y)}{1-y}\bigg)\bigg)
}

\dispSFoutmath{
\Muserfunction{im}[\Muserfunction{SumS}[2,\Mvariable{OPEm}-1]/\Mvariable{OPEm}]
}

\dispSFinmath{
\frac{{S_2}(m-1)}{m}\longrightarrow \Big({y^{m-1}}\multsp \Big(-\frac{1}{2}\multsp {{\log}^2}(y)+\zeta (2)-{{\Mvariable{Li}}_2}(1-y)\Big)
    \Big)
}

\dispSFoutmath{
\Muserfunction{im}[\Muserfunction{SumS}[3,\Mvariable{OPEm}-1]]
}

\dispSFinmath{
{S_3}(m-1)\longrightarrow \bigg({y^{m-1}}\multsp \bigg(\delta (1-y)\multsp \zeta (3)-\frac{{{\log}^2}(y)}{2\multsp (1-y)}\bigg)\bigg)
}

\dispSFoutmath{
\Muserfunction{im}[\Muserfunction{SumS}[1,1,\Mvariable{OPEm}-1]]
}

\dispSFinmath{
{S_{11}}(m-1)\longrightarrow \bigg({y^{m-1}}\multsp {{\bigg(\frac{\log(1-y)}{1-y}\bigg)}_+}\bigg)
}

\dispSFoutmath{
\multsp \Muserfunction{im}[\Muserfunction{SumS}[1,2,\Mvariable{OPEm}-1]]
}

\dispSFinmath{
{S_{12}}(m-1)\longrightarrow \bigg({y^{m-1}}\multsp \bigg(
      -\zeta (3)\multsp \delta (1-y)-\zeta (2)\multsp {{\bigg(\frac{1}{1-y}\bigg)}_+}+\frac{{{\Mvariable{Li}}_2}(1-y)}{1-y}\bigg)\bigg)
}

\dispSFoutmath{
\multsp \Muserfunction{im}[\Muserfunction{SumS}[2,1,\Mvariable{OPEm}-1]]
}

\dispSFinmath{
{S_{21}}(m-1)\longrightarrow \bigg({y^{m-1}}\multsp \bigg(
      2\multsp \zeta (3)\multsp \delta (1-y)-\zeta (2)\multsp {{\bigg(\frac{1}{1-y}\bigg)}_+}+\frac{{{\Mvariable{Li}}_2}(y)}{1-y}\bigg)
      \bigg)
}

\Subsection*{Isolate}

\Subsubsection*{Description}

Isolate[expr] substitutes abbreviations KK[i] for all Plus[...] (sub-sums) in expr. The inserted KK[i] have head HoldForm. Isolate[expr,
  varlist] substitutes KK[i] for all subsums in expr which are free of any occurence of a member of the list varlist. Instead of KK any
  other head or a list of names of the abbreviations may be specified with the option IsolateNames.

\dispSFinmath{
\Muserfunction{im}[\Muserfunction{SumS}[1,1,1,\Mvariable{OPEm}-1]]
}

\dispSFoutmath{
{S_{111}}(m-1)\longrightarrow \Bigg(-\frac{1}{2}\multsp {y^{m-1}}\multsp {{\bigg(\frac{{{\log}^2}(1-y)}{1-y}\bigg)}_+}\Bigg)
}

\Subsubsection*{Examples}

\dispSFinmath{
\Mfunction{Clear}[\Mvariable{im},\Mvariable{list}];
}

\dispSFoutmath{
\Mfunction{Options}[\Mvariable{Isolate}]
}

\dispSFinmath{
\MathBegin{MathArray}{l}
\{\Mvariable{IsolateNames}\rightarrow \Mvariable{KK},
    \Mvariable{HighEnergyPhysics`FeynCalc`IsolatePrint`IsolatePrint}\rightarrow \Mvariable{False},  \\
\noalign{\vspace{0.666667ex}}
   \hspace{1.em} \Mvariable{HighEnergyPhysics`FeynCalc`IsolateSplit`IsolateSplit}\rightarrow \infty \}\\
\MathEnd{MathArray}
}

\dispSFoutmath{
\Mvariable{t0}=\Muserfunction{Isolate}[a+b]
}

\dispSFinmath{
\Muserfunction{KK}(1)
}

\dispSFoutmath{
\Mvariable{t1}=\Muserfunction{Isolate}[(a+b)\multsp f\multsp +\multsp (c+d)\multsp f\multsp +\multsp e,f]
}

\dispSFinmath{
e+f\multsp \Muserfunction{KK}(1)+f\multsp \Muserfunction{KK}(2)
}

\dispSFoutmath{
\Mfunction{StandardForm}[\Mvariable{t1}]
}

\dispSFinmath{
e+f\multsp \Muserfunction{KK}[1]+f\multsp \Muserfunction{KK}[2]
}

\dispSFoutmath{
\{\Mvariable{t0},\multsp \Mvariable{t1},\multsp \Mfunction{ReleaseHold}[\Mvariable{t1}]\}
}

\dispSFinmath{
\{\Muserfunction{KK}(1),e+f\multsp \Muserfunction{KK}(1)+f\multsp \Muserfunction{KK}(2),e+(a+b)\multsp f+(c+d)\multsp f\}
}

\Print{\(\Muserfunction{Isolate}[a[z]\multsp (b+c\multsp (y+z))+d[z]\multsp (y+z),\{a,d\},\Mvariable{IsolateNames}\rightarrow F]\)}

\Print{\(d(z)\multsp F(3)+a(z)\multsp F(4)\)}

\dispSFinmath{
??F
}

\Message{\(\Mvariable{F\multsp is\multsp a\multsp fermion\multsp field.}\)}

\Message{\(\MathBegin{MathArray}[c]{l}
  \MathBegin{MathArray}[c]{l}
  F[3]=y+z \\
  \multsp  \\
  F[4]=b+c\multsp \Mfunction{HoldForm}[F[3]]

     \MathEnd{MathArray}
  \MathEnd{MathArray}\)}

\dispSFoutmath{
\Muserfunction{Isolate}[a-b-c-d-e,\Mvariable{IsolateNames}\rightarrow L,\Mvariable{IsolateSplit}\rightarrow 15]
}

\dispSFinmath{
\MathBegin{MathArray}{l}
\Muserfunction{L2}::\Mvariable{shdw}:\multsp
   \Mvariable{Symbol}\multsp \Mvariable{L2}\multsp \Mvariable{appears}\multsp \Mvariable{in}\multsp \Mvariable{multiple}\multsp
      \Mvariable{contexts}\multsp \{\Mvariable{Global`},\Mvariable{HighEnergyPhysics`Phi`Objects`}\};\multsp   \\
\noalign{\vspace{
   0.666667ex}}
\hspace{2.em} \Mvariable{definitions}\multsp \Mvariable{in}\multsp \Mvariable{context}\multsp \Mvariable{Global`}
   \multsp \Mvariable{may}\multsp \Mvariable{shadow}\multsp \Mvariable{or}\multsp \Mvariable{be}\multsp \Mvariable{shadowed}\multsp
     \Mvariable{by}\multsp \Mvariable{other}\multsp \Mvariable{definitions}.\\
\MathEnd{MathArray}
}

\dispSFoutmath{
\MathBegin{MathArray}{l}
\Muserfunction{L3}::\Mvariable{shdw}:\multsp
   \Mvariable{Symbol}\multsp \Mvariable{L3}\multsp \Mvariable{appears}\multsp \Mvariable{in}\multsp \Mvariable{multiple}\multsp
      \Mvariable{contexts}\multsp \{\Mvariable{Global`},\Mvariable{HighEnergyPhysics`Phi`Objects`}\};\multsp   \\
\noalign{\vspace{
   0.666667ex}}
\hspace{2.em} \Mvariable{definitions}\multsp \Mvariable{in}\multsp \Mvariable{context}\multsp \Mvariable{Global`}
   \multsp \Mvariable{may}\multsp \Mvariable{shadow}\multsp \Mvariable{or}\multsp \Mvariable{be}\multsp \Mvariable{shadowed}\multsp
     \Mvariable{by}\multsp \Mvariable{other}\multsp \Mvariable{definitions}.\\
\MathEnd{MathArray}
}

\dispSFinmath{
L(3)
}

\Subsection*{IsolateHead}

\Subsubsection*{Description}

IsolateHead is equivalent to IsolateNames.

See also:  IsolateNames.

\Subsection*{IsolateNames}

\Subsubsection*{Description}

IsolateNames is an option for Isolate and Collect2. Its default setting is KK. Instead of a symbol the setting may also be a list with
  the names of the abbrevations.

See also:  Isolate, Collect2.

\Subsection*{IsolatePrint}

\Subsubsection*{Description}

IsolatePrint is an option of Isolate. If it is set to OutputForm (or any other *Form) the definitions of the abbreviations are printed
  during the operation of Isolate. The setting IsolatePrint \(\rightarrow \) False suppresses printing.

See also:  Isolate.

\Subsection*{IsolateSplit}

\Subsubsection*{Description}

IsolateSplit is an option for Isolate. Its setting determines the maximum number of characters of FortranForm[expr] which are abbreviated
  by Isolate. If the expression is larger than the indicated number, it is split into smaller pieces and onto each subsum Isolate is
  applied. With the default setting IsolateSplit \(\rightarrow \) Infinity no splitting is done.

See also:  Isolate.

\Subsection*{KeepOnly}

\Subsubsection*{Description}

KeepOnly is an option of OneLoop. It may be set to B0, C0, D0 keeping only the corresponding coefficients. The default setting is False.
  If KeepOnly is set to \{\} then the part of the amplitude which is not coefficient of B0, C0, D0 is kept.

See also:  OneLoop, B0, C0, D0.

\Subsection*{KK}

\Subsubsection*{Description}

KK[i] is the default setting of IsolateNames, which is the head of abbreviations used by Isolate. A KK[i] returned by Isolate is given in
  HoldForm and can be recovered by ReleaseHold[KK[i]].

See also:  Isolate, IsolateNames.

\Subsection*{Kummer}

\Subsubsection*{Description}

Kummer[i][exp] applies Kummer relation number i (i \(=\)1, ... 24, 94,95,96) to all Hypergeometric2F1 in exp. i \(=\) 94 corresponds to
  eq. 9.131.2, i \(=\) 95 to eq. 9.132.1 and i \(=\) 96 to eq. 9.132.2 in Gradsteyn \&{} Ryzhik.

See also:  HypergeometricAC.

\Subsubsection*{Examples}

\dispSFinmath{
\{L[2],L[1]\}
}

\dispSFoutmath{
\{a-b,L(1)\}
}

\dispSFinmath{
\Mfunction{Clear}[\Mvariable{t0},\Mvariable{t1},L]
}

\dispSFoutmath{
\Mfunction{Hypergeometric2F1}[a,b,c,z]==\Muserfunction{Kummer}[2][\Mfunction{Hypergeometric2F1}[a,b,c,z]]
}

\dispSFinmath{
{{\InvisiblePrefixScriptBase }_2}{F_1}(a,b;c;z)==
   {{(1-z)}^{-a-b+c}}\multsp {{\InvisiblePrefixScriptBase }_2}{F_1}(c-a,c-b;c;z)
}

\dispSFoutmath{
\Mfunction{Hypergeometric2F1}[a,b,c,z]==\Muserfunction{Kummer}[3][\Mfunction{Hypergeometric2F1}[a,b,c,z]]
}

\dispSFinmath{
{{\InvisiblePrefixScriptBase }_2}{F_1}(a,b;c;z)==
   {{(1-z)}^{-a}}\multsp {{\InvisiblePrefixScriptBase }_2}{F_1}\big(a,c-b;c;-\frac{z}{1-z}\big)
}

\dispSFoutmath{
\Mfunction{Hypergeometric2F1}[a,b,c,z]==\Muserfunction{Kummer}[4][\Mfunction{Hypergeometric2F1}[a,b,c,z]]
}

\dispSFinmath{
{{\InvisiblePrefixScriptBase }_2}{F_1}(a,b;c;z)==
   {{(1-z)}^{-b}}\multsp {{\InvisiblePrefixScriptBase }_2}{F_1}\big(c-a,b;c;-\frac{z}{1-z}\big)
}

\dispSFoutmath{
\Mfunction{Hypergeometric2F1}[a,b,c,1-z]==\Muserfunction{Kummer}[6][\Mfunction{Hypergeometric2F1}[a,b,c,1-z]]
}

\dispSFinmath{
{{\InvisiblePrefixScriptBase }_2}{F_1}(a,b;c;1-z)==
   {z^{-a-b+c}}\multsp {{\InvisiblePrefixScriptBase }_2}{F_1}(c-b,c-a;c;1-z)
}

\dispSFoutmath{
\MathBegin{MathArray}{l}
\Mfunction{Hypergeometric2F1}[a,b,a+b+1-c,1-z]==  \\
\noalign{\vspace{0.5ex}}
\hspace{1.em} \Muserfunction{Kumme
     r}[6][\Mfunction{Hypergeometric2F1}[a,b,a+b+1-c,1-z]]\\
\MathEnd{MathArray}
}

\dispSFinmath{
{{\InvisiblePrefixScriptBase }_2}{F_1}(a,b;a+b-c+1;1-z)==
   {z^{1-c}}\multsp {{\InvisiblePrefixScriptBase }_2}{F_1}(a-c+1,b-c+1;a+b-c+1;1-z)
}

\dispSFoutmath{
\Mfunction{Hypergeometric2F1}[a,b,c,1-z]==\Muserfunction{Kummer}[7][\Mfunction{Hypergeometric2F1}[a,b,c,1-z]]
}

\dispSFinmath{
{{\InvisiblePrefixScriptBase }_2}{F_1}(a,b;c;1-z)==
   {z^{-a}}\multsp {{\InvisiblePrefixScriptBase }_2}{F_1}\Big(a,c-b;c;-\frac{1-z}{z}\Big)
}

\dispSFoutmath{
\MathBegin{MathArray}{l}
\Mfunction{Hypergeometric2F1}[a,b,a+b+1-c,1-z]==  \\
\noalign{\vspace{0.5ex}}
\hspace{1.em} \Muserfunction{Kumme
     r}[7][\Mfunction{Hypergeometric2F1}[a,b,a+b+1-c,1-z]]\\
\MathEnd{MathArray}
}

\dispSFinmath{
{{\InvisiblePrefixScriptBase }_2}{F_1}(a,b;a+b-c+1;1-z)==
   {z^{-a}}\multsp {{\InvisiblePrefixScriptBase }_2}{F_1}\Big(a,a-c+1;a+b-c+1;-\frac{1-z}{z}\Big)
}

\dispSFoutmath{
\Mfunction{Hypergeometric2F1}[a,b,c,1-z]==\Muserfunction{Kummer}[8][\Mfunction{Hypergeometric2F1}[a,b,c,1-z]]
}

\dispSFinmath{
{{\InvisiblePrefixScriptBase }_2}{F_1}(a,b;c;1-z)==
   {z^{-b}}\multsp {{\InvisiblePrefixScriptBase }_2}{F_1}\Big(c-a,b;c;-\frac{1-z}{z}\Big)
}

\dispSFoutmath{
\MathBegin{MathArray}{l}
\Mfunction{Hypergeometric2F1}[a,b,a+b+1-c,1-z]==  \\
\noalign{\vspace{0.5ex}}
\hspace{1.em} \Muserfunction{Kumme
     r}[8][\Mfunction{Hypergeometric2F1}[a,b,a+b+1-c,1-z]]\\
\MathEnd{MathArray}
}

\dispSFinmath{
{{\InvisiblePrefixScriptBase }_2}{F_1}(a,b;a+b-c+1;1-z)==
   {z^{-b}}\multsp {{\InvisiblePrefixScriptBase }_2}{F_1}\Big(b-c+1,b;a+b-c+1;-\frac{1-z}{z}\Big)
}

\dispSFoutmath{
\Mfunction{Hypergeometric2F1}[a,b,c,z\RawWedge (-1)]==\Muserfunction{Kummer}[10][\Mfunction{Hypergeometric2F1}[a,b,c,z\RawWedge (-1)]]
}

\dispSFinmath{
{{\InvisiblePrefixScriptBase }_2}{F_1}\Big(a,b;c;\frac{1}{z}\Big)==
   {{(1-z)}^{-a-b+c}}\multsp {{(-z)}^{a+b-c}}\multsp {{\InvisiblePrefixScriptBase }_2}{F_1}
     \Big(c-a,c-b;c;\frac{1}{z}\Big)
}

\dispSFoutmath{
\MathBegin{MathArray}{l}
\Mfunction{Hypergeometric2F1}[a,a+1-c,a+1-b,z\RawWedge (-1)]==  \\
\noalign{\vspace{0.5ex}}
\hspace{1.em}
     \Muserfunction{Kummer}[10][\Mfunction{Hypergeometric2F1}[a,a+1-c,a+1-b,z\RawWedge (-1)]]\\
\MathEnd{MathArray}
}

\dispSFinmath{
{{\InvisiblePrefixScriptBase }_2}{F_1}\Big(a,a-c+1;a-b+1;\frac{1}{z}\Big)==
   {{(1-z)}^{-a-b+c}}\multsp {{(-z)}^{a+b-c}}\multsp {{\InvisiblePrefixScriptBase }_2}{F_1}
     \Big(1-b,c-b;a-b+1;\frac{1}{z}\Big)
}

\dispSFoutmath{
\Mfunction{Hypergeometric2F1}[a,b,c,z\RawWedge (-1)]==\Muserfunction{Kummer}[11][\Mfunction{Hypergeometric2F1}[a,b,c,z\RawWedge (-1)]]
}

\dispSFinmath{
{{\InvisiblePrefixScriptBase }_2}{F_1}\Big(a,b;c;\frac{1}{z}\Big)==
   {{\Big(-\frac{1-z}{z}\Big)}^{-a}}\multsp {{(-z)}^a}\multsp
    {{\InvisiblePrefixScriptBase }_2}{F_1}\Big(a,c-b;c;\frac{1}{1-z}\Big)
}

\dispSFoutmath{
\MathBegin{MathArray}{l}
\Mfunction{Hypergeometric2F1}[a,a+1-c,a+1-b,z\RawWedge (-1)]==  \\
\noalign{\vspace{0.5ex}}
\hspace{1.em}
     \Muserfunction{Kummer}[11][\Mfunction{Hypergeometric2F1}[a,a+1-c,a+1-b,z\RawWedge (-1)]]\\
\MathEnd{MathArray}
}

\dispSFinmath{
{{\InvisiblePrefixScriptBase }_2}{F_1}\Big(a,a-c+1;a-b+1;\frac{1}{z}\Big)==
   {{\Big(-\frac{1-z}{z}\Big)}^{-a}}\multsp {{(-z)}^a}\multsp
    {{\InvisiblePrefixScriptBase }_2}{F_1}\Big(a,c-b;a-b+1;\frac{1}{1-z}\Big)
}

\dispSFoutmath{
\Mfunction{Hypergeometric2F1}[a,b,c,z\RawWedge (-1)]==\Muserfunction{Kummer}[12][\Mfunction{Hypergeometric2F1}[a,b,c,z\RawWedge (-1)]]
}

\dispSFinmath{
{{\InvisiblePrefixScriptBase }_2}{F_1}\Big(a,b;c;\frac{1}{z}\Big)==
   {{(1-z)}^{-b}}\multsp {{(-z)}^{b-a}}\multsp {{\InvisiblePrefixScriptBase }_2}{F_1}\Big(b,c-a;c;\frac{1}{1-z}\Big)
}

\dispSFoutmath{
\MathBegin{MathArray}{l}
\Mfunction{Hypergeometric2F1}[a,a+1-c,a+1-b,z\RawWedge (-1)]==  \\
\noalign{\vspace{0.5ex}}
\hspace{1.em}
     \Muserfunction{Kummer}[12][\Mfunction{Hypergeometric2F1}[a,a+1-c,a+1-b,z\RawWedge (-1)]]\\
\MathEnd{MathArray}
}

\dispSFinmath{
{{\InvisiblePrefixScriptBase }_2}{F_1}\Big(a,a-c+1;a-b+1;\frac{1}{z}\Big)==
   {{(1-z)}^{-a+c-1}}\multsp {{(-z)}^{1-c}}\multsp {{\InvisiblePrefixScriptBase }_2}{F_1}
     \Big(a-c+1,1-b;a-b+1;\frac{1}{1-z}\Big)
}

\dispSFoutmath{
\Mfunction{Hypergeometric2F1}[a,b,c,z\RawWedge (-1)]==\Muserfunction{Kummer}[14][\Mfunction{Hypergeometric2F1}[a,b,c,z\RawWedge (-1)]]
}

\dispSFinmath{
{{\InvisiblePrefixScriptBase }_2}{F_1}\Big(a,b;c;\frac{1}{z}\Big)==
   {{(1-z)}^{-a-b+c}}\multsp {{(-z)}^{a+b-c}}\multsp {{\InvisiblePrefixScriptBase }_2}{F_1}
     \Big(c-b,c-a;c;\frac{1}{z}\Big)
}

\dispSFoutmath{
\MathBegin{MathArray}{l}
\Mfunction{Hypergeometric2F1}[b+1-c,b,b+1-a,z\RawWedge (-1)]==  \\
\noalign{\vspace{0.5ex}}
\hspace{1.em}
     \Muserfunction{Kummer}[14][\Mfunction{Hypergeometric2F1}[b+1-c,b,b+1-a,z\RawWedge (-1)]]\\
\MathEnd{MathArray}
}

\dispSFinmath{
{{\InvisiblePrefixScriptBase }_2}{F_1}\Big(b-c+1,b;-a+b+1;\frac{1}{z}\Big)==
   {{(1-z)}^{-a-b+c}}\multsp {{(-z)}^{a+b-c}}\multsp {{\InvisiblePrefixScriptBase }_2}{F_1}
     \Big(1-a,c-a;-a+b+1;\frac{1}{z}\Big)
}

\dispSFoutmath{
\Mfunction{Hypergeometric2F1}[a,b,c,z\RawWedge (-1)]==\Muserfunction{Kummer}[15][\Mfunction{Hypergeometric2F1}[a,b,c,z\RawWedge (-1)]]
}

\dispSFinmath{
{{\InvisiblePrefixScriptBase }_2}{F_1}\Big(a,b;c;\frac{1}{z}\Big)==
   {{(1-z)}^{-b}}\multsp {{(-z)}^b}\multsp {{\InvisiblePrefixScriptBase }_2}{F_1}\Big(b,c-a;c;\frac{1}{1-z}\Big)
}

\dispSFoutmath{
\MathBegin{MathArray}{l}
\Mfunction{Hypergeometric2F1}[b+1-c,b,b+1-a,z\RawWedge (-1)]==  \\
\noalign{\vspace{0.5ex}}
\hspace{1.em}
     \Muserfunction{Kummer}[15][\Mfunction{Hypergeometric2F1}[b+1-c,b,b+1-a,z\RawWedge (-1)]]\\
\MathEnd{MathArray}
}

\dispSFinmath{
{{\InvisiblePrefixScriptBase }_2}{F_1}\Big(b-c+1,b;-a+b+1;\frac{1}{z}\Big)==
   {{(1-z)}^{-b}}\multsp {{(-z)}^b}\multsp {{\InvisiblePrefixScriptBase }_2}{F_1}
     \Big(b,c-a;-a+b+1;\frac{1}{1-z}\Big)
}

\dispSFoutmath{
\Mfunction{Hypergeometric2F1}[a,b,c,z\RawWedge (-1)]==\Muserfunction{Kummer}[16][\Mfunction{Hypergeometric2F1}[a,b,c,z\RawWedge (-1)]]
}

\dispSFinmath{
{{\InvisiblePrefixScriptBase }_2}{F_1}\Big(a,b;c;\frac{1}{z}\Big)==
   {{(1-z)}^{-a}}\multsp {{(-z)}^a}\multsp {{\InvisiblePrefixScriptBase }_2}{F_1}\Big(a,c-b;c;\frac{1}{1-z}\Big)
}

\dispSFoutmath{
\MathBegin{MathArray}{l}
\Mfunction{Hypergeometric2F1}[b+1-c,b,b+1-a,z\RawWedge (-1)]==  \\
\noalign{\vspace{0.5ex}}
\hspace{1.em}
     \Muserfunction{Kummer}[16][\Mfunction{Hypergeometric2F1}[b+1-c,b,b+1-a,z\RawWedge (-1)]]\\
\MathEnd{MathArray}
}

\dispSFinmath{
{{\InvisiblePrefixScriptBase }_2}{F_1}\Big(b-c+1,b;-a+b+1;\frac{1}{z}\Big)==
   {{(1-z)}^{-b+c-1}}\multsp {{(-z)}^{b-c+1}}\multsp {{\InvisiblePrefixScriptBase }_2}{F_1}
     \Big(b-c+1,1-a;-a+b+1;\frac{1}{1-z}\Big)
}

\dispSFoutmath{
\MathBegin{MathArray}{l}
\Mfunction{Hypergeometric2F1}[a+1-c,b+1-c,2-c,z]==  \\
\noalign{\vspace{0.5ex}}
\hspace{1.em} \Muserfunction{Kum
     mer}[18][\Mfunction{Hypergeometric2F1}[a+1-c,b+1-c,2-c,z]]\\
\MathEnd{MathArray}
}

\dispSFinmath{
{{\InvisiblePrefixScriptBase }_2}{F_1}(a-c+1,b-c+1;2-c;z)==
   {{(1-z)}^{-a-b+c}}\multsp {{\InvisiblePrefixScriptBase }_2}{F_1}(1-a,1-b;2-c;z)
}

\dispSFoutmath{
\Mfunction{Hypergeometric2F1}[a,b,c,z]==\Muserfunction{Kummer}[18][\Mfunction{Hypergeometric2F1}[a,b,c,z]]
}

\dispSFinmath{
{{\InvisiblePrefixScriptBase }_2}{F_1}(a,b;c;z)==
   {{(1-z)}^{-a-b+c}}\multsp {{\InvisiblePrefixScriptBase }_2}{F_1}(c-a,c-b;c;z)
}

\dispSFoutmath{
\MathBegin{MathArray}{l}
\Mfunction{Hypergeometric2F1}[a+1-c,b+1-c,2-c,z]==  \\
\noalign{\vspace{0.5ex}}
\hspace{1.em} \Muserfunction{Kum
     mer}[19][\Mfunction{Hypergeometric2F1}[a+1-c,b+1-c,2-c,z]]\\
\MathEnd{MathArray}
}

\dispSFinmath{
{{\InvisiblePrefixScriptBase }_2}{F_1}(a-c+1,b-c+1;2-c;z)==
   {{(1-z)}^{-a+c-1}}\multsp {{\InvisiblePrefixScriptBase }_2}{F_1}\big(a-c+1,1-b;2-c;\frac{z}{z-1}\big)
}

\dispSFoutmath{
\Mfunction{Hypergeometric2F1}[a,b,c,z]==\Muserfunction{Kummer}[19][\Mfunction{Hypergeometric2F1}[a,b,c,z]]
}

\dispSFinmath{
{{\InvisiblePrefixScriptBase }_2}{F_1}(a,b;c;z)==
   {{(1-z)}^{-a}}\multsp {{\InvisiblePrefixScriptBase }_2}{F_1}\big(a,c-b;c;\frac{z}{z-1}\big)
}

\dispSFoutmath{
\MathBegin{MathArray}{l}
\Mfunction{Hypergeometric2F1}[a+1-c,b+1-c,2-c,z]==  \\
\noalign{\vspace{0.5ex}}
\hspace{1.em} \Muserfunction{Kum
     mer}[20][\Mfunction{Hypergeometric2F1}[a+1-c,b+1-c,2-c,z]]\\
\MathEnd{MathArray}
}

\dispSFinmath{
{{\InvisiblePrefixScriptBase }_2}{F_1}(a-c+1,b-c+1;2-c;z)==
   {{(1-z)}^{-b+c-1}}\multsp {{\InvisiblePrefixScriptBase }_2}{F_1}\big(b-c+1,1-a;2-c;\frac{z}{z-1}\big)
}

\dispSFoutmath{
\Mfunction{Hypergeometric2F1}[a,b,c,z]==\Muserfunction{Kummer}[20][\Mfunction{Hypergeometric2F1}[a,b,c,z]]
}

\dispSFinmath{
{{\InvisiblePrefixScriptBase }_2}{F_1}(a,b;c;z)==
   {{(1-z)}^{-b}}\multsp {{\InvisiblePrefixScriptBase }_2}{F_1}\big(b,c-a;c;\frac{z}{z-1}\big)
}

\dispSFoutmath{
\MathBegin{MathArray}{l}
\Mfunction{Hypergeometric2F1}[c-a,c-b,c+1-a-b,1-z]==  \\
\noalign{\vspace{0.5ex}}
\hspace{1.em} \Muserfunction{K
     ummer}[22][\Mfunction{Hypergeometric2F1}[c-a,c-b,c+1-a-b,1-z]]\\
\MathEnd{MathArray}
}

\dispSFinmath{
{{\InvisiblePrefixScriptBase }_2}{F_1}(c-a,c-b;-a-b+c+1;1-z)==
   {z^{1-c}}\multsp {{\InvisiblePrefixScriptBase }_2}{F_1}(1-a,1-b;-a-b+c+1;1-z)
}

\dispSFoutmath{
\Mfunction{Hypergeometric2F1}[a,b,c,1-z]==\Muserfunction{Kummer}[22][\Mfunction{Hypergeometric2F1}[a,b,c,1-z]]
}

\dispSFinmath{
{{\InvisiblePrefixScriptBase }_2}{F_1}(a,b;c;1-z)==
   {z^{-a-b+c}}\multsp {{\InvisiblePrefixScriptBase }_2}{F_1}(c-b,c-a;c;1-z)
}

\dispSFoutmath{
\MathBegin{MathArray}{l}
\Mfunction{Hypergeometric2F1}[c-a,c-b,c+1-a-b,1-z]==  \\
\noalign{\vspace{0.5ex}}
\hspace{1.em} \Muserfunction{K
     ummer}[23][\Mfunction{Hypergeometric2F1}[c-a,c-b,c+1-a-b,1-z]]\\
\MathEnd{MathArray}
}

\dispSFinmath{
{{\InvisiblePrefixScriptBase }_2}{F_1}(c-a,c-b;-a-b+c+1;1-z)==
   {{(1-z)}^{a-c}}\multsp {{\InvisiblePrefixScriptBase }_2}{F_1}\Big(c-a,1-a;-a-b+c+1;1-\frac{1}{1-z}\Big)
}

\dispSFoutmath{
\Mfunction{Hypergeometric2F1}[a,b,c,1-z]==\Muserfunction{Kummer}[23][\Mfunction{Hypergeometric2F1}[a,b,c,1-z]]
}

\dispSFinmath{
{{\InvisiblePrefixScriptBase }_2}{F_1}(a,b;c;1-z)==
   {{(1-z)}^{-a}}\multsp {{\InvisiblePrefixScriptBase }_2}{F_1}\Big(a,c-b;c;1-\frac{1}{1-z}\Big)
}

\dispSFoutmath{
\MathBegin{MathArray}{l}
\Mfunction{Hypergeometric2F1}[c-a,c-b,c+1-a-b,1-z]==  \\
\noalign{\vspace{0.5ex}}
\hspace{1.em} \Muserfunction{K
     ummer}[24][\Mfunction{Hypergeometric2F1}[c-a,c-b,c+1-a-b,1-z]]\\
\MathEnd{MathArray}
}

\dispSFinmath{
{{\InvisiblePrefixScriptBase }_2}{F_1}(c-a,c-b;-a-b+c+1;1-z)==
   {z^{b-c}}\multsp {{\InvisiblePrefixScriptBase }_2}{F_1}\Big(c-b,1-b;-a-b+c+1;-\frac{1-z}{z}\Big)
}

\dispSFoutmath{
\Mfunction{Hypergeometric2F1}[a,b,c,1-z]==\Muserfunction{Kummer}[24][\Mfunction{Hypergeometric2F1}[a,b,c,1-z]]
}

\dispSFinmath{
{{\InvisiblePrefixScriptBase }_2}{F_1}(a,b;c;1-z)==
   {z^{-b}}\multsp {{\InvisiblePrefixScriptBase }_2}{F_1}\Big(b,c-a;c;-\frac{1-z}{z}\Big)
}

\dispSFoutmath{
\Mfunction{Hypergeometric2F1}[a,b,c,z]==\Muserfunction{Kummer}[94][\Mfunction{Hypergeometric2F1}[a,b,c,z]]
}

\dispSFinmath{
\MathBegin{MathArray}{l}
{{\InvisiblePrefixScriptBase }_2}{F_1}(a,b;c;z)==
   \frac{\Gamma (a+b-c)\multsp \Gamma (c)\multsp {{\InvisiblePrefixScriptBase }_2}{F_1}(c-a,c-b;-a-b+c+1;1-z)
       \multsp {{(1-z)}^{-a-b+c}}}{\Gamma (a)\multsp \Gamma (b)}+  \\
\noalign{\vspace{1.36458ex}}
\hspace{2.em} \frac{\Gamma (c)\multsp
     \Gamma (-a-b+c)\multsp {{\InvisiblePrefixScriptBase }_2}{F_1}(a,b;a+b-c+1;1-z)}{\Gamma (c-a)\multsp
     \Gamma (c-b)}\\
\MathEnd{MathArray}
}

\dispSFoutmath{
\Mfunction{Hypergeometric2F1}[a,b,c,z]==\Muserfunction{Kummer}[95][\Mfunction{Hypergeometric2F1}[a,b,c,z]]
}

\Subsection*{Lagrangian}

\Subsubsection*{Description}

Lagrangian["oqu"] gives the unpolarized OPE quark operator. Lagrangian["oqp"] gives the polarized quark OPE operator. Lagrangian["ogu"]
  gives the unpolarized gluon OPE operator. Lagrangian["ogp"] gives the polarized gluon OPE operator. Lagrangian["ogd"] gives the
  sigma-term part of the QCD lagrangian. Lagrangian["QCD"] gives the gluon self interaction part of the QCD lagrangian.

See also:  FeynRule.

\Subsubsection*{Examples}

\dispSFinmath{
\MathBegin{MathArray}{l}
{{\InvisiblePrefixScriptBase }_2}{F_1}(a,b;c;z)==  \\
\noalign{\vspace{1.08333ex}}
   \hspace{1.em} \frac{\Gamma (b-a)\multsp \Gamma (c)\multsp
      {{\InvisiblePrefixScriptBase }_2}{F_1}\big(a,c-b;a-b+1;\frac{1}{1-z}\big)\multsp {{(1-z)}^{-a}}}{\Gamma (b)
      \multsp \Gamma (c-a)}+\frac{\Gamma (a-b)\multsp \Gamma (c)\multsp
      {{\InvisiblePrefixScriptBase }_2}{F_1}\big(b,c-a;-a+b+1;\frac{1}{1-z}\big)\multsp {{(1-z)}^{-b}}}{\Gamma (a)
      \multsp \Gamma (c-b)}\\
\MathEnd{MathArray}
}

\dispSFoutmath{
\Mfunction{Hypergeometric2F1}[a,b,c,z]==\Muserfunction{Kummer}[96][\Mfunction{Hypergeometric2F1}[a,b,c,z]]
}

Twist-2 operator product expansion operators

\dispSFinmath{
\MathBegin{MathArray}{l}
{{\InvisiblePrefixScriptBase }_2}{F_1}(a,b;c;z)==  \\
\noalign{\vspace{1.08333ex}}
   \hspace{1.em} \frac{{{(-1)}^a}\multsp \Gamma (b-a)\multsp \Gamma (c)\multsp
      {{\InvisiblePrefixScriptBase }_2}{F_1}\big(a,a-c+1;a-b+1;\frac{1}{z}\big)\multsp {z^{-a}}}{\Gamma (b)\multsp
      \Gamma (c-a)}+\frac{{{(-1)}^b}\multsp \Gamma (a-b)\multsp \Gamma (c)\multsp
      {{\InvisiblePrefixScriptBase }_2}{F_1}\big(b,b-c+1;-a+b+1;\frac{1}{z}\big)\multsp {z^{-b}}}{\Gamma (a)\multsp
      \Gamma (c-b)}\\
\MathEnd{MathArray}
}

\dispSFoutmath{
\Muserfunction{Lagrangian}["QCD"]
}

\dispSFinmath{
-\frac{1}{4}\multsp F_{\alpha \beta }^{a}.F_{\alpha \beta }^{a}
}

\dispSFoutmath{
\Muserfunction{Lagrangian}["ogu"]
}

\dispSFinmath{
\frac{1}{2}\multsp {{\ImaginaryI }^{m-1}}\multsp F_{\alpha \Delta }^{a}.{{\big(D_{\Delta }^{ab}\big)}^{m-2}}.F_{\alpha \Delta }^{b}
}

\dispSFoutmath{
\Muserfunction{Lagrangian}["ogp"]
}

\dispSFinmath{
\frac{1}{2}\multsp {{\ImaginaryI }^m}\multsp {{\epsilon }^{\alpha \beta \gamma \Delta }}.F_{\beta \gamma }^{a}.
    {{\big(D_{\Delta }^{ab}\big)}^{m-2}}.F_{\alpha \Delta }^{b}
}

\dispSFoutmath{
\Muserfunction{Lagrangian}["oqu"]
}

\Subsection*{LC}

\Subsubsection*{Description}

LC[m,n,r,s] evaluates to 4-dimensional LeviCivita[m,n,r,s] by virtue of applying FeynCalcInternal. LC[m,...][p, ...] evaluates to
  4-dimensional LeviCivita[m,...][p,...] applying FeynCalcInternal.

See also:  LeviCivita, LCD.

\Subsubsection*{Examples}

\dispSFinmath{
{{\ImaginaryI }^m}\multsp \overvar{\psi }{\_}.(\gamma \cdot \Delta ).{{{D_{\Delta }}}^{m-1}}.\psi
}

\dispSFoutmath{
\Muserfunction{Lagrangian}["oqp"]
}

\dispSFinmath{
{{\ImaginaryI }^m}\multsp \overvar{\psi }{\_}.{{\gamma }^5}.(\gamma \cdot \Delta ).{{{D_{\Delta }}}^{m-1}}.\psi
}

\dispSFoutmath{
\Muserfunction{LC}[\mu ,\nu ,\rho ,\sigma ]
}

\dispSFinmath{
{{\epsilon }^{\mu \nu \rho \sigma }}
}

\dispSFoutmath{
\%//\Muserfunction{FCI}
}

\dispSFinmath{
{{\epsilon }^{\mu \nu \rho \sigma }}
}

\dispSFoutmath{
\%//\Mfunction{StandardForm}
}

\dispSFinmath{
\Muserfunction{Eps}[\Muserfunction{LorentzIndex}[\mu ],\Muserfunction{LorentzIndex}[\nu ],\Muserfunction{LorentzIndex}[\rho ],
    \Muserfunction{LorentzIndex}[\sigma ]]
}

\dispSFoutmath{
\Muserfunction{LC}[\mu ,\nu ][p,q]
}

\dispSFinmath{
{{\epsilon }^{\mu \nu pq}}
}

\dispSFoutmath{
\%//\Muserfunction{FCI}//\Mfunction{StandardForm}
}

\Subsection*{LCD}

\Subsubsection*{Description}

LCD[m,n,r,s] evaluates to D-dimensional LeviCivita[m,n,r,s] by virtue of FeynCalcInternal. LCD[m,...][p, ...] evaluates to D-dimensional
  LeviCivita[m,...][p,...] applying FeynCalcInternal.

You need to do SetOptions[Eps, Dimension\(\rightarrow \)D] before LCD can be used as D-dimensional Levi-Civita input function.

See also:  LeviCivita, LC.

\Subsubsection*{Examples}

\dispSFinmath{
\Muserfunction{Eps}[\Muserfunction{LorentzIndex}[\mu ],\Muserfunction{LorentzIndex}[\nu ],\Muserfunction{Momentum}[p,D],
    \Muserfunction{Momentum}[q,D]]
}

\dispSFinmath{
\Muserfunction{Contract}[\Muserfunction{LC}[\mu ,\nu ,\rho ][p]\multsp \Muserfunction{LC}[\mu ,\nu ,\rho ][q]]\multsp
}

\dispSFoutmath{
18\multsp p\cdot q-24\multsp p\cdot q
}

\dispSFinmath{
\Mfunction{SetOptions}[\Mvariable{Eps},\Mvariable{Dimension}\rightarrow D];
}

\dispSFoutmath{
\Muserfunction{LCD}[\mu ,\nu ,\rho ,\sigma ]
}

\dispSFinmath{
{{\epsilon }^{\mu \nu \rho \sigma }}
}

\dispSFoutmath{
\%//\Muserfunction{FCI}
}

\dispSFinmath{
{{\epsilon }^{\mu \nu \rho \sigma }}
}

\dispSFoutmath{
\%//\Mfunction{StandardForm}
}

\dispSFinmath{
\Muserfunction{Eps}[\Muserfunction{LorentzIndex}[\mu ,D],\Muserfunction{LorentzIndex}[\nu ,D],\Muserfunction{LorentzIndex}[\rho ,D],
    \Muserfunction{LorentzIndex}[\sigma ,D]]
}

\dispSFoutmath{
\Muserfunction{LCD}[\mu ,\nu ][p,q]
}

\dispSFinmath{
{{\epsilon }^{\mu \nu pq}}
}

\dispSFoutmath{
\%//\Muserfunction{FCI}//\Mfunction{StandardForm}
}

\dispSFinmath{
\Muserfunction{Eps}[\Muserfunction{LorentzIndex}[\mu ,D],\Muserfunction{LorentzIndex}[\nu ,D],\Muserfunction{Momentum}[p,D],
    \Muserfunction{Momentum}[q,D]]
}

\dispSFoutmath{
\Muserfunction{Factor2}[\Muserfunction{Contract}[\Muserfunction{LCD}[\mu ,\nu ,\rho ][p]\multsp \Muserfunction{LCD}[\mu ,\nu ,\rho ][q]]]
}

\Subsection*{LeftPartialD}

\Subsubsection*{Description}

LeftPartialD[\(\mu \)] denotes \((1-D)\multsp (2-D)\multsp (3-D)\multsp p\cdot q\)acting to the left.

See also:  ExpandPartialD, PartialD, LeftRightPartialD, RightPartialD.

\Subsubsection*{\(\Mfunction{SetOptions}[\Mvariable{Eps},\Mvariable{Dimension}\rightarrow 4]\)}

\dispSFinmath{
\{\Mvariable{Dimension}\rightarrow 4\}
}

\dispSFoutmath{
{{\left( \overvar{\partial }{\leftarrow } \right) }_{\mu }}
}

\dispSFinmath{
\Mvariable{Examples}
}

\dispSFoutmath{
\Muserfunction{QuantumField}[A,\Muserfunction{LorentzIndex}[\mu ]].\Muserfunction{LeftPartialD}[\nu ]
}

\dispSFinmath{
{A_{\mu }}.{{\left( \overvar{\partial }{\leftarrow } \right) }_{\nu }}
}

\dispSFoutmath{
\Muserfunction{ExpandPartialD}[\%]
}

\dispSFinmath{
{{\partial }_{\nu }}A_{\mu }^{ }
}

\dispSFoutmath{
\Mfunction{StandardForm}[\%]
}

\dispSFinmath{
\Muserfunction{QuantumField}[\Muserfunction{PartialD}[\Muserfunction{LorentzIndex}[\nu ]],A,\Muserfunction{LorentzIndex}[\mu ]]
}

\dispSFoutmath{
\Mfunction{StandardForm}[\Muserfunction{LeftPartialD}[\mu ]]
}

\dispSFinmath{
\Muserfunction{LeftPartialD}[\Muserfunction{LorentzIndex}[\mu ]]
}

\dispSFoutmath{
\Muserfunction{QuantumField}[A,\Muserfunction{LorentzIndex}[\mu ]].\Muserfunction{QuantumField}[A,\Muserfunction{LorentzIndex}[\nu ]].
   \Muserfunction{LeftPartialD}[\rho ]
}

\dispSFinmath{
{A_{\mu }}.{A_{\nu }}.{{\left( \overvar{\partial }{\leftarrow } \right) }_{\rho }}
}

\dispSFoutmath{
\Muserfunction{ExpandPartialD}[\%]
}

\Subsection*{LeftRightPartialD}

\Subsubsection*{Description}

LeftRightPartialD[mu] denotes \({A_{\mu }}.{{\partial }_{\rho }}A_{\nu }^{ }+{{\partial }_{\rho }}A_{\mu }^{ }.{A_{\nu }}\), acting to
the left and right. ExplicitPartialD[LeftRightPartialD[\(\Mfunction{StandardForm}[\%]\)]] gives 1/2 (RightPartialD[\(\MathBegin{MathArray}{l}
\Muserfunction{QuantumField}[A,\Muserfunction{LorentzIndex}[\mu ]].  \\
\noalign{\vspace{0.5ex}}
   \hspace{2.em} \Muserfunction{QuantumField}[\Muserfunction{PartialD}[\Muserfunction{LorentzIndex}[\rho ]],A,
      \Muserfunction{LorentzIndex}[\nu ]]+  \\
\noalign{\vspace{0.5ex}}
\hspace{1.em} \Muserfunction{QuantumField}[
    \Muserfunction{PartialD}[\Muserfunction{LorentzIndex}[\rho ]],A,\Muserfunction{LorentzIndex}[\mu ]].  \\
   \noalign{\vspace{0.5ex}}
\hspace{2.em} \Muserfunction{QuantumField}[A,\Muserfunction{LorentzIndex}[\nu ]]\\
\MathEnd{MathArray}\)] - LeftPartialD[\({{\left( \overvar{\partial }{\leftrightarrow } \right) }_{\mu }}\)]).

See also: ExplicitPartialD, ExpandPartialD, PartialD, LeftPartialD, LeftRightPartialD2, RightPartialD.

\Subsubsection*{Examples}

\dispSFinmath{
\mu
}

\dispSFoutmath{
\mu
}

\dispSFinmath{
\mu
}

\dispSFoutmath{
\Muserfunction{LeftRightPartialD}[\mu ]
}

\dispSFinmath{
{{\left( \overvar{\partial }{\leftrightarrow } \right) }_{\mu }}
}

\dispSFoutmath{
\Muserfunction{ExplicitPartialD}[\%]
}

\dispSFinmath{
\frac{1}{2}\multsp \big({{\left( \overvar{\partial }{\rightarrow } \right) }_{\mu }}-
     {{\left( \overvar{\partial }{\leftarrow } \right) }_{\mu }}\big)
}

\dispSFoutmath{
\Muserfunction{LeftRightPartialD}[\mu ].\Muserfunction{QuantumField}[A,\Muserfunction{LorentzIndex}[\nu ]]
}

\dispSFinmath{
{{\left( \overvar{\partial }{\leftrightarrow } \right) }_{\mu }}.{A_{\nu }}
}

\dispSFoutmath{
\Muserfunction{ExpandPartialD}[\%]
}

\dispSFinmath{
\frac{{{\partial }_{\mu }}A_{\nu }^{ }}{2}-\frac{1}{2}\multsp {{\left( \overvar{\partial }{\leftarrow } \right) }_{\mu }}.{A_{\nu }}
}

\dispSFoutmath{
\MathBegin{MathArray}{l}
\Muserfunction{QuantumField}[A,\Muserfunction{LorentzIndex}[\mu ]].  \\
\noalign{\vspace{0.5ex}}
   \hspace{1.em} \Muserfunction{LeftRightPartialD}[\nu ].\Muserfunction{QuantumField}[A,\Muserfunction{LorentzIndex}[\rho ]]\\
   \MathEnd{MathArray}
}

\Subsection*{LeftRightPartialD2}

\Subsubsection*{Description}

LeftRightPartialD2[mu] denotes \({A_{\mu }}.{{\left( \overvar{\partial }{\leftrightarrow } \right) }_{\nu }}.{A_{\rho }}\), acting to the left and
right. ExplicitPartialD[LeftRightPartialD2[\(\Muserfunction{ExpandPartialD}[\%]\)]] gives (RightPartialD[\(\frac{1}{2}\multsp {A_{\mu }}.{{\partial
}_{\nu }}A_{\rho }^{ }-\frac{1}{2}\multsp {{\partial }_{\nu }}A_{\mu }^{ }.{A_{\rho }}\)] \(+\) LeftPartialD[\({{\left( \overvar{\partial
}{\leftrightarrow } \right) }_{\mu }}\)]).

See also:  ExplicitPartialD, ExpandPartialD, PartialD, LeftPartialD, RightPartialD.

\Subsubsection*{Examples}

\dispSFinmath{
\mu
}

\dispSFoutmath{
\mu
}

\dispSFinmath{
\mu
}

\dispSFoutmath{
\Muserfunction{LeftRightPartialD2}[\mu ]
}

\dispSFinmath{
{{\left( \overvar{\partial }{\leftrightarrow } \right) }_{\mu }}
}

\dispSFoutmath{
\Muserfunction{ExplicitPartialD}[\%]
}

\dispSFinmath{
{{\left( \overvar{\partial }{\leftarrow } \right) }_{\mu }}+{{\left( \overvar{\partial }{\rightarrow } \right) }_{\mu }}
}

\dispSFoutmath{
\Muserfunction{LeftRightPartialD2}[\mu ].\Muserfunction{QuantumField}[A,\Muserfunction{LorentzIndex}[\nu ]]
}

\dispSFinmath{
{{\left( \overvar{\partial }{\leftrightarrow } \right) }_{\mu }}.{A_{\nu }}
}

\dispSFoutmath{
\Muserfunction{ExpandPartialD}[\%]
}

\dispSFinmath{
{{\left( \overvar{\partial }{\leftarrow } \right) }_{\mu }}.{A_{\nu }}+{{\partial }_{\mu }}A_{\nu }^{ }
}

\dispSFoutmath{
\MathBegin{MathArray}{l}
\Muserfunction{QuantumField}[A,\Muserfunction{LorentzIndex}[\mu ]].  \\
\noalign{\vspace{0.5ex}}
   \hspace{1.em} \Muserfunction{LeftRightPartialD2}[\nu ].\Muserfunction{QuantumField}[A,\Muserfunction{LorentzIndex}[\rho ]]\\
   \MathEnd{MathArray}
}

\Subsection*{LeviCivita }

\Subsubsection*{Description}

LeviCivita[\({A_{\mu }}.{{\left( \overvar{\partial }{\leftrightarrow } \right) }_{\nu }}.{A_{\rho }}\)] is an input function for the totally antisymmetric
Levi-Civita tensor. It evaluates automatically to the internal representation
  Eps[LorentzIndex[\(\Muserfunction{ExpandPartialD}[\%]\)], LorentzIndex[\({A_{\mu }}.{{\partial }_{\nu }}A_{\rho }^{ }+{{\partial }_{\nu }}A_{\mu
}^{ }.{A_{\rho }}\)], LorentzIndex[\(\mu ,\multsp \nu ,\multsp \rho ,\multsp \sigma \)], LorentzIndex[\(\sigma \)]] (or with a second argument
in LorentzIndex for the Dimension, if the option Dimension of LeviCivita is
  changed). LeviCivita[\(\mu \)][ \(\nu \).] evaluates to Eps[LorentzIndex[\(\rho \)], LorentzIndex[\(\mu ,\multsp \nu \multsp ...\)], ..., Momentum[\(p,\multsp
..\)], ...].

\dispSFinmath{
\mu
}

\dispSFoutmath{
\nu
}

See also:  Eps, Contract, LC, LCD.

\Subsubsection*{Examples}

\dispSFinmath{
p
}

\dispSFoutmath{
\Mfunction{Options}[\Mvariable{LeviCivita}]
}

\dispSFinmath{
\{\Mvariable{Dimension}\rightarrow D\}
}

\dispSFoutmath{
\Muserfunction{LeviCivita}[\alpha ,\beta ,\gamma ,\delta ]
}

\dispSFinmath{
{{\epsilon }^{\alpha \beta \gamma \delta }}
}

\dispSFoutmath{
\Muserfunction{LeviCivita}[\mu ,\nu ][\Mvariable{OPEDelta},p]
}

\dispSFinmath{
{{\epsilon }^{\mu \nu \Delta p}}
}

\dispSFoutmath{
\Muserfunction{LeviCivita}[][p,q,r,s]
}

\dispSFinmath{
{{\epsilon }^{pqrs}}
}

\dispSFoutmath{
\Muserfunction{LeviCivita}[\alpha ,\beta ,\beta ,\delta ]
}

\dispSFinmath{
0
}

\dispSFoutmath{
\Muserfunction{LeviCivita}[\mu ,\nu ][\Mvariable{OPEDelta},p]//\Mfunction{StandardForm}
}

\dispSFinmath{
\Muserfunction{Eps}[\Muserfunction{LorentzIndex}[\mu ],\Muserfunction{LorentzIndex}[\nu ],\Muserfunction{Momentum}[\Mvariable{OPEDelta}],
    \Muserfunction{Momentum}[p]]
}

\dispSFoutmath{
\Mvariable{t1}\multsp =\multsp \Muserfunction{LeviCivita}[\alpha ,\beta ,\gamma ,\rho ].
     \Muserfunction{LeviCivita}[\alpha ,\beta ,\gamma ,\rho ]\multsp
}

\dispSFinmath{
{{\epsilon }^{\alpha \beta \gamma \rho }}.{{\epsilon }^{\alpha \beta \gamma \rho }}
}

\dispSFinmath{
\Muserfunction{Contract}[\Mvariable{t1}]
}

\dispSFoutmath{
-24
}

\dispSFinmath{
\Mfunction{SetOptions}[\Mvariable{LeviCivita},\Mvariable{Dimension}\rightarrow D];
   \Mfunction{SetOptions}[\Mvariable{Eps},\Mvariable{Dimension}\rightarrow D];
}

\dispSFinmath{
\Muserfunction{Contract}[\multsp \Muserfunction{LeviCivita}[\alpha ,\beta ,\gamma ,\rho ].
     \Muserfunction{LeviCivita}[\alpha ,\beta ,\gamma ,\rho ]]//\Muserfunction{Factor2}
}

\dispSFoutmath{
(1-D)\multsp (2-D)\multsp (3-D)\multsp D
}

\dispSFinmath{
\Mfunction{SetOptions}[\Mvariable{LeviCivita},\Mvariable{Dimension}\rightarrow 4];
   \Mfunction{SetOptions}[\Mvariable{Eps},\Mvariable{Dimension}\rightarrow 4];
}

\Subsection*{LeviCivitaSign}

\Subsubsection*{Description}

LeviCivitaSign is an option for DiracTrace and EpsChisholm. It determines the sign in the result of a Dirac trace of four gamma matrices
  and gamma5.

See also:  DiracTrace, EpsChisholm.

\Subsection*{ListIntegrals { }***unfinished***}

\Subsubsection*{Description}

ListIntegrals[exp, \{q1, q2\}] gives a list of basic integrals { }(FeynAmpDenominator's multiplied by scalar products involving q1, q2).
  { }Any non-integer exponent (e.g. the (OPEm-1) in { }ScalarProduct[OPEDelta,q1]\(\RawWedge\)(OPEm-1)) is replaced by OPEm. { }If the
  options Pair is set to True only integrals involving { } scalar products (not counting those with OPEDelta) are given; with Pair
  \(\rightarrow \) False those are excluded.

\Subsubsection*{Examples}

\dispSFinmath{
\Muserfunction{LC}[\alpha ,\beta ,\gamma ,\delta ]
}

\Subsection*{Li2 }

\Subsubsection*{Description}

 Li2 is an abbreviation for the dilog function, i.e., Li2 \(=\) PolyLog[2, \#{}]\&{}.

See also: Li3.

\Subsubsection*{Examples}

\dispSFinmath{
{{\epsilon }^{\alpha \beta \gamma \delta }}
}

\dispSFoutmath{
\Mfunction{Clear}[\Mvariable{t1}]
}

\dispSFinmath{
\{ \}
}


\dispSFinmath{
\Muserfunction{Li2}[x]
}

\dispSFoutmath{
{{\Mvariable{Li}}_2}(x)
}

\Subsection*{Li3 }

\Subsubsection*{Description}

 Li3 is an abbreviation for the trilog function, i.e., Li3 \(=\) PolyLog[3, \#{}]\&{}.

See also: Li2.

\Subsubsection*{Examples}

\dispSFinmath{
\Muserfunction{Li2}//\Mfunction{StandardForm}
}

\dispSFoutmath{
\Mfunction{PolyLog}[2,\#1]\&
}

\dispSFinmath{
-\int \frac{\log [1-x]}{x}\DifferentialD x
}

\dispSFoutmath{
{{\Mvariable{Li}}_2}(x)
}

\dispSFinmath{
\Muserfunction{Li3}[x]
}

\dispSFoutmath{
{{\Mvariable{Li}}_3}(x)
}

\dispSFinmath{
\Muserfunction{Li3}//\Mfunction{StandardForm}
}

\dispSFoutmath{
\Mfunction{PolyLog}[3,\#1]\&
}

\Subsection*{Loop}

\Subsubsection*{Description}

Loop is an option indicating the number of (virtual) loops.

\Subsection*{LorentzIndex}

\Subsubsection*{Description}

LorentzIndex[mu] denotes a four dimensional Lorentz index. For other than four dimensions: LorentzIndex[mu, D] or LorentzIndex[mu], etc.
  LorentzIndex[mu, 4] simplifies to LorentzIndex[mu].

See also:  ChangeDimension, Momentum.

\Subsubsection*{Examples}

This denotes a four-dimensional Lorentz index.

\dispSFinmath{
\Mfunction{D}[\Muserfunction{Li3}[x],x]
}

\dispSFoutmath{
\frac{{{\Mvariable{Li}}_2}(x)}{x}
}

An optional second argument can be given for a dimension different from 4.

\dispSFinmath{
\int \frac{\Muserfunction{Li2}[x]}{x}\DifferentialD x
}

\dispSFoutmath{
{{\Mvariable{Li}}_3}(x)
}

\dispSFinmath{
\Muserfunction{LorentzIndex}[\alpha ]
}

\dispSFoutmath{
\alpha
}

\dispSFinmath{
\Muserfunction{LorentzIndex}[\alpha ,n]
}

\dispSFoutmath{
\alpha
}

\dispSFinmath{
\Muserfunction{LorentzIndex}[\Muserfunction{ComplexIndex}[\alpha ]]
}

\dispSFoutmath{
{{\alpha }^*}
}

Setting \${}LorentzIndices\(=\)True displays the dimension (if different from 4) as a subindex.

\dispSFinmath{
\Muserfunction{ComplexConjugate}[\Muserfunction{LorentzIndex}[\mu ]]
}

\dispSFinmath{
\mu
}

\dispSFoutmath{
\Muserfunction{ComplexConjugate}[\Muserfunction{LorentzIndex}[\Muserfunction{ComplexIndex}[\beta ]]]
}

\dispSFinmath{
{{\beta }^*}
}

\Subsection*{Lower}

\Subsubsection*{Description}

Lower may be used inside LorentzIndex to indicate an covariant LorentzIndex.

See also:  LorentzIndex, Upper, Contract1.

\Subsection*{MakeContext}

\Subsubsection*{Description}

MakeContext[string] constructs the context path of string. MakeContext is invoked at startup of FeynCalc. MakeContext[a, b] construct the
  context path of b defined in the context of a.

See also:  FeynCalc, CheckContext, \${}FeynCalcStuff, Load.

\Subsection*{Mandelstam}

\Subsubsection*{Description}

Mandelstam is an option for DiracTrace, OneLoop, OneLoopSum, Tr and TrickMandelstam. { }A typical setting is Mandelstam \(\rightarrow \)
  \{s, t, u, \(\$LorentzIndices=\Mvariable{True};\) \(+\) \(\{\Muserfunction{LorentzIndex}[\alpha ,D],\Muserfunction{LorentzIndex}[\nu ,n],\Muserfunction{LorentzIndex}[\beta
,4]\}\) \(+\) \(\{{{\alpha }_D},{{\nu }_n},\beta \}\) \(+\) \(\$LorentzIndices\multsp =\multsp \Mvariable{False};\)\}, which stands for { }s \(+\)
t \(+\) u \(=\) \({{{m_1}}^2}\) \(+\) \({{{m_2}}^2}\) \(+\) \({{{m_3}}^2}\) \(+\) \({{{m_4}}^2}\). If other than four-particle processes are calculated
the setting should be: Mandelstam \(\rightarrow \) \{\}.

See also:  DiracTrace, OneLoop, OneLoopSum, Tr, TrickMandelstam, SetMandelstam.

\Subsection*{Map2}

\Subsubsection*{Description}

Map2[f, exp] is equivalent to Map if Nterms[exp] \(>\) 0, otherwise Map2[f, exp] gives f[exp].

See also:  NTerms.

\Subsubsection*{Examples}

\dispSFinmath{
{{{m_1}}^2}
}

\dispSFoutmath{
{{{m_2}}^2}
}

\dispSFinmath{
{{{m_3}}^2}
}

\dispSFoutmath{
{{{m_4}}^2}
}

\dispSFinmath{
\Muserfunction{Map2}[f,a-b]
}

\dispSFoutmath{
f(a)+f(-b)
}

\dispSFinmath{
\Muserfunction{Map2}[f,x]
}

\dispSFoutmath{
f(x)
}

\Subsection*{MemoryAvailable}

\Subsubsection*{Description}

MemoryAvailable is an option of MemSet. It can be set to an integer n, where n is the available amount of main memory in Mega Byte. The
  default setting is \${}MemoryAvailable.

See also:  MemSet, \${}MemoryAvailable.

\Subsection*{MemSet}

\Subsubsection*{Description}

MemSet[f[x\_{}], body] is like f[x\_{}] :\(=\) f[x] \(=\) body, but dependend on the value of the setting of MemoryAvailable
  \(\rightarrow \) memorycut (memorycut - MemoryInUse[]/\(\Muserfunction{Map2}[f,\{a,b,c\}]\)) MemSet[f[x\_{}], body] may evaluate as f[x\_{}] :\(=\)
body.

See also:  MemoryAvailable.

\Subsection*{MetricTensor}

\Subsubsection*{Description}

MetricTensor[\(f(\{a,b,c\})\)] is the metric tensor. The default dimension is 4.

See also:  FeynCalcExternal, FCE, FCI, MT, MTD.

\Subsubsection*{Examples}

\dispSFinmath{
\Muserfunction{Map2}[f,1]
}

\dispSFoutmath{
f(1)
}

\dispSFinmath{
{{10}^6}
}

\dispSFoutmath{
\mu ,\multsp \nu
}

\dispSFinmath{
\Muserfunction{MetricTensor}[\alpha ,\beta ]
}

\dispSFoutmath{
{g^{\alpha \beta }}
}

\dispSFinmath{
\Muserfunction{Contract}[\%\multsp \%]
}

\dispSFoutmath{
4
}

\dispSFinmath{
\Muserfunction{MetricTensor}[\alpha ,\beta ,\Mvariable{Dimension}\rightarrow n]
}

\dispSFoutmath{
\Muserfunction{MetricTensor}(\alpha ,\beta ,\Mvariable{Dimension}\rightarrow n)
}

\dispSFinmath{
\Muserfunction{Contract}[\%\multsp \%]
}

\dispSFoutmath{
n
}

\dispSFinmath{
\Mfunction{StandardForm}[\Muserfunction{MetricTensor}[a,b]]
}

\dispSFoutmath{
\Muserfunction{MetricTensor}[a,b]
}

\dispSFinmath{
\Mfunction{StandardForm}[\Muserfunction{MetricTensor}[a,b,\Mvariable{Dimension}\rightarrow D]]
}

\dispSFoutmath{
\Muserfunction{MetricTensor}[a,b,\Mvariable{Dimension}\rightarrow D]
}

\Subsection*{MLimit}

\Subsubsection*{Description}

MLimit[expr, lims] takes multiple limits of expr using the limits lims.

\Subsubsection*{Examples}

\dispSFinmath{
\Mfunction{StandardForm}[\Muserfunction{FCE}[\Muserfunction{MetricTensor}[a,b]]]
}

\dispSFoutmath{
\Muserfunction{MT}[a,b]
}

\Subsection*{Momentum}

\Subsubsection*{Description}

Momentum[p] is the head of a four momentum (p). The internal representation of a four-dimensional p is Momentum[p]. For other than four
  dimensions: Momentum[p, dim]. Momentum[p, 4] simplifies to Momentum[p].

See also:  DiracGamma, Eps, LorentzIndex, MomentumExpand.

\Subsubsection*{Examples}

This is a four dimensional momentum.

\dispSFinmath{
\Mfunction{StandardForm}[\Muserfunction{FCE}[\Muserfunction{MetricTensor}[a,b,\Mvariable{Dimension}\rightarrow D]]]
}

\dispSFoutmath{
\Muserfunction{MTD}[a,b]
}

As an optional second argument the dimension must be specified if it is different from 4.

\dispSFinmath{
\Muserfunction{MLimit}[y\multsp \log [y]+\sin [x-1]/(x-1),x\rightarrow 1,y\rightarrow 0]
}

\dispSFoutmath{
1
}

The dimension index is supressed in the output.

\dispSFinmath{
\Muserfunction{Momentum}[p]
}

\dispSFoutmath{
p
}

\dispSFinmath{
\Muserfunction{Momentum}[p,D]
}

\dispSFoutmath{
p
}

\dispSFinmath{
\Muserfunction{Momentum}[p,d]//\Mfunction{StandardForm}
}

\dispSFoutmath{
\Muserfunction{Momentum}[p,d]
}

\dispSFinmath{
\Mvariable{a1}=\Muserfunction{Momentum}[-q]
}

\dispSFoutmath{
-q
}

\dispSFinmath{
\Muserfunction{a1}//\Mfunction{StandardForm}
}

\dispSFoutmath{
-\Muserfunction{Momentum}[q]
}

\dispSFinmath{
\Mvariable{a2}\multsp =\multsp \Muserfunction{Momentum}[p-q]\multsp +\multsp \Muserfunction{Momentum}[2q]
}

\dispSFoutmath{
(p-q)+2\multsp q
}

\dispSFinmath{
\Muserfunction{a2}//\Mfunction{StandardForm}
}

\dispSFoutmath{
\Muserfunction{Momentum}[p-q]+2\multsp \Muserfunction{Momentum}[q]
}

\dispSFinmath{
\Muserfunction{a2}//\Muserfunction{MomentumExpand}//\Mfunction{StandardForm}
}

\dispSFoutmath{
\Muserfunction{Momentum}[p]+\Muserfunction{Momentum}[q]
}

\dispSFinmath{
\Muserfunction{a2}//\Muserfunction{MomentumCombine}//\Mfunction{StandardForm}
}

\Subsection*{MomentumCombine}

\Subsubsection*{Description}

MomentumCombine[expr] is the inverse operation to MomentumExpand and ExpandScalarProduct. MomentumCombine combines also Pair`s.

See also: ExpandScalarProduct, Momentum, MomentumExpand, MomentumCombine2.

\Subsubsection*{Examples}

\dispSFinmath{
\Muserfunction{Momentum}[p+q]
}

\dispSFoutmath{
\Muserfunction{ChangeDimension}[\Muserfunction{Momentum}[p],d]//\Mfunction{StandardForm}
}

\dispSFinmath{
\Muserfunction{Momentum}[p,d]
}

\dispSFoutmath{
\Mfunction{Clear}[\Mvariable{a1},\Mvariable{a2}]
}

\dispSFinmath{
\Muserfunction{Momentum}[p]-2\multsp \Muserfunction{Momentum}[q]\multsp //\multsp \Mvariable{MomentumCombine}\multsp //
   \Mfunction{\multsp }\Mvariable{StandardForm}
}

\dispSFoutmath{
\Muserfunction{Momentum}[p-2\multsp q]
}

\dispSFinmath{
\Mvariable{t1}=\Muserfunction{FV}[p,\mu ]\multsp +\multsp 2\multsp \Muserfunction{FV}[q,\mu ]\multsp
}

\dispSFoutmath{
{p^{\mu }}+2\multsp {q^{\mu }}
}

\dispSFinmath{
\Muserfunction{MomentumCombine}[\Mvariable{t1}]
}

\dispSFoutmath{
{{(p+2\multsp q)}^{\mu }}
}

\dispSFinmath{
\Muserfunction{MomentumCombine}[\Mvariable{t1}]//\Mfunction{StandardForm}
}

\dispSFoutmath{
\Muserfunction{Pair}[\Muserfunction{LorentzIndex}[\mu ],\Muserfunction{Momentum}[p+2\multsp q]]
}

\dispSFinmath{
\Muserfunction{MomentumCombine}[\Mvariable{t1}]//\Muserfunction{ExpandScalarProduct}
}

\Subsection*{MomentumCombine2}

\Subsubsection*{Description}

MomentumCombine2[expr] { }is the inverse operation to MomentumExpand and ExpandScalarProduct. MomentumCombine2 combines also FourVectors.

See also: MomentumCombine, MomentumExpand, ExpandScalarProduct, FourVector.

\Subsubsection*{Examples}

\dispSFinmath{
{p^{\mu }}+2\multsp {q^{\mu }}
}

\dispSFoutmath{
\Mfunction{StandardForm}[\%]
}

\dispSFinmath{
\Muserfunction{Pair}[\Muserfunction{LorentzIndex}[\mu ],\Muserfunction{Momentum}[p]]+
   2\multsp \Muserfunction{Pair}[\Muserfunction{LorentzIndex}[\mu ],\Muserfunction{Momentum}[q]]
}

\dispSFoutmath{
\Mfunction{Clear}[\Mvariable{t1}]
}

\Subsection*{MomentumExpand}

\Subsubsection*{Description}

MomentumExpand[expr] expands Momentum[a\(+\)b\(+\) ...] in expr into Momentum[a] \(+\) Momentum[b] \(+\) ... .

See also:  ExpandScalarProduct, MomentumCombine.

\Subsubsection*{Examples}

\dispSFinmath{
3\multsp \Muserfunction{Pair}[\Muserfunction{LorentzIndex}[\mu ],\Muserfunction{Momentum}[p]]+
   2\multsp \Muserfunction{Pair}[\Muserfunction{LorentzIndex}[\mu ],\Muserfunction{Momentum}[q]]
}

\dispSFoutmath{
3\multsp {p^{\mu }}+2\multsp {q^{\mu }}
}

\dispSFinmath{
\Muserfunction{MomentumCombine2}[\%]//\Mfunction{StandardForm}
}

\dispSFoutmath{
\Muserfunction{Pair}[\Muserfunction{LorentzIndex}[\mu ],\Muserfunction{Momentum}[3\multsp p+2\multsp q]]
}

\dispSFinmath{
\Muserfunction{MomentumExpand}[\Muserfunction{Momentum}[p+q]]//\Mfunction{StandardForm}
}

\dispSFoutmath{
\Muserfunction{Momentum}[p]+\Muserfunction{Momentum}[q]
}

\dispSFinmath{
\Muserfunction{ScalarProduct}[p+q,r]
}

\dispSFoutmath{
(p+q)\cdot r
}

\dispSFinmath{
\%//\Mfunction{StandardForm}
}

\dispSFoutmath{
\Muserfunction{Pair}[\Muserfunction{Momentum}[p+q],\Muserfunction{Momentum}[r]]
}

\dispSFinmath{
\Muserfunction{MomentumExpand}[\Muserfunction{ScalarProduct}[p+q,r]]
}

\dispSFoutmath{
(p+q).(r)
}

\dispSFinmath{
\%//\Mfunction{StandardForm}
}

\dispSFoutmath{
\Muserfunction{Pair}[\Muserfunction{Momentum}[p]+\Muserfunction{Momentum}[q],\Muserfunction{Momentum}[r]]
}

\dispSFinmath{
\Muserfunction{MomentumExpand}[\Muserfunction{ScalarProduct}[p+q,r-p]]
}

\dispSFoutmath{
(p+q).(r-p)
}

\Subsection*{MT}

\Subsubsection*{Description}

MT[\(\%//\Mfunction{StandardForm}\)] is the metric tensor in 4 dimensions.

See also:  FeynCalcExternal, FCE, FCI, MetricTensor, MTD.

\Subsubsection*{Examples}

\dispSFinmath{
\Muserfunction{Pair}[\Muserfunction{Momentum}[p]+\Muserfunction{Momentum}[q],-\Muserfunction{Momentum}[p]+\Muserfunction{Momentum}[r]]
}

\dispSFoutmath{
\Muserfunction{Calc}[\%]
}

\dispSFinmath{
-{p^2}-p\cdot q+p\cdot r+q\cdot r
}

\dispSFoutmath{
\mu ,\multsp \nu
}

\dispSFinmath{
\Muserfunction{MT}[\alpha ,\beta ]
}

\dispSFoutmath{
{g^{\alpha \beta }}
}

\dispSFinmath{
\Muserfunction{Contract}[\Muserfunction{MT}[\alpha ,\beta ]\multsp \Muserfunction{MT}[\alpha ,\beta ]]
}

\dispSFoutmath{
4
}

\dispSFinmath{
\Muserfunction{MT}[a,b]//\Mfunction{StandardForm}
}

\dispSFoutmath{
\Muserfunction{MT}[a,b]
}

\Subsection*{MTD}

\Subsubsection*{Description}

MTD[\(\Muserfunction{FCI}[\Muserfunction{MT}[a,b]]//\Mfunction{StandardForm}\)] is the metric tensor in D dimensions.

See also:  FeynCalcExternal, FCE, FCI, MetricTensor, MT.

\Subsubsection*{Examples}

\dispSFinmath{
\Muserfunction{Pair}[\Muserfunction{LorentzIndex}[a],\Muserfunction{LorentzIndex}[b]]
}

\dispSFoutmath{
\Muserfunction{FCE}[\Muserfunction{FCI}[\Muserfunction{MT}[a,b]]]//\Mfunction{StandardForm}
}

\dispSFinmath{
\Muserfunction{MT}[a,b]
}

\dispSFoutmath{
\mu ,\multsp \nu
}

\dispSFinmath{
\Muserfunction{MTD}[\alpha ,\beta ]
}

\dispSFoutmath{
{g^{\alpha \beta }}
}

\dispSFinmath{
\Muserfunction{Contract}[\Muserfunction{MTD}[\alpha ,\beta ]\multsp \Muserfunction{MTD}[\alpha ,\beta ]]
}

\dispSFoutmath{
D
}

\dispSFinmath{
\Muserfunction{MTD}[\alpha ,\beta ]//\Mfunction{StandardForm}
}

\dispSFoutmath{
\Muserfunction{MTD}[\alpha ,\beta ]
}

\Subsection*{NegativeInteger}

\Subsubsection*{Description}

NegativeInteger is a data type. E.g. DataType[n, NegativeInteger] can be set to True.

See also:  DataType.

\Subsection*{Nf}

\Subsubsection*{Description}

Nf denotes the number of flavors.

See also:  Amplitude, CounterTerm, Convolute, AnomalousDimension, SplittingFunction.

\Subsection*{Nielsen}

\Subsubsection*{Description}

Nielsen[i,j, x] denotes Nielsen's polylogarithm.

\dispSFinmath{
\Muserfunction{FCI}[\Muserfunction{MTD}[\alpha ,\beta ]]//\Mfunction{StandardForm}
}

\dispSFoutmath{
\Muserfunction{Pair}[\Muserfunction{LorentzIndex}[\alpha ,D],\Muserfunction{LorentzIndex}[\beta ,D]]
}

See also:  SimplifyPolyLog.

\Subsubsection*{Examples}

\dispSFinmath{
\Muserfunction{FCE}[\Muserfunction{FCI}[\Muserfunction{MTD}[\mu ,\nu ]]]//\Mfunction{StandardForm}
}

\dispSFoutmath{
\Muserfunction{MTD}[\mu ,\nu ]
}

Numerical evaluation is done via N[Nielsen[n\_{},p\_{},x\_{}]] :\(=\) (-1)\(\RawWedge\)(n\(+\)p-1)/(n-1)!/p! NIntegrate[Log[1-x
  t]\(\RawWedge\)p Log[t]\(\RawWedge\)(n-1)/t,\{t,0,1\}];

\dispSFinmath{
\Mfunction{Options}[\Mvariable{Nielsen}]
}

\dispSFoutmath{
\{\Mvariable{PolyLog}\rightarrow \Mvariable{False}\}
}

Some special values are built in.

\dispSFinmath{
\Muserfunction{Nielsen}[1,2,x]
}

\dispSFoutmath{
{S_{12}}(x)
}

\dispSFinmath{
\Mfunction{N}[\Muserfunction{Nielsen}[1,2,0.45]]
}

\dispSFoutmath{
0.0728716
}

\dispSFinmath{
\{\Muserfunction{Nielsen}[1,2,0],\Muserfunction{Nielsen}[1,2,-1],\Muserfunction{Nielsen}[1,2,1/2],\Muserfunction{Nielsen}[1,2,1]\}
}

\dispSFoutmath{
\big\{0,\frac{\zeta (3)}{8},\frac{\zeta (3)}{8},\zeta (3)\big\}
}

\dispSFinmath{
\Muserfunction{Nielsen}[1,2,x,\Mvariable{PolyLog}\rightarrow \Mvariable{True}]
}

\dispSFoutmath{
\frac{1}{2}\multsp \log(x)\multsp {{\log}^2}(1-x)+{{\Mvariable{Li}}_2}(1-x)\multsp \log(1-x)-{{\Mvariable{Li}}_3}(1-x)+\zeta (3)
}

\Subsection*{NonCommFreeQ}

\Subsubsection*{Description}

NonCommFreeQ[exp] yields True if exp contains no non-commutative objects (i.e. those objects which are listed in \${}NonComm) or only
  non-commutative objects inside DiracTrace's or SUNTrace's.

See also:  \${}NonComm, NonCommQ, DiracTrace, SUNTrace.

\Subsection*{NonCommQ}

\Subsubsection*{Description}

NonCommQ[exp] yields True if exp contains non-commutative objects (i.e. those objects which are listed in \${}NonComm) not inside
  DiracTrace's or SUNTrace's.

See also:  \${}NonComm, NonCommFreeQ, DiracTrace, SUNTrace.

\Subsection*{NonCommutative}

\Subsubsection*{Description}

NonCommutative is a data type which may be used, e.g., { }as: DataType[x, NonCommutative] \(=\) True.

See also: DataType, DeclareNonCommutative.

\Subsection*{NTerms}

\Subsubsection*{Description}

NTerms[x] is equivalent to Length if x is a sum; otherwise NTerms[x] returns 1, except NTerms[0] \(\rightarrow \) 0.

\Subsubsection*{Examples}

\dispSFinmath{
\Muserfunction{Nielsen}[1,3,x,\Mvariable{PolyLog}\rightarrow \Mvariable{True}]
}

\dispSFoutmath{
-\frac{1}{6}\multsp \log(x)\multsp {{\log}^3}(1-x)-\frac{1}{2}\multsp {{\Mvariable{Li}}_2}(1-x)\multsp {{\log}^2}(1-x)+
   {{\Mvariable{Li}}_3}(1-x)\multsp \log(1-x)-{{\Mvariable{Li}}_4}(1-x)+\frac{{{\pi }^4}}{90}
}

\dispSFinmath{
\Muserfunction{Nielsen}[3,1,x,\Mvariable{PolyLog}\rightarrow \Mvariable{True}]
}

\dispSFoutmath{
{{\Mvariable{Li}}_4}(x)
}

\dispSFinmath{
\Muserfunction{NTerms}[a-b]
}

\dispSFoutmath{
2
}

\dispSFinmath{
\Muserfunction{NTerms}[a\multsp b\multsp c]
}

\dispSFoutmath{
1
}

\Subsection*{NumberOfMetricTensors}

\Subsubsection*{Description}

NumberOfMetricTensors is an option of Tdec.

See also:  Tdec.

\Subsection*{NumericalFactor}

\Subsubsection*{Description}

NumericalFactor[expr] gives the overall numerical factor of expr.

\Subsection*{NumericQ1}

\Subsubsection*{Description}

NumericQ1[x,\{a,b,..\}] is like NumericQ, but assumes that \{a,b,..\} are numeric quantities.

\Subsubsection*{Examples}

\dispSFinmath{
\Muserfunction{NTerms}[9]
}

\dispSFoutmath{
1
}

\dispSFinmath{
\Muserfunction{NTerms}[0]
}

\dispSFoutmath{
0
}

\dispSFinmath{
\Mfunction{NumericQ}\big[3\multsp a+\log [b]+{c^2}\big]
}

\dispSFoutmath{
\Mvariable{False}
}

\Subsection*{OneLoop { }***unfinished***}

See also the old FeynCalc documentation at http://www.mertig.com/oldfc/. NOTICE: While OneLoop is restricted to 't Hooft Feynman gauge the function
OneLoopSimplify does not have this restriction (but is usually slower). OneLoop handles selfenergies, vertex and box-graphs (those only up to 3rd
rank
  tensor in the integration variable).

WARNING: If you encounter anomalies: \(\Muserfunction{NumericQ1}\big[3\multsp a+\log [b]+{c^2},\{\}\big]\) is calculated in D dimensions has changed
compared to the old FeynCalc version. Please keep in mind that the issue of \(\Mvariable{False}\)schemes is inherintly tricky.

\Subsubsection*{Description}

OneLoop[q, amplitude] calculates the one-loop Feynman diagram amplitude ({\itshape n}-point, where {\itshape n}\(\leq \)4 and the highest tensor
rank of the integration momenta (after cancellation of scalar products) may be 3; unless
  OneLoopSimplify is used).

The argument q denotes the integration variable, i.e., the loop momentum. OneLoop[name, q, amplitude] has as first argument a name of the
  amplitude. If the second argument has head FeynAmp then OneLoop[q, FeynAmp[name, k, expr]] and OneLoop[FeynAmp[name, k, expr]] tranform
  to OneLoop[name, k, expr].

\dispSFinmath{
\Muserfunction{NumericQ1}\big[3\multsp a+\log [b]+{c^2},\{a,b,c\}\big]
}

\dispSFoutmath{
\Mvariable{True}
}

See also:  B0, C0, D0, OneLoopSimplify, TID, TIDL, \${}LimitTo4.

\Subsubsection*{Examples}

\dispSFinmath{
\MathBegin{MathArray}{l}
\Mvariable{The}\multsp \Mvariable{default}\multsp \Mvariable{setting}\multsp \Mvariable{of}\multsp \$West
    \multsp \Mvariable{is}\multsp \Mvariable{True},\multsp   \\
\noalign{\vspace{0.604167ex}}i.e.,\multsp
   \Mvariable{the}\multsp \Mvariable{way}\multsp \Muserfunction{tr}(
     {{\gamma }^{\mu }}{{\gamma }^{\nu }}{{\gamma }^{\rho }}{{\gamma }^{\sigma }}{{\gamma }^{\tau }}{{\gamma }^{\lambda }}{{\gamma }^5})\
   \
\MathEnd{MathArray}
}

\dispSFoutmath{
{{\gamma }^5}
}

\dispSFinmath{
\Mfunction{Options}[\Mvariable{OneLoop}]
}

\dispSFoutmath{
\MathBegin{MathArray}{l}
\{\Mvariable{Apart2}\rightarrow \Mvariable{True},\Mvariable{CancelQP}\rightarrow \Mvariable{True},
    \Mvariable{DenominatorOrder}\rightarrow \Mvariable{False},\Mvariable{Dimension}\rightarrow D,  \\
\noalign{\vspace{0.666667ex}}
   \hspace{1.em} \Mvariable{FinalSubstitutions}\rightarrow \{\},\Mvariable{Factoring}\rightarrow \Mvariable{False},
   \Mvariable{FormatType}\rightarrow \Mvariable{InputForm},\Mvariable{InitialSubstitutions}\rightarrow \{\},  \\
\noalign{\vspace{
   0.666667ex}}
\hspace{1.em} \Mvariable{IntermediateSubstitutions}\rightarrow \{\},\Mvariable{IsolateNames}\rightarrow \Mvariable{False}
   ,\Mvariable{Mandelstam}\rightarrow \{\},\Mvariable{OneLoopSimplify}\rightarrow \Mvariable{False},  \\
\noalign{\vspace{
   0.666667ex}}
\hspace{1.em} \Mvariable{Prefactor}\rightarrow 1,\Mvariable{ReduceGamma}\rightarrow \Mvariable{False},
   \Mvariable{ReduceToScalars}\rightarrow \Mvariable{False},\Mvariable{SmallVariables}\rightarrow \{\},  \\
\noalign{\vspace{
   0.666667ex}}
\hspace{1.em} \Mvariable{WriteOut}\rightarrow \Mvariable{False},
    \Mvariable{WriteOutPaVe}\rightarrow /opt/cvs/HighEnergyPhysics/Phi/Storage/,\Mvariable{Sum}\rightarrow \Mvariable{True}\}\\
   \MathEnd{MathArray}
}

\dispSFinmath{
-\imag /\pi \RawWedge 2\multsp \Muserfunction{FAD}[\{q,m\}]//\Muserfunction{FCI}
}

Remember that FAD[\{q,mf\},\{q-k,mf\}] is a fast possibility to enter \(-\frac{\ImaginaryI }{{{\pi }^2}\multsp ({q^2}-{m^2})}\)

\dispSFinmath{
\Muserfunction{OneLoop}[q,\%]
}

\dispSFoutmath{
{A_0}({m^2})
}

The input to OneLoop may be in 4 dimensions, since the function changes the dimension of the objects automatically to the setting of the
  Dimension option (D by default).

\dispSFinmath{
\Mvariable{mf}/:\Mfunction{MakeBoxes}[\Mvariable{mf},\Mvariable{TraditionalForm}]=
    \Mfunction{InterpretationBox}[\Mfunction{SubscriptBox}["m","f"],\Mvariable{mf}];
}

\dispSFoutmath{
1/(\multsp ({q^2}-{{\Mvariable{mf}}^2})\multsp (\multsp {{(q-k)}^2}-{{\Mvariable{mf}}^2})\multsp ).
}

\dispSFinmath{
\MathBegin{MathArray}{l}
t=\imag \multsp \frac{{{\Mvariable{el}}^2}}{16\multsp {{\pi }^4}(1-D)}
    \Muserfunction{FAD}[\{q,\Mvariable{mf}\},\{q-k,\Mvariable{mf}\}]  \\
\noalign{\vspace{1.02083ex}}
\hspace{3.em} \Muserfunction{DiracT
      race}[(\Mvariable{mf}+\Muserfunction{GS}[q-k]).\Muserfunction{GA}[\mu ].(\Mvariable{mf}+\Muserfunction{GS}[q]).
      \Muserfunction{GA}[\mu ]]\multsp //\multsp \Mvariable{FCI}\\
\MathEnd{MathArray}
}

\dispSFoutmath{
\frac{\ImaginaryI \multsp {{\Mvariable{el}}^2}\multsp \Muserfunction{tr}(
      ({m_f}+\gamma \cdot (q-k)).{{\gamma }^{\mu }}.({m_f}+\gamma \cdot q).{{\gamma }^{\mu }})}{16\multsp (1-D)\multsp {{\pi }^4}\multsp
     ({q^2}-{{{m_f}}^2}).\big({{(q-k)}^2}-{{{m_f}}^2}\big)}
}

\dispSFinmath{
\Muserfunction{OneLoop}[q,t]
}

\dispSFoutmath{
-\frac{{{\Mvariable{el}}^2}\multsp \big(-\frac{8}{3}\multsp {B_0}\big({k^2},{{{m_f}}^2},{{{m_f}}^2}\big)\multsp {{{m_f}}^2}-
        \frac{8\multsp {{{m_f}}^2}}{3}+\frac{8}{3}\multsp {A_0}({{{m_f}}^2})-
        \frac{4}{3}\multsp {B_0}\big({k^2},{{{m_f}}^2},{{{m_f}}^2}\big)\multsp {k^2}+\frac{4\multsp {k^2}}{9}\big)}{16\multsp {{\pi }^2}}
}

\dispSFinmath{
\Mfunction{FullSimplify}[\%]
}

\dispSFoutmath{
\frac{{{\Mvariable{el}}^2}\multsp \big(6\multsp {{{m_f}}^2}-6\multsp {A_0}({{{m_f}}^2})-{k^2}+
       3\multsp {B_0}\big({k^2},{{{m_f}}^2},{{{m_f}}^2}\big)\multsp \big(2\multsp {{{m_f}}^2}+{k^2}\big)\big)}{36\multsp {{\pi }^2}}
}

\dispSFinmath{
\Muserfunction{SP}[k,r]\multsp \Muserfunction{FAD}[\{k,m\}\multsp ,\multsp k\multsp -\multsp p]//\Muserfunction{FCI}
}

\Subsection*{OneLoopSimplify}

\Subsubsection*{Description}

OneLoopSimplify[amp, q] simplifies the one-loop amplitude amp. The second argument denotes the integration momentum.

\dispSFinmath{
\frac{k\cdot r}{\big({k^2}-{m^2}\big).{{(k-p)}^2}}
}

\dispSFoutmath{
\Muserfunction{OneLoop}[k,\%]
}

See also:  OneLoop, TID, TIDL.

\Subsubsection*{Examples}

\dispSFinmath{
\ImaginaryI \multsp {{\pi }^2}\multsp \bigg(-\frac{{B_0}(0,{m^2},{m^2})\multsp p\cdot r\multsp {m^2}}{2\multsp {p^2}}+
     \frac{{B_0}({p^2},0,{m^2})\multsp p\cdot r\multsp {m^2}}{2\multsp {p^2}}-\frac{p\cdot r\multsp {m^2}}{2\multsp {p^2}}+
     \frac{1}{2}\multsp {B_0}({p^2},0,{m^2})\multsp p\cdot r\bigg)
}

\dispSFoutmath{
\Mfunction{Clear}[t]
}

\dispSFinmath{
\Mfunction{Options}[\Mvariable{OneLoopSimplify}]
}

\dispSFoutmath{
\MathBegin{MathArray}{l}
\{\Mvariable{Collecting}\rightarrow \Mvariable{False},\Mvariable{Dimension}\rightarrow D,
    \Mvariable{DimensionalReduction}\rightarrow \Mvariable{False},  \\
\noalign{\vspace{0.666667ex}}
\hspace{1.em} \Mvariable{DiracSimpli
     fy}\rightarrow \Mvariable{True},\Mvariable{FinalSubstitutions}\rightarrow \{\},\Mvariable{IntegralTable}\rightarrow \{\},
   \Mvariable{OPE1Loop}\rightarrow \Mvariable{False},  \\
\noalign{\vspace{0.666667ex}}
\hspace{1.em} \Mvariable{ScalarProductCancel}
     \rightarrow \Mvariable{True},\Mvariable{SUNNToCACF}\rightarrow \Mvariable{True},\Mvariable{SUNTrace}\rightarrow \Mvariable{False}\}\
   \
\MathEnd{MathArray}
}

\dispSFinmath{
\Muserfunction{SP}[k,r]\multsp \Muserfunction{FAD}[\{k,m\}\multsp ,\multsp k\multsp -\multsp p]//\Muserfunction{FCI}
}

\dispSFoutmath{
\frac{k\cdot r}{\big({k^2}-{m^2}\big).{{(k-p)}^2}}
}

\dispSFinmath{
\Muserfunction{OneLoopSimplify}[\%,k]
}

\dispSFoutmath{
\frac{({m^2}+{p^2})\multsp p\cdot r}{2\multsp \big({k^2}-{m^2}\big).{{(k-p)}^2}\multsp {p^2}}-
   \frac{p\cdot r}{2\multsp \big({k^2}-{m^2}\big)\multsp {p^2}}
}

\dispSFinmath{
\Muserfunction{OneLoopSimplify}[\%/.m\rightarrow 0,k]
}

\dispSFoutmath{
\frac{p\cdot r}{2\multsp {k^2}.{{(k-p)}^2}}
}

\dispSFinmath{
\Muserfunction{FAD}[k,k,\multsp k\multsp -\multsp {p_1},\multsp k\multsp -\multsp {p_2}]\multsp \Muserfunction{FVD}[k,\mu ]//
   \Muserfunction{FCI}
}

\dispSFoutmath{
\frac{{k^{\mu }}}{{k^2}.{k^2}.{{(k-{p_1})}^2}.{{(k-{p_2})}^2}}
}

\dispSFinmath{
\Muserfunction{OneLoopSimplify}[\multsp \%,k]
}

\dispSFoutmath{
\MathBegin{MathArray}{l}
\frac{p_{2}^{\mu }\multsp p_{1}^{2}-p_{1}^{\mu }\multsp {p_1}\cdot {p_2}}
    {2\multsp {k^2}.{k^2}.{{(k-{p_1})}^2}\multsp ({{{p_1}\cdot {p_2}}^2}-p_{1}^{2}\multsp p_{2}^{2})}+
   \frac{p_{2}^{\mu }\multsp p_{1}^{2}\multsp {p_1}\cdot {p_2}+p_{1}^{\mu }\multsp p_{2}^{2}\multsp {p_1}\cdot {p_2}-
      p_{1}^{\mu }\multsp p_{1}^{2}\multsp p_{2}^{2}-p_{2}^{\mu }\multsp p_{1}^{2}\multsp p_{2}^{2}}{2\multsp
      {k^2}.{k^2}.{{(k-{p_1})}^2}.{{(k-{p_2})}^2}\multsp ({{{p_1}\cdot {p_2}}^2}-p_{1}^{2}\multsp p_{2}^{2})}-  \\
\noalign{\vspace{
   1.64583ex}}
\hspace{1.em} \frac{p_{2}^{\mu }\multsp {p_1}\cdot {p_2}-p_{1}^{\mu }\multsp p_{2}^{2}}
    {2\multsp {k^2}.{k^2}.{{(k-{p_2})}^2}\multsp ({{{p_1}\cdot {p_2}}^2}-p_{1}^{2}\multsp p_{2}^{2})}-
   \frac{p_{2}^{\mu }\multsp p_{1}^{2}-p_{1}^{\mu }\multsp {p_1}\cdot {p_2}-p_{2}^{\mu }\multsp {p_1}\cdot {p_2}+
      p_{1}^{\mu }\multsp p_{2}^{2}}{2\multsp {k^2}.{{(k-{p_1})}^2}.{{(k-{p_2})}^2}\multsp
      ({{{p_1}\cdot {p_2}}^2}-p_{1}^{2}\multsp p_{2}^{2})}\\
\MathEnd{MathArray}
}

\Subsection*{OneLoopSum}

\Subsubsection*{Description}

OneLoopSum[ FeynAmp[ ... ], FeynAmp[ ... ] , ...] will calculate a list of Feynman amplitudes by replacing FeynAmp step by step by
  OneLoop.

See also:  OneLoop.

\Subsection*{OPE}

\Subsubsection*{Description}

OPE is a convenience variable to separate OPE insertions.\\
OPE is also an option of several input functions like GluonPropagator.

The OPE variable has the property to vanish for higher powers:

\dispSFinmath{
\Muserfunction{FCE}[\%]/.\Muserfunction{SPD}[{p_1}]\rightarrow 0//\Muserfunction{FCI}
}

\dispSFoutmath{
\MathBegin{MathArray}{l}
\frac{p_{1}^{\mu }\multsp p_{2}^{2}}
    {2\multsp {k^2}.{k^2}.{{(k-{p_1})}^2}.{{(k-{p_2})}^2}\multsp {p_1}\cdot {p_2}}-
   \frac{p_{1}^{\mu }}{2\multsp {k^2}.{k^2}.{{(k-{p_1})}^2}\multsp {p_1}\cdot {p_2}}-  \\
\noalign{\vspace{1.60417ex}}
   \hspace{1.em} \frac{p_{2}^{\mu }\multsp {p_1}\cdot {p_2}-p_{1}^{\mu }\multsp p_{2}^{2}}
    {2\multsp {k^2}.{k^2}.{{(k-{p_2})}^2}\multsp {{{p_1}\cdot {p_2}}^2}}-
   \frac{-p_{1}^{\mu }\multsp {p_1}\cdot {p_2}-p_{2}^{\mu }\multsp {p_1}\cdot {p_2}+p_{1}^{\mu }\multsp p_{2}^{2}}
    {2\multsp {k^2}.{{(k-{p_1})}^2}.{{(k-{p_2})}^2}\multsp {{{p_1}\cdot {p_2}}^2}}\\
\MathEnd{MathArray}
}

See also:  OPE1Loop.

\Subsection*{OPEDelta}

\Subsubsection*{Description}

OPEDelta is a lightlike axial vector as used e.g. in the operator product expansion in QCD.

See also:  Twist2QuarkOperator.

\Subsubsection*{Examples}

\dispSFinmath{
\Muserfunction{OneLoopSimplify}[\Muserfunction{FAD}[k-{p_1},k-{p_2}]\multsp \Muserfunction{SP}[k,l]\RawWedge 2,k]
}

\dispSFoutmath{
-\frac{D\multsp {{l\cdot {p_1}}^2}+2\multsp D\multsp l\cdot {p_2}\multsp l\cdot {p_1}-4\multsp l\cdot {p_2}\multsp l\cdot {p_1}+
      D\multsp {{l\cdot {p_2}}^2}-{l^2}\multsp p_{1}^{2}+2\multsp {l^2}\multsp {p_1}\cdot {p_2}-{l^2}\multsp p_{2}^{2}}{4\multsp (1-D)
      \multsp {k^2}.{{(k-{p_1}+{p_2})}^2}}
}

\dispSFinmath{
\{\Mvariable{OPE},\Mvariable{OPE}\RawWedge 2,\Mvariable{OPE}\RawWedge 3\}
}

\dispSFoutmath{
\{\Omega ,0,0\}
}

\dispSFinmath{
\Muserfunction{FourVector}[\Mvariable{OPEDelta},\mu ]
}

\dispSFoutmath{
{{\Delta }_{\mu }}
}

\Subsection*{OPEi, OPEj, OPEk, OPEl, OPEm, OPEn, OPEo}

\Subsubsection*{Description}

OPEi, etc. are variables with DataType PositiveInteger which are used in functions relating to the operator product expansion.

\Subsubsection*{Examples}

\dispSFinmath{
\Muserfunction{Contract}[\%\multsp \%]
}

\dispSFoutmath{
0
}

\dispSFinmath{
\Muserfunction{SP}[\Mvariable{OPEDelta},\Mvariable{OPEDelta}]
}

\dispSFoutmath{
0
}

\dispSFinmath{
\Mvariable{OPEi}
}

\dispSFoutmath{
i
}

Re has been changed:

\dispSFinmath{
\Muserfunction{DataType}[\Mvariable{OPEi},\multsp \Mvariable{OPEj},\Mvariable{OPEk},\Mvariable{OPEl},\multsp \Mvariable{OPEm},\multsp
    \Mvariable{OPEn},\multsp \Mvariable{OPEo},\multsp \Mvariable{PositiveInteger}]
}

\dispSFoutmath{
\{\Mvariable{True},\Mvariable{True},\Mvariable{True},\Mvariable{True},\Mvariable{True},\Mvariable{True},\Mvariable{True}\}
}

\dispSFinmath{
\MathBegin{MathArray}{l}
\Muserfunction{PowerSimplify}[
   \{(-1)\RawWedge (2\Mvariable{OPEi}),(-1)\RawWedge (2\Mvariable{OPEj}),(-1)\RawWedge (2\Mvariable{OPEk}),  \\
   \noalign{\vspace{0.5ex}}
\hspace{2.em} (-1)\RawWedge (2\Mvariable{OPEl}),(-1)\RawWedge (2\Mvariable{OPEm}),
     (-1)\RawWedge (2\Mvariable{OPEn}),(-1)\RawWedge (2\Mvariable{OPEo})\}]\\
\MathEnd{MathArray}
}

\dispSFoutmath{
\{1,1,1,1,1,1,1\}
}

\dispSFinmath{
\{\Mfunction{Re}[\Mvariable{OPEi}]>-3,\multsp \Mfunction{Re}[\Mvariable{OPEi}]>-2,\multsp \Mfunction{Re}[\Mvariable{OPEi}]>-1,\multsp
    \multsp \multsp \Mfunction{Re}[\Mvariable{OPEi}]>0,\multsp \Mfunction{Re}[\Mvariable{OPEi}]>1\}
}

\dispSFoutmath{
\{\Mfunction{Re}(i)>-3,\Mfunction{Re}(i)>-2,\Mfunction{Re}(i)>-1,\Mfunction{Re}(i)>0,\Mfunction{Re}(i)>1\}
}

\Subsection*{OPEInsert { }***unfinished***}

\Subsubsection*{Description}

OPEInsert[diagram\_{}String, name\_{}, rhifile\_{}String] or OPEInsert[diagram\_{}String, name\_{}]. The setting of the option
  EpsContract (4 or D) determines whether the Levi-Civita tensors are contracted in 4 or D dimensions.

See also:  RHI.

\Subsubsection*{Examples}

\dispSFinmath{
\{\Mfunction{Re}[-\Mvariable{OPEi}\multsp +\multsp \Mvariable{OPEm}]\multsp >\multsp 0,\multsp
    \Mfunction{Re}[-\Mvariable{OPEi}\multsp +\multsp \Mvariable{OPEm}]\multsp >\multsp 1,
    \Mfunction{Re}[-\Mvariable{OPEi}\multsp +\multsp \Mvariable{OPEm}]\multsp >\multsp 2\}
}

\Subsection*{OPEInt { }***unfinished***}

\Subsubsection*{Description}

OPEInt[expr, q, p, x]. { }The dimension is changed to the one indicated by the option { }Dimension. The setting of the option EpsContract
  determines { }the dimension in which the Levi-Civita tensors are contracted.

See also:  RHI.

\Subsubsection*{Examples}

\dispSFinmath{
\{\Mfunction{Re}(m)-\Mfunction{Re}(i)>0,\Mfunction{Re}(m)-\Mfunction{Re}(i)>1,\Mfunction{Re}(m)-\Mfunction{Re}(i)>2\}
}

\Subsection*{OPEIntDelta { }***unfinished***}

\Subsubsection*{Description}

OPEIntDelta[expr, x, m] introduces the delta(1-x) (DeltaFunction[1-x]). The {\itshape Mathematica} Integrate function is called and each integration
{ }(from 0 to 1) is recorded for reference (and bug-checking) in the global list
  \${}MIntegrate. \\
Notice that the dimension specified by the option should also be the dimension used in expr. It is replaced in OPEIntDelta by
  (4\(+\)Epsilon).

See also:  RHI.

\Subsubsection*{Examples}

\dispSFinmath{
\{\Mfunction{Re}[\Mvariable{OPEm}]>-3,\multsp \Mfunction{Re}[\Mvariable{OPEm}]>-2,\multsp \Mfunction{Re}[\Mvariable{OPEm}]>-1,\multsp
    \multsp \multsp \Mfunction{Re}[\Mvariable{OPEm}]>0,\multsp \Mfunction{Re}[\Mvariable{OPEm}]>1\}
}

\Subsection*{OPEIntegrate { }***unfinished***}

\Subsubsection*{Description}

OPEIntegrate[expr, q, x]. { }The dimension is changed to the one indicated by the option { }Dimension. The setting of the option
  EpsContract determines { }the dimension in which the Levi-Civita tensors are contracted.

See also:  RHI.

\Subsubsection*{Examples}

\dispSFinmath{
\{\Mfunction{Re}(m)>-3,\Mfunction{Re}(m)>-2,\Mfunction{Re}(m)>-1,\Mfunction{Re}(m)>0,\Mfunction{Re}(m)>1\}
}

\Subsection*{OPEIntegrateDelta { }***unfinished***}

\Subsubsection*{Description}

OPEIntegrateDelta[expr, x, m] introduces the \(\delta \)(1-x) (DeltaFunction[1-x]). The {\itshape Mathematica} Integrate function is called and each
integration { }(from 0 to 1) is recorded for reference (and bug-checking) in the global list
  \${}MIntegrate. \\
Notice that the dimension specified by the option should also be the dimension used in expr. It is replaced in OPEIntegrateDelta by
  (4\(+\)Epsilon).

See also:  RHI.

\Subsubsection*{Examples}

\dispSFinmath{
\{ \}
}

\Subsection*{OPEIntegrate2 { }***unfinished***}

\Subsubsection*{Description}

OPEIntegrate2[exp, k] does special loop (tensorial) integrations. { }Only the residue is calculated.

See also:  OPEIntegrate.

\Subsubsection*{Examples}


\Subsection*{OPESumExplicit}

\Subsubsection*{Description}

OPESumExplicit[exp] calculates OPESum's.

See also:  OPESum, OPESumSimplify.

\Subsubsection*{Examples}

\dispSFinmath{
\{ \}
}


\dispSFinmath{
\{ \}
}


\dispSFinmath{
\{ \}
}


\dispSFinmath{
\{ \}
}


\dispSFinmath{
\{ \}
}

\Subsection*{OPESum}

\Subsubsection*{Description}

OPESum[exp, \{i, 0, m\}] denotes a symbolic sum.The syntax is the same as for Sum.

See also:  OPESumExplicit, OPESumSimplify.

\Subsubsection*{Examples}


\dispSFoutmath{
\Mvariable{t1}=\Muserfunction{OPESum}[A\RawWedge iB\RawWedge (m-i-3),\{i,0,m-3\}]
}

\dispSFinmath{
\sum _{i=0}^{m-3}{A^i}\multsp {B^{-i+m-3}}
}

\dispSFoutmath{
\Muserfunction{OPESumExplicit}[\Mvariable{t1}]
}

\dispSFinmath{
\frac{{A^{m-2}}}{A-B}-\frac{{B^{m-2}}}{A-B}
}

\dispSFoutmath{
\Mvariable{t2}\multsp =\multsp \Muserfunction{OPESum}[a\RawWedge ib\RawWedge (j-i)c\RawWedge (m-j-4),\{i,0,j\},\{j,0,m-4\}]
}

\dispSFinmath{
\sum _{j=0}^{m-4}\multsp (j+1){a^i}\multsp {b^{j-i}}\multsp {c^{-j+m-4}}
}

\dispSFoutmath{
\Muserfunction{OPESumExplicit}[\Mvariable{t2}]
}

\dispSFinmath{
\MathBegin{MathArray}{l}
-\frac{c\multsp {a^{m-2}}}{(a-b)\multsp (a-c)\multsp (b-c)}+
   \frac{b\multsp {a^{m-2}}}{(a-b)\multsp (a-c)\multsp (b-c)}+\frac{c\multsp {b^{m-2}}}{(a-b)\multsp (a-c)\multsp (b-c)}-  \\
   \noalign{\vspace{1.40625ex}}
\hspace{1.em} \frac{a\multsp {b^{m-2}}}{(a-b)\multsp (a-c)\multsp (b-c)}-
   \frac{b\multsp {c^{m-2}}}{(a-b)\multsp (a-c)\multsp (b-c)}+\frac{a\multsp {c^{m-2}}}{(a-b)\multsp (a-c)\multsp (b-c)}\\
   \MathEnd{MathArray}
}

\Subsection*{OPESumSimplify}

\Subsubsection*{Description}

OPESumSimplify[exp] simplifies OPESum's in exp.

See also:  OPESum, OPESumExplicit.

\Subsubsection*{Examples}

\dispSFinmath{
\Mfunction{Clear}[\Mvariable{t1},\Mvariable{t2}];
}

\dispSFoutmath{
\Mvariable{t1}=\Muserfunction{OPESum}[\Muserfunction{SO}[p]\RawWedge \Mvariable{OPEi}
      \Muserfunction{SO}[k]\RawWedge (\Mvariable{OPEm}-\Mvariable{OPEi}-3),\{\Mvariable{OPEi},0,\Mvariable{OPEm}-3\}]
}

\dispSFinmath{
\sum _{i=0}^{m-3}{{(\Delta \cdot k)}^{-i+m-3}}\multsp {{(\Delta \cdot p)}^i}
}

\dispSFoutmath{
\Muserfunction{OPESumExplicit}[\Mvariable{t1}]
}

\dispSFinmath{
\frac{{{(\Delta \cdot k)}^{m-2}}}{\Delta \cdot k-\Delta \cdot p}-\frac{{{(\Delta \cdot p)}^{m-2}}}{\Delta \cdot k-\Delta \cdot p}
}

\dispSFoutmath{
\Mvariable{t2}\multsp =\multsp \Muserfunction{OPESum}[a\RawWedge ib\RawWedge (j-i)c\RawWedge (m-j-4),\{i,0,j\},\{j,0,m-4\}]
}

\dispSFinmath{
\sum _{j=0}^{m-4}\multsp (j+1){a^i}\multsp {b^{j-i}}\multsp {c^{-j+m-4}}
}

\dispSFoutmath{
\Muserfunction{OPESumExplicit}[\Mvariable{t2}]
}

\dispSFinmath{
\MathBegin{MathArray}{l}
-\frac{c\multsp {a^{m-2}}}{(a-b)\multsp (a-c)\multsp (b-c)}+
   \frac{b\multsp {a^{m-2}}}{(a-b)\multsp (a-c)\multsp (b-c)}+\frac{c\multsp {b^{m-2}}}{(a-b)\multsp (a-c)\multsp (b-c)}-  \\
   \noalign{\vspace{1.40625ex}}
\hspace{1.em} \frac{a\multsp {b^{m-2}}}{(a-b)\multsp (a-c)\multsp (b-c)}-
   \frac{b\multsp {c^{m-2}}}{(a-b)\multsp (a-c)\multsp (b-c)}+\frac{a\multsp {c^{m-2}}}{(a-b)\multsp (a-c)\multsp (b-c)}\\
   \MathEnd{MathArray}
}

\dispSFoutmath{
\multsp \Mfunction{Clear}[\Mvariable{t1},\Mvariable{t2}]
}

\Subsection*{OPE1Loop { }***unfinished***}

\Subsubsection*{Description}

OPE1Loop[q1, amp]. { }OPE1Loop[\{q1,q2\}, amp] does sub-loop { }decomposition.

See also:  OPESum.

\Subsubsection*{Examples}

\dispSFinmath{
\Muserfunction{OPESum}[(-\Muserfunction{SOD}[p])\RawWedge (\Mvariable{OPEi}+1)\multsp
     \Muserfunction{SOD}[p-q]\RawWedge (\Mvariable{OPEm}-\Mvariable{OPEi}-2),\{\Mvariable{OPEi},0,\Mvariable{OPEm}\}]
}

\Subsection*{OPE2AI { }***unfinished***}

\Subsubsection*{Description}

OPE2AI[tabname, listtodo, q1,q2,p] is ....

\dispSFinmath{
\sum _{i=0}^{m}{{(-(\Delta \cdot p))}^{i+1}}\multsp {{(\Delta \cdot (p-q))}^{-i+m-2}}
}

\dispSFoutmath{
\Muserfunction{OPESumSimplify}[\%]
}

See also:  OPESum.

\Subsubsection*{Examples}

\dispSFinmath{
-\sum _{i=0}^{m}{{(-1)}^i}\multsp {{(\Delta \cdot p)}^i}\multsp {{(\Delta \cdot (p-q))}^{-i+m-2}}\multsp \Delta \cdot p
}

\Subsection*{OPE2TID { }***unfinished***}

\Subsubsection*{Description}

OPE2TID[exp, k1, k2, p]. The setting of the option EpsContract determines the dimension in which the Levi-Civita tensors are contracted.

\dispSFinmath{
\Muserfunction{OPESumSimplify}[\Muserfunction{OPESum}[\{\Mvariable{OPEi},0,\Mvariable{OPEm}\}]\multsp a\RawWedge \Mvariable{OPEi}]
}

\dispSFoutmath{
\sum _{i=0}^{m}{a^i}
}

See also:  OPESum.

\Subsubsection*{Examples}

\dispSFinmath{
\Muserfunction{OPESumSimplify}[\Muserfunction{OPESum}[\{j,0,i\},\{i,0,m\}]\multsp a\RawWedge (j-i)\multsp b\RawWedge i]
}

\Subsection*{OptionsSelect}

\Subsubsection*{Description}

OptionsSelect[function,opts] returns the option settings of opts accepted by function. { }When an option occurs several times in opts,
  the first setting is selected.

\Subsection*{Pair}

\Subsubsection*{Description}

Pair[x, y] is the head of a special pairing used in the internal representation: x and y may have heads LorentzIndex or Momentum. If both
  x and y have head LorentzIndex, the metric tensor is understood. If x and y have head Momentum, a scalar product is meant. If one of x
  and y has head LorentzIndex and the other Momentum, a Lorentz vector \(\sum _{i=0}^{m}\multsp (i+1){a^{j-i}}\multsp {b^i}\) is understood.

See also:  FourVector, FV, FVD, MetricTensor, MT, MTD, ScalarProduct, SP, SPD.

\Subsubsection*{Examples}

This represents a four-dimensional metric tensor.

\dispSFinmath{
\%//\Mfunction{StandardForm}
}

\dispSFoutmath{
\Muserfunction{OPESum}[{a^{-i+j}}\multsp {b^i},\{i,0,m\},\{j,0,i\}]
}

This is a D-dimensional metric tensor.

\dispSFinmath{
\{ \}
}


\dispSFinmath{
\Mfunction{Options}[\Mvariable{OPE2AI}]
}

\dispSFoutmath{
\{\Mvariable{Directory}\rightarrow /home/rolfm\}
}

\dispSFinmath{
\{ \}
}


\dispSFinmath{
\Mfunction{Options}[\Mvariable{OPE2TID}]
}

\dispSFoutmath{
\{\Mvariable{Uncontract}\rightarrow \Mvariable{False},\Mvariable{Contract}\rightarrow \Mvariable{True},\Mvariable{Dimension}\rightarrow D
    ,\Mvariable{EpsContract}\rightarrow \Mvariable{False}\}
}

\dispSFinmath{
\{ \}
}


\dispSFinmath{
{p^{\mu }}
}

\dispSFoutmath{
\Muserfunction{Pair}[\Muserfunction{LorentzIndex}[\alpha ],\Muserfunction{LorentzIndex}[\beta ]]
}

\dispSFinmath{
{g^{\alpha \beta }}
}

\dispSFoutmath{
\Muserfunction{Pair}[\Muserfunction{LorentzIndex}[\alpha ,D],\Muserfunction{LorentzIndex}[\beta ,D]]
}

\dispSFinmath{
{g^{\alpha \beta }}
}

\dispSFoutmath{
\Muserfunction{Pair}[\Muserfunction{LorentzIndex}[\alpha ],\Muserfunction{Momentum}[p]]
}

\dispSFinmath{
{p^{\alpha }}
}

\dispSFoutmath{
\Muserfunction{Pair}[\Muserfunction{Momentum}[q],\Muserfunction{Momentum}[p]]
}

\dispSFinmath{
p\cdot q
}

\dispSFoutmath{
\Muserfunction{Pair}[\Muserfunction{Momentum}[p],\Muserfunction{Momentum}[p]]
}

\Subsection*{PairCollect}

\Subsubsection*{Description}

PairCollect is an option for DiracTrace specifying if the result is collected with respect to Pair's.

See also:  DiracTrace, Pair.

\Subsection*{PairContract}

\Subsubsection*{Description}

PairContract is like Pair, but with (local) contraction properties.

See also: Pair, Contract.

\Subsection*{PairContract2}

\Subsubsection*{Description}

PairContract2 is like Pair, but with local contraction properties among PairContract2's.

See also: Pair, Contract, PairContract.

\Subsection*{PairContract3}

\Subsubsection*{Description}

PairContract3 is like Pair, but with local contraction properties among PairContract3's.

See also: Pair, Contract, PairContract.

\Subsection*{PartialD}

\Subsubsection*{Description}

PartialD[\({p^2}\)] denotes the four-dimensional \(\Muserfunction{Pair}[\Muserfunction{Momentum}[p-q],\Muserfunction{Momentum}[p]]\) PartialD is
used to denote derivative fields. PartialD[LorentzIndex[\(p\cdot (p-q)\)]] denotes the \(\Muserfunction{Pair}[\Muserfunction{Momentum}[p],\Muserfunction{Momentum}[p]]\RawWedge
2\)-dimensional \({p^4}\)

See also:  ExpandPartialD, LeftPartialD, LeftRightPartialD, RightPartialD.

\Subsubsection*{\(\Muserfunction{Pair}[\Muserfunction{Momentum}[p],\Muserfunction{Momentum}[p]]\RawWedge 3\)}

\dispSFinmath{
{p^6}
}

\dispSFoutmath{
\Muserfunction{Pair}[\Muserfunction{LorentzIndex}[\alpha ,n-4],\Muserfunction{LorentzIndex}[\beta ]]//\Muserfunction{Contract}
}

\dispSFinmath{
0
}

\dispSFoutmath{
\Muserfunction{ExpandScalarProduct}[\Muserfunction{Pair}[\Muserfunction{Momentum}[p-q],\Muserfunction{Momentum}[p]]]
}

\dispSFinmath{
{p^2}-p\cdot q
}

\dispSFoutmath{
\Muserfunction{Pair}[\Muserfunction{Momentum}[-q],\Muserfunction{Momentum}[p]]\multsp +\multsp
   \Muserfunction{Pair}[\Muserfunction{Momentum}[q],\Muserfunction{Momentum}[p]]
}

\Subsection*{PartialDRelations}

\Subsubsection*{Description}

PartialDRelations is an option for ExpandPartialD. It is a list of rules applied by ExpandPartialD at the end.

See also:  PartialD, ExpandPartialD, LeftPartialD, LeftRightPartialD, RightPartialD.

\Subsubsection*{\(0\)}

\dispSFinmath{
\mu
}

\dispSFoutmath{
{{\partial }_{\mu }}.
}

\dispSFinmath{
\mu ,D
}

\dispSFoutmath{
D
}

\Subsection*{PartialFourVector}

\Subsubsection*{Description}

PartialFourVector[exp, FourVector[p, mu]] { }calculates the partial derivative of exp w.r.t. p(mu). PartialFourVector[exp, FourVector[p,
  mu], FourVector[p,nu], ...] { }gives the multiple derivative.

See also:  FourVector, Pair, Contract.

\Subsubsection*{\({{\partial }_{\mu }}.\)}

\dispSFinmath{
\Mvariable{Examples}
}

\dispSFoutmath{
\Muserfunction{QuantumField}[A,\{\mu \}].\Muserfunction{LeftPartialD}[\nu ]
}

\Subsection*{PartialIntegrate}

\Subsubsection*{Description}

 PartialIntegrate[exp, ap, t] does a partial integration of the definite integral Integrate[exp,\{t,0,1\}], with ap the factor that is to
  be integrated and exp/ap the factor that is to be differentiated.

\dispSFinmath{
{A_{\mu }}.{{\left( \overvar{\partial }{\leftarrow } \right) }_{\nu }}
}

\dispSFoutmath{
\Muserfunction{ExpandPartialD}[\%]
}

\Subsubsection*{\({{\partial }_{\nu }}A_{\mu }^{ }\)}

\dispSFinmath{
\Mfunction{StandardForm}[\%]
}

\Message{\(\Muserfunction{QuantumField}[\Muserfunction{PartialD}[\Muserfunction{LorentzIndex}[\nu ]],A,\Muserfunction{LorentzIndex}[\mu ]]\)}

\Message{\(\Mvariable{Examples}\)}

\dispSFoutmath{
\Muserfunction{QuantumField}[A,\{\mu \}].\Muserfunction{QuantumField}[B,\{\mu \}].\Muserfunction{LeftPartialD}[\nu ]
}

\dispSFinmath{
{A_{\mu }}.{B_{\mu }}.{{\left( \overvar{\partial }{\leftarrow } \right) }_{\nu }}
}

\dispSFoutmath{
\Muserfunction{ExpandPartialD}[\%,\multsp \Mvariable{PartialDRelations}\rightarrow \{A\rightarrow C\}]
}

\dispSFinmath{
{C_{\mu }}.{{\partial }_{\nu }}B_{\mu }^{ }+{{\partial }_{\nu }}C_{\mu }^{ }.{B_{\mu }}
}

\dispSFoutmath{
\Mvariable{Examples}
}

\dispSFinmath{
\Muserfunction{PartialFourVector}\big[a\multsp {{\ExponentialE }^{{{\Muserfunction{FourVector}[p,\nu ]}^2}}},
    \Muserfunction{FourVector}[p,\mu ]\big]
}

\dispSFoutmath{
2\multsp a\multsp {{\ExponentialE }^{{p^2}}}\multsp {p^{\mu }}
}

\dispSFinmath{
\Mfunction{Options}[\Mvariable{PartialIntegrate}]
}

\dispSFoutmath{
\{\Mvariable{Integrate}\rightarrow \Mvariable{Integrate}\}
}

\dispSFinmath{
\Mvariable{Examples}
}

\Subsection*{PartitHead}

\Subsubsection*{Description}

PartitHead[expr, h] returns a list \{ex1, h[ex2]\} with ex1 free of expressions with head h, and h[ex2] having head h.

\Subsection*{PauliSigma}

\Subsubsection*{Description}

PauliSigma denotes the vector of the 3 Pauli matrices. PauliSigma[1], PauliSigma[2], PauliSigma[3] give the explicit Pauli matrices.
  PauliSigma[] yields \{PauliSigma[1], PauliSigma[2], PauliSigma[3]\}.

\dispSFinmath{
\Muserfunction{PartialIntegrate}[f[x]g[x],g[x],x]
}

\dispSFoutmath{
\Mfunction{Integrate}::\Mvariable{ilim}:\multsp \Mvariable{Invalid}\multsp \Mvariable{integration}\multsp \Mvariable{variable}\multsp
     \Mvariable{or}\multsp \Mvariable{limit}(s)\multsp \Mvariable{in}\multsp 1.
}

\Subsection*{PaVe}

\Subsubsection*{Description}

 PaVe[ i,j,... \{p10,p12,...\},\{m1\(\RawWedge\)2, mw\(\RawWedge\)2, ...\} ] denotes the invariant (and scalar) Passarino-Veltman
  integrals, i.e. the coefficient functions of the tensor integral decomposition. Joining plist and mlist gives the same conventions as
  for A0, B0, C0, D0. Automatic simlifications are performed for the coefficient functions of two-point integrals and for the scalar
  integrals.

For a more detailed description see the manual (no changes from there basically): http://www.mertig.com/oldfc

See also:  PaVeReduce.

\Subsubsection*{Examples}

Some of the PaVe's reduce to special cases.

\dispSFinmath{
\Mfunction{Integrate}::\Mvariable{ilim}:\multsp \Mvariable{Invalid}\multsp \Mvariable{integration}\multsp \Mvariable{variable}\multsp
     \Mvariable{or}\multsp \Mvariable{limit}(s)\multsp \Mvariable{in}\multsp 0.
}

\dispSFoutmath{
-f(0)\multsp \int g(0)\DifferentialD 0+f(1)\multsp \int g(1)\DifferentialD 1-\big(\int g(x)\DifferentialD x\big)\multsp {f^{\prime }}(x)
}

\dispSFinmath{
f[\Mvariable{x\_}]=\Mfunction{Integrate}[\log [3x+2],x]
}

\dispSFoutmath{
\frac{1}{3}\multsp (-3\multsp x-2)+\frac{1}{3}\multsp (3\multsp x+2)\multsp \log(3\multsp x+2)
}

\Subsection*{PaVeOrderList}

\Subsubsection*{Description}

PaVeOrderList is an option for PaVeOrder and PaVeReduce, specifying in which order the arguments of D0 are to be permuted.

See also: PaVeOrder, PaVeReduce, PaVe, D0.

\Subsection*{PaVeOrder}

\Subsubsection*{Description}

PaVeOrder[expr] orders the arguments of all D0 in expr in a standard way. PaVeOrder[expr, PaVeOrderList \(\rightarrow \) \{ \{..., s, u,
  ...\}, \{... \(g[\Mvariable{x\_}]=\Mfunction{D}\big[\frac{1}{\log [3x+2]},x\big]\), \(-\frac{3}{(3\multsp x+2)\multsp {{\log}^2}(3\multsp x+2)}\)
...\}, ...\}] orders the arguments of all D0 in expr according to the specified ordering lists. The lists may contain only a subsequence
  of the D0-variables.

See also:  PaVeReduce.

\Subsubsection*{Examples}

\dispSFinmath{
\Mfunction{Integrate}[\Muserfunction{PartialIntegrate}[f[x]g[x],f[x],x],\{x,0,1\}]//\Mfunction{FullSimplify}
}

\dispSFoutmath{
-\frac{\log\big(\frac{32}{25}\big)}{\log(5)\multsp \log(8)}
}

\dispSFinmath{
\Mfunction{Integrate}[f[x]g[x],\{x,0,1\}]//\Mfunction{Simplify}
}

\dispSFoutmath{
-\frac{\log\big(\frac{32}{25}\big)}{\log(5)\multsp \log(8)}
}

\dispSFinmath{
\Mfunction{Clear}[f,g]
}

\dispSFoutmath{
\Muserfunction{PauliSigma}[]//\Mfunction{TableForm}
}

\Subsection*{PaVeReduce}

\Subsubsection*{Description}

PaVeReduce[expr] reduces all Passarino-Veltman integrals (i.e. all PaVe's) in expr down to scalar A0, B0, C0 and D0.

\dispSFinmath{
\MathBegin{MathArray}[c]{ll}
  \MathBegin{MathArray}[c]{l}
  0 \\
  1
  \MathEnd{MathArray}&\MathBegin{MathArray}[c]{l}
  1 \\
  0

     \MathEnd{MathArray} \\
  \MathBegin{MathArray}[c]{l}
  0 \\
  -\ImaginaryI
  \MathEnd{MathArray}&\MathBegin{MathArray}[c]{l}

     \ImaginaryI  \\
  0
  \MathEnd{MathArray} \\
  \MathBegin{MathArray}[c]{l}
  1 \\
  0
  \MathEnd{MathArray}&\MathBegin{MathArray}
     [c]{l}
  0 \\
  -1
  \MathEnd{MathArray}
  \MathEnd{MathArray}
}

\dispSFoutmath{
\Muserfunction{PaVe}[0,\{a,b,c,d,e,f\},\{\Mvariable{m1},\Mvariable{m2},\Mvariable{m3},\Mvariable{m4}\}]
}

\dispSFinmath{
{D_0}(a,b,c,d,e,f,\Mvariable{m1},\Mvariable{m2},\Mvariable{m3},\Mvariable{m4})
}

\Print{\(\Muserfunction{PaVe}[0,0,\{\Mvariable{pp}\},\{m\RawWedge 2,M\RawWedge 2\}]\)}

See also:  FRH, PaVeOrder.

\Subsubsection*{Examples}

\dispSFinmath{
\MathBegin{MathArray}{l}
\frac{1}{3}\multsp {B_0}(\Mvariable{pp},{m^2},{M^2})\multsp {m^2}+
   \frac{1}{18}\multsp (3\multsp {m^2}+3\multsp {M^2}-\Mvariable{pp})+  \\
\noalign{\vspace{1.40625ex}}
\hspace{1.em} \frac{1}{6}\multsp
   \bigg({A_0}({M^2})+({m^2}-{M^2}+\Mvariable{pp})\multsp
      \bigg(\frac{({M^2}-{m^2})\multsp ({B_0}(\Mvariable{pp},{m^2},{M^2})-{B_0}(0,{m^2},{M^2}))}{2\multsp \Mvariable{pp}}-
        \frac{1}{2}\multsp {B_0}(\Mvariable{pp},{m^2},{M^2})\bigg)\bigg)\\
\MathEnd{MathArray}
}

\dispSFoutmath{
{{{m_1}}^2}
}

\dispSFinmath{
{{{m_2}}^2}
}

\dispSFoutmath{
\MathBegin{MathArray}{l}
\Muserfunction{PaVeOrder}[\Muserfunction{D0}[
     \Mvariable{me2},\Mvariable{me2},\Mvariable{mw2},\Mvariable{mw2},t,s,\Mvariable{me2},0,\Mvariable{me2},0],  \\
   \noalign{\vspace{0.5ex}}
\hspace{1.em} \Mvariable{PaVeOrderList}\rightarrow \{\Mvariable{me2},\Mvariable{me2},0,0\}]\\
   \MathEnd{MathArray}
}

\dispSFinmath{
{D_0}(\Mvariable{me2},s,\Mvariable{mw2},t,\Mvariable{mw2},\Mvariable{me2},\Mvariable{me2},0,0,\Mvariable{me2})
}

\dispSFoutmath{
\MathBegin{MathArray}{l}
\Muserfunction{PaVeOrder}[\Muserfunction{D0}[
     \Mvariable{me2},\Mvariable{me2},\Mvariable{mw2},\Mvariable{mw2},t,s,\Mvariable{me2},0,\Mvariable{me2},0],  \\
   \noalign{\vspace{0.5ex}}
\hspace{1.em} \Mvariable{PaVeOrderList}\rightarrow \{\Mvariable{me2},\Mvariable{me2},0,0\}]\\
   \MathEnd{MathArray}
}

\dispSFinmath{
{D_0}(\Mvariable{me2},s,\Mvariable{mw2},t,\Mvariable{mw2},\Mvariable{me2},\Mvariable{me2},0,0,\Mvariable{me2})
}

\dispSFoutmath{
\MathBegin{MathArray}{l}
\Mvariable{PaVeOrder}[  \\
\noalign{\vspace{0.5ex}}
\hspace{1.em} \Muserfunction{D0}[
     a,b,c,d,e,f,\Mvariable{m12},\Mvariable{m22},\Mvariable{m32},\Mvariable{m42}]+
    \Muserfunction{D0}[\Mvariable{me2},\Mvariable{me2},\Mvariable{mw2},\Mvariable{mw2},t,s,\Mvariable{me2},0,\Mvariable{me2},0],  \\
   \noalign{\vspace{0.5ex}}
\hspace{1.em} \Mvariable{PaVeOrderList}\rightarrow \{\{\Mvariable{me2},\Mvariable{me2},0,0\},\{f,e\}\}]\\
   \MathEnd{MathArray}
}

\dispSFinmath{
{D_0}(a,d,c,b,f,e,\Mvariable{m22},\Mvariable{m12},\Mvariable{m42},\Mvariable{m32})+
   {D_0}(\Mvariable{me2},s,\Mvariable{mw2},t,\Mvariable{mw2},\Mvariable{me2},\Mvariable{me2},0,0,\Mvariable{me2})
}

\dispSFoutmath{
\Mfunction{Options}[\Mvariable{PaVeReduce}]
}

\dispSFinmath{
\MathBegin{MathArray}{l}
\{\Mvariable{Dimension}\rightarrow \Mvariable{True},\Mvariable{IsolateNames}\rightarrow \Mvariable{False},
    \Mvariable{Mandelstam}\rightarrow \{\},  \\
\noalign{\vspace{0.666667ex}}
\hspace{1.em} \Mvariable{PaVeOrderList}\rightarrow \{\},
    \Mvariable{WriteOutPaVe}\rightarrow /opt/cvs/HighEnergyPhysics/Phi/Storage/\}\\
\MathEnd{MathArray}
}

\dispSFinmath{
?\Mvariable{WriteOutPaVe}
}

\dispSFinmath{
\MathBegin{MathArray}{l}
\Mvariable{WriteOutPaVe\multsp is\multsp an\multsp option\multsp for\multsp PaVeReduce\multsp and\multsp
   OneLoopSum.\multsp If\multsp set\multsp to\multsp }  \\
\noalign{\vspace{0.5ex}}
\hspace{2.em} \Mvariable{a\multsp string,\multsp
   the\multsp results\multsp of\multsp all\multsp Passarino-Veltman\multsp PaVe's\multsp are\multsp stored\multsp in\multsp }  \\
   \noalign{\vspace{0.5ex}}
\hspace{2.em} \Mvariable{files\multsp with\multsp names\multsp generated\multsp from\multsp this\multsp
   string\multsp and\multsp the\multsp arguments\multsp of\multsp PaVe.}\\
\MathEnd{MathArray}
}

\dispSFoutmath{
\Muserfunction{PaVeReduce}[\Muserfunction{PaVe}[1,2,\{s,m\RawWedge 2,m\RawWedge 2\},\{m\RawWedge 2,m\RawWedge 2,M\RawWedge 2\}],
    \Mvariable{IsolateHead}\rightarrow L]
}

\dispSFinmath{
\MathBegin{MathArray}{l}
\frac{(8\multsp {m^2}-3\multsp {M^2}-2\multsp s)\multsp {C_0}({m^2},{m^2},s,{m^2},{M^2},{m^2})\multsp {M^2}}
    {{{(4\multsp {m^2}-s)}^2}}+\frac{({m^2}-{M^2})\multsp {B_0}(0,{m^2},{M^2})}{2\multsp {m^2}\multsp (4\multsp {m^2}-s)}-  \\
   \noalign{\vspace{1.42708ex}}
\hspace{1.em} \frac{(8\multsp {m^4}-10\multsp {M^2}\multsp {m^2}-2\multsp s\multsp {m^2}+{M^2}\multsp s)
      \multsp {B_0}({m^2},{m^2},{M^2})}{2\multsp {m^2}\multsp {{(4\multsp {m^2}-s)}^2}}+
   \frac{(4\multsp {m^2}-6\multsp {M^2}-s)\multsp {B_0}(s,{m^2},{m^2})}{2\multsp {{(4\multsp {m^2}-s)}^2}}+
   \frac{1}{2\multsp (4\multsp {m^2}-s)}\\
\MathEnd{MathArray}
}

\dispSFinmath{
L[10]
}

\dispSFoutmath{
L(10)
}

\dispSFinmath{
\Muserfunction{FRH}[\%]
}

\Subsection*{PD}

\Subsubsection*{Description}

PD is an abbreviation for PropagatorDenominator.

See also:  PropagatorDenominator, IFPD.

\Subsection*{PlusDistribution}

\Subsubsection*{Description}

PlusDistribution[1/(1-x)] denotes a distribution (in the sense of the "\(+\)" prescription).

See also:  Integrate2.

\Subsubsection*{Examples}

\dispSFinmath{
L(10)
}

\dispSFoutmath{
\MathBegin{MathArray}{l}
\Muserfunction{PaVeReduce}[\Muserfunction{PaVe}[
     2,\{\Muserfunction{SmallVariable}[\Mvariable{me2}],\Mvariable{mw2},t\},
      \{\Muserfunction{SmallVariable}[\Mvariable{me2}],0,\Mvariable{mw2}\}],  \\
\noalign{\vspace{0.5ex}}
\hspace{1.em} \Mvariable{WriteO
      utPaVe}\rightarrow "p"]\\
\MathEnd{MathArray}
}

\dispSFinmath{
\frac{{B_0}(0,\Mvariable{mw2},\Mvariable{mw2})}{\Mvariable{mw2}-t}-\frac{{B_0}(t,\Mvariable{mw2},\Mvariable{me2})}{\Mvariable{mw2}-t}+
   \frac{2}{\Mvariable{mw2}-t}
}

\dispSFoutmath{
\MathBegin{MathArray}{l}
\Mfunction{TableForm}[\Mfunction{ReadList}[\Mfunction{If}[\$OperatingSystem==="MacOS",":",""]<>  \\
   \noalign{\vspace{0.5ex}}
\hspace{3.em} "pPaVe2Csmame2mw2tCsmame20mw2.s",\Mvariable{String}]]\\
\MathEnd{MathArray}
}

\dispSFinmath{
\MathBegin{MathArray}[c]{l}
  (\multsp 2/(mw2\multsp -\multsp t)\multsp +\multsp B0[0,\multsp mw2,\multsp mw2]/(mw2\multsp -\multsp
    t)\multsp -\multsp B0[t,\multsp mw2,\multsp SmallVariable[me2]]/ \\
  \multsp \multsp (mw2\multsp -\multsp t) \\
  \multsp \multsp
    )\multsp
  \MathEnd{MathArray}
}

\dispSFoutmath{
\Mvariable{DeleteFile}/@\Mfunction{FileNames}["pPaVe2Csmame2mw2tCsmame20mw2.s"];
}

\dispSFinmath{
\Mvariable{se}=\Muserfunction{SmallVariable}[\Mvariable{ME2}];
}

\dispSFoutmath{
\MathBegin{MathArray}{l}
\Mvariable{d122}=\Muserfunction{PaVeReduce}[
    \Muserfunction{PaVe}[1,2,2,\{\Mvariable{se},\Mvariable{MW2},\Mvariable{MW2},\Mvariable{se},S,T\},
       \{0,\Mvariable{se},0,\Mvariable{se}\}],  \\
\noalign{\vspace{0.5ex}}
\hspace{2.em} \Mvariable{Mandelstam}\rightarrow
     \{S,T,U,2\multsp \Mvariable{MW2}\},\Mvariable{IsolateHead}\rightarrow F]\\
\MathEnd{MathArray}
}

\dispSFinmath{
\MathBegin{MathArray}{l}
\frac{(\Mvariable{MW2}-S)\multsp {T^2}\multsp
      {D_0}(\Mvariable{MW2},\Mvariable{MW2},\Mvariable{ME2},\Mvariable{ME2},T,S,\Mvariable{ME2},0,\Mvariable{ME2},0)\multsp {S^3}}{2
      \multsp {{\big({{\Mvariable{MW2}}^2}-S\multsp U\big)}^3}}+  \\
\noalign{\vspace{1.6875ex}}
\hspace{1.em} \frac{{{(\Mvariable{MW2}-S
          )}^2}\multsp T\multsp {C_0}(\Mvariable{MW2},S,\Mvariable{ME2},\Mvariable{ME2},0,0)\multsp {S^2}}{{{\big(
         {{\Mvariable{MW2}}^2}-S\multsp U\big)}^3}}-  \\
\noalign{\vspace{1.72917ex}}
\hspace{1.em} \frac{(\Mvariable{MW2}-S)\multsp
      {T^2}\multsp {C_0}(T,\Mvariable{ME2},\Mvariable{ME2},\Mvariable{ME2},\Mvariable{ME2},0)\multsp {S^2}}{2\multsp
      {{\big({{\Mvariable{MW2}}^2}-S\multsp U\big)}^3}}+
   \frac{\big({{\Mvariable{MW2}}^2}-4\multsp S\multsp \Mvariable{MW2}+2\multsp {S^2}+S\multsp U\big)\multsp {B_0}(S,0,0)\multsp S}
    {2\multsp (\Mvariable{MW2}-S)\multsp {{\big({{\Mvariable{MW2}}^2}-S\multsp U\big)}^2}}-  \\
\noalign{\vspace{1.22917ex}}
   \hspace{1.em} \big(\big(4\multsp {{\Mvariable{MW2}}^5}-5\multsp S\multsp {{\Mvariable{MW2}}^4}-U\multsp {{\Mvariable{MW2}}^4}-
     16\multsp {S^2}\multsp {{\Mvariable{MW2}}^3}+4\multsp {S^3}\multsp {{\Mvariable{MW2}}^2}+  \\
\noalign{\vspace{0.6875ex}}
   \hspace{6.em} 4\multsp S\multsp {U^2}\multsp {{\Mvariable{MW2}}^2}-4\multsp {S^2}\multsp U\multsp {{\Mvariable{MW2}}^2}+
     4\multsp {S^4}\multsp \Mvariable{MW2}+8\multsp {S^3}\multsp U\multsp \Mvariable{MW2}+{S^2}\multsp {U^3}+{S^3}\multsp {U^2}\big)
    \multsp   \\
\noalign{\vspace{0.708333ex}}
\hspace{4.em} {B_0}(\Mvariable{MW2},0,\Mvariable{ME2})\big)\big/
    \big(2\multsp (\Mvariable{MW2}-S)\multsp {{(4\multsp \Mvariable{MW2}-T)}^2}\multsp {{\big({{\Mvariable{MW2}}^2}-S\multsp U\big)}^2}
     \big)+  \\
\noalign{\vspace{0.833333ex}}
\hspace{1.em} \big(
    (\Mvariable{MW2}+S)\multsp \big(4\multsp {{\Mvariable{MW2}}^3}-9\multsp S\multsp {{\Mvariable{MW2}}^2}-U\multsp {{\Mvariable{MW2}}^2}
       -4\multsp S\multsp U\multsp \Mvariable{MW2}+2\multsp {S^3}+3\multsp S\multsp {U^2}+5\multsp {S^2}\multsp U\big)\multsp
     {B_0}(T,\Mvariable{ME2},\Mvariable{ME2})\big)\big/  \\
\noalign{\vspace{0.875ex}}
\hspace{2.em} \big(
     2\multsp {{(4\multsp \Mvariable{MW2}-T)}^2}\multsp {{\big({{\Mvariable{MW2}}^2}-S\multsp U\big)}^2}\big)-  \\
\noalign{\vspace{
   1.29167ex}}
\hspace{1.em} \frac{1}
    {2\multsp {{(4\multsp \Mvariable{MW2}-T)}^2}\multsp {{\big({{\Mvariable{MW2}}^2}-S\multsp U\big)}^3}}\big(
    (\Mvariable{MW2}+S)\multsp \big(2\multsp {{\Mvariable{MW2}}^6}-8\multsp S\multsp {{\Mvariable{MW2}}^5}-
       2\multsp T\multsp {{\Mvariable{MW2}}^5}+12\multsp {S^2}\multsp {{\Mvariable{MW2}}^4}+  \\
\noalign{\vspace{1.0625ex}}
   \hspace{6.em} 20\multsp S\multsp T\multsp {{\Mvariable{MW2}}^4}-8\multsp {S^3}\multsp {{\Mvariable{MW2}}^3}-
   6\multsp S\multsp {T^2}\multsp {{\Mvariable{MW2}}^3}-36\multsp {S^2}\multsp T\multsp {{\Mvariable{MW2}}^3}+
   2\multsp {S^4}\multsp {{\Mvariable{MW2}}^2}+  \\
\noalign{\vspace{0.6875ex}}
\hspace{6.em} 6\multsp {S^2}\multsp {T^2}\multsp
      {{\Mvariable{MW2}}^2}+20\multsp {S^3}\multsp T\multsp {{\Mvariable{MW2}}^2}+4\multsp {S^2}\multsp {T^3}\multsp \Mvariable{MW2}-
     6\multsp {S^3}\multsp {T^2}\multsp \Mvariable{MW2}-2\multsp {S^4}\multsp T\multsp \Mvariable{MW2}-{S^2}\multsp {T^4}\big)\multsp
   \\
\noalign{\vspace{1.125ex}}
\hspace{4.em} {C_0}(\Mvariable{MW2},\Mvariable{MW2},T,\Mvariable{ME2},0,\Mvariable{ME2})\big)-
   \frac{\Mvariable{MW2}+S}{2\multsp (4\multsp \Mvariable{MW2}-T)\multsp \big({{\Mvariable{MW2}}^2}-S\multsp U\big)}\\
   \MathEnd{MathArray}
}

\dispSFoutmath{
\Muserfunction{Write2}["fctd122.for",\Mvariable{d122res}==\Mvariable{d122},\Mvariable{FormatType}\rightarrow \Mvariable{FortranForm}];
}

\dispSFinmath{
\MathBegin{MathArray}{l}
\Mvariable{TableForm}[  \\
\noalign{\vspace{0.5ex}}
\hspace{1.em} \Mfunction{ReadList}[
    \Mfunction{If}[\$OperatingSystem==="MacOS",":",""]<>"fctd122.for",\Mvariable{String}]]\\
\MathEnd{MathArray}
}

\dispSFoutmath{
\MathBegin{MathArray}[c]{l}
  \multsp \multsp \multsp \multsp \multsp \multsp \multsp \multsp d122res\multsp =\multsp (-5.D-1*(MW2\multsp
    +\multsp S))/ \\
  \multsp \multsp \multsp \multsp \multsp \&\multsp \multsp \multsp ((4D0*MW2\multsp -\multsp 1D0*T)*(MW2**2\multsp
    -\multsp 1D0*S*U))\multsp -\multsp  \\
  \multsp \multsp \multsp \multsp \multsp \&\multsp \multsp (5.D-1*(4D0*MW2**5\multsp -\multsp
    5D0*MW2**4*S\multsp -\multsp  \\
  \multsp \multsp \multsp \multsp \multsp \&\multsp \multsp \multsp \multsp \multsp \multsp \multsp
    16D1*MW2**3*S**2\multsp +\multsp 4D0*MW2**2*S**3\multsp +\multsp  \\
  \multsp \multsp \multsp \multsp \multsp \&\multsp \multsp
    \multsp \multsp \multsp \multsp \multsp 4D0*MW2*S**4\multsp -\multsp 1D0*MW2**4*U\multsp -\multsp  \\
  \multsp \multsp \multsp
    \multsp \multsp \&\multsp \multsp \multsp \multsp \multsp \multsp \multsp 4D0*MW2**2*S**2*U\multsp +\multsp 8D0*MW2*S**3*U\multsp
    +\multsp  \\
  \multsp \multsp \multsp \multsp \multsp \&\multsp \multsp \multsp \multsp \multsp \multsp \multsp
    4D0*MW2**2*S*U**2\multsp +\multsp S**3*U**2\multsp +\multsp S**2*U**3)* \\
  \multsp \multsp \multsp \multsp \multsp \&\multsp \multsp
    \multsp \multsp \multsp B0(MW2,0D0,ME2))/ \\
  \multsp \multsp \multsp \multsp \multsp \&\multsp \multsp \multsp ((MW2\multsp -\multsp
    1D0*S)*(4D0*MW2\multsp -\multsp 1D0*T)**2* \\
  \multsp \multsp \multsp \multsp \multsp \&\multsp \multsp \multsp \multsp \multsp
    (MW2**2\multsp -\multsp 1D0*S*U)**2)\multsp +\multsp  \\
  \multsp \multsp \multsp \multsp \multsp \&\multsp \multsp
    (5.D-1*S*(MW2**2\multsp -\multsp 4D0*MW2*S\multsp +\multsp 2D0*S**2\multsp +\multsp S*U)* \\
  \multsp \multsp \multsp \multsp \multsp
    \&\multsp \multsp \multsp \multsp \multsp B0(S,0D0,0D0))/ \\
  \multsp \multsp \multsp \multsp \multsp \&\multsp \multsp \multsp
    ((MW2\multsp -\multsp 1D0*S)*(MW2**2\multsp -\multsp 1D0*S*U)**2)\multsp +\multsp  \\
  \multsp \multsp \multsp \multsp \multsp
    \&\multsp \multsp (5.D-1*(MW2\multsp +\multsp S)* \\
  \multsp \multsp \multsp \multsp \multsp \&\multsp \multsp \multsp \multsp
    \multsp (4D0*MW2**3\multsp -\multsp 9D0*MW2**2*S\multsp +\multsp 2D0*S**3\multsp -\multsp  \\
  \multsp \multsp \multsp \multsp
    \multsp \&\multsp \multsp \multsp \multsp \multsp \multsp \multsp 1D0*MW2**2*U\multsp -\multsp 4D0*MW2*S*U\multsp +\multsp
    5D0*S**2*U\multsp +\multsp  \\
  \multsp \multsp \multsp \multsp \multsp \&\multsp \multsp \multsp \multsp \multsp \multsp \multsp
    3D0*S*U**2)*B0(T,ME2,ME2))/ \\
  \multsp \multsp \multsp \multsp \multsp \&\multsp \multsp \multsp ((4D0*MW2\multsp -\multsp
    1D0*T)**2*(MW2**2\multsp -\multsp 1D0*S*U)**2)\multsp -\multsp  \\
  \multsp \multsp \multsp \multsp \multsp \&\multsp \multsp
    (5.D-1*(MW2\multsp +\multsp S)* \\
  \multsp \multsp \multsp \multsp \multsp \&\multsp \multsp \multsp \multsp \multsp
    (2D0*MW2**6\multsp -\multsp 8D0*MW2**5*S\multsp +\multsp 12D1*MW2**4*S**2\multsp -\multsp  \\
  \multsp \multsp \multsp \multsp
    \multsp \&\multsp \multsp \multsp \multsp \multsp \multsp \multsp 8D0*MW2**3*S**3\multsp +\multsp 2D0*MW2**2*S**4\multsp -\multsp  \\
      \multsp \multsp \multsp \multsp \multsp \&\multsp \multsp \multsp \multsp \multsp \multsp \multsp 2D0*MW2**5*T\multsp +\multsp
    20D1*MW2**4*S*T\multsp -\multsp  \\
  \multsp \multsp \multsp \multsp \multsp \&\multsp \multsp \multsp \multsp \multsp \multsp
    \multsp 36D1*MW2**3*S**2*T\multsp +\multsp 20D1*MW2**2*S**3*T\multsp -\multsp  \\
  \multsp \multsp \multsp \multsp \multsp \&\multsp
    \multsp \multsp \multsp \multsp \multsp \multsp 2D0*MW2*S**4*T\multsp -\multsp 6D0*MW2**3*S*T**2\multsp +\multsp  \\
  \multsp \multsp
    \multsp \multsp \multsp \&\multsp \multsp \multsp \multsp \multsp \multsp \multsp 6D0*MW2**2*S**2*T**2\multsp -\multsp
    6D0*MW2*S**3*T**2\multsp +\multsp  \\
  \multsp \multsp \multsp \multsp \multsp \&\multsp \multsp \multsp \multsp \multsp \multsp
    \multsp 4D0*MW2*S**2*T**3\multsp -\multsp 1D0*S**2*T**4)* \\
  \multsp \multsp \multsp \multsp \multsp \&\multsp \multsp \multsp
    \multsp \multsp C0(MW2,MW2,T,ME2,0D0,ME2))/ \\
  \multsp \multsp \multsp \multsp \multsp \&\multsp \multsp \multsp ((4D0*MW2\multsp
    -\multsp 1D0*T)**2*(MW2**2\multsp -\multsp 1D0*S*U)**3)\multsp +\multsp  \\
  \multsp \multsp \multsp \multsp \multsp \&\multsp
    \multsp ((MW2\multsp -\multsp 1D0*S)**2*S**2*T*C0(MW2,S,ME2,ME2,0D0,0D0))/ \\
  \multsp \multsp \multsp \multsp \multsp \&\multsp
    \multsp \multsp (MW2**2\multsp -\multsp 1D0*S*U)**3\multsp -\multsp  \\
  \multsp \multsp \multsp \multsp \multsp \&\multsp \multsp
    (5.D-1*(MW2\multsp -\multsp 1D0*S)*S**2*T**2* \\
  \multsp \multsp \multsp \multsp \multsp \&\multsp \multsp \multsp \multsp \multsp
    C0(T,ME2,ME2,ME2,ME2,0D0))/(MW2**2\multsp -\multsp 1D0*S*U)**3\multsp +\multsp  \\
  \multsp \multsp \multsp \multsp \multsp \&\multsp
    \multsp (5.D-1*(MW2\multsp -\multsp 1D0*S)*S**3*T**2* \\
  \multsp \multsp \multsp \multsp \multsp \&\multsp \multsp \multsp \multsp
    \multsp D0(MW2,MW2,ME2,ME2,T,S,ME2,0D0,ME2,0D0))/ \\
  \multsp \multsp \multsp \multsp \multsp \&\multsp \multsp \multsp
    (MW2**2\multsp -\multsp 1D0*S*U)**3 \\
  \multsp \multsp \multsp \multsp \multsp \multsp \multsp \multsp \multsp \multsp \multsp
    \multsp \multsp \multsp \multsp \multsp \multsp \multsp
  \MathEnd{MathArray}
}

\Subsection*{Polarization}

\Subsubsection*{Description}

Polarization[k] \(=\) Polarization[k, I] is the head of a polarization momentum with (incoming) momentum k. A slashed polarization vector
  (e1(k) slash) has to be entered as DiracSlash[Polarization[k]]. The internal representation for a polarization vector e1 corresponding
  to a boson with four momentum k is: Momentum[ Polarization[ k, I ] ]. With this notation transversality of polarization vectors is
  provided, i.e. , Pair[ Momentum[k], Momentum[ Polarization[k, I] ] ] yields 0. Polarization[k,-I] denotes the complex conjugate
  polarization. Polarization is also an option of various functions related to the operator product expansion. The setting 0 denotes the
  unpolarized and 1 the polarized case.

See also:  PolarizationVector, PolarizationSum, DoPolarizationSums.

\Subsubsection*{Examples}

\dispSFinmath{
\Mvariable{DeleteFile}/@\Mfunction{FileNames}["fctd122.for"];\Mfunction{Clear}[\Mvariable{d122},\Mvariable{se}];
}

\dispSFoutmath{
\Muserfunction{PlusDistribution}[1/(1-x)]
}

\dispSFinmath{
{{\Big(\frac{1}{1-x}\Big)}_+}
}

\dispSFoutmath{
\Muserfunction{PlusDistribution}[\log [1-x]/(1-x)]
}

\dispSFinmath{
{{\Big(\frac{\log(1-x)}{1-x}\Big)}_+}
}

\dispSFoutmath{
\Muserfunction{Integrate2}[\Muserfunction{PlusDistribution}[1/(1-x)],\multsp \{x,0,1\}]
}

\dispSFinmath{
0
}

\dispSFoutmath{
\Muserfunction{Integrate2}[\Muserfunction{PlusDistribution}[\log [1-x]/(1-x)],\multsp \{x,0,1\}]
}

\dispSFinmath{
0
}

\dispSFoutmath{
\Muserfunction{Integrate2}[\Muserfunction{PlusDistribution}[\log [1-x]\RawWedge 2/(1-x)],\multsp \{x,0,1\}]
}

\dispSFinmath{
0
}

\dispSFoutmath{
\Muserfunction{PlusDistribution}[\log [x\multsp (1-x)]/(1-x)]
}

\dispSFinmath{
\frac{\log(x)}{1-x}+{{\Big(\frac{\log(1-x)}{1-x}\Big)}_+}
}

\dispSFoutmath{
\Muserfunction{Polarization}[k]
}

\Subsection*{PolarizationSum}

\Subsubsection*{Description}

PolarizationSum[ mu,nu, ... ] defines (as abbreviations) several polarization sums. The first two arguments are the interpreted as
  Lorentz indices, all further ones are momenta. PolarizationSum performs no calculations.

\dispSFinmath{
\varepsilon (k)
}

\dispSFoutmath{
\Muserfunction{Polarization}[k]//\Mfunction{StandardForm}
}

See also:  Polarization.

\Subsubsection*{Examples}

\dispSFinmath{
\Muserfunction{Polarization}[k,\ImaginaryI ]
}

\dispSFoutmath{
\Muserfunction{Polarization}[k]//\Muserfunction{ComplexConjugate}
}

\dispSFinmath{
{{\varepsilon }^*}(k)
}

\dispSFoutmath{
\Muserfunction{Polarization}[k]//\Muserfunction{ComplexConjugate}//\Mfunction{StandardForm}
}

\dispSFinmath{
\Muserfunction{Polarization}[k,-\ImaginaryI ]
}

\dispSFoutmath{
\Muserfunction{DiracSlash}[\Muserfunction{Polarization}[k]]
}

\dispSFinmath{
\gamma \cdot \varepsilon (k)
}

\dispSFoutmath{
\Muserfunction{DiracSlash}[\Muserfunction{Polarization}[k]]//\Mfunction{StandardForm}
}

\Subsection*{PolarizationUncontract}

\Subsubsection*{Description}

PolarizationUncontract does Uncontract on scalar products involving polarization vectors.

See also:  Polarization, Uncontract.

\Subsection*{PolarizationVector}

\Subsubsection*{Description}

PolarizationVector[p, mu] gives a polarization vector.

See also: FourVector, Pair, Polarization.

\Subsubsection*{Examples}

\dispSFinmath{
\Muserfunction{DiracSlash}[\Muserfunction{Polarization}[k,\ImaginaryI ]]
}

\dispSFoutmath{
\Muserfunction{Pair}[\multsp \Muserfunction{Momentum}[k],\multsp
    \Muserfunction{Momentum}[\multsp \Muserfunction{Polarization}[k,\multsp \imag ]\multsp ]\multsp ]
}

\dispSFinmath{
0
}

\dispSFoutmath{
\Mfunction{Options}[\Mvariable{PolarizationSum}]
}

A polarization vector \(\{\Mvariable{Dimension}\rightarrow 4\}\)is a special four-vector.

\dispSFinmath{
\Muserfunction{PolarizationSum}[\mu ,\nu ]
}

\dispSFoutmath{
-{g^{\mu \nu }}
}

The transverality property \(\Muserfunction{PolarizationSum}[\mu ,\nu ,k]\) is built in.

\dispSFinmath{
\frac{{k^{\mu }}\multsp {k^{\nu }}}{{k^2}}-{g^{\mu \nu }}
}

\dispSFoutmath{
\Muserfunction{PolarizationSum}[\mu ,\nu ,k,n]
}

\dispSFinmath{
-{g^{\mu \nu }}-\frac{{k^{\mu }}\multsp {k^{\nu }}\multsp {n^2}}{{{k\cdot n}^2}}+
   \frac{{n^{\mu }}\multsp {k^{\nu }}+{k^{\mu }}\multsp {n^{\nu }}}{k\cdot n}
}

\dispSFoutmath{
\Muserfunction{PolarizationSum}[\mu ,\nu ,k,{p_1}-{p_2}]
}

\dispSFinmath{
-{g^{\mu \nu }}-\frac{{k^{\mu }}\multsp {k^{\nu }}\multsp p_{1}^{2}}{{{(k\cdot {p_1}-k\cdot {p_2})}^2}}+
   \frac{2\multsp {k^{\mu }}\multsp {k^{\nu }}\multsp {p_1}\cdot {p_2}}{{{(k\cdot {p_1}-k\cdot {p_2})}^2}}-
   \frac{{k^{\mu }}\multsp {k^{\nu }}\multsp p_{2}^{2}}{{{(k\cdot {p_1}-k\cdot {p_2})}^2}}+
   \frac{{{({p_1}-{p_2})}^{\mu }}\multsp {k^{\nu }}+{k^{\mu }}\multsp {{({p_1}-{p_2})}^{\nu }}}{k\cdot {p_1}-k\cdot {p_2}}
}

\dispSFoutmath{
\Muserfunction{PolarizationVector}[k,\mu ]
}

\dispSFinmath{
{{\varepsilon }_{\mu }}(k)
}

\dispSFoutmath{
\Mfunction{Conjugate}[\Muserfunction{PolarizationVector}[k,\mu ]]
}

Depending on the alphabetical ordering of the momenta simplifcations are done, e.g., \(\varepsilon _{\mu }^{*}(k)\)

\dispSFinmath{
{{\varepsilon }_{\mu }}(k)
}

\dispSFoutmath{
\Muserfunction{PolarizationVector}[k,\mu]//\Mfunction{StandardForm}
}

\dispSFinmath{
\Muserfunction{Pair}[\Muserfunction{LorentzIndex}[\mu],\Muserfunction{Momentum}[\Muserfunction{Polarization}[k,\ImaginaryI ]]]
}

\dispSFoutmath{
{k^{\mu }}\multsp {{\varepsilon }_{\mu }}(k)=0
}

\dispSFinmath{
\Mvariable{a1}\multsp =\multsp \Muserfunction{PolarizationVector}[k,\mu ]\multsp \Muserfunction{FourVector}[k,\mu ]
}

\Subsection*{PositiveInteger}

\Subsubsection*{Description}

PositiveInteger is a data type. E.g. DataType[OPEm, PositiveInteger] gives True.

See also:  DataType.

\Subsection*{PositiveNumber}

\Subsubsection*{Description}

PositiveNumber is a data type. E.g. DataType[Epsilon, PositiveNumber] \(=\) True (by default).

See also:  DataType.

\Subsection*{PowerFactor}

\Subsubsection*{Description}

PowerFactor[exp] replaces \({k_{\mu }}\multsp {{\varepsilon }_{\mu }}(k)\)\(\Muserfunction{Contract}[\Mvariable{a1}]\)with \(0\).

See also:  PowerSimplify.

\Subsubsection*{Examples}

\dispSFinmath{
\Mvariable{a2}\multsp =\multsp \Muserfunction{PolarizationVector}[k-p,\mu ]\multsp \Muserfunction{FourVector}[k,\mu ]
}

\dispSFoutmath{
{k_{\mu }}\multsp {{\varepsilon }_{\mu }}(k-p)
}

\dispSFinmath{
\Muserfunction{Contract}[\Mvariable{a2}]
}

\dispSFoutmath{
k\cdot (\varepsilon (k-p))
}

\Subsection*{PowerSimplify}

\Subsubsection*{Description}

PowerSimplify[exp] simplifies\(\MathBegin{MathArray}{l}
{p^{\mu }}\multsp {{\varepsilon }_{\mu }}(k-p)=-{{(k-p-k)}^{\mu }}{{\varepsilon }_{\mu }}(k-p)=  \\
   \noalign{\vspace{0.666667ex}}
\hspace{2.em} {k^{\mu }}{{\varepsilon }_{\mu }}(k-p)=
   k\NoBreak \cdot \NoBreak (\NoBreak \varepsilon \NoBreak (\NoBreak k-p\NoBreak )\NoBreak ).\\
\MathEnd{MathArray}\) to \(\Mvariable{a3}\multsp =\multsp \Muserfunction{PolarizationVector}[k-p,\mu ]\multsp \Muserfunction{FourVector}[p,\mu ]\)
\({p_{\mu }}\multsp {{\varepsilon }_{\mu }}(k-p)\) and \(\Muserfunction{Contract}[\Mvariable{a3}]\) to \(k\cdot (\varepsilon (k-p))\) \(\Mfunction{Clear}[\Mvariable{a1},\Mvariable{a2},\Mvariable{a3}]\);
thus assuming that the exponent is an integer (even if it is symbolic). Furthermore \({x^a}\) and \({y^a}\) are expanded and \({{(x\multsp y)}^a}\)
\(\rightarrow \) \({x^a}{y^a}\) and (-1)\(\RawWedge\)({\itshape n} {\itshape m}) \(\rightarrow \) 1 and (-1)\(\RawWedge\)({\itshape n} {\itshape
m}) \(\rightarrow \) \({x^a}\multsp {y^a}\) for {\itshape n} even and odd respectively and \(\Muserfunction{PowerFactor}[\%]\) \(\rightarrow \) \({{(x\multsp
y)}^a}\) and \(\multsp {{(-x)}^a}\) \(\rightarrow \) \({{(-1)}^a}\).

See also:  DataType, OPEm.

\Subsubsection*{Examples}

\dispSFinmath{
{x^a}
}

\dispSFoutmath{
{{(y-x)}^n}
}

\dispSFinmath{
{{(-1)}^n}
}

\dispSFoutmath{
{{(x-y)}^n}
}

\dispSFinmath{
{{(-1)}^{a+n}}
}

\dispSFoutmath{
{{\ImaginaryI }^{a+n}}
}

\dispSFinmath{
{{\ImaginaryI }^{2\multsp m}}
}

\dispSFoutmath{
{{(-1)}^m}
}

\Subsection*{Power2}

\Subsubsection*{Description}

Power2[x, y] represents x\(\RawWedge\)y. { }Sometimes Power2 is more useful than the Mathematica Power. Power2[-a,b] simplifies to
  (-1)\(\RawWedge\)b Power2[a,b] (if no Epsilon is in b ...).

See also:  PowerFactor.

\Subsubsection*{Examples}

\dispSFinmath{
{{(-1)}^m}
}

\dispSFoutmath{
{{(-1)}^{-n}}
}

\dispSFinmath{
{{(-1)}^n}
}

\dispSFoutmath{
{{\ExponentialE }^{\ImaginaryI m\pi }}
}

\Subsection*{PostFortranFile}

\Subsubsection*{Description}

PostFortranFile is an option for Write2 which may be set to a file name (or a list of file names) or a string, { }which will be put at
  the end of the generated Fortran file.

See also:  Write2, PreFortranFile.

\Subsection*{Prefactor}

\Subsubsection*{Description}

Prefactor is an option for OneLoop and OneLoopSum. If set as option of OneLoop, the amplitude is multiplied by Prefactor before
  calculation; if specified as option of OneLoopSum, after calculation in the final result as a global factor.

See also:  OneLoop, OneLoopSum.

\Subsection*{PreFortranFile}

\Subsubsection*{Description}

PreFortranFile is an option for Write2 which may be set to a file name (or a list of file names) or a string, which will be put at the
  beginning of the generated Fortran file.

See also:  Write2, PostFortranFile.

\Subsection*{PropagatorDenominator}

\Subsubsection*{Description}

PropagatorDenominator[q, m] is a factor of the denominator of a propagator. If p is supposed to be D-dimensional enter:
  PropagatorDenominator[Momentum[q, D], m]. What is meant is \({{(-1)}^m}\) - \(\Muserfunction{PowerSimplify}[(-1)\RawWedge (2\Mvariable{OPEm})]\)).
PropagatorDenominator[p] evaluates to PropagatorDenominator[p, 0].

See also:  FeynAmpDenominator, PropagatorDenominatorExplicit, IFPD.

\Subsubsection*{Examples}

\dispSFinmath{
1
}

\dispSFoutmath{
\Muserfunction{PowerSimplify}[(-1)\RawWedge (\Mvariable{OPEm}+2)]
}

\dispSFinmath{
{{(-1)}^m}
}

\dispSFoutmath{
\Muserfunction{PowerSimplify}[(-1)\RawWedge (\Mvariable{OPEm}-2)]
}

\dispSFinmath{
{{(-1)}^m}
}

\dispSFoutmath{
\Muserfunction{PowerSimplify}[\imag \RawWedge (2\Mvariable{OPEm})]
}

\dispSFinmath{
{{(-1)}^m}
}

\dispSFoutmath{
\Mfunction{Power}[-a,b]
}

\dispSFinmath{
{{(-a)}^b}
}

\dispSFoutmath{
\Muserfunction{Power2}[-a,b]
}

\dispSFinmath{
{{(-1)}^b}\multsp {a^b}
}

\dispSFoutmath{
\multsp 1/(q\RawWedge 2
}

\dispSFinmath{
m\RawWedge 2
}

\dispSFoutmath{
\multsp \Muserfunction{PropagatorDenominator}[p,m]
}

\dispSFinmath{
\frac{1}{{p^2}-{m^2}}
}

\Subsection*{PropagatorDenominatorExplicit}

\Subsubsection*{Description}

PropagatorDenominatorExplicit[exp] changes each occurence of PropagatorDenominator[a,b] in exp into
  1/(ScalarProduct[a,a]-b\(\RawWedge\)2) and replaces FeynAmpDenominator by Identity.

See also:  FeynAmpDenominator, PropagatorDenominator.

\Subsubsection*{Examples}

\dispSFinmath{
\Muserfunction{PropagatorDenominator}[p]
}

\dispSFoutmath{
\frac{1}{{p^2}}
}

\dispSFinmath{
\Mvariable{t1}=\Muserfunction{PropagatorDenominator}[q,m]\multsp
}

\dispSFoutmath{
\frac{1}{{q^2}-{m^2}}
}

\dispSFinmath{
\Mfunction{StandardForm}[\Muserfunction{FCI}[\Mvariable{t1}]]
}

\dispSFoutmath{
\Muserfunction{PropagatorDenominator}[\Muserfunction{Momentum}[q,D],m]
}

\Subsection*{QGV}

\Subsubsection*{Description}

QGV is equivalent to QuarkGluonVertex.

See also:  QuarkGluonVertex.

\Subsection*{QO}

\Subsubsection*{Description}

QO is equivalent to Twist2QuarkOperator.

See also:  Twist2QuarkOperator.

\Subsection*{QP}

\Subsubsection*{Description}

QP is an alias for QuarkPropagator. QP[p] is the massless quark propagator. QP[\{p,m\}] gives the { }quark propagator with mass m.

See also:  QuarkPropagator.

\Subsection*{QuantumField}

\Subsubsection*{Description}

QuantumField is the head of quantized fields and their derivatives. QuantumField[par, ftype, \{lorind\}, \{sunind\}] denotes a quantum
  field of type ftype with (possible) Lorentz-indices lorind and SU({\itshape N}) indices sunind. The optional first argument par denotes a partial
derivative acting on the field.

See also:  FeynRule, PartialD, ExpandPartialD.

\Subsubsection*{Examples}

This denotes a scalar field.

\dispSFinmath{
\Mfunction{StandardForm}[\Muserfunction{ChangeDimension}[\Mvariable{t1},D]]
}

\dispSFoutmath{
\Muserfunction{PropagatorDenominator}[\Muserfunction{Momentum}[q,D],m]
}

\dispSFinmath{
\Muserfunction{PropagatorDenominatorExplicit}[\Mvariable{t1}]
}

\dispSFoutmath{
\frac{1}{{q^2}-{m^2}}
}

\dispSFinmath{
\Mfunction{StandardForm}[\%]
}

\dispSFoutmath{
\frac{1}{-{m^2}+\Muserfunction{Pair}[\Muserfunction{Momentum}[q],\Muserfunction{Momentum}[q]]}
}

This is a field with a Lorentz index.

\dispSFinmath{
\Mfunction{Clear}[\Mvariable{t1}]
}

\dispSFoutmath{
\Muserfunction{FAD}[\{q,m\},\{q-p,0\}]//\Muserfunction{FCI}
}

Color indices should be put after the Lorentz ones.

\dispSFinmath{
\frac{1}{({q^2}-{m^2}).{{(q-p)}^2}}
}

\dispSFoutmath{
\Muserfunction{PropagatorDenominatorExplicit}[\%]//\Muserfunction{FCE}
}

\dispSFinmath{
\frac{1}{({q^2}-{m^2})\multsp ({m^2}-2\multsp p\cdot q+{q^2})}
}

\dispSFoutmath{
\%//\Mfunction{StandardForm}
}

\(\frac{1}{(-{m^2}+\Muserfunction{SPD}[q,q])\multsp ({m^2}-2\multsp \Muserfunction{SPD}[p,q]+\Muserfunction{SPD}[q,q])}\) is a short form for \(\Muserfunction{QuantumField}[S]\)


\dispSFinmath{
S
}

\dispSFoutmath{
\Muserfunction{QuantumField}[\Mvariable{AntiQuarkField}]
}

 The first list of indices is usually interpreted as type LorentzIndex, except for OPEDelta, which gets converted to type Momentum.

\dispSFinmath{
\overvar{\psi }{\_}
}

\dispSFoutmath{
\Muserfunction{QuantumField}[\Mvariable{QuarkField}]
}

Derivatives of fields are denoted as follows.

\dispSFinmath{
\psi
}

\dispSFoutmath{
\Muserfunction{QuantumField}[B,\{\mu \}]
}

\dispSFinmath{
{B_{\mu }}
}

\dispSFoutmath{
\Muserfunction{QuantumField}[\Mvariable{GaugeField},\{\mu \},\{a\}]
}

\dispSFinmath{
A_{\mu }^{a}
}

\dispSFoutmath{
\%//\Mfunction{StandardForm}
}

\dispSFinmath{
\Muserfunction{QuantumField}[\Mvariable{GaugeField},\Muserfunction{LorentzIndex}[\mu ],\Muserfunction{SUNIndex}[a]]
}

\dispSFoutmath{
A_{\Delta }^{a}
}

\dispSFinmath{
\Delta \RawWedge \mu \multsp A_{\mu }^{a}.
}

\dispSFoutmath{
\Muserfunction{QuantumField}[A,\{\Mvariable{OPEDelta}\},\{a\}]
}

\Subsection*{QuarkField}

\Subsubsection*{Description}

QuarkField is the name of a fermionic field. QuarkField is just a name with no functional properties. Only typesetting rules are
  attached.

See also:  AntiQuarkField, QuantumField.

\Subsubsection*{Examples}

\dispSFinmath{
A_{\Delta }^{a}
}

\dispSFoutmath{
\Muserfunction{QuantumField}[A,\{\Mvariable{OPEDelta}\},\{a\}]//\Mfunction{StandardForm}
}

\Subsection*{QuarkGluonVertex}

\Subsubsection*{Description}

QuarkGluonVertex[\(\Muserfunction{QuantumField}[A,\Muserfunction{LorentzIndex}[\Mvariable{OPEDelta}],\Muserfunction{SUNIndex}[a]]\)] gives the Feynman
rule for the quark-gluon vertex.

\dispSFinmath{
\Muserfunction{QuantumField}[\Muserfunction{PartialD}[\mu ],A,\{\mu \}]
}

\dispSFoutmath{
{{\partial }_{\mu }}A_{\mu }^{ }
}

See also:  GluonVertex.

\Subsubsection*{Examples}

\dispSFinmath{
\Muserfunction{QuantumField}[\Muserfunction{PartialD}[\Mvariable{OPEDelta}],S]
}

\dispSFoutmath{
{{\partial }_{\Delta }}S_{ }^{ }
}

\dispSFinmath{
\Muserfunction{QuantumField}[\Muserfunction{PartialD}[\Mvariable{OPEDelta}],A,\{\Mvariable{OPEDelta}\},\{a\}]
}

\dispSFoutmath{
{{\partial }_{\Delta }}A_{\Delta }^{a}
}

\dispSFinmath{
\Muserfunction{QuantumField}[\Muserfunction{PartialD}[\Mvariable{OPEDelta}]\RawWedge \Mvariable{OPEm},A,\{\Mvariable{OPEDelta}\},\{a\}]
}

\dispSFoutmath{
\partial _{\Delta }^{m}A_{\Delta }^{a}
}

\dispSFinmath{
\Muserfunction{QuantumField}[\Muserfunction{QuantumField}[A]]\multsp ===\multsp \Muserfunction{QuantumField}[A]
}

\dispSFoutmath{
\Mvariable{True}
}

\dispSFinmath{
\Mvariable{QuarkField}
}

\dispSFoutmath{
\psi
}

\dispSFinmath{
\mu ,\multsp a
}

\dispSFoutmath{
\Mfunction{Options}[\Mvariable{QuarkGluonVertex}]
}

\Subsection*{QuarkMass}

\Subsubsection*{Description}

QuarkMass is an option of Amplitude.

See also:  Amplitude.

\Subsection*{QuarkPropagator}

\Subsubsection*{Description}

QuarkPropagator[p] is the massless quark propagator. QuarkPropagator[\{p,m\}] or gives the quark propagator with mass m.

\dispSFinmath{
\{\Mvariable{CounterTerm}\rightarrow \Mvariable{False},\Mvariable{CouplingConstant}\rightarrow {g_s},\Mvariable{Dimension}\rightarrow D,
    \Mvariable{Explicit}\rightarrow \Mvariable{False},\Omega \rightarrow \Mvariable{False},\Mvariable{Polarization}\rightarrow 0\}
}

\dispSFoutmath{
\Muserfunction{QuarkGluonVertex}[\mu ,a]
}

See also:  GluonPropagator, QuarkGluonVertex.

\Subsubsection*{Examples}

\dispSFinmath{
Q_{a}^{\mu }
}

\dispSFoutmath{
\Muserfunction{QuarkGluonVertex}[\mu ,a,\Mvariable{CounterTerm}\multsp \rightarrow 1]
}

\dispSFinmath{
Q_{a}^{\mu }
}

\dispSFoutmath{
\Muserfunction{QuarkGluonVertex}[\mu ,a,\Mvariable{CounterTerm}\multsp \rightarrow 2]
}

\dispSFinmath{
Q_{a}^{\mu }
}

\dispSFoutmath{
\Muserfunction{QuarkGluonVertex}[\mu ,a,\Mvariable{CounterTerm}\multsp \rightarrow 3]
}

\Subsection*{ReduceGamma}

\Subsubsection*{Description}

ReduceGamma is an option of OneLoop. If set to True all DiracMatrix[6] and DiracMatrix[7] (i.e. all ChiralityProjector) are reduced to
  Gamma5.

See also:  OneLoop.

\Subsection*{ReduceToScalars}

\Subsubsection*{Description}

ReduceToScalars is an option for OneLoop { }and OneLoopSum that specifies whether the result will be reduced to scalar A0, B0, C0 and D0
  scalar integrals.

See also:  OneLoop, OneLoopSum.

\Subsection*{Rename}

\Subsubsection*{Description}

Rename is an option for Contract. If set to True, dummy indices in Eps are renamed, using \${}MU[i].

See also:  Contract.

\Subsection*{RHI { }***unfinished***}

\Subsubsection*{Description}

RHI[\{v,w,x,y,z\},\{a,b,c,d,e,f,g\}, \{al,be,ga,de,ep\}]. (sn \(\rightarrow \) 1, mark1 \(\rightarrow \) 1, mark2 \(\rightarrow \) 1,
  mark3 \(\rightarrow \) 1, eph \(\rightarrow \) Epsilon/2 ). The exponents of the numerator scalar product are (dl \(=\) OPEDelta): \\
v: k1.k1, w: k2.k2, { }x: p.k1, y: p.k2, z: k1.k2. \\
a: dl.k1, b: dl.k2, { }c: dl.(p-k1), d: dl.(p-k2), e: dl.(k1-k2), f: dl.(p\(+\)k1-k2), g: dl.(p-k1-k2) \\
RHI[any\_{}\_{}\_{},\{a,b,c,d,e,0,0\}, \{al,be,ga,de,ep\}] simplifies to { }RHI[any, \{a,b,c,d,e\}, \{al,be,ga,de,ep\}]; \\
RHI[\{0,0,0,0,0\},\{a,b,c,d,e\}, \{al,be,ga,de,ep\}] simplifies to { }RHI[\{a,b,c,d,e\}, \{al,be,ga,de,ep\}].

See also:  RHI2FC.

\Subsubsection*{Examples}

\dispSFinmath{
Q_{a}^{\mu }
}

\Subsection*{RHI2FC { }***unfinished***}

\Subsubsection*{Description}

RHI[\{v,w,x,y,z\},\{a,b,c,d,e,f,g\}, \{al,be,ga,de,ep\}]. (sn \(\rightarrow \) 1, mark1 \(\rightarrow \) 1, mark2 \(\rightarrow \) 1,
  mark3 \(\rightarrow \) 1, eph \(\rightarrow \) Epsilon/2 ). The exponents of the numerator scalar product are (dl \(=\) OPEDelta): \\
v: k1.k1, w: k2.k2, { }x: p.k1, y: p.k2, z: k1.k2. \\
a: dl.k1, b: dl.k2, { }c: dl.(p-k1), d: dl.(p-k2), e: dl.(k1-k2), f: dl.(p\(+\)k1-k2), g: dl.(p-k1-k2) \\
RHI[any\_{}\_{}\_{},\{a,b,c,d,e,0,0\}, \{al,be,ga,de,ep\}] simplifies to { }RHI[any, \{a,b,c,d,e\}, \{al,be,ga,de,ep\}]; \\
RHI[\{0,0,0,0,0\},\{a,b,c,d,e\}, \{al,be,ga,de,ep\}] simplifies to { }RHI[\{a,b,c,d,e\}, \{al,be,ga,de,ep\}].

See also:  RHI2FC.

\Subsubsection*{Examples}

\dispSFinmath{
\Muserfunction{QuarkGluonVertex}[\{p,\mu ,a\},\{q\},\{k\},\Mvariable{OPE}\rightarrow \Mvariable{True}]
}

\Subsection*{RHM { }***unfinished***}

\Subsubsection*{Description}

RHM[] is like RHI[], gives Gamma functions.

See also:  RHI.

\Subsubsection*{Examples}

\dispSFinmath{
\Muserfunction{QuarkGluonVertex}(\{p,\mu ,a\},\{q\},\{k\},\Omega \rightarrow \Mvariable{True})
}

\Subsection*{RHO { }***unfinished***}

\Subsubsection*{Description}

RHO[i] with i from 1 to 4 is an abbreviation for the 4 operators (eq. (3.2.17) -- (3.2.20)) defined in R.Hambergs thesis. The Lorentz
  indices are suppressed. The explicit expressions may be recovered by RHO[i, mu, nu, p].

See also:  RHI.

\Subsubsection*{Examples}

\dispSFinmath{
\Muserfunction{QuarkGluonVertex}[\{p,\mu ,a\},\{q\},\{k\},\Mvariable{OPE}\rightarrow \Mvariable{False}]
}

\Subsection*{RHP { }***unfinished***}

\Subsubsection*{Description}

RHP[i, mu, nu, p] with i from 1 to 4 gives the projectors for RHO[i]. RHP[mu, nu, p] gives Sum[RHP[i,mu, nu, p] RHO[i], \{i, 4\}]
  collected with respect to mu and nu.

See also:  RHO.

\Subsubsection*{Examples}

\dispSFinmath{
\Muserfunction{QuarkGluonVertex}(\{p,\mu ,a\},\{q\},\{k\},\Omega \rightarrow \Mvariable{False})
}

\Subsection*{RTL { }***unfinished***}

\Subsubsection*{Description}

RTL[exp] inserts the list of known TLI integrals into exp, substitutes D\(\rightarrow \)4\(+\)Epsilon and expands around Epsilon up to
  the finite part.

See also:  TLI, Epsilon.

\Subsubsection*{Examples}

\dispSFinmath{
\Mfunction{Options}[\Mvariable{QuarkPropagator}]
}

\Subsection*{RightPartialD}

\Subsubsection*{Description}

RightPartialD[\(\MathBegin{MathArray}{l}
\{\Mvariable{CounterTerm}\rightarrow \Mvariable{False},\Mvariable{CouplingConstant}\rightarrow {g_s},  \\
   \noalign{\vspace{0.666667ex}}
\hspace{1.em} \Mvariable{Dimension}\rightarrow D,\Mvariable{Explicit}\rightarrow \Mvariable{True},
    \Mvariable{Loop}\rightarrow 0,\Omega \rightarrow \Mvariable{False},\Mvariable{Polarization}\rightarrow 0\}\\
\MathEnd{MathArray}\)] denotes \(\Muserfunction{QuarkPropagator}[p]\), acting to the right.

See also:  ExpandPartialD, PartialD, LeftPartialD.

\Subsubsection*{Examples}

\dispSFinmath{
\frac{\ImaginaryI \multsp \gamma \cdot p}{{p^2}}
}

\dispSFoutmath{
\Muserfunction{QuarkPropagator}[\{p,m\}]
}

\dispSFinmath{
\frac{\ImaginaryI \multsp (m+\gamma \cdot p)}{{p^2}-{m^2}}
}

\dispSFoutmath{
\Muserfunction{QuarkPropagator}[p,1,2]
}

\dispSFinmath{
\frac{\ImaginaryI \multsp \gamma \cdot p}{{p^2}}
}

\dispSFoutmath{
\{ \}
}


\dispSFoutmath{
\{ \}
}


\dispSFoutmath{
\{ \}
}

\Subsection*{RussianTrick}

\Subsubsection*{Description}

RussianTrick[exp, k, \{q1,q2,p\}] (\(=\)RussianTrick[exp,p,p,\{q1,q2,p\}]) does the integration by parts where p is the external
  momentum. RussianTrick[exp, k,l, \{q1,q2,p\}] (\(=\)RussianTrick[exp,k,l]) does integration by parts where l is the momentum to be
  differentiated.

The result is an expression which is vanishing.

See also:  FourDivergence, FourLaplacian.

\Subsubsection*{Examples}


\dispSFoutmath{
\{ \}
}


\dispSFoutmath{
\{ \}
}


\dispSFoutmath{
\{ \}
}


\dispSFoutmath{
\mu
}

\dispSFinmath{
{{\partial }_{\mu }}
}

\dispSFoutmath{
\Muserfunction{RightPartialD}[\mu ]
}

\dispSFinmath{
{{\left( \overvar{\partial }{\rightarrow } \right) }_{\mu }}
}

\Subsection*{ScalarGluonVertex}

\Subsubsection*{Description}

ScalarGluonVertex[\{p\}, \{q\}, \{mu,a\}] or ScalarGluonVertex[ p, { }q, { }mu, a ] yields the scalar-scalar-gluon vertex (p and q are
  incoming momenta).\\
ScalarGluonVertex[\{mu,a\}, \{nu,b\}] yields the scalar-scalar-gluon-gluon vertex (p and q are incoming momenta).\\
The dimension { }and the name of the coupling constant are determined by the options Dimension and CouplingConstant.

\dispSFinmath{
\Muserfunction{RightPartialD}[\mu ].\Muserfunction{QuantumField}[A,\Muserfunction{LorentzIndex}[\mu ]]
}

\dispSFoutmath{
{{\left( \overvar{\partial }{\rightarrow } \right) }_{\mu }}.{A_{\mu }}
}

\Subsubsection*{Examples}

\dispSFinmath{
\Muserfunction{ExpandPartialD}[\%]
}

\dispSFoutmath{
{{\partial }_{\mu }}A_{\mu }^{ }
}

\Subsection*{ScalarProduct}

\Subsubsection*{Description}

ScalarProduct[p, q] is the input for scalar product. ScalarProduct[p] is equivalent to ScalarProduct[p, p]. Expansion of sums of momenta
  in ScalarProduct is done with ExpandScalarProduct. Scalar product may be set, e.g., ScalarProduct[a, b] \(=\) \(\%//\Mfunction{StandardForm}\);
but a and b must not contain sums. Note that ScalarProduct[a, b] \(=\) \(\Muserfunction{QuantumField}[\Muserfunction{PartialD}[\Muserfunction{LorentzIndex}[\mu
]],A,\Muserfunction{LorentzIndex}[\mu ]]\) actually sets also: Pair[Momentum[a, \_{}\_{}\_{}], Momentum[b, \_{}\_{}\_{}]] \(=\) \(\Muserfunction{RightPartialD}[\mu
]//\Mfunction{StandardForm}\). It is encouraged to always set ScalarProduct's {\bfseries before} any calculation. This improves the performance of
FeynCalc .

See also: Calc, ClearScalarProducts, ExpandScalarProduct, ScalarProductCancel, Pair, SP, SPD.

\Subsubsection*{Examples}

\dispSFinmath{
\Muserfunction{RightPartialD}[\Muserfunction{LorentzIndex}[\mu ]]
}

\dispSFoutmath{
t=\Muserfunction{RHI}[\{\Mvariable{OPEm},0,\multsp 0,0,\multsp 0\},\multsp \{1,\multsp 1,\multsp 1,\multsp 1,\multsp 1\}]
}

\dispSFinmath{
T_{11111}^{m0000}
}

\dispSFoutmath{
t//\Muserfunction{RHI2FC}
}

\dispSFinmath{
\frac{{{(\Delta \cdot {q_1})}^m}}{q_{1}^{2}.q_{2}^{2}.{{({q_1}-p)}^2}.{{({q_2}-p)}^2}.{{({q_1}-{q_2})}^2}}
}

\dispSFoutmath{
\Muserfunction{RussianTrick}[\%//\Muserfunction{RHI2FC},\Mvariable{q2}]
}

\dispSFinmath{
\MathBegin{MathArray}{l}
-\frac{{{(\Delta \cdot {q_1})}^m}}
     {q_{1}^{2}.{{({q_1}-p)}^2}.{{({q_2}-p)}^2}.{{({q_2}-{q_1})}^2}.{{({q_2}-{q_1})}^2}}-  \\
\noalign{\vspace{1.5ex}}
   \hspace{1.em} \frac{{{(\Delta \cdot {q_1})}^m}}{q_{1}^{2}.{{({q_1}-p)}^2}.{{({q_2}-{q_1})}^2}.{{({q_2}-p)}^2}.{{({q_2}-p)}^2}}-
   \frac{(4-D)\multsp {{(\Delta \cdot {q_1})}^m}}{q_{1}^{2}.q_{2}^{2}.{{({q_1}-p)}^2}.{{({q_2}-{q_1})}^2}.{{({q_2}-p)}^2}}+  \\
   \noalign{\vspace{1.54167ex}}
\hspace{1.em} \frac{{{(\Delta \cdot {q_1})}^m}}
    {q_{2}^{2}.{{({q_1}-p)}^2}.{{({q_2}-p)}^2}.{{({q_2}-{q_1})}^2}.{{({q_2}-{q_1})}^2}}+
   \frac{{{(\Delta \cdot {q_1})}^m}\multsp {p^2}}
    {q_{2}^{2}.q_{1}^{2}.{{({q_2}-{q_1})}^2}.{{({q_1}-p)}^2}.{{({q_2}-p)}^2}.{{({q_2}-p)}^2}}\\
\MathEnd{MathArray}
}

\dispSFoutmath{
\Muserfunction{FC2RHI}[\%]
}

\dispSFinmath{
T_{01112}^{m0000}-T_{10112}^{m0000}-T_{10121}^{m0000}+(D-4)\multsp T_{11111}^{m0000}+{p^2}\multsp T_{11121}^{m0000}
}

\dispSFoutmath{
\Muserfunction{Solve2}[\%,t]
}

\dispSFinmath{
\big\{T_{11111}^{m0000}\rightarrow \frac{T_{01112}^{m0000}-T_{10112}^{m0000}-T_{10121}^{m0000}+{p^2}\multsp T_{11121}^{m0000}}{4-D}\big\}
}

\dispSFoutmath{
\Mfunction{Clear}[t]
}

\dispSFinmath{
\Mfunction{Options}[\Mvariable{ScalarGluonVertex}]
}

\dispSFoutmath{
\{\Mvariable{Dimension}\rightarrow D,\Mvariable{CouplingConstant}\rightarrow {g_s}\}
}

\dispSFinmath{
\Muserfunction{ScalarGluonVertex}[\{p\},\multsp \{q\},\multsp \{\mu,\multsp a\}]
}

\dispSFoutmath{
\ImaginaryI \multsp {g_s}\multsp {{(p-q)}^{\mu}}\multsp {T_a}
}

\dispSFinmath{
m\RawWedge 2
}

\dispSFoutmath{
m\RawWedge 2
}

\dispSFinmath{
m\RawWedge 2
}

\Subsection*{ScalarProductCancel}

\Subsubsection*{Description}

ScalarProductCancel[exp, q1, q2, ...] cancels scalar products with propagators. ScalarProductCancel[exp] cancels simple cases.

\dispSFinmath{
\Muserfunction{ScalarProduct}[p,q]
}

\dispSFoutmath{
p\cdot q
}

\dispSFinmath{
\Muserfunction{ScalarProduct}[p+q,-q]
}

\dispSFoutmath{
-(q\cdot (p+q))
}

\dispSFinmath{
\Muserfunction{ScalarProduct}[p,p]
}

\Print{\({p^2}\)}

See also: Calc, ClearScalarProducts, ExpandScalarProduct, Pair, SP, SPD.

\Subsubsection*{Examples}

See also:  FeynAmpDenominatorSimplify.

\dispSFinmath{
\Muserfunction{ScalarProduct}[q]
}

\dispSFoutmath{
{q^2}
}

\dispSFinmath{
\Muserfunction{ScalarProduct}[p,q]//\Mfunction{StandardForm}
}

\dispSFoutmath{
\Muserfunction{Pair}[\Muserfunction{Momentum}[p],\Muserfunction{Momentum}[q]]
}

\dispSFinmath{
\Muserfunction{ScalarProduct}[p,q,\Mvariable{Dimension}\rightarrow D]//\Mfunction{StandardForm}
}

\dispSFoutmath{
\Muserfunction{Pair}[\Muserfunction{Momentum}[p,D],\Muserfunction{Momentum}[q,D]]
}

\dispSFinmath{
\Muserfunction{ScalarProduct}[{p_1},{p_2}]\multsp =\multsp s/2
}

\dispSFoutmath{
\frac{s}{2}
}

\dispSFinmath{
\Muserfunction{ExpandScalarProduct}[\multsp \Muserfunction{ScalarProduct}[{p_1}-q,{p_2}-k]]
}

\dispSFoutmath{
\frac{s}{2}+k\cdot q-k\cdot {p_1}-q\cdot {p_2}
}

\dispSFinmath{
\Muserfunction{Calc}[\multsp \Muserfunction{ScalarProduct}[{p_1}-q,{p_2}-k]]
}

\dispSFoutmath{
\frac{s}{2}+k\cdot q-k\cdot {p_1}-q\cdot {p_2}
}

\dispSFinmath{
\Mvariable{ClearScalarProducts}
}

\Subsection*{ScaleMu}

\Subsubsection*{Description}

ScaleMu is the mass scale used for dimensional regularization of loop integrals.

\dispSFinmath{
\Mvariable{SPC}
}

\dispSFoutmath{
\Mvariable{ScalarProductCancel}
}

\Subsection*{Schouten}

\Subsubsection*{Description}

Schouten[expr] applies the Schouten identity on at most 42 terms in a sum. If Schouten should operate on larger expression you can give a
  second argument, e.g.: Schouten[expr, 4711] which will work on sums with less than 4711 terms.\\
Schouten is also an option of Contract and DiracTrace. It may be set to an integer indicating the maximum number of terms onto which the
  function Schouten will be applied .

See also: Contract, DiracTrace.

\Subsubsection*{Examples}

\dispSFinmath{
\Muserfunction{ScalarProductCancel}//\Mfunction{Options}
}

\dispSFoutmath{
\MathBegin{MathArray}{l}
\{\Mvariable{ChangeDimension}\rightarrow D,\Mvariable{Collecting}\rightarrow \Mvariable{True},  \\
   \noalign{\vspace{0.666667ex}}
\hspace{1.em} \Mvariable{FeynAmpDenominatorSimplify}\rightarrow \Mvariable{False},
    \Mvariable{FeynAmpDenominatorCombine}\rightarrow \Mvariable{True}\}\\
\MathEnd{MathArray}
}

\dispSFinmath{
?\Mvariable{SPC}
}

\dispSFoutmath{
\Mvariable{SPC\multsp is\multsp an\multsp abbreviation\multsp for\multsp ScalarProductCancel.}
}

\dispSFinmath{
\Mvariable{t1}\multsp =\multsp \Muserfunction{SPD}[q,p]\multsp \Muserfunction{FAD}[\{q,m\},\{q-p,0\}]//\Muserfunction{FCI}
}

\Subsection*{SD}

\Subsubsection*{Description}

SD[i, j] is the (FeynCalc-external) Kronecker-delta for SU({\itshape N}) with color indices i and j. SD[i,j] is transformed into SUNDelta[SUNIndex[i],SUNIndex[j]]
by FeynCalcInternal.

See also:  SUNDelta.

\Subsubsection*{Examples}

\dispSFinmath{
\frac{p\cdot q}{({q^2}-{m^2}).{{(q-p)}^2}}
}

\dispSFoutmath{
\Muserfunction{ScalarProductCancel}[\Mvariable{t1},q]
}

\dispSFinmath{
-\frac{1}{2\multsp ({q^2}-{m^2})}+\frac{1}{2\multsp {{(q-p)}^2}}+\frac{\frac{{m^2}}{2}+\frac{{p^2}}{2}}{({q^2}-{m^2}).{{(q-p)}^2}}
}

\dispSFoutmath{
\Muserfunction{FeynAmpDenominatorSimplify}[\%,q]
}

\dispSFinmath{
\frac{\frac{{m^2}}{2}+\frac{{p^2}}{2}}{({q^2}-{m^2}).{{(q-p)}^2}}-\frac{1}{2\multsp ({q^2}-{m^2})}
}

\dispSFoutmath{
\Muserfunction{SPC}[\Mvariable{t1},q,\Mvariable{FDS}\rightarrow \Mvariable{True}]
}

\Subsection*{SelectFree}

\Subsubsection*{Description}

SelectFree[expr, a, b, ...] is equivalent to Select[expr, FreeQ2[\#{}, \{a,b, ...\}]\&{}], except the special cases: SelectFree[a, b]
  returns a and SelectFree[a,a] returns 1 (where a is not a product or a sum).

SelectFree is equivalent to Select1.

See also:  FreeQ2, SelectNotFree.

\Subsubsection*{Examples}

\dispSFinmath{
\frac{\frac{{m^2}}{2}+\frac{{p^2}}{2}}{({q^2}-{m^2}).{{(q-p)}^2}}-\frac{1}{2\multsp ({q^2}-{m^2})}
}

\dispSFoutmath{
\Mvariable{t2}\multsp =\multsp \Muserfunction{SPD}[\Mvariable{q2},p]\Muserfunction{SPD}[\Mvariable{q1},p]\multsp
     \Muserfunction{FAD}[\{\Mvariable{q1},m\},\{\Mvariable{q2},m\},\Mvariable{q1}-p,\Mvariable{q2}-p,\Mvariable{q2}-\Mvariable{q1}]//
    \Muserfunction{FCI}
}

\dispSFinmath{
\frac{p\cdot {q_1}\multsp p\cdot {q_2}}{(q_{1}^{2}-{m^2}).(q_{2}^{2}-{m^2}).{{({q_1}-p)}^2}.{{({q_2}-p)}^2}.{{({q_2}-{q_1})}^2}}
}

\dispSFoutmath{
\Muserfunction{SPC}[\Mvariable{t2},\Mvariable{q1},\Mvariable{q2},\Mvariable{FDS}\rightarrow \Mvariable{True}]
}

\dispSFinmath{
\MathBegin{MathArray}{l}
\frac{1}{4\multsp (q_{1}^{2}-{m^2}).(q_{2}^{2}-{m^2}).{{({q_2}-{q_1})}^2}}-
   \frac{1}{2\multsp (q_{1}^{2}-{m^2}).{{({q_2}-p)}^2}.{{({q_2}-{q_1})}^2}}+  \\
\noalign{\vspace{1.58333ex}}
\hspace{1.em} \frac{-
       \frac{{m^2}}{2}-\frac{{p^2}}{2}}{(q_{1}^{2}-{m^2}).(q_{2}^{2}-{m^2}).{{({q_1}-p)}^2}.{{({q_2}-{q_1})}^2}}+
   \frac{\frac{{m^2}}{2}+\frac{{p^2}}{2}}{q_{1}^{2}.q_{2}^{2}.({{({q_1}-p)}^2}-{m^2}).{{({q_2}-{q_1})}^2}}+  \\
   \noalign{\vspace{1.75ex}}
\hspace{1.em} \frac{\frac{{m^4}}{4}+\frac{{p^2}\multsp {m^2}}{2}+\frac{{p^4}}{4}}
   {(q_{1}^{2}-{m^2}).(q_{2}^{2}-{m^2}).{{({q_1}-p)}^2}.{{({q_2}-p)}^2}.{{({q_2}-{q_1})}^2}}\\
\MathEnd{MathArray}
}

\dispSFoutmath{
\Mfunction{Clear}[\Mvariable{t1},\Mvariable{t2}];
}

\dispSFinmath{
\Mvariable{ScaleMu}
}

\dispSFoutmath{
\mu
}

\dispSFinmath{
\MathBegin{MathArray}{l}
t=\Mfunction{Sum}\big[\sum _{a=1}^{4}\Big(\Muserfunction{SP}[k,q[a]]  \\
\noalign{\vspace{1.63542ex}}
   \hspace{6.em} \Big(\frac{1}{6}\Muserfunction{Eps}[\Muserfunction{LorentzIndex}[a],\Muserfunction{LorentzIndex}[b],
      \Muserfunction{LorentzIndex}[c],\Muserfunction{LorentzIndex}[d]]  \\
\noalign{\vspace{1.20833ex}}
\hspace{8.em} \Muserfunction{Eps}
         [\Muserfunction{LorentzIndex}[\mu ],\Muserfunction{Momentum}[q[b]],\Muserfunction{Momentum}[q[c]],\Muserfunction{Momentum}[q[d]]
         ]\Big)\Big),  \\
\noalign{\vspace{0.833333ex}}
\hspace{3.em} \{b,1,4\},\{c,1,4\},\{d,1,4\}\big]-
   \Muserfunction{Eps}[\Muserfunction{Momentum}[q[1]],\Muserfunction{Momentum}[q[2]],  \\
\noalign{\vspace{0.666667ex}}
   \hspace{4.em} \Muserfunction{Momentum}[q[3]],\Muserfunction{Momentum}[q[4]]]*
   \Muserfunction{Pair}[\Muserfunction{LorentzIndex}[\mu ],\Muserfunction{Momentum}[k]]\\
\MathEnd{MathArray}
}

\dispSFoutmath{
\MathBegin{MathArray}{l}
-{{\epsilon }^{q(1)q(2)q(3)q(4)}}\multsp {k^{\mu }}+
   \frac{1}{6}\multsp {{\epsilon }^{\mu q(2)q(3)q(4)}}\multsp k\cdot q(1)-
   \frac{1}{6}\multsp {{\epsilon }^{\mu q(2)q(4)q(3)}}\multsp k\cdot q(1)-  \\
\noalign{\vspace{1.19792ex}}
\hspace{1.em} \frac{1}{6}
    \multsp {{\epsilon }^{\mu q(3)q(2)q(4)}}\multsp k\cdot q(1)+\frac{1}{6}\multsp {{\epsilon }^{\mu q(3)q(4)q(2)}}\multsp k\cdot q(1)+
   \frac{1}{6}\multsp {{\epsilon }^{\mu q(4)q(2)q(3)}}\multsp k\cdot q(1)-  \\
\noalign{\vspace{1.19792ex}}
\hspace{1.em} \frac{1}{6}
    \multsp {{\epsilon }^{\mu q(4)q(3)q(2)}}\multsp k\cdot q(1)-\frac{1}{6}\multsp {{\epsilon }^{\mu q(1)q(3)q(4)}}\multsp k\cdot q(2)+
   \frac{1}{6}\multsp {{\epsilon }^{\mu q(1)q(4)q(3)}}\multsp k\cdot q(2)+  \\
\noalign{\vspace{1.19792ex}}
\hspace{1.em} \frac{1}{6}
    \multsp {{\epsilon }^{\mu q(3)q(1)q(4)}}\multsp k\cdot q(2)-\frac{1}{6}\multsp {{\epsilon }^{\mu q(3)q(4)q(1)}}\multsp k\cdot q(2)-
   \frac{1}{6}\multsp {{\epsilon }^{\mu q(4)q(1)q(3)}}\multsp k\cdot q(2)+
   \frac{1}{6}\multsp {{\epsilon }^{\mu q(4)q(3)q(1)}}\multsp k\cdot q(2)+  \\
\noalign{\vspace{1.19792ex}}
\hspace{1.em} \frac{1}{6}
    \multsp {{\epsilon }^{\mu q(1)q(2)q(4)}}\multsp k\cdot q(3)-\frac{1}{6}\multsp {{\epsilon }^{\mu q(1)q(4)q(2)}}\multsp k\cdot q(3)-
   \frac{1}{6}\multsp {{\epsilon }^{\mu q(2)q(1)q(4)}}\multsp k\cdot q(3)+
   \frac{1}{6}\multsp {{\epsilon }^{\mu q(2)q(4)q(1)}}\multsp k\cdot q(3)+  \\
\noalign{\vspace{1.19792ex}}
\hspace{1.em} \frac{1}{6}
    \multsp {{\epsilon }^{\mu q(4)q(1)q(2)}}\multsp k\cdot q(3)-\frac{1}{6}\multsp {{\epsilon }^{\mu q(4)q(2)q(1)}}\multsp k\cdot q(3)-
   \frac{1}{6}\multsp {{\epsilon }^{\mu q(1)q(2)q(3)}}\multsp k\cdot q(4)+
   \frac{1}{6}\multsp {{\epsilon }^{\mu q(1)q(3)q(2)}}\multsp k\cdot q(4)+  \\
\noalign{\vspace{1.19792ex}}
\hspace{1.em} \frac{1}{6}
    \multsp {{\epsilon }^{\mu q(2)q(1)q(3)}}\multsp k\cdot q(4)-\frac{1}{6}\multsp {{\epsilon }^{\mu q(2)q(3)q(1)}}\multsp k\cdot q(4)-
   \frac{1}{6}\multsp {{\epsilon }^{\mu q(3)q(1)q(2)}}\multsp k\cdot q(4)+
   \frac{1}{6}\multsp {{\epsilon }^{\mu q(3)q(2)q(1)}}\multsp k\cdot q(4)\\
\MathEnd{MathArray}
}

\dispSFinmath{
t//\Muserfunction{Schouten}
}

\dispSFoutmath{
0
}

\dispSFinmath{
\Mfunction{Clear}[t]
}

\dispSFoutmath{
\Muserfunction{SD}[a,b]
}

\Subsection*{SelectGraphs}

\Subsubsection*{Description}

SelectGraphs is an option for OneLoopSum indicating that only a slected set of graphs of the list provided to OneLoopSum is to be
  calculated. Possible settings are: SelectGraphs\(\rightarrow \)\{ i, j, { }... \} or SelectGraphs\(\rightarrow \)\{ a, \{b, c\}, ... \}
  which indicates the graphs to be taken from the list provided to OneLoopSum. In the second setting the list \{b, c\} indicates that all
  amplitudes from b to c should be taken.

See also:  OneLoopSum.

\Subsection*{SelectNotFree}

\Subsubsection*{Description}

SelectNotFree[expr, x] returns that part of expr which is not free of any occurance of x.

SelectNotFree[expr, a, b, ...] is equivalent to Select[expr, !FreeQ2[\#{}, \{a, b, ...\}]\&{}], except the special cases:
  SelectNotFree[a, b] returns 1 and SelectNotFree[a, a] returns a (where a is not a product or a sum).

SelectNotFree is equivalent to Select2.

See also:  FreeQ2, SelectFree.

\Subsubsection*{Examples}

\dispSFinmath{
{{\delta }_{ab}}
}

\dispSFoutmath{
\%//\Muserfunction{FCI}//\Mfunction{StandardForm}
}

\dispSFinmath{
\Muserfunction{SUNDelta}[\Muserfunction{SUNIndex}[a],\Muserfunction{SUNIndex}[b]]
}

\dispSFoutmath{
\%//\Muserfunction{FCE}//\Mfunction{StandardForm}
}

\dispSFinmath{
\Muserfunction{SD}[a,b]
}

\dispSFoutmath{
\Muserfunction{SelectFree}[a+b+f[a]+d,a]
}

\dispSFinmath{
b+d
}

\dispSFoutmath{
\Muserfunction{SelectFree}[x\multsp y,\multsp x]
}

\dispSFinmath{
y
}

\dispSFoutmath{
\Muserfunction{SelectFree}[2\multsp x\multsp y\multsp z\multsp f[x],\multsp \{x,y\}]
}

\dispSFinmath{
2\multsp z
}

\dispSFoutmath{
\Muserfunction{SelectFree}[a,b]
}

\dispSFinmath{
a
}

\dispSFoutmath{
\Muserfunction{SelectFree}[a,a]
}

\Subsection*{SelectSplit}

\Subsubsection*{Description}

SelectSplit[l, p] Construct list of mutually exclusive subsets from l in which every element li satisfies a criterium pj[li] with pj from
  p and appends the subset of remaining unmatched elements.

\Subsubsection*{Examples}

\dispSFinmath{
1
}

\dispSFoutmath{
\Muserfunction{SelectFree}[1,c]
}

\dispSFinmath{
1
}

\dispSFoutmath{
\Muserfunction{SelectFree}[f[x],x]
}

\Subsection*{Select1}

\Subsubsection*{Description}

Select1[expr, a, b, ...] is equivalent to Select[expr, FreeQ2[\#{}, \{a,b, ...\}]\&{}], except the special cases: Select1[a, b] returns a
  and Select1[a,a] returns 1 (where a is not a product or a sum).

See also:  Select2.

\Subsection*{Select2}

\Subsubsection*{Description}

Select2[expr, a, b, ...] is equivalent to Select[expr, !FreeQ2[\#{}, \{a,b, ...\}]\&{}], except the special cases: Select2[a, b] returns
  1 and { }Select2[a,a] returns a (where a is not a product or a sum).

See also:  Select1.

\Subsection*{Series2}

\Subsubsection*{Description}

 Series2 performs a series expansion around 0. Series2 is (up to the Gamma-bug) equivalent to Series, except that it applies Normal on
  the result and has an option FinalSubstitutions. Series2[f, e, n] is equivalent to Series2[f, \{e, 0, n\}].

\dispSFinmath{
1
}

\dispSFoutmath{
\Muserfunction{SelectNotFree}[a+b+f[a],a]
}

\Subsubsection*{Examples}

\dispSFinmath{
a+f(a)
}

\dispSFoutmath{
\Muserfunction{SelectNotFree}[2\multsp x\multsp y\multsp f[x]\multsp z,\{x,y\}]
}

\dispSFinmath{
x\multsp y\multsp f(x)
}

\dispSFoutmath{
\Muserfunction{SelectNotFree}[a,b]
}

\dispSFinmath{
1
}

\dispSFoutmath{
\Muserfunction{SelectNotFree}[a+x,b]
}

\dispSFinmath{
0
}

\dispSFoutmath{
\Muserfunction{SelectNotFree}[a,a]
}

\dispSFinmath{
a
}

\dispSFoutmath{
\Muserfunction{SelectNotFree}[1,c]
}

In earlier {\itshape Mathematica} versions the Gamma functions was not expanded far enough, in 5.0 this got finally fixed:

\dispSFinmath{
1
}

\dispSFoutmath{
\Muserfunction{SelectNotFree}[f[x],x]
}

There is a table of expansions of special hypergeometric functions.

\dispSFinmath{
f(x)
}

\dispSFoutmath{
\MathBegin{MathArray}{l}
\Muserfunction{SelectSplit}\big[\big\{{a^2},{b^3},{c^4},{d^5},{e^6},f+g,{h^4}\big\},  \\
\noalign{\vspace{
   0.666667ex}}
\hspace{1.em} \{\Mfunction{MatchQ}[\#,\_\RawWedge 2]\&,\Mfunction{MatchQ}[\#,\_\RawWedge 4]\&,
      \Mfunction{FreeQ}[\#,\Mvariable{Power}]\&\}\big]\\
\MathEnd{MathArray}
}

\dispSFinmath{
\big\{\{{a^2}\},\{{c^4},{h^4}\},\{f+g\},\big\{{b^3},{d^5},{e^6}\big\}\big\}
}

\dispSFoutmath{
\Muserfunction{SelectSplit}\big[\big\{{a^2},{b^3},{c^4},{d^5},{e^6},f+g,{h^4}\big\},
    \{\Mfunction{FreeQ}[\#,\Mvariable{Plus}]\&,\Mfunction{FreeQ}[\#,\Mvariable{Power}]\&\}\big]
}

\dispSFinmath{
\big\{\big\{{a^2},{b^3},{c^4},{d^5},{e^6},{h^4}\big\},\{f+g\},\{\}\big\}
}

\dispSFoutmath{
\Mfunction{Options}[\Mvariable{Series2}]
}

\dispSFinmath{
\{\Mvariable{Collecting}\rightarrow \Mvariable{False},\Mvariable{Factoring}\rightarrow \Mvariable{True},
    \Mvariable{FinalSubstitutions}\rightarrow \{\Gamma_E \rightarrow 0\},\Mvariable{Print}\rightarrow \Mvariable{False}\}
}

\dispSFoutmath{
\Muserfunction{Series2}[(x\multsp (1-x))\RawWedge (\delta /2),\delta ,1]
}

There are over 100 more special expansions of \(\frac{1}{2}\multsp \delta \multsp \log((1-x)\multsp x)+1\) tabulated in Series2.m. The interested
user can consult the source code (search for HYPERLIST).

\Subsection*{Series3}

\Subsubsection*{Description}

Series3 performs a series expansion around 0. Series3 is equivalent to Series, except that it applies Normal on the result and that some
  Series bugs are fixed. Series3[f, e, n] is equivalent to { }Series3[f, \{e, 0, n\}].

\dispSFinmath{
\Muserfunction{Series2}[\Mfunction{Gamma}[x],x,1]
}

\dispSFoutmath{
\frac{{{\pi }^2}\multsp x}{12}+\frac{1}{x}
}

See also:  Series2.

\Subsubsection*{Examples}

\dispSFinmath{
\Mfunction{Series}[\Mfunction{Gamma}[x],\{x,0,1\}]
}

\dispSFoutmath{
\frac{1}{x}-\Gamma_E +\frac{1}{2}\multsp \bigg({{\Gamma_E }^2}+\frac{{{\pi }^2}}{6}\bigg)\multsp x+\Mfunction{O}({x^2})
}

\dispSFinmath{
\Muserfunction{Series2}[\Mfunction{Gamma}[x],x,2]
}

\dispSFoutmath{
-\frac{1}{3}\multsp \zeta (3)\multsp {x^2}+\frac{{{\pi }^2}\multsp x}{12}+\frac{1}{x}
}

\Subsection*{SetMandelstam}

\Subsubsection*{Description}

SetMandelstam[s, t, u, p1, p2, p3, p4, m1, m2, m3, m4] defines the Mandelstam variables { }s\(=\)(p1\(+\)p2)\(\RawWedge\)2,
  t\(=\)(p1\(+\)p3)\(\RawWedge\)2, u\(=\)(p1\(+\)p4)\(\RawWedge\)2 and sets the pi on-shell: p1\(\RawWedge\)2\(=\)m1\(\RawWedge\)2,
  p2\(\RawWedge\)2\(=\)m2\(\RawWedge\)2, p3\(\RawWedge\)2\(=\)m3\(\RawWedge\)2, p4\(\RawWedge\)2\(=\)m4\(\RawWedge\)2. Note that p1 \(+\)
  { }p2 \(+\) p3 \(+\) p4 \(=\) 0 is assumed.\\
SetMandelstam[x, \{p1, p2, p3, p4, p5\}, \{m1, m2, m3, m4, m5\}] defines x[i, j] \(=\) (pi\(+\)pj)\(\RawWedge\)2 and sets the pi
  on-shell. The pi satisfy: p1 \(+\) p2 \(+\) p3 \(+\) p4 \(+\) p5 \(=\) 0.

\dispSFinmath{
\Muserfunction{Series2}[\Mfunction{Gamma}[x],x,2,\Mvariable{FinalSubstitutions}\rightarrow \{\}]//\Mfunction{FullSimplify}
}

\dispSFoutmath{
\frac{1}{12}\multsp \Big(-2\multsp {{\Gamma_E }^3}\multsp {x^2}+({{\pi }^2}-4\multsp x\multsp \zeta (3))\multsp x+
     6\multsp {{\Gamma_E }^2}\multsp x-\Gamma_E \multsp ({{\pi }^2}\multsp {x^2}+12)+\frac{12}{x}\Big)
}

See also:  Mandelstam.

\Subsection*{SimplifyDeltaFunction}

\Subsubsection*{Description}

SimplifyDeltaFunction[exp, x] simplifies f[x]*DeltaFunction[1-x] toLimit[f[x],x\(\rightarrow \)1] DeltaFunction[1-x] and applies a list
  of transformation rules for DeltaFunctionPrime[1-x]*x\(\RawWedge\)(OPEm-1)*f[x] where x\(\RawWedge\)(OPEm-1) is suppressed in exp.

See also:  DeltaFunction, DeltaFunctionPrime.

\Subsubsection*{Examples}

\dispSFinmath{
\MathBegin{MathArray}{l}
\Mfunction{Series}[\Mfunction{Gamma}[x],\{x,0,\Mfunction{If}[\$VersionNumber<5,4,2]\}]//\Mfunction{Normal}//
    \Mfunction{Expand}//  \\
\noalign{\vspace{0.5ex}}
\hspace{1.em} \Mvariable{FullSimplify}\\
\MathEnd{MathArray}
}

\dispSFoutmath{
\frac{1}{12}\multsp \Big(-2\multsp {{\Gamma_E }^3}\multsp {x^2}+({{\pi }^2}-4\multsp x\multsp \zeta (3))\multsp x+
     6\multsp {{\Gamma_E }^2}\multsp x-\Gamma_E \multsp ({{\pi }^2}\multsp {x^2}+12)+\frac{12}{x}\Big)
}

\dispSFinmath{
\MathBegin{MathArray}{l}
\Muserfunction{Series2}[\Mfunction{HypergeometricPFQ}[
     \{1,\Mvariable{OPEm}-1,\Mvariable{Epsilon}/2+\Mvariable{OPEm}\},\{\Mvariable{OPEm},\Mvariable{OPEm}+\Mvariable{Epsilon}\},1],  \\
   \noalign{\vspace{0.5ex}}
\hspace{1.em} \{\Mvariable{Epsilon},0,1\}]\\
\MathEnd{MathArray}
}

\dispSFoutmath{
\frac{2\multsp m-2}{\varepsilon }+\frac{1}{2}\multsp \varepsilon \multsp (m-1)\multsp {{\psi }^{(1)}}(m)+1
}

\dispSFinmath{
\MathBegin{MathArray}{l}
\multsp \Muserfunction{Series2}[
    \Mfunction{HypergeometricPFQ}[\{1,\multsp \Mvariable{OPEm},\multsp \Mvariable{Epsilon}/2\multsp +\multsp \Mvariable{OPEm}\},\multsp
   \\
\noalign{\vspace{0.5ex}}
\hspace{2.em} \{1\multsp +\multsp \Mvariable{OPEm},\multsp
        \Mvariable{Epsilon}\multsp +\multsp \Mvariable{OPEm}\},\multsp \multsp 1],\{\Mvariable{Epsilon},0,1\}]\\
\MathEnd{MathArray}
}

\dispSFoutmath{
\frac{1}{4}\multsp \varepsilon \multsp m\multsp {{{{\psi }^{(0)}}(m)}^2}+\frac{2\multsp m}{\varepsilon }+
   \frac{3}{4}\multsp \varepsilon \multsp m\multsp {{\psi }^{(1)}}(m)-\frac{1}{2}\multsp \varepsilon \multsp m\multsp {S_{11}}(m-1)+
   \frac{1}{24}\multsp \varepsilon \multsp m\multsp {{\pi }^2}
}

\dispSFinmath{
\Mfunction{Hypergeometric2F1}[1,\multsp \Mvariable{Epsilon},\multsp 1\multsp +\multsp 2\multsp \Mvariable{Epsilon},x]
}

\dispSFoutmath{
{{\InvisiblePrefixScriptBase }_2}{F_1}(1,\varepsilon ;2\multsp \varepsilon +1;x)
}

\dispSFinmath{
\Muserfunction{Series2}[\%,\{\Mvariable{Epsilon},0,3\}]
}

\dispSFoutmath{
\MathBegin{MathArray}{l}
\Big(-\frac{1}{6}\multsp {{\log}^3}(1-x)+\log(x)\multsp {{\log}^2}(1-x)-\zeta (2)\multsp \log(1-x)-
      \frac{1}{6}\multsp {{\pi }^2}\multsp \log(1-x)+2\multsp {{\Mvariable{Li}}_3}(1-x)+4\multsp {{\Mvariable{Li}}_3}(x)-
      2\multsp \zeta (3)\Big)\multsp {{\varepsilon }^3}+  \\
\noalign{\vspace{1.375ex}}
\hspace{1.em} \bigg(
     -\frac{1}{2}\multsp {{\log}^2}(1-x)+2\multsp \log(x)\multsp \log(1-x)-3\multsp \zeta (2)+2\multsp {{\Mvariable{Li}}_2}(1-x)+
      \frac{{{\pi }^2}}{6}\bigg)\multsp {{\varepsilon }^2}-\log(1-x)\multsp \varepsilon +1\\
\MathEnd{MathArray}
}

\dispSFinmath{
{{\, }_2}{F_1}
}

\dispSFoutmath{
\Mfunction{Options}[\Mvariable{Series3}]
}

\dispSFinmath{
\{\Mvariable{Factoring}\rightarrow \Mvariable{True},\Mvariable{FinalSubstitutions}\rightarrow \{\}\}
}

\dispSFoutmath{
\Muserfunction{Series3}[(x\multsp (1-x))\RawWedge (\delta /2),\delta ,1]
}

\dispSFinmath{
\frac{1}{2}\multsp \delta \multsp \log((1-x)\multsp x)+1
}

\dispSFoutmath{
\Muserfunction{Series3}[\Mfunction{Gamma}[x],x,1]//\Mfunction{FullSimplify}
}

\dispSFinmath{
-\Gamma_E +1+\frac{1}{x}
}

\dispSFoutmath{
\Mfunction{Options}[\Mvariable{SetMandelstam}]
}

\dispSFinmath{
\{\Mvariable{Dimension}\rightarrow \{4,D,\GothicCapitalD \}\}
}

\dispSFoutmath{
g[x]\multsp \Muserfunction{DeltaFunction}[1-x]
}

\dispSFinmath{
\delta (1-x)\multsp g(x)
}

\dispSFoutmath{
\Muserfunction{SimplifyDeltaFunction}[\multsp \%,x]
}

\dispSFinmath{
\delta (1-x)\multsp \big(\underline{\Mvariable{lim}}\, g(x)\big)
}

\dispSFoutmath{
g[x]\Muserfunction{DeltaFunctionPrime}[1-x]
}

\Subsection*{SimplifyGTI { }***unfinished***}

\Subsubsection*{Description}

SimplifyGTI simplifies GTI's.

See also:  GTI.

\Subsubsection*{Examples}

\dispSFinmath{
{{\delta }^{\prime }}(1-x)\multsp g(x)
}

\Subsection*{SimplifyPolyLog}

\Subsubsection*{Description}

SimplifyPolyLog[y] performs several simplifications assuming that the variables occuring in the Log and PolyLog functions are between 0
  and 1.

See also:  Nielsen.

\Subsubsection*{Examples}

\dispSFinmath{
\Muserfunction{SimplifyDeltaFunction}[\multsp \%,x]
}

\dispSFinmath{
{{\delta }^{\prime }}(1-x)\multsp \big(\underline{\Mvariable{lim}}\, g(x)\big)+
   \delta (1-x)\multsp \big(\underline{\Mvariable{lim}}\, {g^{\prime }}(x)\big)
}

\dispSFoutmath{
x\multsp \log [x]\multsp \Muserfunction{DeltaFunctionPrime}[1-x]
}

\dispSFinmath{
x\multsp {{\delta }^{\prime }}(1-x)\multsp \log(x)
}

\dispSFoutmath{
\Muserfunction{SimplifyDeltaFunction}[\multsp \%,x]
}

\dispSFinmath{
\delta (1-x)
}

\dispSFoutmath{
\Mfunction{PolyLog}[2,1-x]\multsp \Muserfunction{DeltaFunctionPrime}[1-x]
}

\dispSFinmath{
{{\delta }^{\prime }}(1-x)\multsp {{\Mvariable{Li}}_2}(1-x)
}

\dispSFoutmath{
\Muserfunction{SimplifyDeltaFunction}[\multsp \%,x]
}

\dispSFinmath{
-\delta (1-x)
}

\dispSFoutmath{
\log [x]\Mfunction{PolyLog}[2,1-x]\multsp \Muserfunction{DeltaFunctionPrime}[1-x]
}

\dispSFinmath{
{{\delta }^{\prime }}(1-x)\multsp \log(x)\multsp {{\Mvariable{Li}}_2}(1-x)
}

\dispSFoutmath{
\Muserfunction{SimplifyDeltaFunction}[\multsp \%,x]
}

\dispSFinmath{
0
}

\dispSFoutmath{
\Mfunction{PolyLog}[3,1-x]\multsp \Muserfunction{DeltaFunctionPrime}[1-x]
}

\dispSFinmath{
{{\delta }^{\prime }}(1-x)\multsp {{\Mvariable{Li}}_3}(1-x)
}

\dispSFoutmath{
\Muserfunction{SimplifyDeltaFunction}[\multsp \%,x]
}

\dispSFinmath{
-\delta (1-x)
}

\dispSFoutmath{
\{ \}
}


\dispSFoutmath{
\Muserfunction{sip}[\Mvariable{y\_}]\multsp :=\multsp y\multsp ==\multsp \Muserfunction{SimplifyPolyLog}[y]
}

\dispSFinmath{
\Muserfunction{sip}[\Mfunction{PolyLog}[2,1/x]]
}

\dispSFoutmath{
{{\Mvariable{Li}}_2}\Big(\frac{1}{x}\Big)==-\frac{1}{2}\multsp {{\log}^2}(x)+\log(1-x)\multsp \log(x)+
    \ImaginaryI \multsp \pi \multsp \log(x)+\zeta (2)+{{\Mvariable{Li}}_2}(1-x)
}

\dispSFinmath{
\Muserfunction{sip}[\Mfunction{PolyLog}[2,x]]
}

\dispSFoutmath{
{{\Mvariable{Li}}_2}(x)==\zeta (2)-\log(1-x)\multsp \log(x)-{{\Mvariable{Li}}_2}(1-x)
}

\dispSFinmath{
\Muserfunction{sip}[\Mfunction{PolyLog}[2,1-x\RawWedge 2]]
}

\dispSFoutmath{
{{\Mvariable{Li}}_2}(1-{x^2})==-\zeta (2)-2\multsp \log(x)\multsp \log(x+1)+2\multsp {{\Mvariable{Li}}_2}(1-x)-
    2\multsp {{\Mvariable{Li}}_2}(-x)
}

\dispSFinmath{
\Muserfunction{sip}[\Mfunction{PolyLog}[2,x\RawWedge 2]]
}

\dispSFoutmath{
{{\Mvariable{Li}}_2}({x^2})==2\multsp \zeta (2)-2\multsp \log(1-x)\multsp \log(x)-2\multsp {{\Mvariable{Li}}_2}(1-x)+
    2\multsp {{\Mvariable{Li}}_2}(-x)
}

\dispSFinmath{
\Muserfunction{sip}[\Mfunction{PolyLog}[2,-x/(1-x)]]
}

\dispSFoutmath{
{{\Mvariable{Li}}_2}\big(-\frac{x}{1-x}\big)==-\frac{1}{2}\multsp {{\log}^2}(1-x)+\log(x)\multsp \log(1-x)-\zeta (2)+
    {{\Mvariable{Li}}_2}(1-x)
}

\dispSFinmath{
\Muserfunction{sip}[\Mfunction{PolyLog}[2,x/(x-1)]]
}

\dispSFoutmath{
{{\Mvariable{Li}}_2}\big(\frac{x}{x-1}\big)==-\frac{1}{2}\multsp {{\log}^2}(1-x)+\log(x)\multsp \log(1-x)-\zeta (2)+
    {{\Mvariable{Li}}_2}(1-x)
}

\dispSFinmath{
\Muserfunction{sip}[\Muserfunction{Nielsen}[1,2,-x/(1-x)]]
}

\dispSFoutmath{
{S_{12}}\big(-\frac{x}{1-x}\big)=={S_{12}}(x)-\frac{1}{6}\multsp {{\log}^3}(1-x)
}

\dispSFinmath{
\Muserfunction{sip}[\Mfunction{PolyLog}[3,-1/x]]
}

\dispSFoutmath{
{{\Mvariable{Li}}_3}\Big(-\frac{1}{x}\Big)==\frac{{{\log}^3}(x)}{6}+\zeta (2)\multsp \log(x)+{{\Mvariable{Li}}_3}(-x)
}

\dispSFinmath{
\Muserfunction{sip}[\Mfunction{PolyLog}[3,1-x]]
}

\Subsection*{Simplify2 { }***unfinished***}

\Subsubsection*{Description}

Simplify2 is a special ordering function.

\Subsubsection*{Examples}

\dispSFinmath{
\Mvariable{True}
}

\Subsection*{SmallDelta}

\Subsubsection*{Description}

SmallDelta denotes some small positive number.

\Subsection*{SmallEpsilon}

\Subsubsection*{Description}

SmallEpsilon denotes some small positive number.

\Subsection*{SmallVariable}

\Subsubsection*{Description}

SmallVariable[me] is the head of small (negligible) variables. This means any mass with this head can be neglected if it appears in a
  sum, but not as an argument of Passarino-Veltman (PaVe) functions or PropagatorDenominator.

See also:  PaVe, PaVeReduce, PropagatorDenominator.

\Subsection*{SmallVariables}

\Subsubsection*{Description}

SmallVariables is an option for OneLoop. "SmallVariables\(\rightarrow \)\{Melectron\}" i.e. will substitute "SmallVariable[Melectron]" {
  }for all Melectron's in the calculation.

See also:  OneLoop.

\Subsection*{SMP}

\Subsubsection*{Description}

SMP[par] substitutes a symbol for the { }Standard Model parameter par. { }SMP[] gives the list of substitutions. par should be a string;
  e.g., SMP["SW"] gives { }sw (in the Global` context).

\dispSFinmath{
\Muserfunction{sip}[\Mfunction{PolyLog}[3,x\RawWedge 2]]
}

\dispSFoutmath{
{{\Mvariable{Li}}_3}({x^2})==-2\multsp \log(1-x)\multsp {{\log}^2}(x)+4\multsp \zeta (2)\multsp \log(x)-
    4\multsp {{\Mvariable{Li}}_2}(1-x)\multsp \log(x)-4\multsp {S_{12}}(1-x)+4\multsp {{\Mvariable{Li}}_3}(-x)+4\multsp \zeta (3)
}

See also: SMVertex.

\Subsubsection*{Examples}

\dispSFinmath{
\Muserfunction{sip}[\Mfunction{PolyLog}[3,-x/(1-x)]]
}

\dispSFoutmath{
\MathBegin{MathArray}{l}
{{\Mvariable{Li}}_3}\big(-\frac{x}{1-x}\big)==
   \frac{1}{6}\multsp {{\log}^3}(1-x)-\frac{1}{2}\multsp \log(x)\multsp {{\log}^2}(1-x)+\frac{1}{2}\multsp {{\log}^2}(x)\multsp \log(1-x)
    +  \\
\noalign{\vspace{0.90625ex}}
\hspace{2.em} \zeta (2)\multsp \log(1-x)-\zeta (2)\multsp \log(x)+{S_{12}}(1-x)+
   \log(x)\multsp {{\Mvariable{Li}}_2}(1-x)-{{\Mvariable{Li}}_3}(1-x)\\
\MathEnd{MathArray}
}

\Subsection*{SMVertex}

\Subsubsection*{Description}

SMVertex["AWW", p,mu, q,nu, k,rho] gives the photon-W-W vertex (p,mu correspond to the photon, q,nu to the (incoming) W\(+\) and k,rho to
  the (incoming) W-. All momenta are flowing into the vertex. { }SMVertex["HHH", \_{}\_{}\_{}] give the three-higgs coupling.

\dispSFinmath{
\Muserfunction{sip}[\Mfunction{PolyLog}[3,1-1/x]]
}

\dispSFoutmath{
\MathBegin{MathArray}{l}
{{\Mvariable{Li}}_3}\Big(1-\frac{1}{x}\Big)==  \\
\noalign{\vspace{1.35417ex}}
\hspace{1.em} \frac{{{\log}^3}(x
      )}{6}-\frac{1}{2}\multsp {{\log}^2}(1-x)\multsp \log(x)+{{\Mvariable{Li}}_2}(1-x)\multsp \log(x)+{S_{12}}(1-x)+{S_{12}}(x)-
   \log(1-x)\multsp {{\Mvariable{Li}}_2}(1-x)-\zeta (3)\\
\MathEnd{MathArray}
}

\Subsubsection*{Examples}

\dispSFinmath{
\Muserfunction{sip}[\Mfunction{PolyLog}[4,-x/(1-x)]]
}

\dispSFoutmath{
\MathBegin{MathArray}{l}
{{\Mvariable{Li}}_4}\big(-\frac{x}{1-x}\big)==
   -\frac{1}{24}\multsp {{\log}^4}(1-x)+\frac{1}{2}\multsp \log(x)\multsp {{\log}^3}(1-x)-
    \frac{1}{2}\multsp {{\log}^2}(x)\multsp {{\log}^2}(1-x)-  \\
\noalign{\vspace{1.19792ex}}
\hspace{2.em} \frac{1}{2}\multsp \zeta (2)
    \multsp {{\log}^2}(1-x)+\frac{1}{2}\multsp {{\Mvariable{Li}}_2}(1-x)\multsp {{\log}^2}(1-x)+\zeta (2)\multsp \log(x)\multsp \log(1-x)
   -{S_{12}}(1-x)\multsp \log(1-x)-  \\
\noalign{\vspace{0.90625ex}}
\hspace{2.em} {S_{12}}(x)\multsp \log(1-x)-
   \log(x)\multsp {{\Mvariable{Li}}_2}(1-x)\multsp \log(1-x)+\zeta (3)\multsp \log(1-x)-{S_{13}}(x)+{S_{22}}(x)-{{\Mvariable{Li}}_4}(x)\\
   \MathEnd{MathArray}
}

\Subsection*{Smu { }***unfinished***}

\Subsubsection*{Description}

Smu { }is ...

\Subsection*{Sn}

\Subsubsection*{Description}

Sn is \(\Muserfunction{sip}[\log [a+b/c]]\)/\(\log\Big(a+\frac{b}{c}\Big)==\log\Big(\frac{b+a\multsp c}{c}\Big)\).

\Subsection*{SO}

\Subsubsection*{Description}

SO[q] is a four-dimensional scalar product of OPEDelta with q. It is transformed into Pair[Momentum[q], Momentum[OPEDelta] by FCI.

See also: FCI, OPEDelta, Pair, ScalarProduct, SOD.

\Subsubsection*{Examples}

\dispSFinmath{
\Muserfunction{sip}[\log [1/x]]
}

\dispSFoutmath{
\log\Big(\frac{1}{x}\Big)==-\log(x)
}

\dispSFinmath{
\Muserfunction{sip}[\Mfunction{ArcTanh}[x]]
}

\dispSFoutmath{
{{\tanh}^{-1}}(x)==\frac{1}{2}\multsp \log\Big(-\frac{x+1}{1-x}\Big)
}

\dispSFinmath{
\Muserfunction{sip}[\Mfunction{ArcSinh}[x]]
}

\dispSFoutmath{
{{\sinh}^{-1}}(x)==\log\big(x+{\sqrt{{x^2}+1}}\big)
}

\Subsection*{SOD}

\Subsubsection*{Description}

SOD[q] is a D-dimensional scalar product of OPEDelta with q. It is transformed into Pair[Momentum[q,D], Momentum[OPEDelta,D] by
  FeynCalcInternal.

See also:  OPEDelta, Pair, ScalarProduct, SOD.

\Subsubsection*{Examples}

\dispSFinmath{
\Muserfunction{sip}[\Mfunction{ArcCosh}[x]]
}

\dispSFoutmath{
{{\cosh}^{-1}}(x)==\log\big(x+{\sqrt{{x^2}-1}}\big)
}

\dispSFinmath{
\Mfunction{Clear}[\Mvariable{sip}]
}

\dispSFoutmath{
\{ \}
}


\dispSFoutmath{
\Muserfunction{SMP}[]//\Mfunction{InputForm}
}

\Subsection*{Solve2}

\Subsubsection*{Description}

Solve2 is equivalent to Solve, except that it works only for linear equations (and returns just a list) and accepts the options Factoring
  and FinalSubstitutions. Solve2 uses the "high school algorithm" and factors intermediate results. Therefore it can be drastically more
  useful than Solve.

\dispSFinmath{
\{\Mvariable{EL}\multsp :>\multsp e,\multsp \Mvariable{CW}\multsp :>\multsp \Mvariable{cw},\multsp
    \Mvariable{ME}\multsp :>\multsp \Mvariable{me},\multsp \Mvariable{MH}\multsp :>\multsp \Mvariable{mh},\multsp
    \Mvariable{MW}\multsp :>\multsp \Mvariable{mw},\multsp \Mvariable{SW}\multsp :>\multsp \Mvariable{sw}\}
}

\dispSFoutmath{
\Muserfunction{SMP}["ME"]
}

\Subsubsection*{Examples}

\dispSFinmath{
\Mvariable{me}
}

\dispSFoutmath{
\Muserfunction{SMVertex}//\Mfunction{Options}
}

If no equation sign is given the polynomials are supposed to be 0.

\dispSFinmath{
\{\Mvariable{Dimension}\rightarrow 4,\Mvariable{Explicit}\rightarrow \Mvariable{True}\}
}

\dispSFoutmath{
\Muserfunction{SMVertex}["AWW",p,\mu ,\multsp q,\nu ,\multsp k,\rho ]
}

\dispSFinmath{
-\ImaginaryI \multsp e\multsp ({{(k-q)}^{\mu }}\multsp {g^{\nu \rho }}-{g^{\mu \rho }}\multsp {{(k-p)}^{\nu }}-
     {g^{\mu \nu }}\multsp {{(p-q)}^{\rho }})
}

\dispSFoutmath{
{{\pi }^{n/2}}
}

\dispSFinmath{
{{(2\multsp \pi )}^n}
}

\dispSFoutmath{
\Muserfunction{SO}[p]
}

\dispSFinmath{
\Delta \cdot p
}

\dispSFoutmath{
\Muserfunction{SO}[p-q]
}

\Subsection*{Solve3}

\Subsubsection*{Description}

Solve3 is equivalent to Solve, except that it works only for linear equations (and returns just a list) and uses the "high school
  algorithm" and is sometimes better than Solve for systems involving rational polyonials.

\dispSFinmath{
\Delta \cdot (p-q)
}

\dispSFoutmath{
\Muserfunction{SO}[p]//\Muserfunction{FCI}//\Mfunction{StandardForm}
}

See also: Solve2.

\Subsubsection*{Examples}

\dispSFinmath{
\Muserfunction{Pair}[\Muserfunction{Momentum}[\Mvariable{OPEDelta}],\Muserfunction{Momentum}[p]]
}

\dispSFoutmath{
\Muserfunction{SOD}[p]
}

\Subsection*{SP}

\Subsubsection*{Description}

SP[a, b] denotes a four-dimensional scalar product. SP[a, b] is transformed into ScalarProduct[a, b] by FeynCalcInternal. SP[p] is the
  same as SP[p, p] \(\Delta \cdot p\)).

See also:  Calc, ExpandScalarProduct, ScalarProduct.

\Subsubsection*{Examples}

\dispSFinmath{
\Muserfunction{SOD}[p-q]
}

\dispSFoutmath{
\Delta \cdot (p-q)
}

\dispSFinmath{
\Muserfunction{SOD}[p]//\Muserfunction{FCI}//\Mfunction{StandardForm}
}

\dispSFoutmath{
\Muserfunction{Pair}[\Muserfunction{Momentum}[\Mvariable{OPEDelta},D],\Muserfunction{Momentum}[p,D]]
}

\dispSFinmath{
\Mfunction{Options}[\Mvariable{Solve2}]
}

\dispSFoutmath{
\{\Mvariable{Factoring}\rightarrow \Mvariable{Factor2},\Mvariable{FinalSubstitutions}\rightarrow \{\}\}
}

\dispSFinmath{
\Muserfunction{Solve2}[\{2\multsp x==b-w/2,y-d==p\},\{x,y\}]
}

\dispSFoutmath{
\big\{x\rightarrow \frac{1}{4}\multsp (2\multsp b-w),y\rightarrow d+p\big\}
}

\dispSFinmath{
\Muserfunction{Solve2}[x+y,x]
}

\dispSFoutmath{
\{x\rightarrow -y\}
}

\dispSFinmath{
\Muserfunction{Solve2}[x+y,x,\Mvariable{FinalSubstitutions}\rightarrow \{y\rightarrow h\}]
}

\dispSFoutmath{
\{x\rightarrow -h\}
}

\dispSFinmath{
\Muserfunction{Solve2}[\{2\multsp x==b-w/2,y-d==p\},\{x,y\},\Mvariable{Factoring}\rightarrow \Mvariable{Expand}]
}

\dispSFoutmath{
\big\{x\rightarrow \frac{b}{2}-\frac{w}{4},y\rightarrow d+p\big\}
}

\Subsection*{SPC}

\Subsubsection*{Description}

SPC is an abbreviation for ScalarProductCancel.

See also:  ScalarProductCancel.

\Subsection*{SPD}

\Subsubsection*{Description}

SPD[a, b] denotes a D-dimensional scalar product. SPD[a, b] is transformed into ScalarProduct[a, b,Dimension\(\rightarrow \)D] by
  FeynCalcInternal. SPD[p] is the same as SPD[p,p] \(\Mfunction{Solve}[\{2\multsp x==b-w/2,y-d==p\},\{x,y\}]\)).

See also:  PD, Calc, ExpandScalarProduct, ScalarProduct.

\Subsubsection*{Examples}

\dispSFinmath{
\big\{\big\{x\rightarrow \frac{1}{4}\multsp (2\multsp b-w),y\rightarrow d+p\big\}\big\}
}

\dispSFoutmath{
\Mfunction{Options}[\Mvariable{Solve3}]
}

\dispSFinmath{
\{\Mvariable{Factoring}\rightarrow \Mvariable{False},\Mvariable{FinalSubstitutions}\rightarrow \{\}\}
}

\dispSFoutmath{
\Muserfunction{Solve3}[\{2\multsp x==b-w/2,y-d==p\},\{x,y\}]
}

\dispSFinmath{
\big\{x\rightarrow \frac{1}{4}\multsp (2\multsp b-w),y\rightarrow d+p\big\}
}

\dispSFoutmath{
(={p^2}
}

\dispSFinmath{
\Muserfunction{SP}[p,q]\multsp +\multsp \Muserfunction{SP}[q]
}

\dispSFoutmath{
p\cdot q+{q^2}
}

\dispSFinmath{
\Muserfunction{SP}[p-q,q+2p]
}

\dispSFoutmath{
(p-q)\cdot (2\multsp p+q)
}

\dispSFinmath{
\Muserfunction{Calc}[\multsp \Muserfunction{SP}[p-q,q+2p]\multsp ]
}

\dispSFoutmath{
2\multsp {p^2}-p\cdot q-{q^2}
}

\dispSFinmath{
\Muserfunction{ExpandScalarProduct}[\Muserfunction{SP}[p-q]]
}

\dispSFoutmath{
{p^2}-2\multsp p\cdot q+{q^2}
}

\dispSFinmath{
\Muserfunction{SP}[a,b]//\Mfunction{StandardForm}
}

\dispSFoutmath{
\Muserfunction{SP}[a,b]
}

\Subsection*{Spinor}

\Subsubsection*{Description}

Spinor[p, m, o] is the head of Dirac spinors. Which of the spinors {\itshape u}, {\itshape v}, \(\Muserfunction{SP}[a,b]//\Muserfunction{FCI}//\Mfunction{StandardForm}\)
\(\Muserfunction{Pair}[\Muserfunction{Momentum}[a],\Muserfunction{Momentum}[b]]\)or \(\Muserfunction{SP}[a,b]//\Muserfunction{FCI}//\Muserfunction{FCE}//\Mfunction{StandardForm}\)is
understood, depends on the sign of the momentum (p) argument and the relative position of DiracSlash[p]: Spinor[sign p, mass] is that
  spinor which yields: sign*mass*Spinor[p, mass] if the Dirac equation is applied (by DiracEquation or DiracSimplify).\\
The optional argument o can be used for additional degrees of freedom. If no optional argument o is supplied, a 1 is subsituted in.\\
Spinors of fermions of mass {\itshape m} are normalized to have square \(\Muserfunction{SP}[a,b]\){\itshape u}\(=\)2 {\itshape m} and { }\((={p^2}\){\itshape
v\(=\)}-2 {\itshape m}.

See also:  FermionSpinSum, DiracSimplify, SpinorU, SpinorV, SpinorUBar, SpinorVBar.

\Subsubsection*{Examples}

\dispSFinmath{
\Muserfunction{SPD}[p,q]\multsp +\multsp \Muserfunction{SPD}[q]
}

\dispSFoutmath{
p\cdot q+{q^2}
}

\dispSFinmath{
\Muserfunction{SPD}[p-q,q+2p]
}

\dispSFoutmath{
(p-q)\cdot (2\multsp p+q)
}

FeynCalc uses covariant normalization (as opposed to e.g. the normalization used in Bjorken\&{}Drell).

\dispSFinmath{
\Muserfunction{Calc}[\multsp \Muserfunction{SPD}[p-q,q+2p]\multsp ]
}

\dispSFoutmath{
2\multsp {p^2}-p\cdot q-{q^2}
}

\dispSFinmath{
\Muserfunction{ExpandScalarProduct}[\Muserfunction{SPD}[p-q]]
}

\dispSFoutmath{
{p^2}-2\multsp p\cdot q+{q^2}
}

\dispSFinmath{
\Muserfunction{SPD}[a,b]//\Mfunction{StandardForm}
}

\dispSFoutmath{
\Muserfunction{SPD}[a,b]
}

By convention, ChangeDimension does not operate on momenta in Spinor's (but on e.g. { }DiracSlash[Momentum[p]]).

\dispSFinmath{
\Muserfunction{SPD}[a,b]//\Muserfunction{FCI}//\Mfunction{StandardForm}
}

\dispSFoutmath{
\Muserfunction{Pair}[\Muserfunction{Momentum}[a,D],\Muserfunction{Momentum}[b,D]]
}

\dispSFinmath{
\Muserfunction{SPD}[a,b]//\Muserfunction{FCI}//\Muserfunction{FCE}//\Mfunction{StandardForm}
}

\dispSFoutmath{
\Muserfunction{SPD}[a,b]
}

SmallVariable's are discarded by Spinor.

\dispSFinmath{
\Muserfunction{FCE}[\Muserfunction{ChangeDimension}[\Muserfunction{SP}[p,q],\multsp D]]//\Mfunction{StandardForm}
}

\dispSFoutmath{
\Muserfunction{SPD}[p,q]
}

\Subsection*{SpinorCollect}

\Subsubsection*{Description}

SpinorCollect is an option for FermionSpinSum. If set to False the { }argument of FermionSpinSum has to be already collected w.r.t.
  Spinor.

See also: { }FermionSpinSum, Spinor.

\Subsection*{SpinorUBar}

\Subsubsection*{Description}

SpinorUBar[p, m] denotes a \(\overvar{u}{\_}\)-spinor.

See also:  Spinor, SpinorU, SpinorV, SpinorVBar.

\Subsubsection*{Examples}

One argument only assumes a massless spinor.

\dispSFinmath{
,\multsp
}

\dispSFoutmath{
\overvar{v}{\_}\multsp
}

\dispSFinmath{
\overvar{u}{\_}
}

\dispSFoutmath{
\overvar{v}{\_}\multsp
}

\dispSFinmath{
\Muserfunction{Spinor}[p]
}

\dispSFoutmath{
\varphi (p)
}

\dispSFinmath{
\Muserfunction{Spinor}[p,m]
}

\dispSFoutmath{
\varphi (p,m)
}

\dispSFinmath{
\Muserfunction{Spinor}[p,m].\Muserfunction{Spinor}[p,m]//\Muserfunction{DiracSimplify}
}

\dispSFoutmath{
2\multsp m
}

\Subsection*{SpinorU}

\Subsubsection*{Description}

SpinorU[p, m, optarg] denotes a u-spinor.

See also:  Spinor, SpinorUBar, SpinorV, SpinorVBar.

\Subsubsection*{Examples}

\dispSFinmath{
\Muserfunction{DiracSimplify}[\Muserfunction{Spinor}[-p,m].\Muserfunction{DiracSlash}[p]]
}

\dispSFoutmath{
-m\multsp \varphi (-p,m)
}

\dispSFinmath{
\Muserfunction{Spinor}[p]//\Mfunction{StandardForm}
}

\dispSFoutmath{
\Muserfunction{Spinor}[\Muserfunction{Momentum}[p],0,1]
}

\dispSFinmath{
\Muserfunction{ChangeDimension}[\Muserfunction{Spinor}[p],D]//\Mfunction{StandardForm}
}

\dispSFoutmath{
\Muserfunction{Spinor}[\Muserfunction{Momentum}[p],0,1]
}

\dispSFinmath{
\Muserfunction{Spinor}[p,m]//\Mfunction{StandardForm}
}

\dispSFoutmath{
\Muserfunction{Spinor}[\Muserfunction{Momentum}[p],m,1]
}

\dispSFinmath{
\Muserfunction{Spinor}[p,\Muserfunction{SmallVariable}[m]]//\Mfunction{StandardForm}
}

\dispSFoutmath{
\Muserfunction{Spinor}[\Muserfunction{Momentum}[p],0,1]
}

\Subsection*{SpinorVBar}

\Subsubsection*{Description}

SpinorVBar[p, m] denotes a \(\overvar{u}{\_}\)(p,m)-spinor.

See also:  Spinor, SpinorU, SpinorV, SpinorUBar.

\Subsubsection*{Examples}

\dispSFinmath{
\Muserfunction{SpinorUBar}[p]
}

\dispSFoutmath{
\overvar{u}{\_}(p)
}

\dispSFinmath{
\Muserfunction{SpinorUBar}[p,m]
}

\dispSFoutmath{
\overvar{u}{\_}(p,m)
}

\dispSFinmath{
\Muserfunction{SpinorUBar}[p,m]//\Mfunction{StandardForm}
}

\dispSFoutmath{
\Muserfunction{SpinorUBar}[p,m]
}

\dispSFinmath{
\Muserfunction{SpinorUBar}[p,m]//\Muserfunction{FCI}//\Mfunction{StandardForm}
}

\dispSFoutmath{
\Muserfunction{Spinor}[\Muserfunction{Momentum}[p],m,1]
}

\dispSFinmath{
\Muserfunction{SpinorUBar}[p,m]//\Muserfunction{FCI}//\Muserfunction{FCE}//\Mfunction{StandardForm}
}

\dispSFoutmath{
\Muserfunction{Spinor}[\Muserfunction{Momentum}[p],m,1]
}

\Subsection*{SpinorV}

\Subsubsection*{Description}

SpinorV[p, m, optarg] denotes a v-spinor.

See also:  Spinor, SpinorUBar, SpinorU, SpinorVBar.

\Subsubsection*{Examples}

\dispSFinmath{
\Muserfunction{SpinorU}[p]
}

\dispSFoutmath{
u(p)
}

\dispSFinmath{
\Muserfunction{SpinorU}[p,m]
}

\dispSFoutmath{
u(p,m)
}

\dispSFinmath{
\Muserfunction{SpinorU}[p,m]//\Mfunction{StandardForm}
}

\dispSFoutmath{
\Muserfunction{SpinorU}[p,m]
}

\dispSFinmath{
\Muserfunction{SpinorU}[p,m]//\Muserfunction{FCI}//\Mfunction{StandardForm}
}

\dispSFoutmath{
\Muserfunction{Spinor}[\Muserfunction{Momentum}[p],m,1]
}

\dispSFinmath{
\Muserfunction{SpinorU}[p,m]//\Muserfunction{FCI}//\Muserfunction{FCE}//\Mfunction{StandardForm}
}

\dispSFoutmath{
\Muserfunction{Spinor}[\Muserfunction{Momentum}[p],m,1]
}

\Subsection*{SpinPolarizationSum}

\Subsubsection*{Description}

SpinPolarizationSum is an option for FermionSpinSum. The set (pure) function acts on the usual spin sum.

See also: { }FermionSpinSum.

\Subsection*{SPL}

\Subsubsection*{Description}

SPL is an abbreviation for SimplifyPolyLog.

See also:  SimplifyPolyLog.

\Subsection*{SplittingFunction}

\Subsubsection*{Description}

SplittingFunction[pxy] is a database of splitting functions in the \(\overvar{v}{\_}\) scheme.

\dispSFinmath{
\Muserfunction{SpinorVBar}[p]
}

\dispSFoutmath{
\overvar{v}{\_}(p)
}

See also:  AnomalousDimension.

\Subsubsection*{Examples}

Unpolarized case:

\dispSFinmath{
\Muserfunction{SpinorVBar}[p,m]
}

In general the argument should be a string, but if the variables Pqq, etc. have no value, you can omit the "".

\dispSFinmath{
\overvar{v}{\_}(p,m)
}

\dispSFoutmath{
\Muserfunction{SpinorVBar}[p,m]//\Mfunction{StandardForm}
}

\dispSFinmath{
\Muserfunction{SpinorVBar}[p,m]
}

\dispSFoutmath{
\Muserfunction{SpinorVBar}[p,m]//\Muserfunction{FCI}//\Mfunction{StandardForm}
}

\dispSFinmath{
\Muserfunction{Spinor}[-\Muserfunction{Momentum}[p],m,1]
}

\dispSFoutmath{
\Muserfunction{SpinorVBar}[p,m]//\Muserfunction{FCI}//\Muserfunction{FCE}//\Mfunction{StandardForm}
}

\dispSFinmath{
\Muserfunction{Spinor}[-\Muserfunction{Momentum}[p],m,1]
}

\dispSFoutmath{
\Muserfunction{SpinorV}[p]
}

\dispSFinmath{
v(p)
}

\dispSFoutmath{
\Muserfunction{SpinorV}[p,m]
}

\dispSFinmath{
v(p,m)
}

\dispSFoutmath{
\Muserfunction{SpinorV}[p,m]//\Mfunction{StandardForm}
}

\dispSFinmath{
\Muserfunction{SpinorV}[p,m]
}

\dispSFoutmath{
\Muserfunction{SpinorV}[p,m]//\Muserfunction{FCI}//\Mfunction{StandardForm}
}

\dispSFinmath{
\Muserfunction{Spinor}[-\Muserfunction{Momentum}[p],m,1]
}

Polarized case:

\dispSFinmath{
\Muserfunction{SpinorV}[p,m]//\Muserfunction{FCI}//\Muserfunction{FCE}//\Mfunction{StandardForm}
}

\dispSFinmath{
\Muserfunction{Spinor}[-\Muserfunction{Momentum}[p],m,1]
}

\dispSFoutmath{
\overvar{\Mvariable{MS}}{\_}
}

\dispSFinmath{
\Mfunction{Options}[\Mvariable{SplittingFunction}]
}

\dispSFoutmath{
\{\Mvariable{Polarization}\rightarrow 1\}
}

\dispSFinmath{
\Mfunction{SetOptions}[\Mvariable{SplittingFunction},\Mvariable{Polarization}\rightarrow 0];
}

\dispSFoutmath{
\Muserfunction{SplittingFunction}[\Mvariable{Pqq}]
}

\dispSFinmath{
{C_F}\multsp \Big(-4\multsp x+6\multsp \delta (1-x)+8\multsp {{\Big(\frac{1}{1-x}\Big)}_+}-4\Big)
}

\dispSFoutmath{
\Muserfunction{SplittingFunction}[\Mvariable{Pqg}]
}

\dispSFinmath{
{T_f}\multsp (16\multsp {x^2}-16\multsp x+8)
}

\dispSFoutmath{
\Muserfunction{SplittingFunction}[\Mvariable{Pgq}]
}

\dispSFinmath{
{C_F}\multsp \Big(4\multsp x-8+\frac{8}{x}\Big)
}

\dispSFoutmath{
\Muserfunction{SplittingFunction}[\Mvariable{Pgg}]
}

\dispSFinmath{
8\multsp {C_A}\multsp \Big(-{x^2}+x+\frac{11}{12}\multsp \delta (1-x)+{{\Big(\frac{1}{1-x}\Big)}_+}-2+\frac{1}{x}\Big)-
   \frac{8}{3}\multsp {N_f}\multsp {T_f}\multsp \delta (1-x)
}

\dispSFoutmath{
\Muserfunction{SplittingFunction}[\Mvariable{aqq}]
}

\dispSFinmath{
{C_F}\multsp \Big(2\multsp x+(7-4\multsp \zeta (2))\multsp \delta (1-x)+(2\multsp x+2)\multsp \log((1-x)\multsp x)-
     4\multsp \Big(\frac{\log(x)}{1-x}+{{\Big(\frac{\log(1-x)}{1-x}\Big)}_+}\Big)-4\Big)
}

\dispSFoutmath{
\Muserfunction{SplittingFunction}[\Mvariable{agq}]
}

\dispSFinmath{
{C_F}\multsp \Big(-2\multsp x+\Big(-2\multsp x+4-\frac{4}{x}\Big)\multsp \log((1-x)\multsp x)+2-\frac{4}{x}\Big)
}

\dispSFoutmath{
\Muserfunction{SplittingFunction}[\Mvariable{aqg}]
}

\dispSFinmath{
{T_f}\multsp ((-8\multsp {x^2}+8\multsp x-4)\multsp \log((1-x)\multsp x)-4)
}

\dispSFoutmath{
\Muserfunction{SplittingFunction}[\Mvariable{agg}]
}

\dispSFinmath{
\Mfunction{SetOptions}[\Mvariable{SplittingFunction},\Mvariable{Polarization}\rightarrow 1];
}

\dispSFoutmath{
\Muserfunction{SplittingFunction}[\Mvariable{Pqq}]
}

\dispSFinmath{
{C_F}\multsp \Big(-4\multsp x+6\multsp \delta (1-x)+8\multsp {{\Big(\frac{1}{1-x}\Big)}_+}-4\Big)
}

\dispSFoutmath{
\Muserfunction{SplittingFunction}[\Mvariable{Pqg}]
}

\dispSFinmath{
{T_f}\multsp (16\multsp x-8)
}

\dispSFoutmath{
\Muserfunction{SplittingFunction}[\Mvariable{Pgq}]
}

\dispSFinmath{
{C_F}\multsp (8-4\multsp x)
}

\dispSFoutmath{
\Muserfunction{SplittingFunction}[\Mvariable{Pgg}]
}

\dispSFinmath{
{C_A}\multsp \Big(-16\multsp x+\frac{22}{3}\multsp \delta (1-x)+8\multsp {{\Big(\frac{1}{1-x}\Big)}_+}+8\Big)-
   \frac{8}{3}\multsp {N_f}\multsp {T_f}\multsp \delta (1-x)
}

\dispSFoutmath{
\Muserfunction{SplittingFunction}[\Mvariable{aqq}]
}

\dispSFinmath{
{C_F}\multsp \Big(8\multsp (1-x)+2\multsp x+(7-4\multsp \zeta (2))\multsp \delta (1-x)+(2\multsp x+2)\multsp \log((1-x)\multsp x)-
     4\multsp \log(x)\multsp {{\Big(\frac{1}{1-x}\Big)}_+}-4\multsp {{\Big(\frac{\log(1-x)}{1-x}\Big)}_+}-4\Big)
}

\dispSFoutmath{
\Muserfunction{SplittingFunction}[\Mvariable{agq}]
}

\Subsection*{StandardMatrixElement}

\Subsubsection*{Description}

StandardMatrixElement[ ... ] is the head for matrix element abbreviations.

See also:  OneLoop.

\Subsection*{SubContext}

\Subsubsection*{Description}

SubContext[fun] gives the sub-directory (context) in HighEnergyPhysics.

See also:  FeynCalc, Load.

\Subsection*{SubLoop}

\Subsubsection*{Description}

SubLoop is an option for OPE1Loop. If set to True, sub 1-loop tensorintegral decomposition is performed.

See also:  OPE1Loop.

\Subsection*{SumP}

\Subsubsection*{Description}

SumP[k, m] is \({C_F}\multsp (-4\multsp x+(2\multsp x-4)\multsp \log((1-x)\multsp x)+2)\)

See also:  SumS, SumT.

\Subsubsection*{Examples}

\dispSFinmath{
\Muserfunction{SplittingFunction}[\Mvariable{agqd}]
}

\dispSFoutmath{
{C_F}\multsp ((2\multsp x-4)\multsp \log((1-x)\multsp x)-2)
}

\dispSFinmath{
\Muserfunction{SplittingFunction}[\Mvariable{aqg}]
}

\dispSFoutmath{
{T_f}\multsp ((4-8\multsp x)\multsp \log((1-x)\multsp x)-4)
}

\dispSFinmath{
\Muserfunction{SplittingFunction}[\Mvariable{aqgd}]
}

\dispSFoutmath{
{T_f}\multsp ((4-8\multsp x)\multsp \log((1-x)\multsp x)-4)
}

\dispSFinmath{
\Muserfunction{SplittingFunction}[\Mvariable{agg}]
}

\dispSFoutmath{
{C_A}\multsp \Big(\Big(\frac{67}{9}-4\multsp \zeta (2)\Big)\multsp \delta (1-x)+(8\multsp x-4)\multsp \log((1-x)\multsp x)-
      4\multsp \Big(\frac{\log(x)}{1-x}+{{\Big(\frac{\log(1-x)}{1-x}\Big)}_+}\Big)+2\Big)-\frac{20}{9}\multsp {T_f}\multsp \delta (1-x)
}

\dispSFinmath{
\Muserfunction{SplittingFunction}[\Mvariable{aggd}]
}

\dispSFoutmath{
{C_A}\multsp \Big(\Big(\frac{67}{9}-4\multsp \zeta (2)\Big)\multsp \delta (1-x)+(8\multsp x-4)\multsp \log((1-x)\multsp x)-
      4\multsp \Big(\frac{\log(x)}{1-x}+{{\Big(\frac{\log(1-x)}{1-x}\Big)}_+}\Big)+2\Big)-\frac{20}{9}\multsp {T_f}\multsp \delta (1-x)
}

\dispSFinmath{
\Muserfunction{SplittingFunction}[\Mvariable{PQQS}]
}

\dispSFoutmath{
{C_F}\multsp {T_f}\multsp \big(-16\multsp (x+1)\multsp {{\log}^2}(x)+(48\multsp x-16)\multsp \log(x)+16\multsp (1-x)\big)
}

\dispSFinmath{
\Muserfunction{SplittingFunction}[\Mvariable{PQQNS}]
}

\dispSFoutmath{
\MathBegin{MathArray}{l}
\bigg(-4\multsp (x+1)\multsp {{\log}^2}(x)-8\multsp \Big(2\multsp x+\frac{3}{1-x}\Big)\multsp \log(x)-  \\
   \noalign{\vspace{1.33333ex}}
\hspace{4.em} \frac{16\multsp ({x^2}+1)\multsp \log(1-x)\multsp \log(x)}{1-x}-40\multsp (1-x)+
      \delta (1-x)\multsp (-24\multsp \zeta (2)+48\multsp \zeta (3)+3)\bigg)\multsp C_{F}^{2}+  \\
\noalign{\vspace{1.33333ex}}
   \hspace{1.em} {N_f}\multsp \bigg(\frac{88\multsp x}{9}+\Big(-\frac{16\multsp \zeta (2)}{3}-\frac{2}{3}\Big)\multsp \delta (1-x)-
      \frac{8\multsp ({x^2}+1)\multsp \log(x)}{3\multsp (1-x)}-\frac{80}{9}\multsp {{\Big(\frac{1}{1-x}\Big)}_+}-\frac{8}{9}\bigg)\multsp
     {C_F}-  \\
\noalign{\vspace{1.42708ex}}
\hspace{1.em} 8\multsp \Big({C_F}-\frac{{C_A}}{2}\Big)\multsp
    \bigg(4\multsp (1-x)+2\multsp (x+1)\multsp \log(x)+
      \frac{({x^2}+1)\multsp \big({{\log}^2}(x)-4\multsp \log(x+1)\multsp \log(x)-2\multsp \zeta (2)-4\multsp {{\Mvariable{Li}}_2}(-x)
          \big)}{x+1}\bigg)\multsp {C_F}+  \\
\noalign{\vspace{1.64583ex}}
\hspace{1.em} {C_A}\multsp
   \bigg(\frac{4\multsp ({x^2}+1)\multsp {{\log}^2}(x)}{1-x}-\frac{4}{3}\multsp \Big(5\multsp x-\frac{22}{1-x}+5\Big)\multsp \log(x)+
     \frac{4}{9}\multsp (53-187\multsp x)+  \\
\noalign{\vspace{1.5625ex}}
\hspace{4.em} 8\multsp (x+1)\multsp \zeta (2)+
     \Big(\frac{536}{9}-16\multsp \zeta (2)\Big)\multsp {{\Big(\frac{1}{1-x}\Big)}_+}+
     \delta (1-x)\multsp \Big(\frac{88\multsp \zeta (2)}{3}-24\multsp \zeta (3)+\frac{17}{3}\Big)\bigg)\multsp {C_F}\\
   \MathEnd{MathArray}
}

\Subsection*{SumS}

\Subsubsection*{Description}

SumS[1, m] is the harmonic number \(\Muserfunction{SplittingFunction}[\Mvariable{PQG}]\) \(\MathBegin{MathArray}{l}
4\multsp {C_F}\multsp {T_f}\multsp
   \big((8\multsp x-4)\multsp {{\log}^2}(1-x)+(16-16\multsp x)\multsp \log(1-x)+  \\
\noalign{\vspace{0.645833ex}}
\hspace{4.em} (
        8-16\multsp x)\multsp \log(x)\multsp \log(1-x)+(4\multsp x-2)\multsp {{\log}^2}(x)+54\multsp x-16\multsp x\multsp \zeta (2)+
      8\multsp \zeta (2)-18\multsp \log(x)-44\big)+  \\
\noalign{\vspace{0.6875ex}}
\hspace{1.em} 4\multsp {C_A}\multsp {T_f}\multsp
   \big((4-8\multsp x)\multsp {{\log}^2}(1-x)+(16\multsp x-16)\multsp \log(1-x)+(-8\multsp x-4)\multsp {{\log}^2}(x)-44\multsp x-
     8\multsp \zeta (2)+  \\
\noalign{\vspace{0.625ex}}
\hspace{4.em} (32\multsp x+4)\multsp \log(x)+
    (-16\multsp x-8)\multsp \log(x)\multsp \log(x+1)+(-16\multsp x-8)\multsp {{\Mvariable{Li}}_2}(-x)+48\big)\\
\MathEnd{MathArray}\) SumS[1,1,m] is \(\Muserfunction{SplittingFunction}[\Mvariable{PGQ}]\)({\itshape i})/{\itshape i}. SumS[k,l,m] is \(\MathBegin{MathArray}{l}
({C_A}\multsp {C_F}).\Big((16-8\multsp x)\multsp {{\log}^2}(1-x)+
     \Big(\frac{8\multsp x}{3}+\frac{80}{3}\Big)\multsp \log(1-x)+(16\multsp x-32)\multsp \log(x)\multsp \log(1-x)+
     (8\multsp x+16)\multsp {{\log}^2}(x)+  \\
\noalign{\vspace{1.33333ex}}
\hspace{4.em} \frac{280\multsp x}{9}+
      16\multsp x\multsp \zeta (2)+(32-104\multsp x)\multsp \log(x)+(16\multsp x+32)\multsp \log(x)\multsp \log(x+1)+
      (16\multsp x+32)\multsp {{\Mvariable{Li}}_2}(-x)+\frac{328}{9}\Big)+  \\
\noalign{\vspace{0.895833ex}}
\hspace{1.em} C_{F}^{2}.
   \big((8\multsp x-16)\multsp {{\log}^2}(1-x)+(-8\multsp x-16)\multsp \log(1-x)+(8-4\multsp x)\multsp {{\log}^2}(x)+32\multsp x-
     (32\multsp x+64)\multsp \log(x)+  \\
\noalign{\vspace{0.958333ex}}
\hspace{4.em} (36\multsp x+48)\multsp \log(x)-68\big)+
   ({C_F}\multsp {T_f}).\Big(-\frac{32\multsp x}{9}+\Big(\frac{32\multsp x}{3}-\frac{64}{3}\Big)\multsp \log(1-x)-\frac{128}{9}\Big)\\
   \MathEnd{MathArray}\). SumS[r, n] represents Sum[Sign[r]\(\RawWedge\)i/i\(\RawWedge\)Abs[r], \{i, 1, n\}]. SumS[r,s, n] is
  Sum[Sign[r]\(\RawWedge\)k/k\(\RawWedge\)Abs[r] Sign[s]\(\RawWedge\)j/j\(\RawWedge\)Abs[s], \{k, 1, n\}, \{j, 1, k\}], etc.

\dispSFinmath{
\Muserfunction{SplittingFunction}[\Mvariable{PGG}]
}

\dispSFoutmath{
\MathBegin{MathArray}{l}
\Big(\Big(-\frac{8}{x+1}+32+\frac{8}{1-x}\Big)\multsp {{\log}^2}(x)+
    \Big(\frac{232}{3}-\frac{536\multsp x}{3}\Big)\multsp \log(x)+\Big(64\multsp x-\frac{32}{1-x}-32\Big)\multsp \log(1-x)\multsp \log(x)
    +  \\
\noalign{\vspace{1.33333ex}}
\hspace{4.em} \Big(64\multsp x+\frac{32}{x+1}+32\Big)\multsp \log(x+1)\multsp \log(x)-
   \frac{388\multsp x}{9}+\frac{64}{3}\multsp \delta (1-x)+
   \zeta (2)\multsp \Big(64\multsp x-16\multsp {{\Big(\frac{1}{1-x}\Big)}_+}+\frac{16}{x+1}\Big)+  \\
\noalign{\vspace{1.32292ex}}
   \hspace{4.em} \frac{536}{9}\multsp {{\Big(\frac{1}{1-x}\Big)}_+}+
      \Big(64\multsp x+\frac{32}{x+1}+32\Big)\multsp {{\Mvariable{Li}}_2}(-x)+24\multsp \delta (1-x)\multsp \zeta (3)-\frac{148}{9}\Big)
     \multsp C_{A}^{2}+  \\
\noalign{\vspace{1.32292ex}}
\hspace{1.em} {T_f}\multsp
    \Big(\frac{608\multsp x}{9}-\frac{32}{3}\multsp \delta (1-x)+\Big(-\frac{32\multsp x}{3}-\frac{32}{3}\Big)\multsp \log(x)-
      \frac{160}{9}\multsp {{\Big(\frac{1}{1-x}\Big)}_+}-\frac{448}{9}\Big)\multsp {C_A}+  \\
\noalign{\vspace{1.05208ex}}
   \hspace{1.em} {C_F}\multsp {T_f}\multsp \big((-16\multsp x-16)\multsp {{\log}^2}(x)+(16\multsp x-80)\multsp \log(x)+80\multsp x-
     8\multsp \delta (1-x)-80\big)\\
\MathEnd{MathArray}
}

See also:  SumP, SumT.

\Subsubsection*{Examples}

\dispSFinmath{
{2^{k-1}}\sum _{i=1}^{2m}(1+{{(-1)}^i})/{i^k}
}

\dispSFoutmath{
\Muserfunction{SumP}[1,m-1]
}

\dispSFinmath{
S_{1}^{'}\NoBreak (\NoBreak m-1\NoBreak )
}

\dispSFoutmath{
\Muserfunction{SumP}[2,m-1]
}

\dispSFinmath{
S_{2}^{'}\NoBreak (\NoBreak m-1\NoBreak )
}

\dispSFoutmath{
\Muserfunction{SumP}[1,m]
}

\dispSFinmath{
S_{1}^{'}\NoBreak (\NoBreak m\NoBreak )
}

\dispSFoutmath{
\Muserfunction{SumP}[1,4]
}

\dispSFinmath{
\frac{25}{12}
}

\dispSFoutmath{
\sum _{i=1}^{8}(1+(-1)\RawWedge i)/i
}

\dispSFinmath{
\frac{25}{12}
}

\dispSFinmath{
\Muserfunction{Explicit}[\Muserfunction{SumP}[1,n/2]]
}

\dispSFoutmath{
\frac{1}{2}\multsp (1-{{(-1)}^n})\multsp {S_1}\Big(\frac{n-1}{2}\Big)+\frac{1}{2}\multsp (1+{{(-1)}^n})\multsp {S_1}\big(\frac{n}{2}\big)
}

\dispSFinmath{
\%/.n\rightarrow 8
}

\dispSFinmath{
\frac{25}{12}
}

\dispSFoutmath{
\sum _{i=1}^{m}\multsp i\RawWedge (-1)
}

\dispSFinmath{
(={S_1}(m)\multsp ).
}

\dispSFoutmath{
\sum _{i=1}^{m}{S_1}
}

\dispSFinmath{
\sum _{i=1}^{m}\multsp {S_l}(i)\big/{i^k}
}

\dispSFoutmath{
\Mfunction{Options}[\Mvariable{SumS}]
}

\dispSFinmath{
\{\Mvariable{Reduce}\rightarrow \Mvariable{False}\}
}

\dispSFoutmath{
\Muserfunction{SumS}[1,m-1]
}

\dispSFinmath{
{S_1}(m-1)
}

\dispSFoutmath{
\Muserfunction{SumS}[2,m-1]
}

\dispSFinmath{
{S_2}(m-1)
}

\dispSFoutmath{
\Muserfunction{SumS}[-1,m]
}

\Subsection*{SumT}

\Subsubsection*{Description}

SumT[1, m] is the alternative harmonic number \({S_{-1}}(m)\) SumT[r, n] represents Sum[(-1)\(\RawWedge\)i/i\(\RawWedge\)r, \{i,1,n\}], SumT[r,s,
n] is Sum[1/k\(\RawWedge\)r
  (-1)\(\RawWedge\)j/j\(\RawWedge\)s, \{k, 1, n\}, \{j, 1, k\}].

See also:  SumP, SumS.

\Subsubsection*{Examples}

\dispSFinmath{
\Muserfunction{SumS}[1,m,\Mvariable{Reduce}\rightarrow \Mvariable{True}]
}

\dispSFoutmath{
{S_1}(m-1)+\frac{1}{m}
}

\dispSFinmath{
\Muserfunction{SumS}[3,m+2,\Mvariable{Reduce}\rightarrow \Mvariable{True}]
}

\dispSFoutmath{
{S_3}(m+1)+\frac{1}{{{(m+2)}^3}}
}

\dispSFinmath{
\Mfunction{SetOptions}[\Mvariable{SumS},\Mvariable{Reduce}\rightarrow \Mvariable{True}];
}

\dispSFoutmath{
\Muserfunction{SumS}[3,m+2]
}

\dispSFinmath{
{S_3}(m-1)+\frac{1}{{m^3}}+\frac{1}{{{(m+1)}^3}}+\frac{1}{{{(m+2)}^3}}
}

\dispSFoutmath{
\Mfunction{SetOptions}[\Mvariable{SumS},\Mvariable{Reduce}\rightarrow \Mvariable{False}];
}

\dispSFinmath{
\Muserfunction{SumS}[1,4]
}

\dispSFoutmath{
\frac{25}{12}
}

\dispSFinmath{
\sum _{i=1}^{4}1/i
}

\dispSFoutmath{
\frac{25}{12}
}

\dispSFinmath{
\Muserfunction{SumS}[1,2,m-1]
}

\dispSFoutmath{
{S_{12}}(m-1)
}

\dispSFinmath{
\Muserfunction{SumS}[1,1,1,11]
}

\dispSFoutmath{
\frac{31276937512951}{4260000729600}
}

\dispSFinmath{
\Muserfunction{SumS}[-1,4]
}

\dispSFoutmath{
-\frac{7}{12}
}

\dispSFinmath{
\Muserfunction{SumT}[1,4]
}

\dispSFoutmath{
-\frac{7}{12}
}

\dispSFinmath{
\sum _{i=1}^{m}\multsp (-1)\RawWedge i/i\multsp
}

\dispSFoutmath{
\Muserfunction{SumT}[1,m-1]
}

\dispSFinmath{
{{\left( \overvar{S}{\RawTilde } \right) }_1}(m-1)
}

\dispSFoutmath{
\Muserfunction{SumT}[2,m-1]
}

\dispSFinmath{
{{\left( \overvar{S}{\RawTilde } \right) }_2}(m-1)
}

\dispSFoutmath{
\Muserfunction{SumT}[1,m]
}

\dispSFinmath{
{{\left( \overvar{S}{\RawTilde } \right) }_1}(m)
}

\dispSFoutmath{
\Muserfunction{SumT}[1,m,\Mvariable{Reduce}\rightarrow \Mvariable{True}]
}

\dispSFinmath{
{{\left( \overvar{S}{\RawTilde } \right) }_1}(m-1)+\frac{{{(-1)}^m}}{m}
}

\dispSFoutmath{
\Muserfunction{SumT}[1,4]
}

\Subsection*{SUND}

\Subsubsection*{Description}

SUND[a,b,c] are the symmetric SU({\itshape N}) \(-\frac{7}{12}\)

See also:  SUNDelta, SUNF, SUNSimplify.

\Subsubsection*{Examples}

\dispSFinmath{
\sum _{i=1}^{4}(-1)\RawWedge i/i
}

\dispSFoutmath{
-\frac{7}{12}
}

\dispSFinmath{
\Muserfunction{SumT}[1,2,m-1]
}

\dispSFoutmath{
{{\left( \overvar{S}{\RawTilde } \right) }_{12}}(m-1)
}

\dispSFinmath{
\Muserfunction{SumT}[1,2,42]
}

\dispSFoutmath{
-\frac{38987958697055013360489864298703621429610152138683927}{10512121660702378405316004964483761080879190528000000}
}

\dispSFinmath{
\Muserfunction{SumT}[1,4]
}

\dispSFoutmath{
-\frac{7}{12}
}

\dispSFinmath{
\Muserfunction{SumS}[-1,4]
}

\dispSFoutmath{
-\frac{7}{12}
}

\dispSFinmath{
\sum _{i=1}^{m-1}(-1)\RawWedge i/i
}

\dispSFoutmath{
\frac{1}{2}\multsp {{(-1)}^m}\multsp \Big({{\psi }^{(0)}}\big(\frac{m}{2}\big)-{{\psi }^{(0)}}\Big(\frac{m+1}{2}\Big)\Big)-
   \frac{\log(4)}{2}
}

\dispSFinmath{
\Muserfunction{SumT}[1,2,12]
}

\dispSFoutmath{
-\frac{57561743656913}{21300003648000}
}

\dispSFinmath{
\Muserfunction{SumS}[1,-2,42]
}

\dispSFoutmath{
-\frac{38987958697055013360489864298703621429610152138683927}{10512121660702378405316004964483761080879190528000000}
}

\dispSFinmath{
\Mfunction{Array}[\Mvariable{SumT},6]
}

\dispSFoutmath{
\big\{-1,-\frac{5}{8},-\frac{179}{216},-\frac{1207}{1728},-\frac{170603}{216000},-\frac{155903}{216000}\big\}
}

\dispSFinmath{
\Mfunction{Array}[\Muserfunction{SumS}[-2,1,\#1]\&,6]
}

\dispSFoutmath{
\big\{-1,-\frac{5}{8},-\frac{179}{216},-\frac{1207}{1728},-\frac{170603}{216000},-\frac{155903}{216000}\big\}
}

\dispSFinmath{
{d_{\Mvariable{abc}}}.
}

\Subsection*{SUNDelta}

\Subsubsection*{Description}

SUNDelta[a, b] is the Kronecker-delta for SU({\itshape N}) with color indices a and b.

See also: ExplicitSUNIndex, SD, SUNF, SUNIndex, SUNSimplify, Trick.

\Subsubsection*{Examples}

\dispSFinmath{
\Muserfunction{SUND}[a,b,c]
}

\dispSFoutmath{
{d_{abc}}
}

\dispSFinmath{
\Mvariable{tt}=\Muserfunction{SUND}[a,b,c,\Mvariable{Explicit}\rightarrow \Mvariable{True}]
}

\dispSFoutmath{
2\multsp \Muserfunction{tr}({T_a}.{T_b}.{T_c})+2\multsp \Muserfunction{tr}({T_b}.{T_a}.{T_c})
}

\dispSFinmath{
\Muserfunction{SUND}[c,a,b]
}

\dispSFoutmath{
{d_{abc}}
}

\dispSFinmath{
\Muserfunction{SUND}[a,b,b]
}

\dispSFoutmath{
{d_{abb}}
}

\dispSFinmath{
\Muserfunction{SUNSimplify}[\Muserfunction{SUND}[a,b,c]\multsp \Muserfunction{SUND}[a,b,c]]
}

\dispSFoutmath{
-2\multsp (4-C_{A}^{2})\multsp {C_F}
}

\dispSFinmath{
\Muserfunction{SUNSimplify}[\Muserfunction{SUND}[a,b,c]\multsp \Muserfunction{SUND}[a,b,c],
     \Mvariable{SUNNToCACF}\rightarrow \Mvariable{False}]//\Muserfunction{Factor2}
}

\dispSFoutmath{
\frac{(1-{N^2})\multsp (4-{N^2})}{N}
}

\dispSFinmath{
\Muserfunction{SUNSimplify}[\Muserfunction{SUND}[a,b,c]\multsp \Muserfunction{SUND}[e,b,c],
     \Mvariable{SUNNToCACF}\rightarrow \Mvariable{False}]//\Muserfunction{Factor2}
}

\dispSFoutmath{
-\frac{(4-{N^2})\multsp {{\delta }_{ae}}}{N}
}

Symbolic arguments to SUNDelta are transformed into the data type SUNIndex and integer arguments are transformed to ExplicitSUNIndex. The
  difference is that SUNSimplify will only sum over symbolic indices.

\dispSFinmath{
\Muserfunction{SUND}[a,b,c]//\Mfunction{StandardForm}
}

\dispSFoutmath{
\Muserfunction{SUND}[a,b,c]
}

\dispSFinmath{
\Muserfunction{SUND}[a,b,c]//\Muserfunction{FCI}//\Mfunction{StandardForm}
}

\dispSFoutmath{
\Muserfunction{SUND}[\Muserfunction{SUNIndex}[a],\Muserfunction{SUNIndex}[b],\Muserfunction{SUNIndex}[c]]
}

\dispSFinmath{
\Muserfunction{SUND}[a,b,c]//\Muserfunction{FCI}//\Muserfunction{FCE}//\Mfunction{StandardForm}
}

\dispSFoutmath{
\Muserfunction{SUND}[a,b,c]
}

\Subsection*{SUNDeltaContract}

\Subsubsection*{Description}

SUNDeltaContract[expr] substitues for all SUNDelta in expr SUNDeltaContract, contracts the SUN(N) indices and resubstitutes SUNDelta.
  SUNDeltaContract[i, j] is the Kronecker-delta for SU({\itshape N}) with contraction properties. SUNDeltaContract wraps also the head SUNIndex around
its arguments.

See also:  SUNDelta, SUNIndex.

\Subsection*{SUNF}

\Subsubsection*{Description}

SUNF[a,b,c] are the structure constants of SU({\itshape N}). The arguments a,b,c should be of symbolic type.

\dispSFinmath{
\Mfunction{Clear}[\Mvariable{tt}]
}

\dispSFoutmath{
\Muserfunction{SUNDelta}[a,b]
}

See also:  SUND, SUNDelta, SUNIndex, SUNSimplify, SUNT, Trick.

\Subsubsection*{Examples}

\dispSFinmath{
{{\delta }_{ab}}
}

\dispSFoutmath{
\Muserfunction{Trick}[\multsp \Muserfunction{SUNDelta}[a,b]\multsp \Muserfunction{SUNDelta}[b,c]\multsp ]
}

\dispSFinmath{
{{\delta }_{ac}}
}

\dispSFoutmath{
\Muserfunction{SUNDelta}[\Muserfunction{SUNIndex}[a],\Muserfunction{SUNIndex}[b]]
}

\dispSFinmath{
{{\delta }_{ab}}
}

\dispSFoutmath{
\Muserfunction{SUNDelta}[a,b]//\Mfunction{StandardForm}
}

\dispSFinmath{
\Muserfunction{SUNDelta}[a,b]
}

\dispSFoutmath{
\Muserfunction{SUNDelta}[a,b]//\Muserfunction{FCI}//\Mfunction{StandardForm}
}

\dispSFinmath{
\Muserfunction{SUNDelta}[\Muserfunction{SUNIndex}[a],\Muserfunction{SUNIndex}[b]]
}

\dispSFoutmath{
\Muserfunction{SUNDelta}[a,b]//\Muserfunction{FCI}//\Muserfunction{FCE}//\Mfunction{StandardForm}
}

This is a consequence of the usual choice for the normalization of the \(\Muserfunction{SD}[a,b]\)

\dispSFinmath{
\Muserfunction{SD}[a,b]//\Muserfunction{FCI}//\Mfunction{StandardForm}
}

\dispSFoutmath{
\Muserfunction{SUNDelta}[\Muserfunction{SUNIndex}[a],\Muserfunction{SUNIndex}[b]]
}

\dispSFinmath{
\Muserfunction{SUNDelta}[a,2]\Muserfunction{SUNDelta}[a,b]\Muserfunction{SUNDelta}[c,2]//\Muserfunction{SUNSimplify}
}

\dispSFoutmath{
{{\delta }_{2b}}\multsp {{\delta }_{2c}}
}

\dispSFinmath{
\%//\Mfunction{StandardForm}
}

\dispSFoutmath{
\MathBegin{MathArray}{l}
\Muserfunction{SUNDelta}[\Muserfunction{ExplicitSUNIndex}[2],\Muserfunction{SUNIndex}[b]]\multsp   \\
   \noalign{\vspace{0.5ex}}
\hspace{1.em} \Muserfunction{SUNDelta}[\Muserfunction{ExplicitSUNIndex}[2],\Muserfunction{SUNIndex}[c]]\\
   \MathEnd{MathArray}
}

\dispSFinmath{
\Muserfunction{SUNDelta}[1,2]//\Muserfunction{FCI}//\Mfunction{StandardForm}
}

\dispSFoutmath{
\Muserfunction{SUNDelta}[\Muserfunction{ExplicitSUNIndex}[1],\Muserfunction{ExplicitSUNIndex}[2]]
}

\dispSFinmath{
\Mfunction{Options}[\Mvariable{SUNF}]
}

\dispSFoutmath{
\{\Mvariable{Explicit}\rightarrow \Mvariable{False}\}
}

\dispSFinmath{
\Mvariable{t1}=\Muserfunction{SUNF}[a,b,c]x+\Muserfunction{SUNF}[b,a,c]y
}

\dispSFoutmath{
x\multsp {f_{abc}}+y\multsp {f_{bac}}
}

\dispSFinmath{
\Muserfunction{Calc}[\Mvariable{t1}]
}

\dispSFoutmath{
x\multsp {f_{abc}}-y\multsp {f_{abc}}
}

\dispSFinmath{
\Muserfunction{SUNSimplify}[\Mvariable{t1}]
}

\dispSFoutmath{
(x-y)\multsp {f_{abc}}
}

\Subsection*{SUNFJacobi}

\Subsubsection*{Description}

SUNFJacobi is an option for SUNSimplify, indicating whether the Jacobi identity should be used.

See also: { }SUNF, SUNSimplify.

\Subsubsection*{Examples}

\dispSFinmath{
\Muserfunction{SUNF}[a,a,b]
}

\dispSFoutmath{
{f_{aab}}
}

\dispSFinmath{
\Muserfunction{SUNF}[a,a,b]//\Muserfunction{Calc}
}

\dispSFoutmath{
0
}

\Subsection*{SUNIndex}

\Subsubsection*{Description}

SUNIndex is the head of SU({\itshape N}) indices.

See also: ExplicitSUNIndex, SUNDelta, SUNF.

\Subsubsection*{Examples}

\dispSFinmath{
{T_a}\multsp \Mvariable{generators}\Mvariable{in}\multsp (\Mvariable{see}\multsp \multsp \Mvariable{SUNT}).
}

\dispSFoutmath{
\Muserfunction{SUNF}[a,b,c,\Mvariable{Explicit}\rightarrow \Mvariable{True}]
}

\dispSFinmath{
2\multsp \ImaginaryI \multsp (\Muserfunction{tr}({T_a}.{T_c}.{T_b})-\Muserfunction{tr}({T_a}.{T_b}.{T_c}))
}

\dispSFoutmath{
\Muserfunction{SUNSimplify}[\Muserfunction{SUNF}[a,b,c]\multsp \Muserfunction{SUNF}[a,b,d]]
}

\dispSFinmath{
{C_A}\multsp {{\delta }_{cd}}
}

\dispSFoutmath{
\Muserfunction{SUNSimplify}[\Muserfunction{SUNF}[a,b,c],\Mvariable{Explicit}\rightarrow \Mvariable{True}]
}

\Subsection*{SUNIndexRename}

\Subsubsection*{Description}

SUNIndexRename is an option of SUNSimplify. If set to False, no automatic renaming of dummy indices is done.

See also:  SUNSimplify, SUNIndex.

\Subsection*{SUNN}

\Subsubsection*{Description}

SUNN denotes the number of colors. Trick[SUNDelta[a, a]] yields (\(-2\multsp \ImaginaryI \multsp (\Muserfunction{tr}({T_a}.{T_b}.{T_c})-\Muserfunction{tr}({T_b}.{T_a}.{T_c}))\)
-1).

See also:  SUNSimplify, Trick, SUNIndex, CA, CF.

\Subsubsection*{Examples}

\dispSFinmath{
\Muserfunction{SUNF}[a,b,c]//\Mfunction{StandardForm}
}

\dispSFoutmath{
\Muserfunction{SUNF}[a,b,c]
}

\Subsection*{SUNNToCACF}

\Subsubsection*{Description}

SUNNToCACF is an option of SUNSimplify. If set to True, the Casimir operator eigenvalues CA (\(=\)N) and CF (\(=\)(N\(\RawWedge\)2-1)/(2
  N)) are introduced.

See also:  SUNSimplify, Trick, SUNN, CA, CF.

\Subsubsection*{Examples}

\dispSFinmath{
\Muserfunction{SUNF}[a,b,c]//\Muserfunction{FCI}//\Mfunction{StandardForm}
}

\dispSFoutmath{
\Muserfunction{SUNF}[\Muserfunction{SUNIndex}[a],\Muserfunction{SUNIndex}[b],\Muserfunction{SUNIndex}[c]]
}

\Subsection*{SUNSimplify}

\Subsubsection*{Description}

SUNSimplify simplifies products of SUNT (and complex conjugated) matrices. Basic renaming of dummy indices is done. If the option
  SUNTrace is set to False, then any SUNT-matrices are taken out of DiracTrace[...]; otherwise a color-trace is taken (by SUNTrace)
  before taking the SUN-objects in front of DiracTrace[...].

\dispSFinmath{
\Muserfunction{SUNF}[a,b,c]//\Muserfunction{FCI}//\Muserfunction{FCE}//\Mfunction{StandardForm}
}

\dispSFoutmath{
\Muserfunction{SUNF}[a,b,c]
}

See also:  Trick.

\Subsubsection*{Examples}

\dispSFinmath{
\Muserfunction{SUNF}[b,a,c]
}

\dispSFoutmath{
{f_{bac}}
}

\dispSFinmath{
\Muserfunction{SUNF}[b,a,c]//\Muserfunction{FCI}
}

\dispSFoutmath{
-{f_{abc}}
}

\dispSFinmath{
\Muserfunction{SUNF}[a,b,c]\Muserfunction{SUNF}[e,f,c]//
   \Muserfunction{SUNSimplify}[\#,\Mvariable{SUNFJacobi}\rightarrow \Mvariable{False}]\&
}

\dispSFoutmath{
{f_{abc}}\multsp {f_{cef}}
}

\dispSFinmath{
\Muserfunction{SUNF}[a,b,c]\Muserfunction{SUNF}[e,f,c]//
   \Muserfunction{SUNSimplify}[\#,\Mvariable{SUNFJacobi}\rightarrow \Mvariable{True}]\&
}

\dispSFoutmath{
{f_{ace}}\multsp {f_{bcf}}-{f_{acf}}\multsp {f_{bce}}
}

\dispSFinmath{
\Muserfunction{SUNIndex}[i]
}

\dispSFoutmath{
i
}

\dispSFinmath{
\%//\Mfunction{StandardForm}
}

\dispSFoutmath{
\Muserfunction{SUNIndex}[i]
}

\dispSFinmath{
\Muserfunction{SUNDelta}[i,j]//\Muserfunction{FCI}//\Mfunction{StandardForm}
}

\dispSFoutmath{
\Muserfunction{SUNDelta}[\Muserfunction{SUNIndex}[i],\Muserfunction{SUNIndex}[j]]
}

\dispSFinmath{
{{\Mvariable{SUNN}}^2}
}

\dispSFoutmath{
\Muserfunction{SUNSimplify}[\Muserfunction{SUNDelta}[\Muserfunction{SUNIndex}[a],\multsp \Muserfunction{SUNIndex}[a]],
    \Mvariable{SUNNToCACF}\rightarrow \Mvariable{False}]
}

\dispSFinmath{
{N^2}-1
}

\dispSFoutmath{
\Muserfunction{SUNSimplify}[\Muserfunction{SUNDelta}[\Muserfunction{SUNIndex}[a],\multsp \Muserfunction{SUNIndex}[a]],
    \Mvariable{SUNNToCACF}\rightarrow \Mvariable{True}]
}

\dispSFinmath{
2\multsp {C_A}\multsp {C_F}
}

\dispSFoutmath{
\Mfunction{Options}[\Mvariable{SUNSimplify}]
}

\dispSFinmath{
\MathBegin{MathArray}{l}
\{\Mvariable{Expanding}\rightarrow \Mvariable{False},\Mvariable{Explicit}\rightarrow \Mvariable{False},
    \Mvariable{Factoring}\rightarrow \Mvariable{False},  \\
\noalign{\vspace{0.666667ex}}
\hspace{1.em} \Mvariable{SUNIndexRename}
     \rightarrow \Mvariable{True},\Mvariable{SUNFJacobi}\rightarrow \Mvariable{False},\Mvariable{SUNNToCACF}\rightarrow \Mvariable{True},
    \Mvariable{SUNTrace}\rightarrow \Mvariable{False}\}\\
\MathEnd{MathArray}
}

\dispSFoutmath{
\Mvariable{t1}=\Muserfunction{SUNDelta}[a,b]\multsp \Muserfunction{SUNDelta}[b,c]
}

\dispSFinmath{
{{\delta }_{ab}}\multsp {{\delta }_{bc}}
}

\dispSFoutmath{
\Muserfunction{SUNSimplify}[\Mvariable{t1}]
}

\dispSFinmath{
{{\delta }_{ac}}
}

\dispSFoutmath{
\Mvariable{t2}=\Muserfunction{SUNT}[a].\Muserfunction{SUNT}[a]
}

\dispSFinmath{
{T_a}.{T_a}
}

\dispSFoutmath{
\Muserfunction{SUNSimplify}[\Mvariable{t2}]
}

\dispSFinmath{
{C_F}
}

\dispSFoutmath{
\Muserfunction{SUNSimplify}[\Mvariable{t2},\Mvariable{SUNNToCACF}\rightarrow \Mvariable{False}]
}

\dispSFinmath{
\frac{{N^2}-1}{2\multsp N}
}

\dispSFoutmath{
\Mvariable{t3}=\Muserfunction{SUNF}[a,r,s]\Muserfunction{SUNF}[b,r,s]
}

\dispSFinmath{
{f_{ars}}\multsp {f_{brs}}
}

\dispSFoutmath{
\Muserfunction{SUNSimplify}[\Mvariable{t3}]
}

\dispSFinmath{
{C_A}\multsp {{\delta }_{ab}}
}

\dispSFoutmath{
\Mvariable{t4}=\Muserfunction{SUNF}[a,b,c]\multsp .\multsp \Muserfunction{SUNF}[a,b,c]
}

\dispSFinmath{
{f_{abc}}.{f_{abc}}
}

\dispSFoutmath{
\Muserfunction{SUNSimplify}[\Mvariable{t4}]
}

\dispSFinmath{
2\multsp C_{A}^{2}\multsp {C_F}
}

\dispSFoutmath{
\Mvariable{t5}=\Muserfunction{SUNF}[a,b,c]\multsp \Muserfunction{SUNF}[d,b,c]
}

\dispSFinmath{
{f_{abc}}\multsp {f_{dbc}}
}

\dispSFoutmath{
\Muserfunction{SUNSimplify}[\Mvariable{t5}]
}

\dispSFinmath{
{C_A}\multsp {{\delta }_{ad}}
}

\dispSFoutmath{
\Mvariable{t6}=\Muserfunction{SUNF}[a,b,c]\multsp \Muserfunction{SUND}[d,b,c]
}

\dispSFinmath{
{d_{bcd}}\multsp {f_{abc}}
}

\dispSFoutmath{
\Muserfunction{SUNSimplify}[\Mvariable{t6},\Mvariable{Explicit}\rightarrow \Mvariable{True}]
}

\dispSFinmath{
0
}

\dispSFoutmath{
\Muserfunction{SUNSimplify}[\Muserfunction{SUND}[a,b,c]\multsp \Muserfunction{SUND}[a,b,c],
     \Mvariable{SUNNToCACF}\rightarrow \Mvariable{False}]//\Muserfunction{Factor2}
}

\dispSFinmath{
\frac{(1-{N^2})\multsp (4-{N^2})}{N}
}

\dispSFoutmath{
\Muserfunction{SUNSimplify}[\Muserfunction{SUND}[a,b,c]\multsp \Muserfunction{SUND}[e,b,c],
     \Mvariable{SUNNToCACF}\rightarrow \Mvariable{False}]//\Mfunction{FullSimplify}
}

\dispSFinmath{
\frac{({N^2}-4)\multsp {{\delta }_{ae}}}{N}
}

\Subsection*{SUNT}

\Subsubsection*{Description}

SUNT[a] is the SU({\itshape N}) \(\Muserfunction{SUNSimplify}[\Muserfunction{SUNF}[a,b,c],\Mvariable{Explicit}\rightarrow \Mvariable{True}]\) generator
in the fundamental representation.

See also:  CA, CF, SUND, SUNDelta, SUNF, SUNSimplify.

\Subsubsection*{Examples}

\dispSFinmath{
-2\multsp \ImaginaryI \multsp (\Muserfunction{tr}({T_a}.{T_b}.{T_c})-\Muserfunction{tr}({T_b}.{T_a}.{T_c}))
}

\dispSFoutmath{
\Muserfunction{SUNSimplify}[\Muserfunction{SUND}[a,b,c],\Mvariable{Explicit}\rightarrow \Mvariable{True}]
}

Since \(2\multsp (\Muserfunction{tr}({T_a}.{T_b}.{T_c})+\Muserfunction{tr}({T_b}.{T_a}.{T_c}))\) is a noncommutative object, products have to separated
by a dot (.).

\dispSFinmath{
\Muserfunction{SUNSimplify}[\Muserfunction{SUNF}[a,b,c]\multsp \Muserfunction{SUNT}[c,b,a]]
}

\dispSFoutmath{
-\frac{1}{2}\multsp \ImaginaryI \multsp {C_A}\multsp {C_F}
}

\dispSFinmath{
\Mvariable{t7}=\Muserfunction{SUNF}[a,b,e]\Muserfunction{SUNF}[c,d,e]+\Muserfunction{SUNF}[a,b,z]\Muserfunction{SUNF}[c,d,z]
}

\dispSFoutmath{
{f_{abe}}\multsp {f_{cde}}+{f_{abz}}\multsp {f_{cdz}}
}

\dispSFinmath{
\Muserfunction{SUNSimplify}[\Mvariable{t7},\Mvariable{Explicit}\rightarrow \Mvariable{False}]
}

\dispSFoutmath{
2\multsp {f_{abe}}\multsp {f_{cde}}
}

\dispSFinmath{
\Muserfunction{SUNSimplify}[\Mvariable{t7},\Mvariable{Explicit}\rightarrow \Mvariable{False},
    \Mvariable{SUNIndexRename}\rightarrow \Mvariable{False}]
}

\dispSFoutmath{
{f_{abe}}\multsp {f_{cde}}+{f_{abz}}\multsp {f_{cdz}}
}

\dispSFinmath{
\Muserfunction{SUNSimplify}[1-\Muserfunction{SUNDelta}[i,i]]
}

\dispSFoutmath{
2-C_{A}^{2}
}

\dispSFinmath{
\Mvariable{t8}=\Muserfunction{DiracTrace}[f[\Muserfunction{SUNIndex}[a]]
     \Muserfunction{DiracMatrix}[\mu ].\Muserfunction{DiracMatrix}[\nu ]]
}

\dispSFoutmath{
\Muserfunction{tr}({{\gamma }^{\mu }}.{{\gamma }^{\nu }}\multsp f(a))
}

The normalizaton of the generators is chosen in the standard way, therefore \(\Muserfunction{SUNSimplify}[\Mvariable{t8},\Mvariable{SUNTrace}\rightarrow
\Mvariable{False}]\)

\dispSFinmath{
\Muserfunction{tr}({{\gamma }^{\mu }}.{{\gamma }^{\nu }}\multsp f(a))
}

\dispSFoutmath{
\Muserfunction{SUNSimplify}[\Mvariable{t8},\Mvariable{SUNTrace}\rightarrow \Mvariable{True}]
}

In case you want \({C_A}\multsp \Muserfunction{tr}({{\gamma }^{\mu }}.{{\gamma }^{\nu }})\multsp f(a)\), you need to include a factor 2Tf inside
the trace.

\dispSFinmath{
\Mfunction{Clear}[\Mvariable{t1},\Mvariable{t2},\Mvariable{t3},\Mvariable{t4},\Mvariable{t5},\Mvariable{t6},\Mvariable{t7},\Mvariable{t8}
   ]
}

\dispSFoutmath{
{T_a}
}

\dispSFinmath{
\Muserfunction{SUNT}[a]
}

\dispSFoutmath{
{T_a}
}

\dispSFinmath{
{T_a}
}

\dispSFoutmath{
\Muserfunction{SUNT}[a].\multsp \Muserfunction{SUNT}[b].\multsp \Muserfunction{SUNT}[c]
}

\dispSFinmath{
{T_a}.{T_b}.{T_c}
}

\dispSFoutmath{
\Muserfunction{SUNT}[a,b,c,d]
}

\Subsection*{SUNTrace}

\Subsubsection*{Description}

SUNTrace[expr] calculates the color-trace.

\dispSFinmath{
{T_a}{T_b}{T_c}{T_d}
}

\dispSFoutmath{
\Muserfunction{SUNSimplify}[\Muserfunction{SUNT}[a,b,a],\Mvariable{SUNNToCACF}\rightarrow \Mvariable{False}]
}

See also:  SUNSimplify, Tr.

\Subsubsection*{Examples}

\dispSFinmath{
-\frac{{T_b}}{2\multsp N}
}

\dispSFoutmath{
\Muserfunction{SUNSimplify}[\Muserfunction{SUNT}[a,b,b,a]]
}

\dispSFinmath{
C_{F}^{2}
}

\dispSFoutmath{
\Muserfunction{SUNSimplify}[\Muserfunction{SUNT}[a,b,a]]
}

\dispSFinmath{
-\frac{1}{2}\multsp ({C_A}-2\multsp {C_F})\multsp {T_b}
}

\dispSFoutmath{
\Muserfunction{SUNSimplify}[\Muserfunction{SUNT}[a,b,a],\Mvariable{SUNNToCACF}\rightarrow \Mvariable{False}]
}

\dispSFinmath{
-\frac{{T_b}}{2\multsp N}
}

\dispSFoutmath{
\Muserfunction{tr}({T_a}{T_b})\multsp =\multsp 1/2\multsp {{\delta }_{\Mvariable{ab}}}.
}

\dispSFinmath{
\Muserfunction{SUNTrace}[\Muserfunction{SUNT}[a,b]]
}

\dispSFoutmath{
\frac{{{\delta }_{ab}}}{2}
}

\dispSFinmath{
{T_f}
}

\dispSFoutmath{
\Muserfunction{SUNTrace}[2\multsp \Mvariable{Tf}\multsp \Muserfunction{SUNT}[a,b]]
}

\dispSFinmath{
{T_f}\multsp {{\delta }_{ab}}
}

\dispSFoutmath{
\Muserfunction{SUNTrace}[\Muserfunction{SUNT}[a,b]]//\Mfunction{StandardForm}
}

\dispSFinmath{
\frac{1}{2}\multsp \Muserfunction{SUNDelta}[\Muserfunction{SUNIndex}[a],\Muserfunction{SUNIndex}[b]]
}

\dispSFoutmath{
\Muserfunction{SUNT}[a]//\Muserfunction{FCI}//\Mfunction{StandardForm}
}

\dispSFinmath{
\Muserfunction{SUNT}[\Muserfunction{SUNIndex}[a]]
}

\dispSFoutmath{
\Muserfunction{SUNT}[a]//\Muserfunction{FCI}//\Muserfunction{FCE}//\Mfunction{StandardForm}
}

\dispSFinmath{
\Muserfunction{SUNT}[a]
}

\dispSFoutmath{
\Mfunction{Options}[\Mvariable{SUNTrace}]
}

\dispSFinmath{
\{\Mvariable{Explicit}\rightarrow \Mvariable{False}\}
}

\dispSFoutmath{
\Muserfunction{SUNTrace}[\Muserfunction{SUNT}[a,b]]
}

\dispSFinmath{
\frac{{{\delta }_{ab}}}{2}
}

\dispSFoutmath{
\Muserfunction{SUNTrace}[\Muserfunction{SUNT}[a,b,c]]
}

\Subsection*{SymbolicSum2}

\Subsubsection*{Description}

SymbolicSum2 is similar to SymbolicSum (Algegra`SymbolicSum`SymbolicSum was a function to do symbolic summation. It was obsolete from
  version 3 - all functionality is now autoloaded by Sum), but extended to several double sums.

\Subsection*{SymbolicSum3}

\Subsubsection*{Description}

SymbolicSymbolicSum3 is similar to SymbolicSum (Algegra`SymbolicSum`SymbolicSum was a function to do symbolic summation. It was obsolete
  from version 3 - all functionality is now autoloaded by Sum), but extended to several double sums.

\Subsection*{Symmetrize}

\Subsubsection*{Description}

Symmetrize[expr, \{a1, a2, ...\}] antisymmetrizes expr with respect to the variables a1,a2, ...

See also: AntiSymmetrize.

\Subsubsection*{Examples}

\dispSFinmath{
\Muserfunction{tr}({T_a}.{T_b}.{T_c})
}

\dispSFoutmath{
\Muserfunction{SUNTrace}[\Muserfunction{SUNT}[a,b,c],\Mvariable{Explicit}\rightarrow \Mvariable{True}]
}

\dispSFinmath{
\frac{{d_{abc}}}{4}+\frac{1}{4}\multsp \ImaginaryI \multsp {f_{abc}}
}

\dispSFoutmath{
\Muserfunction{SUNTrace}[\Muserfunction{SUNT}[a,b,c,d]]
}

\Subsection*{TBox, Tbox}

\Subsubsection*{Description}

TBox[a, b, ...] or Tbox[a, b, ...] produces a RowBox[\{a,b, ...\}] where a, b, ... are boxed in TraditionalForm. TBox and Tbox are used
  internally by FeynCalc to produce the typeset output in TraditionalForm

\Subsection*{Tdec}

\Subsubsection*{Description}

Tdec[\{q,mu\}, \{p\}]; Tdec[\{\{qi, mu\}, \{qj, nu\}, ...\}, \{p1, p2, ...\}] or Tdec[exp, \{\{qi, mu\}, \{qj, nu\}, ...\}, \{p1, p2,
  ...\}] calculates the tensorial decomposition formulas. The more common ones are saved in TIDL.

\dispSFinmath{
\Muserfunction{tr}({T_a}.{T_b}.{T_c}.{T_d})
}

\dispSFoutmath{
\Mvariable{t1}=\Muserfunction{SUNTrace}[\Muserfunction{SUNT}[a,b,c,d],\Mvariable{Explicit}\rightarrow \Mvariable{True}]
}

See also:  TID, TIDL, OneLoopSimplify.

\Subsubsection*{Examples}

Check that \(\frac{1}{8}\multsp {d_{ead}}\multsp {d_{ebc}}-\frac{1}{8}\multsp \ImaginaryI \multsp {f_{ade}}\multsp {d_{ebc}}-
   \frac{1}{8}\multsp {d_{eac}}\multsp {d_{ebd}}+\frac{1}{8}\multsp {d_{eab}}\multsp {d_{ecd}}+
   \frac{{{\delta }_{ad}}\multsp {{\delta }_{bc}}}{4\multsp N}-\frac{{{\delta }_{ac}}\multsp {{\delta }_{bd}}}{4\multsp N}+
   \frac{{{\delta }_{ab}}\multsp {{\delta }_{cd}}}{4\multsp N}+\frac{1}{8}\multsp \ImaginaryI \multsp {d_{ead}}\multsp {f_{bce}}\)

\dispSFinmath{
\Muserfunction{SUNSimplify}[\Mvariable{t1},\Mvariable{Explicit}\rightarrow \Mvariable{True}]
}

\dispSFoutmath{
\Muserfunction{tr}({T_a}.{T_b}.{T_c}.{T_d})
}

\dispSFinmath{
\Mvariable{t2}=\Muserfunction{SUNTrace}[\Muserfunction{SUNT}[a,b,c,d,e],\Mvariable{Explicit}\rightarrow \Mvariable{True}]
}

\dispSFoutmath{
\MathBegin{MathArray}{l}
\frac{{{\delta }_{ab}}\multsp \big(\frac{{d_{cde}}}{4}+\frac{1}{4}\multsp \ImaginaryI \multsp {f_{cde}}\big)}
    {2\multsp N}+  \\
\noalign{\vspace{1.36458ex}}
\hspace{1.em} \frac{1}{2}\multsp {d_{\Mvariable{c110}ab}}\multsp
   \Big(-\frac{1}{8}\multsp {d_{\Mvariable{c111}ce}}\multsp {d_{\Mvariable{c111}\Mvariable{c110}d}}+
     \frac{1}{8}\multsp {d_{\Mvariable{c111}cd}}\multsp {d_{\Mvariable{c111}\Mvariable{c110}e}}+
     \frac{1}{8}\multsp {d_{\Mvariable{c111}c\Mvariable{c110}}}\multsp {d_{\Mvariable{c111}de}}-
     \frac{{{\delta }_{ce}}\multsp {{\delta }_{\Mvariable{c110}d}}}{4\multsp N}+  \\
\noalign{\vspace{1.33333ex}}
\hspace{4.em}
        \frac{{{\delta }_{cd}}\multsp {{\delta }_{\Mvariable{c110}e}}}{4\multsp N}+
      \frac{{{\delta }_{c\Mvariable{c110}}}\multsp {{\delta }_{de}}}{4\multsp N}+
      \frac{1}{8}\multsp \ImaginaryI \multsp {d_{\Mvariable{c111}\Mvariable{c110}e}}\multsp {f_{cd\Mvariable{c111}}}-
      \frac{1}{8}\multsp \ImaginaryI \multsp {d_{\Mvariable{c111}cd}}\multsp {f_{\Mvariable{c110}e\Mvariable{c111}}}\Big)+  \\
   \noalign{\vspace{1.33333ex}}
\hspace{1.em} \frac{1}{2}\multsp \ImaginaryI \multsp {f_{ab\Mvariable{c110}}}\multsp
   \Big(-\frac{1}{8}\multsp {d_{\Mvariable{c112}ce}}\multsp {d_{\Mvariable{c112}\Mvariable{c110}d}}+
     \frac{1}{8}\multsp {d_{\Mvariable{c112}cd}}\multsp {d_{\Mvariable{c112}\Mvariable{c110}e}}+
     \frac{1}{8}\multsp {d_{\Mvariable{c112}c\Mvariable{c110}}}\multsp {d_{\Mvariable{c112}de}}-  \\
\noalign{\vspace{1.33333ex}}
   \hspace{4.em} \frac{{{\delta }_{ce}}\multsp {{\delta }_{\Mvariable{c110}d}}}{4\multsp N}+
    \frac{{{\delta }_{cd}}\multsp {{\delta }_{\Mvariable{c110}e}}}{4\multsp N}+
    \frac{{{\delta }_{c\Mvariable{c110}}}\multsp {{\delta }_{de}}}{4\multsp N}+
    \frac{1}{8}\multsp \ImaginaryI \multsp {d_{\Mvariable{c112}\Mvariable{c110}e}}\multsp {f_{cd\Mvariable{c112}}}-
    \frac{1}{8}\multsp \ImaginaryI \multsp {d_{\Mvariable{c112}cd}}\multsp {f_{\Mvariable{c110}e\Mvariable{c112}}}\Big)\\
   \MathEnd{MathArray}
}

This calculates integral transformation for any \(\Muserfunction{SUNSimplify}[\Mvariable{t2},\Mvariable{Explicit}\rightarrow \Mvariable{True}]\)
\(\Muserfunction{tr}({T_a}.{T_b}.{T_c}.{T_d}.{T_e})\).

\dispSFinmath{
\Muserfunction{SUNSimplify}[\Muserfunction{SUNF}[a,b,c]\multsp \Muserfunction{SUND}[d,b,c]]
}

\dispSFinmath{
0
}

\dispSFoutmath{
\Muserfunction{SUNSimplify}[\Muserfunction{SUNF}[a,b,c]\multsp \Muserfunction{SUND}[a,b,d]]
}

\dispSFinmath{
0
}

\dispSFoutmath{
\Muserfunction{SUNSimplify}[\Muserfunction{SUNF}[a,b,c]\multsp \Muserfunction{SUND}[a,d,c]]
}

\dispSFinmath{
0
}

\Subsection*{TensorFunction}

\Subsubsection*{Description}

Tensorfunction[t, mu, nu, ...] transform into t[LorentzIndex[mu], LorentzIndex[nu], ...], i.e., it can be used as an unspecified
  tensoriell function t. A symmetric tensor can be obtained by Tensorfunction[\{t, "S"\}, mu, nu, ...], and an antisymmteric one by
  Tensorfunction[\{t, "A"\}, mu, nu, ...].

See also:  Symmetrize.

\Subsubsection*{Examples}

\dispSFinmath{
\Muserfunction{SUNSimplify}[\Muserfunction{SUND}[a,b,c]\multsp \Muserfunction{SUND}[d,b,c]]
}

\dispSFoutmath{
-(4-C_{A}^{2})\multsp ({C_A}-2\multsp {C_F})\multsp {{\delta }_{ad}}
}

\dispSFinmath{
\Muserfunction{Symmetrize}[f[a,b],\{a,b\}]
}

\dispSFoutmath{
\frac{1}{2}\multsp (f(a,b)+f(b,a))
}

\dispSFinmath{
\Muserfunction{Symmetrize}[f[x,y,z],\{x,y,z\}]
}

\dispSFoutmath{
\frac{1}{6}\multsp (f(x,y,z)+f(x,z,y)+f(y,x,z)+f(y,z,x)+f(z,x,y)+f(z,y,x))
}

\dispSFinmath{
\Mfunction{Options}[\Mvariable{Tdec}]
}

\dispSFoutmath{
\{\Mvariable{Dimension}\rightarrow D,\Mvariable{Factoring}\rightarrow \Mvariable{Factor2},
    \Mvariable{FeynCalcExternal}\rightarrow \Mvariable{True},\Mvariable{List}\rightarrow \Mvariable{True},
    \Mvariable{NumberOfMetricTensors}\rightarrow \infty \}
}

\dispSFinmath{
\int {d^D}f(p,q){q^{\mu }}=\multsp \frac{{p^{\mu }}}{{p^2}}\int {d^D}f(p,q)p\cdot q
}

\dispSFoutmath{
\Muserfunction{Tdec}[\{q,\mu \},\{p\}]
}

\dispSFinmath{
\big\{\{\Mvariable{X1}\rightarrow {m^2},\Mvariable{X2}\rightarrow p\cdot q\},\frac{\Mvariable{X2}\multsp {p^{\mu }}}{\Mvariable{X1}}
   \big\}
}

\dispSFoutmath{
\%[[2]]/.\%[[1]]
}

\dispSFinmath{
\frac{{p^{\mu }}\multsp p\cdot q}{{m^2}}
}

\dispSFoutmath{
\int {d^D}{q_1}{d^D}{q_2}{d^D}{q_3}
}

\dispSFinmath{
f(p,{q_{1,}}{q_2},{q_3})\multsp q_{1}^{\mu }q_{2}^{\nu }q_{3}^{\rho }
}

\dispSFoutmath{
t=\Muserfunction{Tdec}[\{\{{q_1},\mu \},\{{q_2},\nu \},\{{q_3},\rho \}\},\{p\}];
}

\dispSFinmath{
t[[2]]/.t[[1]]
}

\Subsection*{Tf}

\Subsubsection*{Description}

Tf is the color factor \(\MathBegin{MathArray}{l}
\frac{(1-{D^2})\multsp {p^{\rho }}\multsp {g^{\mu \nu }}\multsp p\cdot {q_3}\multsp
      (p\cdot {q_1}\multsp p\cdot {q_2}-{m^2}\multsp {q_1}\cdot {q_2})}{{{(1-D)}^2}\multsp (D+1)\multsp {m^4}}+
   \frac{(1-{D^2})\multsp {p^{\nu }}\multsp {g^{\mu \rho }}\multsp p\cdot {q_2}\multsp
      (p\cdot {q_1}\multsp p\cdot {q_3}-{m^2}\multsp {q_1}\cdot {q_3})}{{{(1-D)}^2}\multsp (D+1)\multsp {m^4}}+  \\
   \noalign{\vspace{1.52083ex}}
\hspace{1.em} \frac{(1-{D^2})\multsp {p^{\mu }}\multsp {g^{\nu \rho }}\multsp p\cdot {q_1}\multsp
      (p\cdot {q_2}\multsp p\cdot {q_3}-{m^2}\multsp {q_2}\cdot {q_3})}{{{(1-D)}^2}\multsp (D+1)\multsp {m^4}}-
   \frac{1}{{{(1-D)}^2}\multsp (D+1)\multsp {m^6}}((1-{D^2})\multsp {p^{\mu }}\multsp {p^{\nu }}\multsp {p^{\rho }}\multsp   \\
   \noalign{\vspace{0.958333ex}}
\hspace{4.em} (-(p\cdot {q_3})\multsp {q_1}\cdot {q_2}\multsp {m^2}-
      p\cdot {q_2}\multsp {q_1}\cdot {q_3}\multsp {m^2}-p\cdot {q_1}\multsp {q_2}\cdot {q_3}\multsp {m^2}+
      D\multsp p\cdot {q_1}\multsp p\cdot {q_2}\multsp p\cdot {q_3}+2\multsp p\cdot {q_1}\multsp p\cdot {q_2}\multsp p\cdot {q_3}))\\
   \MathEnd{MathArray}\). It is 1/2 for SU({\itshape N}).

See also: Tr2, GluonPropagator, GluonSelfEnergy.

\Subsection*{TFi}

\Subsubsection*{Description}

\dispSFinmath{
\Muserfunction{Contract}[\%\multsp \Muserfunction{FVD}[p,\mu ]\Muserfunction{FVD}[p,\nu ]\multsp \Muserfunction{FVD}[p,\rho ]]//
   \Mfunction{Factor}
}

\Message{\(\MathBegin{MathArray}{l}
\frac{1}{(D-1)\multsp {m^6}}  \\
\noalign{\vspace{1.01042ex}}
\hspace{1.em} (
   {p^4}\multsp (p\cdot {q_3}\multsp {q_1}\cdot {q_2}\multsp {m^4}+p\cdot {q_2}\multsp {q_1}\cdot {q_3}\multsp {m^4}+
      p\cdot {q_1}\multsp {q_2}\cdot {q_3}\multsp {m^4}-3\multsp p\cdot {q_1}\multsp p\cdot {q_2}\multsp p\cdot {q_3}\multsp {m^2}-
      {p^2}\multsp p\cdot {q_3}\multsp {q_1}\cdot {q_2}\multsp {m^2}-  \\
\noalign{\vspace{0.604167ex}}
\hspace{5.em} {p^2}\multsp
       p\cdot {q_2}\multsp {q_1}\cdot {q_3}\multsp {m^2}-{p^2}\multsp p\cdot {q_1}\multsp {q_2}\cdot {q_3}\multsp {m^2}+
      D\multsp {p^2}\multsp p\cdot {q_1}\multsp p\cdot {q_2}\multsp p\cdot {q_3}+
      2\multsp {p^2}\multsp p\cdot {q_1}\multsp p\cdot {q_2}\multsp p\cdot {q_3}))\\
\MathEnd{MathArray}\)}

TFi[d, pp, \{\{n1,m1\},\{n2,m2\},\{n3,m3\},\{n4,m4\},\{n5,m5\}\}] is the 2-loop d-dimensional integral 1/( (q1\(\RawWedge\)2 -
  m1\(\RawWedge\)2)\(\RawWedge\)n1 (q2\(\RawWedge\)2 - m2\(\RawWedge\)2)\(\RawWedge\)n2 ((q1-p)\(\RawWedge\)2 -
  m3\(\RawWedge\)2)\(\RawWedge\)n3 *((q2-p)\(\RawWedge\)2 - m4\(\RawWedge\)2)\(\RawWedge\)n4 * ((q1-q2)\(\RawWedge\)2 -
  m5\(\RawWedge\)2)\(\RawWedge\)n5 ) . TFi[d, pp, \{x,y,z,v,w\}, \{\{n1,m1\},\{n2,m2\},\{n3,m3\},\{n4,m4\},\{n5,m5\}\}] has as additional
  factors in the numerator (q1\(\RawWedge\)2)\(\RawWedge\)x*(q2\(\RawWedge\)2)\(\RawWedge\)y*(q1.p)\(\RawWedge\)z*
  (q2.p)\(\RawWedge\)v*(q1.q2)\(\RawWedge\)w. TFi[d, pp, dp, \{a,b\}, \{\{n1,m1\},\{n2,m2\},\{n3,m3\},\{n4,m4\},\{n5,m5\}\}] has as
  additional factors in the numerator (OPEDelta.q1)\(\RawWedge\)a * (OPEDelta.q2)\(\RawWedge\)b; dp is (OPEDelta.p).

TFi is similar to TFI from the TARCER package, see  hep-ph/9801383.

The function TarcerRecurse from the TARCER package recognize TFi (as well as TFI, which is defined in the HighEnergyPhysics`Tarcer`
  context).

See also: ToTFi, FromTFi.

\Subsubsection*{Examples}

\dispSFinmath{
\Mfunction{Clear}[t];
}

\dispSFoutmath{
\Muserfunction{Tensorfunction}[t,\mu ,\nu ,\tau ]
}

\dispSFinmath{
\Muserfunction{Tensorfunction}(t,\mu ,\nu ,\tau )
}

\dispSFoutmath{
\%//\Mfunction{StandardForm}
}

\Subsection*{ThreeVector}

\Subsubsection*{Description}

ThreeVector[p] is the three dimensional vector p.

See also: DotProduct, CrossProduct.

\Subsection*{TID}

\Subsubsection*{Description}

TID[amp,q] performs a tensorial decomposition.

\dispSFinmath{
\Muserfunction{Tensorfunction}[t,\mu ,\nu ,\tau ]
}

\dispSFoutmath{
\Muserfunction{Contract}[\Muserfunction{FV}[p,\mu ]\multsp \%]
}

See also:  OneLoopSimplify, TIDL.

\Subsubsection*{Examples}

\dispSFinmath{
{p^{\mu }}\multsp \Muserfunction{Tensorfunction}(t,\mu ,\nu ,\tau )
}

\dispSFoutmath{
\%//\Mfunction{StandardForm}
}

\dispSFinmath{
\Muserfunction{Pair}[\Muserfunction{LorentzIndex}[\mu ],\Muserfunction{Momentum}[p]]\multsp
   \Muserfunction{Tensorfunction}[t,\mu ,\nu ,\tau ]
}

\dispSFoutmath{
\Muserfunction{Tensorfunction}[\{f,"S"\},\alpha ,\beta ]
}

\dispSFinmath{
\Muserfunction{Tensorfunction}(\{f,S\},\alpha ,\beta )
}

\dispSFoutmath{
\Muserfunction{Tensorfunction}[\{f,"S"\},\beta ,\alpha ]
}

\dispSFinmath{
\Muserfunction{Tensorfunction}(\{f,S\},\beta ,\alpha )
}

\dispSFoutmath{
\%//\Mfunction{StandardForm}
}

\dispSFinmath{
\Muserfunction{Tensorfunction}[\{f,S\},\beta ,\alpha ]
}

\dispSFoutmath{
\Mfunction{Attributes}[f]
}

\Subsection*{TIDL}

\Subsubsection*{Description}

 TIDL is a database of tensorial reduction formalae.

See also:  TID

\Subsubsection*{Examples}

\Subsubsection*{\(\{\}\)}

In any n-dimensional integral \(\Mfunction{ClearAttributes}[f,\Mvariable{Orderless}]\)can be replaced by

\dispSFinmath{
{T_f}
}

\dispSFoutmath{
?\Mvariable{TFi}
}

\Subsubsection*{\(\Mfunction{Information}::\Mvariable{notfound}:\multsp \Mvariable{Symbol}\multsp \Mvariable{TFi}
    \multsp \Mvariable{not}\multsp \Mvariable{found}.\)}

\dispSFinmath{
\Muserfunction{TFi}[D,M\RawWedge 2,\{\{2,\Mvariable{m1}\},\{1,\Mvariable{m2}\},\{3,\Mvariable{m3}\},\{1,\Mvariable{m4}\},
      \{1,\Mvariable{m5}\}\}]
}

\dispSFoutmath{
F_{\{2,\Mvariable{m1}\}\{1,\Mvariable{m2}\}\{3,\Mvariable{m3}\}\{1,\Mvariable{m4}\}\{1,\Mvariable{m5}\}}^{(D)}
}

\Subsubsection*{\(\Muserfunction{TFi}[D,M\RawWedge 2,\{\{2,M\},1,1,1,1\}]\)}

\dispSFinmath{
F_{\{2,M\}\{1,0\}\{1,0\}\{1,0\}\{1,0\}}^{(D)}
}

\dispSFoutmath{
\Mfunction{Options}[\Mvariable{TID}]
}

\Subsubsection*{\(\MathBegin{MathArray}{l}
\{\Mvariable{Collecting}\rightarrow \Mvariable{True},\Mvariable{Contract}\rightarrow \Mvariable{False},
    \Mvariable{Dimension}\rightarrow D,\Mvariable{ChangeDimension}\rightarrow D,  \\
\noalign{\vspace{0.666667ex}}
\hspace{1.em}
     \Mvariable{DimensionalReduction}\rightarrow \Mvariable{False},\Mvariable{FeynAmpDenominatorCombine}\rightarrow \Mvariable{True},  \\
   \noalign{\vspace{0.666667ex}}
\hspace{1.em} \Mvariable{FeynAmpDenominatorSimplify}\rightarrow \Mvariable{False},
    \Mvariable{Isolate}\rightarrow \Mvariable{False},\Mvariable{ScalarProductCancel}\rightarrow \Mvariable{True}\}\\
\MathEnd{MathArray}\)}

\dispSFinmath{
\Muserfunction{FAD}[k,\multsp k\multsp -\multsp {p_1},\multsp k\multsp -\multsp {p_2}]\Muserfunction{FVD}[k,\mu ]//\Muserfunction{FCI}
}

\dispSFoutmath{
\frac{{k^{\mu }}}{{k^2}.{{(k-{p_1})}^2}.{{(k-{p_2})}^2}}
}

\Subsubsection*{\(\Mvariable{Factor}/@\Muserfunction{TID}[\%,k]//\Muserfunction{FCE}\)}

\dispSFinmath{
\MathBegin{MathArray}{l}
-\frac{\frac{1}
        {([ {k^2} ])\multsp ([ {{(k-{p_1})}^2} ])}\multsp
       \big(p_{2}^{\mu }\multsp p_{1}^{2}-\frac{s\multsp p_{1}^{\mu }}{2}\big)}{2\multsp
       \big(p_{1}^{2}\multsp p_{2}^{2}-\frac{{s^2}}{4}\big)}-
   \frac{\frac{1}{([ {k^2} ])\multsp ([ {{(k-{p_1})}^2} ])\multsp
         ([ {{(k-{p_2})}^2} ])}\multsp
      \big(\frac{1}{2}\multsp s\multsp p_{2}^{\mu }\multsp p_{1}^{2}-p_{1}^{\mu }\multsp p_{2}^{2}\multsp p_{1}^{2}-
        p_{2}^{\mu }\multsp p_{2}^{2}\multsp p_{1}^{2}+\frac{1}{2}\multsp s\multsp p_{1}^{\mu }\multsp p_{2}^{2}\big)}{2\multsp
      \big(p_{1}^{2}\multsp p_{2}^{2}-\frac{{s^2}}{4}\big)}-  \\
\noalign{\vspace{2.0625ex}}
\hspace{1.em} \frac{\frac{1}
       {([ {k^2} ])\multsp ([ {{(k-{p_2})}^2} ])}\multsp
      \big(\frac{s\multsp p_{2}^{\mu }}{2}-p_{1}^{\mu }\multsp p_{2}^{2}\big)}{2\multsp
      \big(\frac{{s^2}}{4}-p_{1}^{2}\multsp p_{2}^{2}\big)}+
   \frac{\frac{1}{([ {{(k-{p_1})}^2} ])\multsp ([ {{(k-{p_2})}^2} ])}\multsp
      \big(-\frac{1}{2}\multsp s\multsp p_{1}^{\mu }+p_{2}^{2}\multsp p_{1}^{\mu }-\frac{s\multsp p_{2}^{\mu }}{2}+
        p_{2}^{\mu }\multsp p_{1}^{2}\big)}{2\multsp \big(p_{1}^{2}\multsp p_{2}^{2}-\frac{{s^2}}{4}\big)}\\
\MathEnd{MathArray}
}

\Subsubsection*{\(\Muserfunction{FAD}[k,\multsp k\multsp -\multsp p]\Muserfunction{FVD}[k,\mu ]//\Muserfunction{FCI}\)}

\dispSFinmath{
\frac{{k^{\mu }}}{{k^2}.{{(k-p)}^2}}
}

\Subsubsection*{\(\Muserfunction{TID}[\%,k]\)}

\dispSFinmath{
\frac{{p^{\mu }}}{2\multsp {k^2}.{{(k-p)}^2}}-\frac{{p^{\mu }}}{2\multsp {k^2}\multsp {p^2}}+
   \frac{{p^{\mu }}}{2\multsp {{(k-p)}^2}\multsp {p^2}}
}

\dispSFoutmath{
\Muserfunction{TID}[\%\%,k,\Mvariable{FeynAmpDenominatorSimplify}\rightarrow \Mvariable{True}]
}

\Subsubsection*{\(\frac{{p^{\mu }}}{2\multsp {k^2}.{{(k-p)}^2}}\)}

\dispSFinmath{
{B_{\mu }}-\Mvariable{type}
}

\Subsubsection*{\(\int {d^n} \multsp {q^{\mu }}\multsp f(q,p)\multsp \Mvariable{the}\multsp {q^{\mu }}\multsp \)}

\dispSFinmath{
\Muserfunction{TIDL}[\{q,\mu \},\{p\}]
}

\Subsubsection*{\(\frac{{p^{\mu }}\multsp p\cdot q}{{p^2}}\)}

\dispSFinmath{
{B_{\Mvariable{\mu \nu }}}-\Mvariable{type}
}

\dispSFinmath{
\Muserfunction{TIDL}[\{\{q,\mu \},\{q,\nu \}\},\{p\}]
}

\dispSFoutmath{
\frac{{g^{\mu \nu }}\multsp ({{p\cdot q}^2}-{p^2}\multsp {q^2})}{(1-D)\multsp {p^2}}-
   \frac{{p^{\mu }}\multsp {p^{\nu }}\multsp (D\multsp {{p\cdot q}^2}-{p^2}\multsp {q^2})}{(1-D)\multsp {p^4}}
}

\Subsection*{TimedIntegrate}

\Subsubsection*{Description}

 TimedIntegrate[exp, vars] is like Integrate, but stops after the number of seconds specified by the option Timing. Options of Integrate
  can be given and are passed on.

\dispSFinmath{
{C_{\mu }}-\Mvariable{type}
}

\dispSFoutmath{
\Muserfunction{TIDL}[\{q,\mu \},\{{p_1},{p_2}\}]
}

\Subsubsection*{Examples}

This should reach to be done

\dispSFinmath{
\frac{p_{2}^{\mu }\multsp q\cdot {p_1}\multsp {p_1}\cdot {p_2}-p_{2}^{\mu }\multsp q\cdot {p_2}\multsp p_{1}^{2}}
    {{{{p_1}\cdot {p_2}}^2}-p_{1}^{2}\multsp p_{2}^{2}}+
   \frac{p_{1}^{\mu }\multsp q\cdot {p_2}\multsp {p_1}\cdot {p_2}-p_{1}^{\mu }\multsp q\cdot {p_1}\multsp p_{2}^{2}}
    {{{{p_1}\cdot {p_2}}^2}-p_{1}^{2}\multsp p_{2}^{2}}
}

\dispSFoutmath{
{C_{\Mvariable{\mu \nu }}}-\Mvariable{type}
}

This shouldn't

\dispSFinmath{
\Muserfunction{TIDL}[\{\{q,\mu \},\{q,\nu \}\},\{{p_1},{p_2}\}]
}

\dispSFoutmath{
\MathBegin{MathArray}{l}
\frac{{g^{\mu \nu }}\multsp (p_{2}^{2}\multsp {{q\cdot {p_1}}^2}-
        2\multsp q\cdot {p_2}\multsp {p_1}\cdot {p_2}\multsp q\cdot {p_1}+{q^2}\multsp {{{p_1}\cdot {p_2}}^2}+
        {{q\cdot {p_2}}^2}\multsp p_{1}^{2}-{q^2}\multsp p_{1}^{2}\multsp p_{2}^{2})}{(D-2)\multsp
      ({{{p_1}\cdot {p_2}}^2}-p_{1}^{2}\multsp p_{2}^{2})}+  \\
\noalign{\vspace{1.125ex}}
\hspace{1.em} (
   p_{2}^{\mu }\multsp p_{2}^{\nu }\multsp (D\multsp {{{p_1}\cdot {p_2}}^2}\multsp {{q\cdot {p_1}}^2}-
      2\multsp {{{p_1}\cdot {p_2}}^2}\multsp {{q\cdot {p_1}}^2}+p_{1}^{2}\multsp p_{2}^{2}\multsp {{q\cdot {p_1}}^2}-
      2\multsp D\multsp q\cdot {p_2}\multsp p_{1}^{2}\multsp {p_1}\cdot {p_2}\multsp q\cdot {p_1}+
      2\multsp q\cdot {p_2}\multsp p_{1}^{2}\multsp {p_1}\cdot {p_2}\multsp   \\
\noalign{\vspace{0.770833ex}}
\hspace{7.em} q\cdot {p_1}
        +D\multsp {{q\cdot {p_2}}^2}\multsp p_{1}^{4}-{{q\cdot {p_2}}^2}\multsp p_{1}^{4}+
        {q^2}\multsp p_{1}^{2}\multsp {{{p_1}\cdot {p_2}}^2}-{q^2}\multsp p_{1}^{4}\multsp p_{2}^{2}))\big/
    \big((D-2)\multsp {{({{{p_1}\cdot {p_2}}^2}-p_{1}^{2}\multsp p_{2}^{2})}^2}\big)+  \\
\noalign{\vspace{0.708333ex}}
   \hspace{1.em} (p_{2}^{\mu }\multsp p_{1}^{\nu }\multsp
    (-{q^2}\multsp {{{p_1}\cdot {p_2}}^3}+D\multsp q\cdot {p_1}\multsp q\cdot {p_2}\multsp {{{p_1}\cdot {p_2}}^2}-
      D\multsp {{q\cdot {p_2}}^2}\multsp p_{1}^{2}\multsp {p_1}\cdot {p_2}+{{q\cdot {p_2}}^2}\multsp p_{1}^{2}\multsp {p_1}\cdot {p_2}-
   \\
\noalign{\vspace{0.583333ex}}
\hspace{6.em} D\multsp {{q\cdot {p_1}}^2}\multsp p_{2}^{2}\multsp {p_1}\cdot {p_2}+
       {{q\cdot {p_1}}^2}\multsp p_{2}^{2}\multsp {p_1}\cdot {p_2}+{q^2}\multsp p_{1}^{2}\multsp p_{2}^{2}\multsp {p_1}\cdot {p_2}+
       D\multsp q\cdot {p_1}\multsp q\cdot {p_2}\multsp p_{1}^{2}\multsp p_{2}^{2}-
       2\multsp q\cdot {p_1}\multsp q\cdot {p_2}\multsp p_{1}^{2}\multsp p_{2}^{2}))\big/  \\
\noalign{\vspace{0.625ex}}
   \hspace{2.em} \big((D-2)\multsp {{({{{p_1}\cdot {p_2}}^2}-p_{1}^{2}\multsp p_{2}^{2})}^2}\big)+
   (p_{1}^{\mu }\multsp p_{2}^{\nu }\multsp (-{q^2}\multsp {{{p_1}\cdot {p_2}}^3}+
       D\multsp q\cdot {p_1}\multsp q\cdot {p_2}\multsp {{{p_1}\cdot {p_2}}^2}-
       D\multsp {{q\cdot {p_2}}^2}\multsp p_{1}^{2}\multsp {p_1}\cdot {p_2}+  \\
\noalign{\vspace{0.604167ex}}
\hspace{6.em} {{q\cdot
        {p_2}}^2}\multsp p_{1}^{2}\multsp {p_1}\cdot {p_2}-D\multsp {{q\cdot {p_1}}^2}\multsp p_{2}^{2}\multsp {p_1}\cdot {p_2}+
   {{q\cdot {p_1}}^2}\multsp p_{2}^{2}\multsp {p_1}\cdot {p_2}+{q^2}\multsp p_{1}^{2}\multsp p_{2}^{2}\multsp {p_1}\cdot {p_2}+  \\
   \noalign{\vspace{0.604167ex}}
\hspace{6.em} D\multsp q\cdot {p_1}\multsp q\cdot {p_2}\multsp p_{1}^{2}\multsp p_{2}^{2}-
        2\multsp q\cdot {p_1}\multsp q\cdot {p_2}\multsp p_{1}^{2}\multsp p_{2}^{2}))\big/
    \big((D-2)\multsp {{({{{p_1}\cdot {p_2}}^2}-p_{1}^{2}\multsp p_{2}^{2})}^2}\big)+  \\
\noalign{\vspace{0.708333ex}}
   \hspace{1.em} (p_{1}^{\mu }\multsp p_{1}^{\nu }\multsp
    (D\multsp {{{p_1}\cdot {p_2}}^2}\multsp {{q\cdot {p_2}}^2}-2\multsp {{{p_1}\cdot {p_2}}^2}\multsp {{q\cdot {p_2}}^2}+
      p_{1}^{2}\multsp p_{2}^{2}\multsp {{q\cdot {p_2}}^2}-
      2\multsp D\multsp q\cdot {p_1}\multsp {p_1}\cdot {p_2}\multsp p_{2}^{2}\multsp q\cdot {p_2}+
      2\multsp q\cdot {p_1}\multsp {p_1}\cdot {p_2}\multsp   \\
\noalign{\vspace{0.666667ex}}
\hspace{7.em} p_{2}^{2}\multsp q\cdot {p_2}
       +D\multsp {{q\cdot {p_1}}^2}\multsp p_{2}^{4}-{{q\cdot {p_1}}^2}\multsp p_{2}^{4}-{q^2}\multsp p_{1}^{2}\multsp p_{2}^{4}+
       {q^2}\multsp {{{p_1}\cdot {p_2}}^2}\multsp p_{2}^{2}))\big/
   \big((D-2)\multsp {{({{{p_1}\cdot {p_2}}^2}-p_{1}^{2}\multsp p_{2}^{2})}^2}\big)\\
\MathEnd{MathArray}
}

\Subsection*{TLI { }***unfinished***}

\Subsubsection*{Description}

TLI[\{v,w,x,y,z\},\{a,b,c,d,e\}, \{al,be,ga,de,ep\}]. { }The exponents of the numerator scalar product are (dl \(=\) OPEDelta): { }v:
  k1.k1, w: k2.k2, { }x: p.k1, y: p.k2, z: k1.k2. { }a: dl.k1, b: dl.k2, { }c: dl.(p-k1), d: dl.(p-k2), e: dl.(k1-k2).\\
TLI[\{mu1, ...\}, \{nu1, ...\}][\{v,w,x,y,z\},\{a,b,c,d,e\}, \{al,be,ga,de,ep\}] is a tensor integral, where mu1 belongs to k1 and nu1 to
  k2.\\
TLI[any\_{}\_{}\_{},\{a,b,c,d,e,0,0\}, \{al,be,ga,de,ep\}] simplifies to { }TLI[any, \{a,b,c,d,e\}, \{al,be,ga,de,ep\}].\\
TLI[\{0,0,0,0,0\},\{a,b,c,d,e\}, \{al,be,ga,de,ep\}] simplifies to { }TLI[\{a,b,c,d,e\}, \{al,be,ga,de,ep\}].

\dispSFinmath{
{C_{\Mvariable{\mu \nu \rho }}}-\Mvariable{type}
}

\dispSFoutmath{
\Muserfunction{TIDL}[\{\{q,\mu \},\{q,\nu \},\{q,\rho \}\},\{p,k\}];
}

See also:  TLI2FC.

\Subsubsection*{Examples}

\dispSFinmath{
{C_{\Mvariable{\mu \nu \rho \sigma }}}-\Mvariable{type}
}

\Subsection*{TLIFP { }***unfinished***}

\Subsubsection*{Description}

TLIFP[exp] does Feynman-Parametrizations of TLI's in exp.

\dispSFinmath{
\Muserfunction{TIDL}[\{\{q,\mu \},\{q,\nu \},\{q,\rho \},\{q,\sigma \}\},\{p,k\}];
}

\dispSFoutmath{
{D_{\mu }}-\Mvariable{type}
}

See also:  TLI.

\Subsubsection*{Examples}

\dispSFinmath{
\Muserfunction{TIDL}[\{q,\mu \},\{{p_1},{p_2},{p_3}\}]
}

\Subsection*{TLIHYP { }***unfinished***}

\Subsubsection*{Description}

TLIHYP[exp] expresses TLI's in exp. in terms of hypergeometric functions, where possible.

\dispSFinmath{
\MathBegin{MathArray}{l}
(p_{3}^{\mu }\multsp (q\cdot {p_3}\multsp {{{p_1}\cdot {p_2}}^2}-
      q\cdot {p_2}\multsp {p_1}\cdot {p_3}\multsp {p_1}\cdot {p_2}-  \\
\noalign{\vspace{0.604167ex}}
\hspace{6.em} q\cdot {p_1}\multsp
        {p_2}\cdot {p_3}\multsp {p_1}\cdot {p_2}-q\cdot {p_3}\multsp p_{1}^{2}\multsp p_{2}^{2}+
       q\cdot {p_1}\multsp {p_1}\cdot {p_3}\multsp p_{2}^{2}+q\cdot {p_2}\multsp p_{1}^{2}\multsp {p_2}\cdot {p_3}))/  \\
   \noalign{\vspace{0.583333ex}}
\hspace{2.em} (p_{3}^{2}\multsp {{{p_1}\cdot {p_2}}^2}-
      2\multsp {p_1}\cdot {p_3}\multsp {p_2}\cdot {p_3}\multsp {p_1}\cdot {p_2}+p_{1}^{2}\multsp {{{p_2}\cdot {p_3}}^2}+
      {{{p_1}\cdot {p_3}}^2}\multsp p_{2}^{2}-p_{1}^{2}\multsp p_{2}^{2}\multsp p_{3}^{2})-  \\
\noalign{\vspace{0.6875ex}}
   \hspace{1.em} (p_{2}^{\mu }\multsp (-(q\cdot {p_2})\multsp {{{p_1}\cdot {p_3}}^2}+
      q\cdot {p_3}\multsp {p_1}\cdot {p_2}\multsp {p_1}\cdot {p_3}+q\cdot {p_1}\multsp {p_2}\cdot {p_3}\multsp {p_1}\cdot {p_3}-  \\
   \noalign{\vspace{0.604167ex}}
\hspace{6.em} q\cdot {p_3}\multsp p_{1}^{2}\multsp {p_2}\cdot {p_3}+
       q\cdot {p_2}\multsp p_{1}^{2}\multsp p_{3}^{2}-q\cdot {p_1}\multsp {p_1}\cdot {p_2}\multsp p_{3}^{2}))/  \\
\noalign{\vspace{
   0.583333ex}}
\hspace{2.em} (p_{3}^{2}\multsp {{{p_1}\cdot {p_2}}^2}-
      2\multsp {p_1}\cdot {p_3}\multsp {p_2}\cdot {p_3}\multsp {p_1}\cdot {p_2}+p_{1}^{2}\multsp {{{p_2}\cdot {p_3}}^2}+
      {{{p_1}\cdot {p_3}}^2}\multsp p_{2}^{2}-p_{1}^{2}\multsp p_{2}^{2}\multsp p_{3}^{2})+  \\
\noalign{\vspace{0.6875ex}}
   \hspace{1.em} (p_{1}^{\mu }\multsp (q\cdot {p_1}\multsp {{{p_2}\cdot {p_3}}^2}-
      q\cdot {p_3}\multsp {p_1}\cdot {p_2}\multsp {p_2}\cdot {p_3}-q\cdot {p_2}\multsp {p_1}\cdot {p_3}\multsp {p_2}\cdot {p_3}+  \\
   \noalign{\vspace{0.645833ex}}
\hspace{6.em} q\cdot {p_3}\multsp {p_1}\cdot {p_3}\multsp p_{2}^{2}+
       q\cdot {p_2}\multsp {p_1}\cdot {p_2}\multsp p_{3}^{2}-q\cdot {p_1}\multsp p_{2}^{2}\multsp p_{3}^{2}))/  \\
\noalign{\vspace{
   0.604167ex}}
\hspace{2.em} (p_{3}^{2}\multsp {{{p_1}\cdot {p_2}}^2}-
    2\multsp {p_1}\cdot {p_3}\multsp {p_2}\cdot {p_3}\multsp {p_1}\cdot {p_2}+p_{1}^{2}\multsp {{{p_2}\cdot {p_3}}^2}+
    {{{p_1}\cdot {p_3}}^2}\multsp p_{2}^{2}-p_{1}^{2}\multsp p_{2}^{2}\multsp p_{3}^{2})\\
\MathEnd{MathArray}
}

\dispSFoutmath{
{D_{\Mvariable{\mu \nu }}}-\Mvariable{type}
}

See also:  TLI.

\Subsubsection*{Examples}

\dispSFinmath{
\Muserfunction{TIDL}[\{\{q,\mu \},\{q,\nu \}\},\{p,k,r\}];
}

\Subsection*{TLI2}

\Subsubsection*{Description}

like TLI, but without any functional properties.

See also:  TLI.

\Subsection*{TLI2FC { }***unfinished***}

\Subsubsection*{Description}

TLI2FC[exp] transforms all TLI-integrals in exp to the FAD form.

\dispSFinmath{
{D_{\Mvariable{\mu \nu \rho }}}-\Mvariable{type}
}

\dispSFoutmath{
\Muserfunction{TIDL}[\{\{q,\mu \},\{q,\nu \},\{q,\rho \}\},\{p,k,r\}];
}

See also: { }TLI. { }FC2TLI.

\Subsubsection*{Examples}

\dispSFinmath{
{D_{\Mvariable{\mu \nu \rho \sigma }}}-\Mvariable{type}
}

\Subsection*{ToDistribution}

\Subsubsection*{Description}

ToDistribution[exp,x] replaces (1-x)\(\RawWedge\)(a Epsilon - 1) in exp by 1/(a Epsilon) DeltaFunction[1-x] \(+\) 1/(1-x) \(+\) a Epsilon
  Log[1-x]/(1-x) \(+\) 1/2 a\(\RawWedge\)2 Epsilon\(\RawWedge\)2 Log[1-x]\(\RawWedge\)2/(1-x)] and (1-x)\(\RawWedge\)(a Epsilon - 2) in
  exp by -1/(a Epsilon) DeltaFunctionPrime[1-x] \(+\) 1/(1-x)\(\RawWedge\)2 \(+\) (a Epsilon) Log[1-x]/(1-x)\(\RawWedge\)2 \(+\)
  a\(\RawWedge\)2 Epsilon\(\RawWedge\)2/2 Log[1-x]\(\RawWedge\)2/(1-x)\(\RawWedge\)2 \(+\) a\(\RawWedge\)3 Epsilon\(\RawWedge\)3/6
  Log[1-x]\(\RawWedge\)3/(1-x)\(\RawWedge\)2.

\dispSFinmath{
\Muserfunction{TIDL}[\{\{q,\mu \},\{q,\nu \},\{q,\rho \},\{q,\sigma \}\},\{p,k,r,s\}];
}

\dispSFoutmath{
\{\Mfunction{Length}[\%],\multsp \Mfunction{LeafCount}[\%],\Mfunction{ByteCount}[\%]\}
}

See also:  PlusDistribution.

\Subsubsection*{Examples}

\dispSFinmath{
\{4,29,424\}
}

\dispSFoutmath{
\Mfunction{Options}[\Mvariable{TimedIntegrate}]
}

\dispSFinmath{
\{\Mvariable{Timing}\rightarrow 10,\Mvariable{Assumptions}\rightarrow \varepsilon >0,
    \Mvariable{Integrate}\rightarrow \Mvariable{Integrate},\Mvariable{Expand}\rightarrow \Mvariable{True}\}
}

\dispSFoutmath{
\Muserfunction{TimedIntegrate}[\log [x\RawWedge 5],\{x,0,1\},\Mvariable{Timing}\rightarrow 1]
}

\dispSFinmath{
\bigg(\int _{0}^{1}\DifferentialD x\VeryThinSpace \bigg).\log\big({x^5}\big)
}

\dispSFoutmath{
\Muserfunction{TimedIntegrate}[\log [\cos [x\RawWedge 5]],\{x,0,1\},\Mvariable{Timing}\rightarrow 10,
    \Mvariable{Integrate}\rightarrow \Mvariable{int}]
}

\dispSFinmath{
\Muserfunction{int}\big(\log\big(\cos\big({x^5}\big)\big),\{x,0,1\},\Mvariable{Assumptions}\rightarrow \varepsilon >0\big)
}

\dispSFoutmath{
\Mfunction{Options}[\Mvariable{TLI}]
}

\dispSFinmath{
\{\Mvariable{EpsilonOrder}\rightarrow 0,\Mvariable{Momentum}\rightarrow p\}
}

\dispSFoutmath{
\{ \}
}


\dispSFoutmath{
\Mfunction{Options}[\Mvariable{TLIFP}]
}

\dispSFinmath{
\{\Mvariable{FeynmanParameterNames}\rightarrow \{x,t,u,s,y\},\Mvariable{GammaExpand}\rightarrow \Mvariable{True},
    \Mvariable{Momentum}\rightarrow p,\Mvariable{Print}\rightarrow \Mvariable{True}\}
}

\dispSFoutmath{
\{ \}
}

\Subsection*{ToHypergeometric}

\Subsubsection*{Description}

ToHypergeometric[\(\), t] returns \(\Mfunction{Options}[\Mvariable{TLIHYP}]\). Remember that Re {\itshape b} \(>\)0 and Re ({\itshape c}-{\itshape
b}) \(>\) 0 should hold (need not be set in {\itshape Mathematica}).

See also:  HypergeometricAC, HypergeometricIR, HypergeometricSE.

\Subsubsection*{Examples}

\dispSFinmath{
\{\Mvariable{FeynmanParameterNames}\rightarrow x,\Mvariable{Momentum}\rightarrow p\}
}

\dispSFoutmath{
\{ \}
}


\dispSFoutmath{
\Mfunction{Options}[\Mvariable{TLI2FC}]
}

\dispSFinmath{
\{\Mvariable{Dimension}\rightarrow D,\Mvariable{FeynCalcExternal}\rightarrow \Mvariable{False},
    \Mvariable{Momentum}\rightarrow \{{q_1},{q_2},p\}\}
}

\dispSFoutmath{
\{ \}
}


\dispSFoutmath{
\Mfunction{Options}[\Mvariable{ToDistribution}]
}

\Subsection*{ToLarin}

\Subsubsection*{Description}

ToLarin[exp] translates all \(\{\Mvariable{PlusDistribution}\rightarrow \Mvariable{Identity}\}\) in exp into \(\Mfunction{SetOptions}[\Mvariable{ToDistribution},\Mvariable{PlusDistribution}\rightarrow
\Mvariable{PlusDistribution}]\)

See also:  Eps.

\Subsubsection*{Examples}

\dispSFinmath{
\{\Mvariable{PlusDistribution}\rightarrow \Mvariable{PlusDistribution}\}
}

\dispSFoutmath{
\Muserfunction{ToDistribution}[(1-x)\RawWedge (\Mvariable{Epsilon}-1),x]
}

\dispSFinmath{
\frac{1}{6}\multsp {{\bigg(\frac{{{\log}^3}(1-x)}{1-x}\bigg)}_+}\multsp {{\varepsilon }^3}+
   \frac{1}{2}\multsp {{\bigg(\frac{{{\log}^2}(1-x)}{1-x}\bigg)}_+}\multsp {{\varepsilon }^2}+
   {{\Big(\frac{\log(1-x)}{1-x}\Big)}_+}\multsp \varepsilon +{{\Big(\frac{1}{1-x}\Big)}_+}+\frac{\delta (1-x)}{\varepsilon }
}

\dispSFoutmath{
\Mfunction{SetOptions}[\Mvariable{ToDistribution},\Mvariable{PlusDistribution}\rightarrow \Mvariable{Identity}]
}

\Subsection*{ToTFi}

\Subsubsection*{Description}

ToTFi[expr, q1, q2, p] translates FeynCalc 2-loop self energy type integrals into the TFi notatation, which can be used as input for the
  function TarcerRecurse from the TARCER package. See TFi for details on the conventions.

See also: TFi, FromTFi.

\Subsubsection*{Examples}

\dispSFinmath{
\{\Mvariable{PlusDistribution}\rightarrow \Mvariable{Identity}\}
}

\dispSFoutmath{
\Muserfunction{ToDistribution}[(1-x)\RawWedge (\Mvariable{Epsilon}-2),x]
}

\dispSFinmath{
\frac{{{\varepsilon }^3}\multsp {{\log}^3}(1-x)}{6\multsp {{(1-x)}^2}}+
   \frac{{{\varepsilon }^2}\multsp {{\log}^2}(1-x)}{2\multsp {{(1-x)}^2}}+\frac{\varepsilon \multsp \log(1-x)}{{{(1-x)}^2}}-
   \frac{{{\delta }^{\prime }}(1-x)}{\varepsilon }+\frac{1}{{{(1-x)}^2}}
}

\dispSFoutmath{
\Muserfunction{Series2}\big[\Mfunction{Integrate}\big[{{(1-x)}^{\Mvariable{Epsilon}-2}},\{x,0,1\},
      \Mvariable{GenerateConditions}\rightarrow \Mvariable{False}\big],\Mvariable{Epsilon},3\big]
}

\dispSFinmath{
-{{\varepsilon }^3}-{{\varepsilon }^2}-\varepsilon -1
}

\dispSFoutmath{
\Mvariable{Options}@\Mvariable{Integrate}
}

\dispSFinmath{
\{\Mvariable{Assumptions}\RuleDelayed \$Assumptions,\Mvariable{GenerateConditions}\rightarrow \Mvariable{Automatic},
    \Mvariable{PrincipalValue}\rightarrow \Mvariable{False}\}
}

\dispSFoutmath{
\Muserfunction{Integrate2}[\Muserfunction{ToDistribution}[(1-x)\RawWedge (\Mvariable{Epsilon}-2),x],\{x,0,1\}]
}

\dispSFinmath{
-{{\varepsilon }^3}-{{\varepsilon }^2}-\varepsilon -1
}

\dispSFoutmath{
{t^b}\multsp {{(1-t)}^c}\multsp {{(1+t\multsp z)}^a}
}

\dispSFinmath{
\MathBegin{MathArray}{l}
\Gamma (b+1)\multsp \Gamma (c+1)/\Gamma (b+c+2)\multsp   \\
\noalign{\vspace{0.666667ex}}
\hspace{1.em} {{
       }_2}{F_1}(-a,b+1;b+c+2;-z)\\
\MathEnd{MathArray}
}

\dispSFoutmath{
\Muserfunction{ToHypergeometric}\big[{t^b}\multsp {{(1-t)}^c}\multsp {{(1+t\multsp z)}^a},t\big]
}

\Subsection*{Tr}

\Subsubsection*{Description}

Tr[exp] calculates the Dirac trace of exp. Depending on the setting of the option SUNTrace also a trace over SU({\itshape N}) objects is performed.

\dispSFinmath{
\frac{\Gamma (b+1)\multsp \Gamma (c+1)\multsp {{\InvisiblePrefixScriptBase }_2}{F_1}(-a,b+1;b+c+2;-z)}
   {\Gamma (b+c+2)}
}

\dispSFoutmath{
\Muserfunction{ToHypergeometric}\big[w\multsp {t^{b-1}}\multsp {{(1-t)}^{c-b-1}}\multsp {{(1-t\multsp z)}^{-a}},t\big]
}

See also: DiracSimplify, DiracTrace, FermionSpinSum, SUNTrace.

\Subsubsection*{Examples}

\dispSFinmath{
\frac{w\multsp \Gamma (b)\multsp \Gamma (c-b)\multsp {{\InvisiblePrefixScriptBase }_2}{F_1}(a,b;c;z)}{\Gamma (c)}
}

\dispSFoutmath{
\Muserfunction{ToHypergeometric}\big[{t^b}\multsp {{(1-t)}^c}\multsp {{(u+t\multsp z)}^a},t\big]
}

\dispSFinmath{
\frac{{u^a}\multsp \Gamma (b+1)\multsp \Gamma (c+1)\multsp
     {{\InvisiblePrefixScriptBase }_2}{F_1}\big(-a,b+1;b+c+2;-\frac{z}{u}\big)}{\Gamma (b+c+2)}
}

\dispSFoutmath{
\Muserfunction{ToHypergeometric}\big[w\multsp {t^{b-1}}\multsp {{(1-t)}^{c-b-1}}\multsp {{(u-t\multsp z)}^{-a}},t\big]
}

\dispSFinmath{
\frac{{u^{-a}}\multsp w\multsp \Gamma (b)\multsp \Gamma (c-b)\multsp
     {{\InvisiblePrefixScriptBase }_2}{F_1}\big(a,b;c;\frac{z}{u}\big)}{\Gamma (c)}
}

\dispSFoutmath{
{{\gamma }^{\mu }}\multsp {{\gamma }^5}
}

\dispSFinmath{
-\ImaginaryI /6\multsp {{\varepsilon }^{\Mvariable{\mu \nu \lambda \sigma }}}{{\gamma }^{\nu }}\multsp {{\gamma }^{\lambda }}\multsp
   {{\gamma }^{\sigma }}.
}

\dispSFoutmath{
\Muserfunction{DiracMatrix}[\nu ].\Muserfunction{DiracMatrix}[\mu ].\Muserfunction{DiracMatrix}[5]
}

\dispSFinmath{
{{\gamma }^{\nu }}.{{\gamma }^{\mu }}.{{\gamma }^5}
}

\dispSFoutmath{
\Muserfunction{ToLarin}[\%]
}

\dispSFinmath{
-\frac{1}{6}\multsp \ImaginaryI \multsp {{\gamma }^{\nu }}.{{\gamma }^{\Mvariable{du1}}}.{{\gamma }^{\Mvariable{du2}}}.
    {{\gamma }^{\Mvariable{du3}}}\multsp {{\epsilon }^{\mu \Mvariable{du1}\Mvariable{du2}\Mvariable{du3}}}
}

\dispSFoutmath{
\Muserfunction{FAD}[\Mvariable{q1},\Mvariable{q1}-p,\{\Mvariable{q2},M\},\{\Mvariable{q2}-p,m\},\Mvariable{q1}-\Mvariable{q2}]
}

\dispSFinmath{
\frac{1}{([ q_{1}^{2} ])\multsp ([ {{({q_1}-p)}^2} ])\multsp
     ([ {{({q_1}-{q_2})}^2} ])\multsp
     ([ q_{2}^{2} - {M^2} ])\multsp
     ([ {{({q_2}-p)}^2} - {m^2} ])}
}

\Print{\(\Muserfunction{ToTFi}[\%,\Mvariable{q1},\Mvariable{q2},p]\)}

\${}West\(=\)False should be set before any calculation is done (because intermediate results from trace calculations are cached).

\dispSFinmath{
F_{\{1,0\}\{1,M\}\{1,0\}\{1,m\}\{1,0\}}^{(D)}
}

\dispSFoutmath{
\%//\Mfunction{StandardForm}
}

\dispSFinmath{
\Muserfunction{TFi}[D,{m^2},\{\{1,0\},\{1,M\},\{1,0\},\{1,m\},\{1,0\}\}]
}

\dispSFoutmath{
\Muserfunction{SOD}[\Mvariable{q1}]\multsp \Muserfunction{SOD}[\Mvariable{q2}]
    \Muserfunction{FAD}[\Mvariable{q1},\Mvariable{q1}-p,\{\Mvariable{q2},M\},\{\Mvariable{q2}-p,m\},\Mvariable{q1}-\Mvariable{q2}]//
   \Muserfunction{FCI}
}

\dispSFinmath{
\frac{\Delta \cdot {q_1}\multsp \Delta \cdot {q_2}}
   {q_{1}^{2}.{{({q_1}-p)}^2}.(q_{2}^{2}-{M^2}).({{({q_2}-p)}^2}-{m^2}).{{({q_1}-{q_2})}^2}}
}

\dispSFoutmath{
\Muserfunction{ToTFi}[\%]
}

\dispSFinmath{
F_{\{1,0\}\{1,M\}\{1,0\}\{1,m\}\{1,0\}}^{(D)\multsp 11}
}

\dispSFoutmath{
\%//\Mfunction{StandardForm}
}

\dispSFinmath{
\Muserfunction{TFi}[D,{m^2},\Muserfunction{SOD}[p],\{1,1\},\{\{1,0\},\{1,M\},\{1,0\},\{1,m\},\{1,0\}\}]
}

\dispSFoutmath{
\Mfunction{Options}[\Mvariable{Tr}]
}

\dispSFinmath{
\MathBegin{MathArray}{l}
\{\Mvariable{DiracTraceEvaluate}\rightarrow \Mvariable{True},\Mvariable{Factoring}\rightarrow \Mvariable{False},
    \Mvariable{FeynCalcExternal}\rightarrow \Mvariable{False},  \\
\noalign{\vspace{0.666667ex}}
\hspace{1.em} \Mvariable{LeviCivitaSign}
    \rightarrow -1,\Mvariable{Mandelstam}\rightarrow \{\},\Mvariable{PairCollect}\rightarrow \Mvariable{False},
   \Mvariable{Schouten}\rightarrow 442,  \\
\noalign{\vspace{0.666667ex}}
\hspace{1.em} \Mvariable{SUNTrace}\rightarrow \Mvariable{False}
    ,\Mvariable{TraceOfOne}\rightarrow 4,\Mvariable{SUNNToCACF}\rightarrow \Mvariable{False}\}\\
\MathEnd{MathArray}
}

\dispSFoutmath{
\Muserfunction{GA}[\mu ,\nu ]
}

\dispSFinmath{
{{\gamma }^{\mu }}.{{\gamma }^{\nu }}
}

\dispSFoutmath{
\Muserfunction{Tr}[\%]
}

\dispSFinmath{
4\multsp {g^{\mu \nu }}
}

\dispSFoutmath{
\Muserfunction{Tr}[(\Muserfunction{GSD}[p]+m).\Muserfunction{GAD}[\mu ].(\Muserfunction{GSD}[q]-m).\Muserfunction{GAD}[\nu ]]
}

\dispSFinmath{
4\multsp (-{g^{\mu \nu }}\multsp {m^2}+{q^{\mu }}\multsp {p^{\nu }}+{p^{\mu }}\multsp {q^{\nu }}-{g^{\mu \nu }}\multsp p\cdot q)
}

\dispSFoutmath{
\Muserfunction{Tr}[\Muserfunction{GA}[\mu ,\nu ,\rho ,\sigma ,5]]
}

\dispSFinmath{
-4\multsp \ImaginaryI \multsp {{\epsilon }^{\mu \nu \rho \sigma }}
}

\dispSFoutmath{
\Muserfunction{Tr}[\Muserfunction{GAD}[\mu ,\nu ,\rho ,\sigma ,\tau ,\xi ,5]]
}

\dispSFinmath{
\MathBegin{MathArray}{l}
4\multsp (\ImaginaryI \multsp {{\epsilon }^{\xi \rho \sigma \tau }}\multsp {g^{\mu \nu }}-
     \ImaginaryI \multsp {{\epsilon }^{\nu \rho \sigma \tau }}\multsp {g^{\mu \xi }}+
     \ImaginaryI \multsp {{\epsilon }^{\nu \xi \sigma \tau }}\multsp {g^{\mu \rho }}-
     \ImaginaryI \multsp {{\epsilon }^{\nu \xi \rho \tau }}\multsp {g^{\mu \sigma }}+  \\
\noalign{\vspace{0.604167ex}}
   \hspace{3.em} \ImaginaryI \multsp {{\epsilon }^{\nu \xi \rho \sigma }}\multsp {g^{\mu \tau }}+
   \ImaginaryI \multsp {{\epsilon }^{\mu \rho \sigma \tau }}\multsp {g^{\nu \xi }}-
   \ImaginaryI \multsp {{\epsilon }^{\mu \xi \sigma \tau }}\multsp {g^{\nu \rho }}+
   \ImaginaryI \multsp {{\epsilon }^{\mu \xi \rho \tau }}\multsp {g^{\nu \sigma }}-
   \ImaginaryI \multsp {{\epsilon }^{\mu \xi \rho \sigma }}\multsp {g^{\nu \tau }}-  \\
\noalign{\vspace{0.604167ex}}
\hspace{3.em}
      \ImaginaryI \multsp {{\epsilon }^{\mu \nu \sigma \tau }}\multsp {g^{\xi \rho }}+
    \ImaginaryI \multsp {{\epsilon }^{\mu \nu \rho \tau }}\multsp {g^{\xi \sigma }}-
    \ImaginaryI \multsp {{\epsilon }^{\mu \nu \rho \sigma }}\multsp {g^{\xi \tau }}+
    \ImaginaryI \multsp {{\epsilon }^{\mu \nu \xi \tau }}\multsp {g^{\rho \sigma }}-
    \ImaginaryI \multsp {{\epsilon }^{\mu \nu \xi \sigma }}\multsp {g^{\rho \tau }}+
    \ImaginaryI \multsp {{\epsilon }^{\mu \nu \xi \rho }}\multsp {g^{\sigma \tau }})\\
\MathEnd{MathArray}
}

\dispSFoutmath{
\$West
}

\dispSFinmath{
\Mvariable{True}
}

\dispSFoutmath{
?\$West
}

\dispSFinmath{
\MathBegin{MathArray}{l}
\Mvariable{If\multsp \$West\multsp is\multsp set\multsp to\multsp True\multsp (which\multsp is\multsp the\multsp
   default),\multsp traces\multsp involving\multsp }  \\
\noalign{\vspace{0.5ex}}
\hspace{2.em} \Mvariable{more\multsp than\multsp
   4\multsp Dirac\multsp matrices\multsp and\multsp gamma5\multsp are\multsp calculated\multsp recursively\multsp }  \\
   \noalign{\vspace{0.5ex}}
\hspace{2.em} \Mvariable{according\multsp to\multsp formula\multsp (A.5)\multsp from\multsp Comp.\multsp
   Phys.\multsp Comm\multsp 77\multsp (1993)\multsp 286-}  \\
\noalign{\vspace{0.5ex}}
\hspace{2.em} 298,\multsp which\multsp is\multsp
   based\multsp on\multsp the\multsp Breitenlohner\multsp Maison\multsp gamma5\multsp -\multsp scheme.\\
\MathEnd{MathArray}
}

\dispSFoutmath{
\$West=\Mvariable{False}
}

\dispSFinmath{
\Mvariable{False}
}

\Subsection*{TraceDimension}

\Subsubsection*{Description}

TraceDimension is an option for FeynCalc2FORM. If set to 4: trace, if set to n: tracen.

See also:  FeynCalc2FORM.

\Subsection*{TraceOfOne}

\Subsubsection*{Description}

TraceOfOne is an option for Tr and DiracTrace. Its setting determines the value of the unit trace.

See also: Tr, DiracTrace.

\Subsubsection*{Examples}

\dispSFinmath{
\Muserfunction{Tr}[\Muserfunction{GAD}[{{\nu }_1},{{\nu }_2},{{\nu }_3},{{\nu }_4},{{\nu }_5},{{\nu }_6}].\Muserfunction{GA}[5]]
}

\dispSFoutmath{
0
}

\dispSFinmath{
\Muserfunction{Tr}[\Muserfunction{GS}[p,q,r,s]]
}

\dispSFoutmath{
4\multsp (p\cdot s\multsp q\cdot r-p\cdot r\multsp q\cdot s+p\cdot q\multsp r\cdot s)
}

\Subsection*{Trick}

\Subsubsection*{Description}

Trick[exp] performs several basic simplifications without expansion. Trick[exp] uses Contract, DotSimplify and SUNDeltaContract.

See also:  Calc, Contract, DiracTrick, DotSimplify, DiracTrick.

\Subsubsection*{\(\Muserfunction{Tr}[(\Muserfunction{GS}[p]+m).\Muserfunction{GA}[\mu ].(\Muserfunction{GS}[q]+m).\Muserfunction{GA}[\mu ],
    \Mvariable{Factoring}\rightarrow \Mvariable{True}]\)}

This calculates \(8\multsp (2\multsp {m^2}-p\cdot q)\)and \(\Muserfunction{Tr}[\Muserfunction{GA}[\alpha ,\beta ],\Mvariable{FCE}\rightarrow \Mvariable{True}]\)
in D dimensions.

\dispSFinmath{
4\multsp {g^{\alpha \beta }}
}

\dispSFoutmath{
\%//\Mfunction{StandardForm}
}

\dispSFinmath{
4\multsp \Muserfunction{MT}[\alpha ,\beta ]
}

\dispSFoutmath{
\Muserfunction{Tr}[\Muserfunction{GAD}[\mu ].\Muserfunction{GA}[\alpha ].\Muserfunction{GA}[\beta ].\Muserfunction{GAD}[\mu ],
    \Mvariable{FCE}\rightarrow \Mvariable{True}]
}

\dispSFinmath{
4\multsp D\multsp {g^{\alpha \beta }}
}

\dispSFoutmath{
\%//\Mfunction{StandardForm}
}

\dispSFinmath{
4\multsp D\multsp \Muserfunction{MT}[\alpha ,\beta ]
}

\dispSFoutmath{
t=\Muserfunction{GA}[\mu ,\nu ]\multsp \Muserfunction{SUNT}[b].\Muserfunction{SUNT}[c]\multsp \Muserfunction{SUNDelta}[c,b]
}

\dispSFinmath{
{{\gamma }^{\mu }}.{{\gamma }^{\nu }}\multsp {T_b}.{T_c}\multsp {{\delta }_{bc}}
}

\dispSFoutmath{
\Muserfunction{Tr}[t,\Mvariable{SUNTrace}\rightarrow \Mvariable{False},\Mvariable{SUNNToCACF}\rightarrow \Mvariable{True}]
}

\Subsection*{TrickIntegrate}

\Subsubsection*{Description}

 TrickIntegrate[(1-t)\(\RawWedge\)(a Epsilon -1) g[t], t] does an integration trick for the definite integral of ((1-t)\(\RawWedge\)(a
  Epsilon -1) g[t]) { }from 0 to 1. TrickIntegrate[(1-t)\(\RawWedge\)(a Epsilon -1) g[t], t] gives { }g[1]/a/Epsilon \(+\)
  Hold[Integrate][(1-t)\(\RawWedge\)(a Epsilon-1) (g[t]-g[1]),\{t,0,1\}]. TrickIntegrate[t\(\RawWedge\)(a Epsilon -1) f[t], t] gives
  f[0]/a/Epsilon \(+\) Hold[Integrate][t\(\RawWedge\)(a Epsilon-1) (f[t]-f[0]),\{t,0,1\}], provided g[1] and f[0] do exist.

\Subsubsection*{Examples}

\dispSFinmath{
4\multsp {C_F}\multsp {g^{\mu \nu }}
}

\dispSFoutmath{
\Muserfunction{Tr}[t,\Mvariable{SUNTrace}\rightarrow \Mvariable{True},\Mvariable{SUNNToCACF}\rightarrow \Mvariable{True}]
}

\Subsection*{TrickMandelstam}

\Subsubsection*{Description}

TrickMandelstam[expr, \{s, t, u, \(4\multsp {C_A}\multsp {C_F}\multsp {g^{\mu \nu }}\) \(+\) \(\Muserfunction{Tr}[1,\Mvariable{SUNTrace}\rightarrow
\Mvariable{False},\Mvariable{SUNNToCACF}\rightarrow \Mvariable{True}]\)\(+\) \(4\)\(+\) \(\Muserfunction{Tr}[1,\Mvariable{SUNTrace}\rightarrow \Mvariable{True},\Mvariable{SUNNToCACF}\rightarrow
\Mvariable{True}]\)\}] simplifies all sums in expr so that one of the Mandelstam variables s, t or u is eliminated by the relation {\itshape s} \(+\)
{\itshape t} \(+\) {\itshape u} \(=\) \(4\multsp {C_A}\) \(+\) \(\Mfunction{Clear}[t];\)\(+\) \(\Muserfunction{Tr}[1,\Mvariable{SUNTrace}\rightarrow
\Mvariable{False},\Mvariable{SUNNToCACF}\rightarrow \Mvariable{True},
    \Mvariable{TraceOfOne}\rightarrow \Mvariable{tr1}]\)\(+\) \(\Mvariable{tr1}\) . The trick is that the resulting sum has the most short number
of terms.

\Subsubsection*{Examples}

\dispSFinmath{
\Muserfunction{Tr}[1,\Mvariable{SUNTrace}\rightarrow \Mvariable{True},\Mvariable{SUNNToCACF}\rightarrow \Mvariable{True},
    \Mvariable{TraceOfOne}\rightarrow \Mvariable{tr2}]
}

\dispSFoutmath{
{C_A}\multsp \Mvariable{tr2}
}

\dispSFinmath{
\Mvariable{Examples}
}

\dispSFoutmath{
{g^{\mu \multsp \nu }}{{\gamma }_{\mu }}
}

\Subsection*{Tr2}

\Subsubsection*{Description}

If exp contains DiracTrace's, Tr2[exp] simplifies exp and does the Dirac traces unless more that 4 gamma matrices and DiracGamma[5]
  occur. Tr2[exp] also separates the color-strucure, and takes the color trace if Tf occurs in exp. If exp does not contain DiracTrace's,
  Tr2[exp] takes the Dirac trace.

\dispSFinmath{
g_{\nu }^{\nu }
}

\dispSFoutmath{
\Muserfunction{Trick}[\{\Muserfunction{GA}[\mu ]\multsp \Muserfunction{MT}[\mu ,\nu ],\multsp \Muserfunction{MTD}[\nu ,\nu ]\}]
}

See also: Tr, Tf, DiracTrace, DiracSimplify, SUNTrace.

\Subsubsection*{Examples}

\dispSFinmath{
\{{{\gamma }^{\nu }},D\}
}

\dispSFoutmath{
\Muserfunction{FV}[p+r,\mu ]\multsp \Muserfunction{MT}[\mu ,\nu ]\multsp \Muserfunction{FV}[q-p,\nu ]
}

\dispSFinmath{
({{q-p}^{\nu }})\multsp ({{p+r}^{\mu }})\multsp {g^{\mu \nu }}
}

\dispSFoutmath{
\Muserfunction{Trick}[\%]
}

\dispSFinmath{
-{p^2}+p\cdot q-p\cdot r+q\cdot r
}

\dispSFoutmath{
\Muserfunction{Trick}[c.b.a\multsp .\multsp \Muserfunction{GA}[d].\Muserfunction{GA}[e]]
}

\dispSFinmath{
a\multsp b\multsp c\multsp {{\gamma }^d}.{{\gamma }^e}
}

\dispSFoutmath{
\%//\Muserfunction{FCE}//\Mfunction{StandardForm}
}

\dispSFinmath{
a\multsp b\multsp c\multsp \Muserfunction{GA}[d].\Muserfunction{GA}[e]
}

\dispSFoutmath{
\Muserfunction{TrickIntegrate}[(1-t)\RawWedge (a\multsp \Mvariable{Epsilon}-1)\multsp g[t],t]
}

\dispSFinmath{
\frac{\underline{\Mvariable{lim}}\, g(1-t)}{a\multsp \varepsilon }+
   \Mfunction{Hold}[\Mvariable{Integrate}]\big[{t^{a\multsp \varepsilon -1}}\multsp \big(g(1-t)-\underline{\Mvariable{lim}}\, g(1-t)\big)
     ,\{t,0,1\}\big]
}

\dispSFoutmath{
m_{1}^{2}
}

\Subsection*{Twist2AlienOperator { }***unfinished*** (unclear usage def.)}

\Subsubsection*{Description}

Twist2AlienOperator[p, 0] : (7); { } Twist2AlienOperator[p1,p2,\{p3,mu,a\}, 0] (p1: incoming quark momentum, p3: incoming gluon
  (count1)).

\Subsubsection*{Examples}

\dispSFinmath{
m_{2}^{2}
}

\dispSFoutmath{
m_{3}^{2}
}

\dispSFinmath{
m_{4}^{2}
}

\dispSFoutmath{
m_{1}^{2}
}

\Subsection*{Twist2CounterOperator { }***unfinished*** (unclear usage def.)}

\Subsubsection*{Description}

Twist2CounterOperator[p,mu,nu,a,b,5]; Twist2CounterOperator[p, 7] : (7); { }Twist2CounterOperator[p1,p2,\{p3,mu,a\}, 1] (p1: incoming
  quark momentum, p3: incoming gluon (count1)).

\Subsubsection*{Examples}

\dispSFinmath{
m_{2}^{2}
}

\dispSFoutmath{
m_{3}^{2}
}

\dispSFinmath{
m_{4}^{2}
}

\dispSFoutmath{
\Muserfunction{TrickMandelstam}\big[(s+t-u)\multsp \big(2M_{W}^{2}-t-u\big),\big\{s,t,u,2M_{W}^{2}\big\}\big]//\Muserfunction{Factor2}
}

\dispSFinmath{
-2\multsp s\multsp (u-M_{W}^{2})
}

\dispSFoutmath{
\Muserfunction{TrickMandelstam}[M\RawWedge 2\multsp s\multsp -\multsp s\RawWedge 2\multsp +\multsp M\RawWedge 2\multsp t\multsp -\multsp
     st\multsp +\multsp M\RawWedge 2\multsp u\multsp -\multsp su,\multsp \{s,t,u,2M\RawWedge 2\}]
}

\Subsection*{Twist2GluonOperator}

\Subsubsection*{Description}

Twist2GluonOperator[\{p, mu, a\}, \{nu, b\}] or Twist2GluonOperator[p, \{mu, a\}, \{nu, b\}] or Twist2GluonOperator[p, mu,a, nu,b] yields
  the 2-gluon operator (p is ingoing momentum corresponding to Lorentz index mu). Twist2GluonOperator[\{p,mu,a\}, \{q,nu,b\}, \{k,la,c\}]
  or Twist2GluonOperator[ p,mu,a , q,nu,b , k,la,c ] gives the 3-gluon operator. Twist2GluonOperator[\{p,mu,a\}, \{q,nu,b\}, \{k,la,c\},
  \{s,si,d\}] or Twist2GluonOperator[p,mu,a , q,nu,b , k,la,c , s,si,d] yields the 4-Gluon operator. The dimension is determined by the
  option and Dimension. The setting of the option Polarization (unpolarized: 0; polarized: 1) determines whether the unpolarized or
  polarized feynman rule is returned. With the setting Explicit to False the color-structure and the (1\(+\)(-1)\(\RawWedge\)OPEm) (for
  polarized: (1-(-1)\(\RawWedge\)OPEm)) is extracted and the color indices are omitted in the arguments of Twist2GluonOperator.

See also:  Twist2QuarkOperator.

\Subsubsection*{Polarized case, zero-momentum insertion }

\dispSFinmath{
2\multsp {M^2}\multsp ({M^2}-s)
}

\dispSFoutmath{
\Mfunction{Options}[\Mvariable{Tr2}]
}

2-gluon legs

\dispSFinmath{
\{\Mvariable{Factoring}\rightarrow \Mvariable{False}\}
}

\dispSFoutmath{
\Muserfunction{DiracTrace}[a]//\Muserfunction{Tr2}
}

\dispSFinmath{
4\multsp a
}

\dispSFoutmath{
\Muserfunction{Tr}[a]
}

\dispSFinmath{
4\multsp a
}

\dispSFoutmath{
\Muserfunction{DiracTrace}[a[\Muserfunction{SUNIndex}[i]]\Mvariable{Tf}]//\Muserfunction{Tr2}
}

\dispSFinmath{
4\multsp {C_A}\multsp {T_f}\multsp a(i)
}

\dispSFoutmath{
\Muserfunction{DiracTrace}[a[\Muserfunction{SUNIndex}[i]]]//\Muserfunction{Tr2}
}

3-gluon legs

\dispSFinmath{
4\multsp a(i)
}

\dispSFoutmath{
\Muserfunction{DiracTrace}[a\multsp \Mvariable{Tf}]//\Muserfunction{Tr2}
}

\dispSFinmath{
4\multsp a\multsp {T_f}
}

\dispSFoutmath{
\Muserfunction{Tr}[a\multsp \Mvariable{Tf},\Mvariable{SUNTrace}\rightarrow \Mvariable{True}]
}

\dispSFinmath{
4\multsp a\multsp {C_A}\multsp {T_f}
}

\dispSFoutmath{
\Muserfunction{Twist2AlienOperator}[p,\multsp 0]
}

4-gluon legs

\dispSFinmath{
\frac{2\multsp {C_F}\multsp g_{s}^{2}\multsp \big(\frac{2}{m}-\frac{1}{m+1}-\frac{2}{m-1}\big)\multsp {S_n}\multsp \gamma \cdot \Delta
     \multsp {{(\Delta \cdot p)}^{m-1}}}{\varepsilon }
}

\dispSFoutmath{
\Muserfunction{Twist2AlienOperator}[\Mvariable{p1},\Mvariable{p2},\{\Mvariable{p3},\mu,a\},\multsp 0]
}

\dispSFinmath{
\MathBegin{MathArray}{l}
\frac{1}{\varepsilon }\bigg(\ImaginaryI \multsp (1+{{(-1)}^m})\multsp g_{s}^{3}\multsp {S_n}\multsp
     {T_a}.(\gamma \cdot \Delta )\multsp {{\Delta }^{\mu}}\multsp   \\
\noalign{\vspace{1.51042ex}}
\hspace{3.em} \bigg(
     \Big(\frac{1}{m-1}-\frac{1}{m}\Big)\multsp {{(-(\Delta \cdot {p_3}))}^{m-2}}+
      \Big(\frac{2}{m}-\frac{1}{m+1}-\frac{2}{m-1}\Big)\multsp
       \bigg(\frac{{{(\Delta \cdot {p_1})}^{m-1}}}{\Delta \cdot {p_1}+\Delta \cdot {p_3}}-
         \frac{{{(-(\Delta \cdot {p_3}))}^{m-1}}}{\Delta \cdot {p_1}+\Delta \cdot {p_3}}\bigg)\bigg)\bigg)\\
\MathEnd{MathArray}
}

\dispSFoutmath{
\Muserfunction{Twist2CounterOperator}[p,\mu,\nu,a,b,5]
}

\dispSFinmath{
\MathBegin{MathArray}{l}
-\frac{1}{2\multsp \varepsilon }\Big((1+{{(-1)}^m})\multsp {C_A}\multsp g_{s}^{2}\multsp {S_n}\multsp   \\
   \noalign{\vspace{1.33333ex}}
\hspace{4.em} \Big(\Big(\frac{8}{m}-\frac{24}{m+1}+\frac{24}{m+2}-\frac{4}{m-1}\Big)\multsp
     {{\Delta }^{\mu}}\multsp {{\Delta }^{\nu}}\multsp {p^2}\multsp {{(\Delta \cdot p)}^{m-2}}+
    \Big(-\frac{6}{m}+\frac{16}{m+1}-\frac{16}{m+2}+\frac{6}{m-1}\Big)\multsp   \\
\noalign{\vspace{1.33333ex}}
\hspace{7.em} (
        {p^{\mu}}\multsp {{\Delta }^{\nu}}+{{\Delta }^{\mu}}\multsp {p^{\nu}})\multsp {{(\Delta \cdot p)}^{m-1}}+
      \Big(\frac{8}{m}-\frac{12}{m+1}+\frac{8}{m+2}-\frac{8}{m-1}\Big)\multsp {g^{\mu\nu}}\multsp {{(\Delta \cdot p)}^m}\Big)\multsp
    {{\delta }_{ab}}\Big)\\
\MathEnd{MathArray}
}

\dispSFoutmath{
\Muserfunction{Twist2CounterOperator}[p,7]
}

\Subsubsection*{Unpolarized case, zero-momentum insertion}

\dispSFinmath{
\frac{(1+{{(-1)}^m})\multsp {C_F}\multsp g_{s}^{2}\multsp \big(\frac{2}{m}-\frac{1}{m+1}-\frac{2}{m-1}\big)\multsp {S_n}\multsp
     \gamma \cdot \Delta \multsp {{(\Delta \cdot p)}^{m-1}}}{\varepsilon }
}

\dispSFoutmath{
\Muserfunction{Twist2CounterOperator}[\Mvariable{p1},\Mvariable{p2},\{\Mvariable{p3},\mu,a\},1]
}

2-gluon legs

\dispSFinmath{
-\frac{(1+{{(-1)}^m})\multsp ({C_A}-2\multsp {C_F})\multsp g_{s}^{3}\multsp \big(\frac{2}{m}-\frac{1}{m+1}-\frac{2}{m-1}\big)\multsp
      {S_n}\multsp \gamma \cdot \Delta \multsp {{\Delta }^{\mu}}\multsp
      \big(\frac{{{(\Delta \cdot {p_1})}^{m-1}}}{\Delta \cdot {p_1}+\Delta \cdot {p_2}}-
        \frac{{{(-(\Delta \cdot {p_2}))}^{m-1}}}{\Delta \cdot {p_1}+\Delta \cdot {p_2}}\big)\multsp {T_a}}{2\multsp \varepsilon }
}

\dispSFoutmath{
\MathBegin{MathArray}{l}
\Mfunction{SetOptions}[\Mvariable{Twist2GluonOperator},\multsp
    \Mvariable{Polarization}\multsp \rightarrow \multsp 1,  \\
\noalign{\vspace{0.5ex}}
\hspace{1.em} \Mvariable{Explicit}\rightarrow
     \Mvariable{False},\multsp \Mvariable{ZeroMomentumInsertion}\multsp \rightarrow \multsp \Mvariable{True}]\\
\MathEnd{MathArray}
}

\dispSFinmath{
\{\Mvariable{CouplingConstant}\rightarrow {g_s},\Mvariable{Dimension}\rightarrow D,\Mvariable{Polarization}\rightarrow 1,
    \Mvariable{Explicit}\rightarrow \Mvariable{False},\Mvariable{ZeroMomentumInsertion}\rightarrow \Mvariable{True}\}
}

\dispSFoutmath{
\Muserfunction{Twist2GluonOperator}[p,\{\mu ,a\},\{\nu ,b\}]
}

\dispSFinmath{
\ImaginaryI \multsp (1-{{(-1)}^m})\multsp {{\delta }_{ab}}\multsp
   \big(\Mfunction{O}_{\mu \VeryThinSpace \nu }^{\Mvariable{G2}}\Mfunction{(}p)\big)
}

\dispSFoutmath{
\Muserfunction{Twist2GluonOperator}[p,\{\mu ,a\},\{\nu ,b\},\Mvariable{Explicit}\rightarrow \Mvariable{True}]
}

\dispSFinmath{
\ImaginaryI \multsp {{\epsilon }^{\mu \nu \Delta p}}\multsp (1-{{(-1)}^m})\multsp {{(\Delta \cdot p)}^{m-1}}\multsp {{\delta }_{ab}}
}

\dispSFoutmath{
\Muserfunction{Twist2GluonOperator}[p,\{\mu \},\{\nu \}]
}

3-gluon legs

\dispSFinmath{
\Mfunction{O}_{\mu \VeryThinSpace \nu }^{\Mvariable{G2}}\Mfunction{(}p)
}

\dispSFoutmath{
\Muserfunction{Twist2GluonOperator}[p,\{\mu \},\{\nu \},\Mvariable{Explicit}\rightarrow \Mvariable{True}]
}

\dispSFinmath{
{{\epsilon }^{\mu \nu \Delta p}}\multsp {{(\Delta \cdot p)}^{m-1}}
}

\dispSFoutmath{
\Muserfunction{Twist2GluonOperator}[\{p,\mu ,a\},\{q,\nu ,b\},\{r,\rho ,c\}]
}

\dispSFinmath{
{g_s}\multsp (1-{{(-1)}^m})\multsp {f_{abc}}\multsp \big(
    \Mfunction{O}_{\nu \VeryThinSpace \rho \VeryThinSpace \mu }^{\Mvariable{G3}}\Mfunction{(}q,r,p)\big)
}

\dispSFoutmath{
\Muserfunction{Twist2GluonOperator}[\{p,\mu \},\{q,\nu \},\{r,\rho \}]
}

4-gluon legs

\dispSFinmath{
\Mfunction{O}_{\nu \VeryThinSpace \rho \VeryThinSpace \mu }^{\Mvariable{G3}}\Mfunction{(}q,r,p)
}

\dispSFoutmath{
\Muserfunction{Twist2GluonOperator}[\{p,\mu \},\{q,\nu \},\{r,\rho \},\Mvariable{Explicit}\rightarrow \Mvariable{True}]
}

\dispSFinmath{
\MathBegin{MathArray}{l}
-({{\epsilon }^{\mu \rho p\Delta }}\multsp {{\Delta }^{\nu }}-
       {{\epsilon }^{\mu \nu p\Delta }}\multsp {{\Delta }^{\rho }})\multsp {{(\Delta \cdot p)}^{m-2}}-
   ({{\epsilon }^{\nu \rho \Delta q}}\multsp {{\Delta }^{\mu }}+{{\epsilon }^{\mu \nu \Delta q}}\multsp {{\Delta }^{\rho }})\multsp
    {{(\Delta \cdot q)}^{m-2}}-  \\
\noalign{\vspace{1.0625ex}}
\hspace{1.em} (
     {{\epsilon }^{\nu \rho \Delta r}}\multsp {{\Delta }^{\mu }}-{{\epsilon }^{\mu \rho \Delta r}}\multsp {{\Delta }^{\nu }})\multsp
    {{(\Delta \cdot r)}^{m-2}}+\bigg(\sum _{i=0}^{m-3}{{(-1)}^i}\multsp {{(\Delta \cdot p)}^i}\multsp {{(\Delta \cdot q)}^{-i+m-3}}\bigg)
    \multsp {{\Delta }^{\rho }}\multsp ({{\epsilon }^{\nu p\Delta q}}\multsp {{\Delta }^{\mu }}+
      {{\epsilon }^{\mu \nu \Delta q}}\multsp \Delta \cdot p)-  \\
\noalign{\vspace{1.5625ex}}
\hspace{1.em} \bigg(
     \sum _{i=0}^{m-3}{{(-1)}^i}\multsp {{(\Delta \cdot p)}^i}\multsp {{(\Delta \cdot r)}^{-i+m-3}}\bigg)\multsp {{\Delta }^{\nu }}
    \multsp ({{\epsilon }^{\rho p\Delta r}}\multsp {{\Delta }^{\mu }}+{{\epsilon }^{\mu \rho \Delta r}}\multsp \Delta \cdot p)-  \\
   \noalign{\vspace{1.5625ex}}
\hspace{1.em} \bigg(\sum _{i=0}^{m-3}
     {{(-1)}^i}\multsp {{(\Delta \cdot q)}^i}\multsp {{(\Delta \cdot r)}^{-i+m-3}}\bigg)\multsp {{\Delta }^{\mu }}\multsp
   ({{\epsilon }^{\rho \Delta qr}}\multsp {{\Delta }^{\nu }}-{{\epsilon }^{\nu \rho \Delta r}}\multsp \Delta \cdot q)\\
   \MathEnd{MathArray}
}

\dispSFoutmath{
\Muserfunction{Twist2GluonOperator}[\{p,\mu ,a\},\{q,\nu ,b\},\{r,\rho ,c\},\{s,\sigma ,d\}]
}

\dispSFinmath{
\MathBegin{MathArray}{l}
\ImaginaryI \multsp (1-{{(-1)}^m})\multsp g_{s}^{2}\multsp   \\
\noalign{\vspace{0.604167ex}}
   \hspace{1.em} \big(-{f_{ab\Mvariable{c135}}}\multsp {f_{c\Mvariable{c135}d}}\multsp
     \big(\Mfunction{O}_{\mu \VeryThinSpace \nu \VeryThinSpace \rho \VeryThinSpace \sigma }^{\Mvariable{G4}}\Mfunction{(}p,q,r,s)\big)-
    {f_{ac\Mvariable{c135}}}\multsp {f_{b\Mvariable{c135}d}}\multsp
     \big(\Mfunction{O}_{\mu \VeryThinSpace \rho \VeryThinSpace \nu \VeryThinSpace \sigma }^{\Mvariable{G4}}\Mfunction{(}p,r,q,s)\big)+
    {f_{a\Mvariable{c135}d}}\multsp {f_{bc\Mvariable{c135}}}\multsp
     \big(\Mfunction{O}_{\rho \VeryThinSpace \nu \VeryThinSpace \mu \VeryThinSpace \sigma }^{\Mvariable{G4}}\Mfunction{(}r,q,p,s)\big)
     \big)\\
\MathEnd{MathArray}
}

\dispSFoutmath{
\Muserfunction{Twist2GluonOperator}[\{p,\mu \},\{q,\nu \},\{r,\rho \},\{s,\sigma \}]
}

\Subsubsection*{Advanced Examples}

The setting All for Explicit performs the sums.

\dispSFinmath{
\Mfunction{O}_{\mu \VeryThinSpace \nu \VeryThinSpace \rho \VeryThinSpace \sigma }^{\Mvariable{G4}}\Mfunction{(}p,q,r,s)
}

\dispSFinmath{
\Muserfunction{Twist2GluonOperator}[\{p,\mu \},\{q,\nu \},\{r,\rho \},\{s,\sigma \},\Mvariable{Explicit}\rightarrow \Mvariable{True}]
}

\Subsubsection*{Polarized case, non-zero-momentum insertion}

\dispSFinmath{
\MathBegin{MathArray}{l}
({{\epsilon }^{\Delta \nu \rho \sigma }}\multsp {{\Delta }^{\mu }}-
      {{\epsilon }^{\Delta \mu \rho \sigma }}\multsp {{\Delta }^{\nu }})\multsp {{(\Delta \cdot r+\Delta \cdot s)}^{m-2}}-
   \bigg(\sum _{i=0}^{m-3}{{(-(\Delta \cdot p)-\Delta \cdot q)}^i}\multsp {{(\Delta \cdot s)}^{-i+m-3}}\bigg)\multsp
    ({{\epsilon }^{\nu \sigma \Delta s}}\multsp {{\Delta }^{\mu }}-{{\epsilon }^{\mu \sigma \Delta s}}\multsp {{\Delta }^{\nu }})\multsp
    {{\Delta }^{\rho }}+  \\
\noalign{\vspace{1.5625ex}}
\hspace{1.em} \bigg(
     \sum _{i=0}^{m-3}{{(\Delta \cdot p+\Delta \cdot q)}^{-i+m-3}}\multsp {{(-(\Delta \cdot r))}^i}\bigg)\multsp
    ({{\epsilon }^{\rho \nu r\Delta }}\multsp {{\Delta }^{\mu }}-{{\epsilon }^{\rho \mu r\Delta }}\multsp {{\Delta }^{\nu }})\multsp
    {{\Delta }^{\sigma }}-  \\
\noalign{\vspace{1.5625ex}}
\hspace{1.em} \bigg(
     \sum _{i=0}^{m-3}{{(-(\Delta \cdot p))}^i}\multsp {{(\Delta \cdot r+\Delta \cdot s)}^{-i+m-3}}\bigg)\multsp {{\Delta }^{\nu }}
    \multsp ({{\epsilon }^{\mu \sigma \Delta p}}\multsp {{\Delta }^{\rho }}-
      {{\epsilon }^{\mu \rho \Delta p}}\multsp {{\Delta }^{\sigma }})+  \\
\noalign{\vspace{1.5625ex}}
\hspace{1.em} \bigg(
     \sum _{i=0}^{m-3}{{(-(\Delta \cdot q))}^i}\multsp {{(\Delta \cdot r+\Delta \cdot s)}^{-i+m-3}}\bigg)\multsp {{\Delta }^{\mu }}
    \multsp ({{\epsilon }^{\nu \sigma \Delta q}}\multsp {{\Delta }^{\rho }}-
      {{\epsilon }^{\nu \rho \Delta q}}\multsp {{\Delta }^{\sigma }})-  \\
\noalign{\vspace{1.66667ex}}
\hspace{1.em} \Bigg(
     \sum _{j=0}^{m-4}\multsp (j+1){{(\Delta \cdot p)}^{-j+m-4}}\multsp {{(-(\Delta \cdot r)-\Delta \cdot s)}^{j-i}}\multsp
       {{(-(\Delta \cdot s))}^i}\Bigg)\multsp {{\Delta }^{\nu }}\multsp {{\Delta }^{\rho }}\multsp
    (-{{\epsilon }^{\Delta \sigma ps}}\multsp {{\Delta }^{\mu }}-{{\epsilon }^{\mu \sigma \Delta s}}\multsp \Delta \cdot p)+  \\
   \noalign{\vspace{1.67708ex}}
\hspace{1.em} \Bigg(\sum _{j=0}^{m-4}\multsp (j+1)
      {{(\Delta \cdot q)}^{-j+m-4}}\multsp {{(-(\Delta \cdot r)-\Delta \cdot s)}^{j-i}}\multsp {{(-(\Delta \cdot s))}^i}\Bigg)\multsp
    {{\Delta }^{\mu }}\multsp {{\Delta }^{\rho }}\multsp
    (-{{\epsilon }^{\Delta \sigma qs}}\multsp {{\Delta }^{\nu }}-{{\epsilon }^{\nu \sigma \Delta s}}\multsp \Delta \cdot q)+  \\
   \noalign{\vspace{1.67708ex}}
\hspace{1.em} \Bigg(\sum _{j=0}^{m-4}\multsp (j+1)
      {{(\Delta \cdot p)}^{-j+m-4}}\multsp {{(-(\Delta \cdot r))}^i}\multsp {{(-(\Delta \cdot r)-\Delta \cdot s)}^{j-i}}\Bigg)\multsp
    {{\Delta }^{\nu }}\multsp {{\Delta }^{\sigma }}\multsp
    (-{{\epsilon }^{\Delta \mu pr}}\multsp {{\Delta }^{\rho }}-{{\epsilon }^{\mu \rho \Delta p}}\multsp \Delta \cdot r)-  \\
   \noalign{\vspace{1.67708ex}}
\hspace{1.em} \Bigg(\sum _{j=0}^{m-4}\multsp (j+1)
     {{(\Delta \cdot q)}^{-j+m-4}}\multsp {{(-(\Delta \cdot r))}^i}\multsp {{(-(\Delta \cdot r)-\Delta \cdot s)}^{j-i}}\Bigg)\multsp
   {{\Delta }^{\mu }}\multsp {{\Delta }^{\sigma }}\multsp
   (-{{\epsilon }^{\Delta \nu qr}}\multsp {{\Delta }^{\rho }}-{{\epsilon }^{\nu \rho \Delta q}}\multsp \Delta \cdot r)\\
   \MathEnd{MathArray}
}

\dispSFoutmath{
\Mfunction{SetOptions}[\Mvariable{Twist2GluonOperator},\Mvariable{Polarization}\rightarrow 0,
    \Mvariable{ZeroMomentumInsertion}\rightarrow \Mvariable{True}]
}

2-gluon legs

\dispSFinmath{
\{\Mvariable{CouplingConstant}\rightarrow {g_s},\Mvariable{Dimension}\rightarrow D,\Mvariable{Polarization}\rightarrow 0,
    \Mvariable{Explicit}\rightarrow \Mvariable{False},\Mvariable{ZeroMomentumInsertion}\rightarrow \Mvariable{True}\}
}

\dispSFoutmath{
\Muserfunction{Twist2GluonOperator}[p,\{\mu ,a\},\{\nu ,b\}]
}

3-gluon legs

\dispSFinmath{
\frac{1}{2}\multsp ({{(-1)}^m}+1)\multsp {{\delta }_{ab}}\multsp
   \big(\Mfunction{O}_{\mu \VeryThinSpace \nu }^{\Mvariable{G2}}\Mfunction{(}p)\big)
}

\dispSFoutmath{
\Muserfunction{Twist2GluonOperator}[p,\{\mu ,a\},\{\nu ,b\},\Mvariable{Explicit}\rightarrow \Mvariable{True}]
}

4-gluon legs

\dispSFinmath{
\frac{1}{2}\multsp ({g^{\mu \nu }}\multsp {{\Delta \cdot p}^2}-
     ({p^{\mu }}\multsp {{\Delta }^{\nu }}+{{\Delta }^{\mu }}\multsp {p^{\nu }})\multsp \Delta \cdot p+
     {{\Delta }^{\mu }}\multsp {{\Delta }^{\nu }}\multsp {p^2})\multsp ({{(-1)}^m}+1)\multsp {{(\Delta \cdot p)}^{m-2}}\multsp
   {{\delta }_{ab}}
}

\dispSFoutmath{
\Muserfunction{Twist2GluonOperator}[p,\{\mu \},\{\nu \}]
}

\Subsubsection*{Unpolarized case, non-zero-momentum insertion}

\dispSFinmath{
\Mfunction{O}_{\mu \VeryThinSpace \nu }^{\Mvariable{G2}}\Mfunction{(}p)
}

\dispSFoutmath{
\Muserfunction{Twist2GluonOperator}[p,\{\mu \},\{\nu \},\Mvariable{Explicit}\rightarrow \Mvariable{True}]
}

2-gluon legs

\dispSFinmath{
({g^{\mu \nu }}\multsp {{\Delta \cdot p}^2}-({p^{\mu }}\multsp {{\Delta }^{\nu }}+{{\Delta }^{\mu }}\multsp {p^{\nu }})\multsp
      \Delta \cdot p+{{\Delta }^{\mu }}\multsp {{\Delta }^{\nu }}\multsp {p^2})\multsp {{(\Delta \cdot p)}^{m-2}}
}

\dispSFoutmath{
\Muserfunction{Twist2GluonOperator}[\{p,\mu ,a\},\{q,\nu ,b\},\{r,\rho ,c\}]
}

3-gluon legs

\dispSFinmath{
-\frac{1}{2}\multsp \ImaginaryI \multsp {g_s}\multsp ({{(-1)}^m}+1)\multsp {f_{abc}}\multsp
   \big(\Mfunction{O}_{\nu \VeryThinSpace \rho \VeryThinSpace \mu }^{\Mvariable{G3}}\Mfunction{(}q,r,p)\big)
}

\dispSFinmath{
\Muserfunction{Twist2GluonOperator}[\{p,\mu \},\{q,\nu \},\{r,\rho \}]
}

\dispSFoutmath{
\Mfunction{O}_{\nu \VeryThinSpace \rho \VeryThinSpace \mu }^{\Mvariable{G3}}\Mfunction{(}q,r,p)
}

\dispSFinmath{
\Muserfunction{Twist2GluonOperator}[\{p,\mu \},\{q,\nu \},\{r,\rho \},\Mvariable{Explicit}\rightarrow \Mvariable{True}]
}

\dispSFoutmath{
\MathBegin{MathArray}{l}
\bigg(\sum _{i=0}^{m-3}{{(-(\Delta \cdot p))}^i}\multsp {{(\Delta \cdot q)}^{-i+m-3}}\bigg)\multsp
    {{\Delta }^{\rho }}\multsp ({q^{\mu }}\multsp {{\Delta }^{\nu }}\multsp \Delta \cdot p-
      {g^{\mu \nu }}\multsp \Delta \cdot q\multsp \Delta \cdot p+{{\Delta }^{\mu }}\multsp {p^{\nu }}\multsp \Delta \cdot q-
      {{\Delta }^{\mu }}\multsp {{\Delta }^{\nu }}\multsp p\cdot q)+  \\
\noalign{\vspace{1.5625ex}}
\hspace{1.em} \bigg(
     \sum _{i=0}^{m-3}{{(\Delta \cdot p)}^{-i+m-3}}\multsp {{(-(\Delta \cdot r))}^i}\bigg)\multsp {{\Delta }^{\nu }}\multsp
    ({r^{\mu }}\multsp {{\Delta }^{\rho }}\multsp \Delta \cdot p-{g^{\mu \rho }}\multsp \Delta \cdot r\multsp \Delta \cdot p+
      {{\Delta }^{\mu }}\multsp {p^{\rho }}\multsp \Delta \cdot r-{{\Delta }^{\mu }}\multsp {{\Delta }^{\rho }}\multsp p\cdot r)+  \\
   \noalign{\vspace{1.5625ex}}
\hspace{1.em} \bigg(\sum _{i=0}^{m-3}{{(-(\Delta \cdot q))}^i}\multsp {{(\Delta \cdot r)}^{-i+m-3}}\bigg)
    \multsp {{\Delta }^{\mu }}\multsp ({r^{\nu }}\multsp {{\Delta }^{\rho }}\multsp \Delta \cdot q-
      {g^{\nu \rho }}\multsp \Delta \cdot r\multsp \Delta \cdot q+{{\Delta }^{\nu }}\multsp {q^{\rho }}\multsp \Delta \cdot r-
      {{\Delta }^{\nu }}\multsp {{\Delta }^{\rho }}\multsp q\cdot r)+  \\
\noalign{\vspace{1.10417ex}}
\hspace{1.em} (
     {{\Delta }^{\mu }}\multsp ({p^{\nu }}\multsp {{\Delta }^{\rho }}-{{\Delta }^{\nu }}\multsp {p^{\rho }})+
      ({g^{\mu \rho }}\multsp {{\Delta }^{\nu }}-{g^{\mu \nu }}\multsp {{\Delta }^{\rho }})\multsp \Delta \cdot p)\multsp
    {{(\Delta \cdot p)}^{m-2}}+  \\
\noalign{\vspace{0.604167ex}}
\hspace{1.em} (
     {{\Delta }^{\nu }}\multsp ({{\Delta }^{\mu }}\multsp {q^{\rho }}-{q^{\mu }}\multsp {{\Delta }^{\rho }})+
      ({g^{\mu \nu }}\multsp {{\Delta }^{\rho }}-{{\Delta }^{\mu }}\multsp {g^{\nu \rho }})\multsp \Delta \cdot q)\multsp
    {{(\Delta \cdot q)}^{m-2}}+(({r^{\mu }}\multsp {{\Delta }^{\nu }}-{{\Delta }^{\mu }}\multsp {r^{\nu }})\multsp {{\Delta }^{\rho }}+
      ({{\Delta }^{\mu }}\multsp {g^{\nu \rho }}-{g^{\mu \rho }}\multsp {{\Delta }^{\nu }})\multsp \Delta \cdot r)\multsp
    {{(\Delta \cdot r)}^{m-2}}\\
\MathEnd{MathArray}
}

4-gluon legs

\dispSFinmath{
\Muserfunction{Twist2GluonOperator}[\{p,\mu ,a\},\{q,\nu ,b\},\{r,\rho ,c\},\{s,\sigma ,d\}]
}

\dispSFoutmath{
\MathBegin{MathArray}{l}
\frac{1}{2}\multsp (1+{{(-1)}^m})\multsp g_{s}^{2}\multsp   \\
\noalign{\vspace{1.01042ex}}
\hspace{1.em} \big(
   -{f_{ab\Mvariable{c136}}}\multsp {f_{c\Mvariable{c136}d}}\multsp
     \big(\Mfunction{O}_{\mu \VeryThinSpace \nu \VeryThinSpace \rho \VeryThinSpace \sigma }^{\Mvariable{G4}}\Mfunction{(}p,q,r,s)\big)-
    {f_{ac\Mvariable{c136}}}\multsp {f_{b\Mvariable{c136}d}}\multsp
     \big(\Mfunction{O}_{\mu \VeryThinSpace \rho \VeryThinSpace \nu \VeryThinSpace \sigma }^{\Mvariable{G4}}\Mfunction{(}p,r,q,s)\big)+
    {f_{a\Mvariable{c136}d}}\multsp {f_{bc\Mvariable{c136}}}\multsp
     \big(\Mfunction{O}_{\rho \VeryThinSpace \nu \VeryThinSpace \mu \VeryThinSpace \sigma }^{\Mvariable{G4}}\Mfunction{(}r,q,p,s)\big)
     \big)\\
\MathEnd{MathArray}
}

Suppress the lengthy output.

\dispSFinmath{
\Muserfunction{Twist2GluonOperator}[\{p,\mu \},\{q,\nu \},\{r,\rho \},\{s,\sigma \}]
}

\dispSFoutmath{
\Mfunction{O}_{\mu \VeryThinSpace \nu \VeryThinSpace \rho \VeryThinSpace \sigma }^{\Mvariable{G4}}\Mfunction{(}p,q,r,s)
}

The setting Explicit \(\rightarrow \) All performs the sums.

\dispSFinmath{
\Muserfunction{Twist2GluonOperator}[\{p,\mu \},\{q,\nu \},\{r,\rho \},\{s,\sigma \},\Mvariable{Explicit}\rightarrow \Mvariable{True}]
}

\dispSFinmath{
\MathBegin{MathArray}{l}
({{\Delta }^{\mu }}\multsp {g^{\nu \sigma }}\multsp {{\Delta }^{\rho }}-
      {g^{\mu \sigma }}\multsp {{\Delta }^{\nu }}\multsp {{\Delta }^{\rho }}-
      {{\Delta }^{\mu }}\multsp {g^{\nu \rho }}\multsp {{\Delta }^{\sigma }}+
      {g^{\mu \rho }}\multsp {{\Delta }^{\nu }}\multsp {{\Delta }^{\sigma }})\multsp {{(\Delta \cdot r+\Delta \cdot s)}^{m-2}}+  \\
   \noalign{\vspace{1.0625ex}}
\hspace{1.em} \bigg(\sum _{i=0}^{m-3}
      {{(-(\Delta \cdot p))}^i}\multsp {{(\Delta \cdot r+\Delta \cdot s)}^{-i+m-3}}\bigg)\multsp {{\Delta }^{\nu }}\multsp
    ({{\Delta }^{\mu }}\multsp {p^{\rho }}\multsp {{\Delta }^{\sigma }}-
      {g^{\mu \rho }}\multsp \Delta \cdot p\multsp {{\Delta }^{\sigma }}-
      {{\Delta }^{\mu }}\multsp {{\Delta }^{\rho }}\multsp {p^{\sigma }}+
      {g^{\mu \sigma }}\multsp {{\Delta }^{\rho }}\multsp \Delta \cdot p)-  \\
\noalign{\vspace{1.5625ex}}
\hspace{1.em} \bigg(
     \sum _{i=0}^{m-3}{{(-(\Delta \cdot q))}^i}\multsp {{(\Delta \cdot r+\Delta \cdot s)}^{-i+m-3}}\bigg)\multsp {{\Delta }^{\mu }}
    \multsp ({{\Delta }^{\nu }}\multsp {q^{\rho }}\multsp {{\Delta }^{\sigma }}-
      {g^{\nu \rho }}\multsp \Delta \cdot q\multsp {{\Delta }^{\sigma }}-
      {{\Delta }^{\nu }}\multsp {{\Delta }^{\rho }}\multsp {q^{\sigma }}+
      {g^{\nu \sigma }}\multsp {{\Delta }^{\rho }}\multsp \Delta \cdot q)+  \\
\noalign{\vspace{1.5625ex}}
\hspace{1.em} \bigg(
     \sum _{i=0}^{m-3}{{(\Delta \cdot p+\Delta \cdot q)}^{-i+m-3}}\multsp {{(-(\Delta \cdot r))}^i}\bigg)\multsp {{\Delta }^{\sigma }}
    \multsp ({r^{\mu }}\multsp {{\Delta }^{\nu }}\multsp {{\Delta }^{\rho }}-
      {{\Delta }^{\mu }}\multsp {r^{\nu }}\multsp {{\Delta }^{\rho }}+{{\Delta }^{\mu }}\multsp {g^{\nu \rho }}\multsp \Delta \cdot r-
      {g^{\mu \rho }}\multsp {{\Delta }^{\nu }}\multsp \Delta \cdot r)+  \\
\noalign{\vspace{1.5625ex}}
\hspace{1.em} \bigg(
     \sum _{i=0}^{m-3}{{(-(\Delta \cdot p)-\Delta \cdot q)}^i}\multsp {{(\Delta \cdot s)}^{-i+m-3}}\bigg)\multsp {{\Delta }^{\rho }}
    \multsp ({s^{\mu }}\multsp {{\Delta }^{\nu }}\multsp {{\Delta }^{\sigma }}-
      {{\Delta }^{\mu }}\multsp {s^{\nu }}\multsp {{\Delta }^{\sigma }}+{{\Delta }^{\mu }}\multsp {g^{\nu \sigma }}\multsp \Delta \cdot s
      -{g^{\mu \sigma }}\multsp {{\Delta }^{\nu }}\multsp \Delta \cdot s)-  \\
\noalign{\vspace{1.66667ex}}
\hspace{1.em} \Bigg(
    \sum _{j=0}^{m-4}\multsp (j+1){{(\Delta \cdot p)}^{-j+m-4}}\multsp {{(-(\Delta \cdot r))}^i}\multsp
      {{(-(\Delta \cdot r)-\Delta \cdot s)}^{j-i}}\Bigg)\multsp   \\
\noalign{\vspace{1.17708ex}}
\hspace{2.em} {{\Delta }^{\nu }}
    \multsp {{\Delta }^{\sigma }}\multsp (-{r^{\mu }}\multsp {{\Delta }^{\rho }}\multsp \Delta \cdot p+
      {g^{\mu \rho }}\multsp \Delta \cdot r\multsp \Delta \cdot p-{{\Delta }^{\mu }}\multsp {p^{\rho }}\multsp \Delta \cdot r+
      {{\Delta }^{\mu }}\multsp {{\Delta }^{\rho }}\multsp p\cdot r)+  \\
\noalign{\vspace{1.16667ex}}
\hspace{1.em} \Bigg(
    \sum _{j=0}^{m-4}\multsp (j+1){{(\Delta \cdot p)}^{-j+m-4}}\multsp {{(-(\Delta \cdot r)-\Delta \cdot s)}^{j-i}}\multsp
      {{(-(\Delta \cdot s))}^i}\Bigg)\multsp {{\Delta }^{\nu }}\multsp {{\Delta }^{\rho }}\multsp   \\
\noalign{\vspace{1.17708ex}}
   \hspace{2.em} (-{s^{\mu }}\multsp {{\Delta }^{\sigma }}\multsp \Delta \cdot p+
      {g^{\mu \sigma }}\multsp \Delta \cdot s\multsp \Delta \cdot p-{{\Delta }^{\mu }}\multsp {p^{\sigma }}\multsp \Delta \cdot s+
      {{\Delta }^{\mu }}\multsp {{\Delta }^{\sigma }}\multsp p\cdot s)+  \\
\noalign{\vspace{1.16667ex}}
\hspace{1.em} \Bigg(
    \sum _{j=0}^{m-4}\multsp (j+1){{(\Delta \cdot q)}^{-j+m-4}}\multsp {{(-(\Delta \cdot r))}^i}\multsp
      {{(-(\Delta \cdot r)-\Delta \cdot s)}^{j-i}}\Bigg)\multsp {{\Delta }^{\mu }}\multsp {{\Delta }^{\sigma }}\multsp   \\
   \noalign{\vspace{1.67708ex}}
\hspace{2.em} (-{r^{\nu }}\multsp {{\Delta }^{\rho }}\multsp \Delta \cdot q+
      {g^{\nu \rho }}\multsp \Delta \cdot r\multsp \Delta \cdot q-{{\Delta }^{\nu }}\multsp {q^{\rho }}\multsp \Delta \cdot r+
      {{\Delta }^{\nu }}\multsp {{\Delta }^{\rho }}\multsp q\cdot r)-
   \Bigg(\sum _{j=0}^{m-4}\multsp (j+1){{(\Delta \cdot q)}^{-j+m-4}}\multsp {{(-(\Delta \cdot r)-\Delta \cdot s)}^{j-i}}\multsp
       {{(-(\Delta \cdot s))}^i}\Bigg)\multsp   \\
\noalign{\vspace{1.17708ex}}
\hspace{2.em} {{\Delta }^{\mu }}\multsp
   {{\Delta }^{\rho }}\multsp (-{s^{\nu }}\multsp {{\Delta }^{\sigma }}\multsp \Delta \cdot q+
     {g^{\nu \sigma }}\multsp \Delta \cdot s\multsp \Delta \cdot q-{{\Delta }^{\nu }}\multsp {q^{\sigma }}\multsp \Delta \cdot s+
     {{\Delta }^{\nu }}\multsp {{\Delta }^{\sigma }}\multsp q\cdot s)\\
\MathEnd{MathArray}
}

\dispSFoutmath{
\Mfunction{SetOptions}[\Mvariable{Twist2GluonOperator},\multsp \Mvariable{Polarization}\multsp \rightarrow \multsp 1,\multsp
     \Mvariable{Explicit}\rightarrow \Mvariable{All}];
}

\Subsection*{Twist2QuarkOperator}

\Subsubsection*{Description}

Twist2QuarkOperator[p] or Twist2QuarkOperator[p,\_{},\_{}] yields the quark-antiquark operator (p is momentum in the direction of the
  incoming quark). Twist2QuarkOperator[\{p,q\}] yields the 2-quark operator for non-zero momentum insertion (p is momentum in the
  direction of the incoming quark). Twist2QuarkOperator[\{p1,\_{}\_{}\_{}\}, \{p2,\_{}\_{}\_{}\}, \{p3, mu, a\}] or
  Twist2QuarkOperator[p1,\_{},\_{}, p2,\_{},\_{}, p3,mu,a] is the quark-antiquark-gluon operator, where p1 is the incoming quark, p2 the
  incoming antiquark and p3 denotes the incoming gluon momentum. Twist2QuarkOperator[\{p1\}, \{p2\}, \{p3, mu, a\}, \{p4, nu, b\}] or
  Twist2QuarkOperator[\{p1,\_{}\_{}\_{}\}, \{p2,\_{}\_{}\_{}\}, \{p3, mu, a\}, \{p4, nu, b\}] or Twist2QuarkOperator[p1,\_{},\_{},
  p2,\_{},\_{}, p3,mu,a, p4, nu, b] gives the Quark-Quark-Gluon-Gluon-operator. The setting of the option Polarization (unpolarized: 0;
  polarized: 1) determines whether the unpolarized or polarized operator is returned

See also:  Twist2GluonOperator.

\Subsubsection*{Examples}

\Subsubsection*{Polarized case, zero-momentum insertion}

\dispSFinmath{
\Muserfunction{Twist2GluonOperator}[\{p,\mu ,a\},\{q,\nu ,b\},\{r,\rho ,c\}];
}

\dispSFoutmath{
\Mfunction{SetOptions}[\Mvariable{Twist2GluonOperator},\multsp \Mvariable{Polarization}\multsp \rightarrow \multsp 1,
    \Mvariable{ZeroMomentumInsertion}\rightarrow \Mvariable{False}]
}

Quark-antiquark operator.

\dispSFinmath{
\{\Mvariable{CouplingConstant}\rightarrow {g_s},\Mvariable{Dimension}\rightarrow D,\Mvariable{Polarization}\rightarrow 1,
    \Mvariable{Explicit}\rightarrow \Mvariable{All},\Mvariable{ZeroMomentumInsertion}\rightarrow \Mvariable{False}\}
}

\dispSFoutmath{
\Muserfunction{Twist2GluonOperator}[\{p,q\},\{\mu ,a\},\{\nu ,b\}]
}

Quark-antiquark-gluon operator.

\dispSFinmath{
\frac{1}{2}\multsp \ImaginaryI \multsp (1-{{(-1)}^m})\multsp
   (({{\epsilon }^{\nu \Delta pq}}\multsp {{\Delta }^{\mu }}-{{\epsilon }^{\mu \nu \Delta q}}\multsp \Delta \cdot p)\multsp
      {{(\Delta \cdot p)}^{m-2}}+({{\epsilon }^{\mu \nu \Delta p}}\multsp \Delta \cdot q-
        {{\epsilon }^{\mu \Delta pq}}\multsp {{\Delta }^{\nu }})\multsp {{(\Delta \cdot q)}^{m-2}})\multsp {{\delta }_{ab}}
}

\dispSFoutmath{
\Muserfunction{Twist2GluonOperator}[\{p,\mu ,a\},\{q,\nu ,b\},\{r,\rho ,c\}]
}

Quark-antiquark-gluon-gluon operator.

\dispSFinmath{
\MathBegin{MathArray}[p]{l}
{g_s}\multsp \bigg(-{{\epsilon }^{\nu \rho \Delta p}}\multsp {{\Delta }^{\mu }}\multsp
      {{(\Delta \cdot p)}^{m-2}}+{{\epsilon }^{\mu \rho \Delta q}}\multsp {{\Delta }^{\nu }}\multsp {{(\Delta \cdot q)}^{m-2}}-
     {{\epsilon }^{\mu \nu \Delta r}}\multsp {{\Delta }^{\rho }}\multsp {{(\Delta \cdot r)}^{m-2}}-  \\
\noalign{\vspace{1.5625ex}}
   \hspace{3.em} {{\epsilon }^{\rho \Delta pr}}\multsp
    \bigg(\sum _{i=0}^{m-3}{{(\Delta \cdot p)}^{-i+m-3}}\multsp {{(\Delta \cdot p+\Delta \cdot q)}^i}\bigg)\multsp {{\Delta }^{\mu }}
    \multsp {{\Delta }^{\nu }}+{{\epsilon }^{\rho \Delta qr}}\multsp
    \bigg(\sum _{i=0}^{m-3}{{(\Delta \cdot q)}^{-i+m-3}}\multsp {{(\Delta \cdot p+\Delta \cdot q)}^i}\bigg)\multsp {{\Delta }^{\mu }}
    \multsp {{\Delta }^{\nu }}+  \\
\noalign{\vspace{1.5625ex}}
\hspace{3.em} {{\epsilon }^{\nu \Delta pq}}\multsp
    \bigg(\sum _{i=0}^{m-3}{{(\Delta \cdot p)}^{-i+m-3}}\multsp {{(\Delta \cdot p+\Delta \cdot r)}^i}\bigg)\multsp {{\Delta }^{\mu }}
    \multsp {{\Delta }^{\rho }}+{{\epsilon }^{\nu \Delta qr}}\multsp
    \bigg(\sum _{i=0}^{m-3}{{(\Delta \cdot r)}^{-i+m-3}}\multsp {{(\Delta \cdot p+\Delta \cdot r)}^i}\bigg)\multsp {{\Delta }^{\mu }}
    \multsp {{\Delta }^{\rho }}+  \\
\noalign{\vspace{1.5625ex}}
\hspace{3.em} {{\epsilon }^{\mu \Delta pq}}\multsp
    \bigg(\sum _{i=0}^{m-3}{{(\Delta \cdot q)}^{-i+m-3}}\multsp {{(\Delta \cdot q+\Delta \cdot r)}^i}\bigg)\multsp {{\Delta }^{\nu }}
    \multsp {{\Delta }^{\rho }}-{{\epsilon }^{\mu \Delta pr}}\multsp
    \bigg(\sum _{i=0}^{m-3}{{(\Delta \cdot r)}^{-i+m-3}}\multsp {{(\Delta \cdot q+\Delta \cdot r)}^i}\bigg)\multsp {{\Delta }^{\nu }}
    \multsp {{\Delta }^{\rho }}+  \\
\noalign{\vspace{1.5625ex}}
\hspace{3.em} \frac{1}{{{(\Delta \cdot p+\Delta \cdot q)}^2}}
   \bigg({{\epsilon }^{\mu \rho \Delta r}}\multsp {{\Delta }^{\nu }}\multsp
     \bigg({{(\Delta \cdot p+\Delta \cdot q)}^m}+\bigg(
         \sum _{i=0}^{m-3}{{(\Delta \cdot p)}^{-i+m-3}}\multsp {{(\Delta \cdot p+\Delta \cdot q)}^i}\bigg)\multsp {{\Delta \cdot p}^3}+
   \\
\noalign{\vspace{1.5625ex}}
\hspace{8.em} \bigg(\sum _{i=0}^{m-3}
           {{(\Delta \cdot p)}^{-i+m-3}}\multsp {{(\Delta \cdot p+\Delta \cdot q)}^i}\bigg)\multsp \Delta \cdot p\multsp
         {{\Delta \cdot q}^2}+2\multsp \bigg(\sum _{i=0}^{m-3}{{(\Delta \cdot p)}^{-i+m-3}}\multsp {{(\Delta \cdot p+\Delta \cdot q)}^i}
          \bigg)\multsp {{\Delta \cdot p}^2}\multsp \Delta \cdot q\bigg)\bigg)-  \\
\noalign{\vspace{1.5625ex}}
\hspace{3.em} \frac{1}
    {{{(\Delta \cdot p+\Delta \cdot q)}^2}}\bigg({{\epsilon }^{\nu \rho \Delta r}}\multsp {{\Delta }^{\mu }}\multsp
     \bigg({{(\Delta \cdot p+\Delta \cdot q)}^m}+\bigg(
         \sum _{i=0}^{m-3}{{(\Delta \cdot q)}^{-i+m-3}}\multsp {{(\Delta \cdot p+\Delta \cdot q)}^i}\bigg)\multsp {{\Delta \cdot q}^3}+
   \\
\noalign{\vspace{1.5625ex}}
\hspace{8.em} 2\multsp
         \bigg(\sum _{i=0}^{m-3}{{(\Delta \cdot q)}^{-i+m-3}}\multsp {{(\Delta \cdot p+\Delta \cdot q)}^i}\bigg)\multsp \Delta \cdot p
         \multsp {{\Delta \cdot q}^2}+\bigg(\sum _{i=0}^{m-3}{{(\Delta \cdot q)}^{-i+m-3}}\multsp {{(\Delta \cdot p+\Delta \cdot q)}^i}
          \bigg)\multsp {{\Delta \cdot p}^2}\multsp \Delta \cdot q\bigg)\bigg)-  \\
\noalign{\vspace{1.5625ex}}
\hspace{3.em} \frac{1}
    {{{(\Delta \cdot p+\Delta \cdot r)}^2}}\bigg({{\epsilon }^{\mu \nu \Delta q}}\multsp {{\Delta }^{\rho }}\multsp
     \bigg({{(\Delta \cdot p+\Delta \cdot r)}^m}+\bigg(
         \sum _{i=0}^{m-3}{{(\Delta \cdot p)}^{-i+m-3}}\multsp {{(\Delta \cdot p+\Delta \cdot r)}^i}\bigg)\multsp {{\Delta \cdot p}^3}+
   \\
\noalign{\vspace{1.5625ex}}
\hspace{8.em} \bigg(\sum _{i=0}^{m-3}
           {{(\Delta \cdot p)}^{-i+m-3}}\multsp {{(\Delta \cdot p+\Delta \cdot r)}^i}\bigg)\multsp \Delta \cdot p\multsp
         {{\Delta \cdot r}^2}+2\multsp \bigg(\sum _{i=0}^{m-3}{{(\Delta \cdot p)}^{-i+m-3}}\multsp {{(\Delta \cdot p+\Delta \cdot r)}^i}
          \bigg)\multsp {{\Delta \cdot p}^2}\multsp \Delta \cdot r\bigg)\bigg)-  \\
\noalign{\vspace{1.5625ex}}
\hspace{3.em} \frac{1}
    {{{(\Delta \cdot p+\Delta \cdot r)}^2}}\bigg({{\epsilon }^{\nu \rho \Delta q}}\multsp {{\Delta }^{\mu }}\multsp
     \bigg({{(\Delta \cdot p+\Delta \cdot r)}^m}+\bigg(
         \sum _{i=0}^{m-3}{{(\Delta \cdot r)}^{-i+m-3}}\multsp {{(\Delta \cdot p+\Delta \cdot r)}^i}\bigg)\multsp {{\Delta \cdot r}^3}+
   \\
\noalign{\vspace{1.5625ex}}
\hspace{8.em} 2\multsp
         \bigg(\sum _{i=0}^{m-3}{{(\Delta \cdot r)}^{-i+m-3}}\multsp {{(\Delta \cdot p+\Delta \cdot r)}^i}\bigg)\multsp \Delta \cdot p
         \multsp {{\Delta \cdot r}^2}+\bigg(\sum _{i=0}^{m-3}{{(\Delta \cdot r)}^{-i+m-3}}\multsp {{(\Delta \cdot p+\Delta \cdot r)}^i}
          \bigg)\multsp {{\Delta \cdot p}^2}\multsp \Delta \cdot r\bigg)\bigg)-  \\
\noalign{\vspace{1.5625ex}}
\hspace{3.em} \frac{1}
    {{{(\Delta \cdot q+\Delta \cdot r)}^2}}\bigg({{\epsilon }^{\mu \nu \Delta p}}\multsp {{\Delta }^{\rho }}\multsp
     \bigg({{(\Delta \cdot q+\Delta \cdot r)}^m}+\bigg(
         \sum _{i=0}^{m-3}{{(\Delta \cdot q)}^{-i+m-3}}\multsp {{(\Delta \cdot q+\Delta \cdot r)}^i}\bigg)\multsp {{\Delta \cdot q}^3}+
   \\
\noalign{\vspace{1.5625ex}}
\hspace{8.em} \bigg(\sum _{i=0}^{m-3}
           {{(\Delta \cdot q)}^{-i+m-3}}\multsp {{(\Delta \cdot q+\Delta \cdot r)}^i}\bigg)\multsp \Delta \cdot q\multsp
         {{\Delta \cdot r}^2}+2\multsp \bigg(\sum _{i=0}^{m-3}{{(\Delta \cdot q)}^{-i+m-3}}\multsp {{(\Delta \cdot q+\Delta \cdot r)}^i}
          \bigg)\multsp {{\Delta \cdot q}^2}\multsp \Delta \cdot r\bigg)\bigg)+  \\
\noalign{\vspace{1.5625ex}}
\hspace{3.em} \frac{1}
    {{{(\Delta \cdot q+\Delta \cdot r)}^2}}\bigg({{\epsilon }^{\mu \rho \Delta p}}\multsp {{\Delta }^{\nu }}\multsp
     \bigg({{(\Delta \cdot q+\Delta \cdot r)}^m}+\bigg(
         \sum _{i=0}^{m-3}{{(\Delta \cdot r)}^{-i+m-3}}\multsp {{(\Delta \cdot q+\Delta \cdot r)}^i}\bigg)\multsp {{\Delta \cdot r}^3}+
   \\
\noalign{\vspace{1.5625ex}}
\hspace{8.em} 2\multsp
         \bigg(\sum _{i=0}^{m-3}{{(\Delta \cdot r)}^{-i+m-3}}\multsp {{(\Delta \cdot q+\Delta \cdot r)}^i}\bigg)\multsp \Delta \cdot q
         \multsp {{\Delta \cdot r}^2}+\bigg(\sum _{i=0}^{m-3}{{(\Delta \cdot r)}^{-i+m-3}}\multsp {{(\Delta \cdot q+\Delta \cdot r)}^i}
          \bigg)\multsp {{\Delta \cdot q}^2}\multsp \Delta \cdot r\bigg)\bigg)-  \\
\noalign{\vspace{1.58333ex}}
\hspace{3.em} \frac
        {{{\epsilon }^{\mu \nu \rho \Delta }}\multsp (\Delta \cdot q\multsp \Delta \cdot r\multsp {{(\Delta \cdot p)}^m}+
          \Delta \cdot q\multsp {{(\Delta \cdot r)}^m}\multsp \Delta \cdot p+
          {{(\Delta \cdot q)}^m}\multsp \Delta \cdot r\multsp \Delta \cdot p)}{\Delta \cdot q\multsp \Delta \cdot r\multsp
        \Delta \cdot p}\bigg)\multsp {f_{abc}}\\
\MathEnd{MathArray}
}

\dispSFoutmath{
\Muserfunction{Twist2GluonOperator}[\{p,\mu ,a\},\{q,\nu ,b\},\{r,\rho ,c\},\{s,\sigma ,d\},
    \Mvariable{Explicit}\rightarrow \Mvariable{False}]
}

\Subsubsection*{Unpolarized case, zero-momentum insertion}

\dispSFinmath{
\MathBegin{MathArray}{l}
\ImaginaryI \multsp (1-{{(-1)}^m})\multsp g_{s}^{2}\multsp   \\
\noalign{\vspace{0.604167ex}}
   \hspace{1.em} \big(-{f_{ab\Mvariable{c137}}}\multsp {f_{c\Mvariable{c137}d}}\multsp
     \big(\Mfunction{O}_{\mu \VeryThinSpace \nu \VeryThinSpace \rho \VeryThinSpace \sigma }^{\Mvariable{G4}}\Mfunction{(}p,q,r,s)\big)-
    {f_{ac\Mvariable{c137}}}\multsp {f_{b\Mvariable{c137}d}}\multsp
     \big(\Mfunction{O}_{\mu \VeryThinSpace \rho \VeryThinSpace \nu \VeryThinSpace \sigma }^{\Mvariable{G4}}\Mfunction{(}p,r,q,s)\big)+
    {f_{a\Mvariable{c137}d}}\multsp {f_{bc\Mvariable{c137}}}\multsp
     \big(\Mfunction{O}_{\rho \VeryThinSpace \nu \VeryThinSpace \mu \VeryThinSpace \sigma }^{\Mvariable{G4}}\Mfunction{(}r,q,p,s)\big)
     \big)\\
\MathEnd{MathArray}
}

\dispSFoutmath{
\Mfunction{SetOptions}[\Mvariable{Twist2GluonOperator},\multsp \Mvariable{Polarization}\multsp \rightarrow \multsp 0,
    \Mvariable{ZeroMomentumInsertion}\rightarrow \Mvariable{False}]
}

Quark-antiquark operator.

\dispSFinmath{
\{\Mvariable{CouplingConstant}\rightarrow {g_s},\Mvariable{Dimension}\rightarrow D,\Mvariable{Polarization}\rightarrow 0,
    \Mvariable{Explicit}\rightarrow \Mvariable{All},\Mvariable{ZeroMomentumInsertion}\rightarrow \Mvariable{False}\}
}

\dispSFoutmath{
\Muserfunction{Twist2GluonOperator}[\{p,q\},\{\mu ,a\},\{\nu ,b\}]
}

Quark-antiquark-gluon operator.

\dispSFinmath{
-\frac{1}{2}\multsp (1+{{(-1)}^m})\multsp ({q^{\mu }}\multsp {{\Delta }^{\nu }}\multsp \Delta \cdot p-
     {g^{\mu \nu }}\multsp \Delta \cdot q\multsp \Delta \cdot p+
     {{\Delta }^{\mu }}\multsp ({p^{\nu }}\multsp \Delta \cdot q-{{\Delta }^{\nu }}\multsp p\cdot q))\multsp
   ({{(\Delta \cdot p)}^{m-2}}+{{(\Delta \cdot q)}^{m-2}})\multsp {{\delta }_{ab}}
}

\dispSFoutmath{
\Muserfunction{Twist2GluonOperator}[\{p,\mu ,a\},\{q,\nu ,b\},\{r,\rho ,c\}];
}

Quark-antiquark-gluon-gluon operator.

\dispSFinmath{
\Muserfunction{Twist2GluonOperator}[\{p,\mu \},\{q,\nu \},\{r,\rho \}]
}

\dispSFoutmath{
-\frac{1}{2}\multsp (1+{{(-1)}^m})\multsp ({{\mu }^q}\multsp {{\Delta }^r}\multsp \Delta \cdot p-
     {g^{qr}}\multsp \Delta \cdot \mu \multsp \Delta \cdot p+
     {{\Delta }^q}\multsp ({p^r}\multsp \Delta \cdot \mu -{{\Delta }^r}\multsp p\cdot \mu ))\multsp
   ({{(\Delta \cdot p)}^{m-2}}+{{(\Delta \cdot \mu )}^{m-2}})\multsp {{\delta }_{\nu \rho }}
}

This shows the FeynCalcExternal (FCE) form.

\dispSFinmath{
\Muserfunction{Twist2GluonOperator}[\{p,\mu \},\{q,\nu \},\{r,\rho \},\Mvariable{Explicit}\rightarrow \Mvariable{True}]
}

\dispSFoutmath{
-\frac{1}{2}\multsp (1+{{(-1)}^m})\multsp ({{\mu }^q}\multsp {{\Delta }^r}\multsp \Delta \cdot p-
     {g^{qr}}\multsp \Delta \cdot \mu \multsp \Delta \cdot p+
     {{\Delta }^q}\multsp ({p^r}\multsp \Delta \cdot \mu -{{\Delta }^r}\multsp p\cdot \mu ))\multsp
   ({{(\Delta \cdot p)}^{m-2}}+{{(\Delta \cdot \mu )}^{m-2}})\multsp {{\delta }_{\nu \rho }}
}

\dispSFinmath{
\Muserfunction{Twist2GluonOperator}[\{p,\mu ,a\},\{q,\nu ,b\},\{r,\rho ,c\},\{s,\sigma ,d\},
    \Mvariable{Explicit}\rightarrow \Mvariable{False}]
}

\dispSFoutmath{
\MathBegin{MathArray}{l}
\frac{1}{2}\multsp (1+{{(-1)}^m})\multsp g_{s}^{2}\multsp   \\
\noalign{\vspace{1.01042ex}}
\hspace{1.em} \big(
   -{f_{ab\Mvariable{c138}}}\multsp {f_{c\Mvariable{c138}d}}\multsp
     \big(\Mfunction{O}_{\mu \VeryThinSpace \nu \VeryThinSpace \rho \VeryThinSpace \sigma }^{\Mvariable{G4}}\Mfunction{(}p,q,r,s)\big)-
    {f_{ac\Mvariable{c138}}}\multsp {f_{b\Mvariable{c138}d}}\multsp
     \big(\Mfunction{O}_{\mu \VeryThinSpace \rho \VeryThinSpace \nu \VeryThinSpace \sigma }^{\Mvariable{G4}}\Mfunction{(}p,r,q,s)\big)+
    {f_{a\Mvariable{c138}d}}\multsp {f_{bc\Mvariable{c138}}}\multsp
     \big(\Mfunction{O}_{\rho \VeryThinSpace \nu \VeryThinSpace \mu \VeryThinSpace \sigma }^{\Mvariable{G4}}\Mfunction{(}r,q,p,s)\big)
     \big)\\
\MathEnd{MathArray}
}

\dispSFinmath{
\Mfunction{Short}[\Muserfunction{Twist2GluonOperator}[\{p,\mu \},\{q,\nu \},\{r,\rho \},\{s,\sigma \},
      \Mvariable{Explicit}\rightarrow \Mvariable{True}],6]
}

\dispSFoutmath{
\MathBegin{MathArray}{l}
({{\Delta }^{\mu }}\multsp {g^{\nu \sigma }}\multsp {{\Delta }^{\rho }}-
      {g^{\mu \sigma }}\multsp {{\Delta }^{\nu }}\multsp {{\Delta }^{\rho }}-
      {{\Delta }^{\mu }}\multsp {g^{\nu \rho }}\multsp {{\Delta }^{\sigma }}+
      {g^{\mu \rho }}\multsp {{\Delta }^{\nu }}\multsp {{\Delta }^{\sigma }})\multsp {{(\Delta \cdot r+\Delta \cdot s)}^{m-2}}+  \\
   \noalign{\vspace{1.0625ex}}
\hspace{1.em} \bigg(\sum _{i=0}^{m-3}
      {{(-(\Delta \cdot p))}^i}\multsp {{(\Delta \cdot r+\Delta \cdot s)}^{-i+m-3}}\bigg)\multsp {{\Delta }^{\nu }}\multsp
    ({{\Delta }^{\mu }}\multsp {p^{\rho }}\multsp {{\Delta }^{\sigma }}-
      {g^{\mu \rho }}\multsp \Delta \cdot p\multsp {{\Delta }^{\sigma }}-
      {{\Delta }^{\mu }}\multsp {{\Delta }^{\rho }}\multsp {p^{\sigma }}+
      {g^{\mu \sigma }}\multsp {{\Delta }^{\rho }}\multsp \Delta \cdot p)-  \\
\noalign{\vspace{1.5625ex}}
\hspace{1.em} \bigg(
     \sum _{i=0}^{m-3}{{(-(\Delta \cdot q))}^i}\multsp {{(\Delta \cdot r+\Delta \cdot s)}^{-i+m-3}}\bigg)\multsp {{\Delta }^{\mu }}
    \multsp ({{\Delta }^{\nu }}\multsp {q^{\rho }}\multsp {{\Delta }^{\sigma }}-
      {g^{\nu \rho }}\multsp \Delta \cdot q\multsp {{\Delta }^{\sigma }}-
      {{\Delta }^{\nu }}\multsp {{\Delta }^{\rho }}\multsp {q^{\sigma }}+
      {g^{\nu \sigma }}\multsp {{\Delta }^{\rho }}\multsp \Delta \cdot q)+  \\
\noalign{\vspace{1.5625ex}}
\hspace{1.em} \bigg(
     \sum _{i=0}^{m-3}{{(\Delta \cdot p+\Delta \cdot q)}^{-i+m-3}}\multsp {{(-(\Delta \cdot r))}^i}\bigg)\multsp {{\Delta }^{\sigma }}
    \multsp ({r^{\mu }}\multsp {{\Delta }^{\nu }}\multsp {{\Delta }^{\rho }}-
      {{\Delta }^{\mu }}\multsp {r^{\nu }}\multsp {{\Delta }^{\rho }}+{{\Delta }^{\mu }}\multsp {g^{\nu \rho }}\multsp \Delta \cdot r-
      {g^{\mu \rho }}\multsp {{\Delta }^{\nu }}\multsp \Delta \cdot r)+  \\
\noalign{\vspace{1.5625ex}}
\hspace{1.em} \bigg(
     \sum _{i=0}^{m-3}{{(\LeftSkeleton 1\RightSkeleton )}^i}\multsp {{(\Delta \cdot \LeftSkeleton 1\RightSkeleton )}^{-i+m-3}}\bigg)
     \multsp {{\LeftSkeleton 1\RightSkeleton }^{\LeftSkeleton 1\RightSkeleton }}\multsp (\LeftSkeleton 1\RightSkeleton )-
   \LeftSkeleton 1\RightSkeleton +  \\
\noalign{\vspace{1.66667ex}}
\hspace{1.em} \Bigg(
    \sum _{j=0}^{m-4}\multsp (j+1){{(\Delta \cdot p)}^{-j+m-4}}\multsp {{(-(\Delta \cdot r)-\Delta \cdot s)}^{j-i}}\multsp
      {{(-(\Delta \cdot s))}^i}\Bigg)\multsp   \\
\noalign{\vspace{1.17708ex}}
\hspace{2.em} {{\Delta }^{\nu }}\multsp
    {{\Delta }^{\rho }}\multsp (-{s^{\mu }}\multsp {{\Delta }^{\sigma }}\multsp \Delta \cdot p+
      {g^{\mu \sigma }}\multsp \Delta \cdot s\multsp \Delta \cdot p-{{\Delta }^{\mu }}\multsp {p^{\sigma }}\multsp \Delta \cdot s+
      {{\Delta }^{\mu }}\multsp {{\Delta }^{\sigma }}\multsp p\cdot s)+  \\
\noalign{\vspace{1.16667ex}}
\hspace{1.em} \Bigg(
    \sum _{j=0}^{m-4}\multsp (j+1){{(\Delta \cdot q)}^{-j+m-4}}\multsp {{(-(\Delta \cdot r))}^i}\multsp
      {{(-(\Delta \cdot r)-\Delta \cdot s)}^{j-i}}\Bigg)\multsp {{\Delta }^{\mu }}\multsp {{\Delta }^{\sigma }}\multsp   \\
   \noalign{\vspace{1.67708ex}}
\hspace{2.em} (-{r^{\nu }}\multsp {{\Delta }^{\rho }}\multsp \Delta \cdot q+
      {g^{\nu \rho }}\multsp \Delta \cdot r\multsp \Delta \cdot q-{{\Delta }^{\nu }}\multsp {q^{\rho }}\multsp \Delta \cdot r+
      {{\Delta }^{\nu }}\multsp {{\Delta }^{\rho }}\multsp q\cdot r)-
   \Bigg(\sum _{j=0}^{m-4}\multsp (j+1){{(\Delta \cdot q)}^{-j+m-4}}\multsp {{(-(\Delta \cdot r)-\Delta \cdot s)}^{j-i}}\multsp
       {{(-(\Delta \cdot s))}^i}\Bigg)\multsp   \\
\noalign{\vspace{1.17708ex}}
\hspace{2.em} {{\Delta }^{\mu }}\multsp
    {{\Delta }^{\rho }}\multsp (-{s^{\nu }}\multsp {{\Delta }^{\sigma }}\multsp \Delta \cdot q+
      {g^{\nu \sigma }}\multsp \Delta \cdot s\multsp \Delta \cdot q-{{\Delta }^{\nu }}\multsp {q^{\sigma }}\multsp \Delta \cdot s+
      {{\Delta }^{\nu }}\multsp {{\Delta }^{\sigma }}\multsp q\cdot s)\\
\MathEnd{MathArray}
}

\Subsubsection*{Polarized case, non-zero momentum insertion}

\dispSFinmath{
\Mfunction{SetOptions}[\Mvariable{Twist2GluonOperator},\multsp \Mvariable{Polarization}\multsp \rightarrow \multsp 1,\multsp
     \Mvariable{Explicit}\rightarrow \Mvariable{All}];
}

\dispSFoutmath{
\Mfunction{Short}[\Muserfunction{Twist2GluonOperator}[\{p,\mu ,a\},\{q,\nu ,b\},\{r,\rho ,c\}],5]
}

With the setting ZeroMomentumInsertion \(\rightarrow \) False a non-zero momentum is assumed to flow into the operator vertex.

This is the Feynman rule associated with the quark-antiquark operator, where p is the momentum of the incoming quark and q the momentum
  of the antiquark. The momentum flowing into the operator vertex is thus -p-q.

\dispSFinmath{
\MathBegin{MathArray}{l}
{g_s}\multsp \bigg(-{{\epsilon }^{\nu \rho \Delta p}}\multsp {{\Delta }^{\mu }}\multsp
      {{(\Delta \cdot p)}^{m-2}}+\LeftSkeleton 20\RightSkeleton +  \\
\noalign{\vspace{1.5625ex}}
\hspace{3.em} \frac{1}
    {{{(\Delta \cdot q+\Delta \cdot r)}^2}}\bigg({{\epsilon }^{\mu \rho \Delta p}}\multsp {{\Delta }^{\nu }}\multsp
     \bigg({{(\Delta \cdot q+\Delta \cdot r)}^m}+\bigg(
         \sum _{i=0}^{m-3}{{(\Delta \cdot r)}^{-i+m-3}}\multsp {{(\Delta \cdot q+\Delta \cdot r)}^i}\bigg)\multsp {{\Delta \cdot r}^3}+
   \\
\noalign{\vspace{1.5625ex}}
\hspace{8.em} 2\multsp
         \bigg(\sum _{i=0}^{m-3}{{(\Delta \cdot r)}^{-i+m-3}}\multsp {{(\Delta \cdot q+\Delta \cdot r)}^i}\bigg)\multsp \Delta \cdot q
         \multsp {{\Delta \cdot r}^2}+\bigg(\sum _{i=0}^{m-3}{{(\Delta \cdot r)}^{-i+m-3}}\multsp {{(\Delta \cdot q+\Delta \cdot r)}^i}
          \bigg)\multsp {{\Delta \cdot q}^2}\multsp \Delta \cdot r\bigg)\bigg)-  \\
\noalign{\vspace{1.58333ex}}
\hspace{3.em} \frac
        {{{\epsilon }^{\mu \nu \rho \Delta }}\multsp (\Delta \cdot q\multsp \Delta \cdot r\multsp {{(\Delta \cdot p)}^m}+
          \Delta \cdot q\multsp {{(\Delta \cdot r)}^m}\multsp \Delta \cdot p+
          {{(\Delta \cdot q)}^m}\multsp \Delta \cdot r\multsp \Delta \cdot p)}{\Delta \cdot q\multsp \Delta \cdot r\multsp
        \Delta \cdot p}\bigg)\multsp {f_{abc}}\\
\MathEnd{MathArray}
}

\dispSFoutmath{
\Mfunction{SetOptions}[\Mvariable{Twist2QuarkOperator},\Mvariable{Polarization}\rightarrow 1,
    \Mvariable{ZeroMomentumInsertion}\rightarrow \Mvariable{True}]
}

This is the quark-antiquark-gluon operator vertex.

\dispSFinmath{
\{\Mvariable{CouplingConstant}\rightarrow {g_s},\Mvariable{Dimension}\rightarrow D,\Mvariable{Explicit}\rightarrow \Mvariable{True},
    \Mvariable{Polarization}\rightarrow 1,\Mvariable{ZeroMomentumInsertion}\rightarrow \Mvariable{True}\}
}

\dispSFoutmath{
\Mvariable{t1}=\Muserfunction{Twist2QuarkOperator}[p]
}

This shows the FeynCalcExternal form.

\dispSFinmath{
-(\gamma \cdot \Delta ).{{\gamma }^5}\multsp {{(\Delta \cdot p)}^{m-1}}
}

\dispSFoutmath{
\Mvariable{t2}=\Muserfunction{Twist2QuarkOperator}[\{p\},\{q\},\{k,\mu ,a\}]
}

\Subsubsection*{Unpolarized case, non-zero momentum insertion}

\dispSFinmath{
-{g_s}\multsp (\gamma \cdot \Delta ).{{\gamma }^5}.{T_a}\multsp
   \bigg(\sum _{i=0}^{m-2}{{(-1)}^i}\multsp {{(\Delta \cdot p)}^{-i+m-2}}\multsp {{(\Delta \cdot q)}^i}\bigg)\multsp {{\Delta }^{\mu }}
}

\dispSFoutmath{
\Mvariable{t3}=\Muserfunction{Twist2QuarkOperator}[\{p\},\{q\},\{k,\mu ,a\},\{r,\nu ,b\}]
}

Quark-antiquark operator.

\dispSFinmath{
\MathBegin{MathArray}{l}
-g_{s}^{2}\multsp (\gamma \cdot \Delta ).{{\gamma }^5}.
    \Bigg({T_b}.{T_a}\multsp \Bigg(\sum _{j=0}^{m-3}\multsp (j+1)
         {{(-1)}^j}\multsp {{(\Delta \cdot p)}^i}\multsp {{(\Delta \cdot q)}^{-j+m-3}}\multsp {{(\Delta \cdot p+\Delta \cdot r)}^{j-i}}
          \Bigg)-  \\
\noalign{\vspace{1.67708ex}}
\hspace{4.em} {{(-1)}^m}\multsp {T_a}.{T_b}\multsp
       \Bigg(\sum _{j=0}^{m-3}\multsp (j+1){{(-1)}^j}\multsp {{(\Delta \cdot p)}^{-j+m-3}}\multsp {{(\Delta \cdot q)}^i}\multsp
          {{(\Delta \cdot q+\Delta \cdot r)}^{j-i}}\Bigg)\Bigg)\multsp {{\Delta }^{\mu }}\multsp {{\Delta }^{\nu }}\\
\MathEnd{MathArray}
}

\dispSFoutmath{
\Mfunction{SetOptions}[\Mvariable{Twist2QuarkOperator},\multsp \Mvariable{Polarization}\rightarrow 0,
    \Mvariable{ZeroMomentumInsertion}\rightarrow \Mvariable{True}]
}

Quark-antiquark-gluon operator.

\dispSFinmath{
\{\Mvariable{CouplingConstant}\rightarrow {g_s},\Mvariable{Dimension}\rightarrow D,\Mvariable{Explicit}\rightarrow \Mvariable{True},
    \Mvariable{Polarization}\rightarrow 0,\Mvariable{ZeroMomentumInsertion}\rightarrow \Mvariable{True}\}
}

\dispSFoutmath{
\Mvariable{t4}=\Muserfunction{Twist2QuarkOperator}[p]
}

\dispSFinmath{
\gamma \cdot \Delta \multsp {{(\Delta \cdot p)}^{m-1}}
}

\Subsection*{Twist3QuarkOperator}

\Subsubsection*{Description}

Twist3QuarkOperator[p] or { }Twist3QuarkOperator[p,\_{},\_{}] { }yields the { }2-quark operator (p is momentum in the direction of the
  fermion number flow). Twist3QuarkOperator[\{p1,\_{}\_{}\_{}\}, \{p2,\_{}\_{}\_{}\}, \{p3, mu, a\}] or Twist3QuarkOperator[p1,\_{},\_{},
  { }p2,\_{},\_{}, { }p3,mu,a] Quark-Quark-Gluon-operator, where p1 is the incoming quark, p2 the incoming antiquark and p3 denotes the
  (incoming) gluon momentum. { }Twist3QuarkOperator[\{p1,\_{}\_{}\_{}\}, \{p2,\_{}\_{}\_{}\}, \{p3, mu, a\}, \{p4, nu, b\}] or
  Twist3QuarkOperator[p1,\_{},\_{}, { }p2,\_{},\_{}, { }p3,mu,a, p4, nu, b] { }gives the Quark-Quark-Gluon-Gluon-operator. The setting of
  the option Polarization (unpolarized: 0; polarized: 1) determines whether the uppolarized or polarized operator is returned.

\dispSFinmath{
\Mvariable{t5}=\Muserfunction{Twist2QuarkOperator}[\{p\},\{q\},\{k,\mu ,a\}]
}

\dispSFoutmath{
{g_s}\multsp (\gamma \cdot \Delta ).{T_a}\multsp \bigg(
    \sum _{i=0}^{m-2}{{(-1)}^i}\multsp {{(\Delta \cdot p)}^{-i+m-2}}\multsp {{(\Delta \cdot q)}^i}\bigg)\multsp {{\Delta }^{\mu }}
}

See also:  Twist2QuarkOperator, Twist2GluonOperator.

\Subsubsection*{Examples}

\dispSFinmath{
\Mvariable{t6}=\Muserfunction{Twist2QuarkOperator}[\{p\},\{q\},\{k,\mu ,a\},\{r,\nu ,b\}]
}

\dispSFoutmath{
\MathBegin{MathArray}{l}
-{{(-1)}^m}\multsp g_{s}^{2}\multsp
   (\gamma \cdot \Delta ).\bigg({T_a}.{T_b}\multsp \bigg(
        \sum _{i=0}^{m-3}\multsp (i+1){{(-(\Delta \cdot p))}^{-i+m-3}}\multsp {{(\Delta \cdot q)}^j}\multsp
          {{(k\cdot \Delta +\Delta \cdot q)}^{i-j}}\bigg)+  \\
\noalign{\vspace{1.5625ex}}
\hspace{4.em} {T_b}.{T_a}\multsp
       \bigg(\sum _{i=0}^{m-3}\multsp (i+1){{(-(\Delta \cdot p))}^{-i+m-3}}\multsp {{(\Delta \cdot q)}^j}\multsp
          {{(\Delta \cdot q+\Delta \cdot r)}^{i-j}}\bigg)\bigg)\multsp {{\Delta }^{\mu }}\multsp {{\Delta }^{\nu }}\\
\MathEnd{MathArray}
}

\Subsection*{Twist4GluonOperator}

\Subsubsection*{Description}

Twist4GluonOperator[\{oa, ob, oc, od\}, \{p1,la1,a1\}, \{p2,la2,a2\}, \{p3,la3,a3\}, \{p4,la4,a4\}].

See also:  Twist2QuarkOperator, Twist3QuarkOperator, Twist2GluonOperator.

\Subsubsection*{Examples}

\dispSFinmath{
\Muserfunction{Twist2QuarkOperator}[p]//\Muserfunction{FCE}//\Mfunction{StandardForm}
}

\dispSFoutmath{
\Muserfunction{GSD}[\Mvariable{OPEDelta}]\multsp {{\Muserfunction{SOD}[p]}^{-1+\Mvariable{OPEm}}}
}

\Subsection*{TwoLoopSimplify { }***unfinished*** (examples missing)}

\Subsubsection*{Description}

TwoLoopSimplify[amplitude,\{qu1,qu2\}] simplifies the 2-loop amplitude (qu1 and qu2 denote the integration momenta).
  TwoLoopSimplify[amplitude] transforms to TwoLoopSimplify[amplitude, \{Global`q1, Global`q2\}], i.e., the integration momenta in
  amplitude must be named q1 and q2.

See also:  OneLoopSimplify.

\Subsubsection*{Examples}

\dispSFinmath{
\Muserfunction{Twist2QuarkOperator}[\{p\},\{q\},\{k,\mu ,a\},\Mvariable{Explicit}\rightarrow \Mvariable{All},
    \Mvariable{Polarization}\rightarrow 0]
}

\Subsection*{Uncontract}

\Subsubsection*{Description}

Uncontract[exp, q1, q2, ...] uncontracts Eps and DiracGamma. Uncontract[exp, q1, q2, Pair \(\rightarrow \) \{p\}] uncontracts also p.q1
  and p.q2; the option Pair \(\rightarrow \) All uncontracts all momenta except OPEDelta.

\dispSFinmath{
-{g_s}\multsp (\gamma \cdot \Delta ).{T_a}\multsp {{\Delta }^{\mu }}\multsp
   \bigg(\frac{{{(\Delta \cdot p)}^{m-1}}}{\Delta \cdot p+\Delta \cdot q}+
     \frac{{{(-1)}^m}\multsp {{(\Delta \cdot q)}^{m-1}}}{\Delta \cdot p+\Delta \cdot q}\bigg)
}

\dispSFoutmath{
\Muserfunction{Twist2QuarkOperator}[\{p\},\{q\},\{k,\mu ,a\},\{r,\nu ,b\},\Mvariable{Explicit}\rightarrow \Mvariable{All},
    \Mvariable{Polarization}\rightarrow 0]
}

See also:  Contract.

\Subsubsection*{Examples}

\dispSFinmath{
\MathBegin{MathArray}{l}
-g_{s}^{2}\multsp (\gamma \cdot \Delta ).
    \bigg({T_a}.{T_b}\multsp \bigg(\frac{{{(-1)}^{m-1}}\multsp {{(\Delta \cdot p)}^{m-1}}}
          {(\Delta \cdot p+\Delta \cdot q)\multsp (k\cdot \Delta +\Delta \cdot p+\Delta \cdot q)}-
         \frac{{{(\Delta \cdot q)}^{m-1}}}{k\cdot \Delta \multsp (\Delta \cdot p+\Delta \cdot q)}+
         \frac{{{(k\cdot \Delta +\Delta \cdot q)}^{m-1}}}{k\cdot \Delta \multsp (k\cdot \Delta +\Delta \cdot p+\Delta \cdot q)}\bigg)+
   \\
\noalign{\vspace{1.64583ex}}
\hspace{4.em} {T_b}.{T_a}\multsp
       \bigg(\frac{{{(-1)}^{m-1}}\multsp {{(\Delta \cdot p)}^{m-1}}}
          {(\Delta \cdot p+\Delta \cdot q)\multsp (\Delta \cdot p+\Delta \cdot q+\Delta \cdot r)}-
         \frac{{{(\Delta \cdot q)}^{m-1}}}{(\Delta \cdot p+\Delta \cdot q)\multsp \Delta \cdot r}+
         \frac{{{(\Delta \cdot q+\Delta \cdot r)}^{m-1}}}{\Delta \cdot r\multsp (\Delta \cdot p+\Delta \cdot q+\Delta \cdot r)}\bigg)
       \bigg)\multsp {{\Delta }^{\mu }}\multsp {{\Delta }^{\nu }}\multsp {{(-1)}^m}\\
\MathEnd{MathArray}
}

\dispSFoutmath{
\Mfunction{SetOptions}[\Mvariable{Twist2QuarkOperator},\Mvariable{Polarization}\rightarrow 1,
    \Mvariable{ZeroMomentumInsertion}\rightarrow \Mvariable{False}]
}

\dispSFinmath{
\{\Mvariable{CouplingConstant}\rightarrow {g_s},\Mvariable{Dimension}\rightarrow D,\Mvariable{Explicit}\rightarrow \Mvariable{True},
    \Mvariable{Polarization}\rightarrow 1,\Mvariable{ZeroMomentumInsertion}\rightarrow \Mvariable{False}\}
}

\dispSFoutmath{
\Mvariable{t1}=\Muserfunction{Twist2QuarkOperator}[\{p,q\}]
}

\dispSFinmath{
{2^{1-m}}\multsp {{\gamma }^5}.(\gamma \cdot \Delta )\multsp {{(\Delta \cdot p-\Delta \cdot q)}^{m-1}}
}

\dispSFoutmath{
\Mvariable{t2}=\Muserfunction{Twist2QuarkOperator}[\{p\},\{q\},\{k,\mu ,a\}]
}

\dispSFinmath{
{2^{2-m}}\multsp {g_s}\multsp {T_a}.{{\gamma }^5}.(\gamma \cdot \Delta )\multsp
   \bigg(\sum _{i=0}^{m-2}{{(-(k\cdot \Delta )+\Delta \cdot p-\Delta \cdot q)}^{-i+m-2}}\multsp
      {{(k\cdot \Delta +\Delta \cdot p-\Delta \cdot q)}^i}\bigg)\multsp {{\Delta }^{\mu }}
}

\dispSFoutmath{
\Mfunction{StandardForm}[\Muserfunction{FCE}[\Mvariable{t2}]]
}

\dispSFinmath{
\MathBegin{MathArray}{l}
{2^{2-\Mvariable{OPEm}}}\multsp \Mvariable{Gstrong}\multsp
   \Muserfunction{SUNT}[a].\Muserfunction{GA}[5].\Muserfunction{GSD}[\Mvariable{OPEDelta}]\multsp
   \Muserfunction{FVD}[\Mvariable{OPEDelta},\mu ]\multsp   \\
\noalign{\vspace{0.572917ex}}
\hspace{1.em} \Muserfunction{OPESum}\big[
   {{(-\Muserfunction{SOD}[k]+\Muserfunction{SOD}[p]-\Muserfunction{SOD}[q])}^{-2-\Mvariable{OPEi}+\Mvariable{OPEm}}}\multsp
     {{(\Muserfunction{SOD}[k]+\Muserfunction{SOD}[p]-\Muserfunction{SOD}[q])}^{\Mvariable{OPEi}}},  \\
\noalign{\vspace{
   0.666667ex}}
\hspace{2.em} \{\Mvariable{OPEi},0,-2+\Mvariable{OPEm}\}\big]\\
\MathEnd{MathArray}
}

\dispSFoutmath{
\Mfunction{SetOptions}[\Mvariable{Twist2QuarkOperator},\Mvariable{Polarization}\rightarrow 1,
    \Mvariable{ZeroMomentumInsertion}\rightarrow \Mvariable{False}]
}

By default scalar products are not uncontracted.

\dispSFinmath{
\{\Mvariable{CouplingConstant}\rightarrow {g_s},\Mvariable{Dimension}\rightarrow D,\Mvariable{Explicit}\rightarrow \Mvariable{True},
    \Mvariable{Polarization}\rightarrow 1,\Mvariable{ZeroMomentumInsertion}\rightarrow \Mvariable{False}\}
}

\dispSFoutmath{
\Mvariable{t4}=\Muserfunction{Twist2QuarkOperator}[\{p,q\}]
}

With the option Pair\(\rightarrow \)All they are ``uncontracted ''.

\dispSFinmath{
{2^{1-m}}\multsp {{\gamma }^5}.(\gamma \cdot \Delta )\multsp {{(\Delta \cdot p-\Delta \cdot q)}^{m-1}}
}

\dispSFoutmath{
\Mvariable{t5}=\Muserfunction{Twist2QuarkOperator}[\{p\},\{q\},\{k,\mu ,a\}]
}

\dispSFinmath{
{2^{2-m}}\multsp {g_s}\multsp {T_a}.{{\gamma }^5}.(\gamma \cdot \Delta )\multsp
   \bigg(\sum _{i=0}^{m-2}{{(-(k\cdot \Delta )+\Delta \cdot p-\Delta \cdot q)}^{-i+m-2}}\multsp
      {{(k\cdot \Delta +\Delta \cdot p-\Delta \cdot q)}^i}\bigg)\multsp {{\Delta }^{\mu }}
}

\dispSFoutmath{
\Mfunction{Clear}[\Mvariable{t1},\Mvariable{t2},\Mvariable{t3},\Mvariable{t4},\Mvariable{t5},\Mvariable{t6}];
}

\dispSFinmath{
\Muserfunction{Twist3QuarkOperator}//\Mfunction{Options}
}

\Subsection*{UnDeclareNonCommutative}

\Subsubsection*{Description}

UnDeclareNonCommutative[a, b, ...] undeclares a,b, ... to be noncommutative, i.e., DataType[a,b, ..., NonCommutative] is set to False.

See also:  DataType, DeclareNonCommutative.

\Subsubsection*{Examples}

\dispSFinmath{
\{\Mvariable{CouplingConstant}\rightarrow {g_s},\Mvariable{Dimension}\rightarrow D,\Mvariable{Polarization}\rightarrow 1\}
}

As a side-effect of DeclareNonCommutative x is declared to be of DataType NonCommutative.

\dispSFinmath{
\Muserfunction{Twist3QuarkOperator}[p]
}

\dispSFoutmath{
{{(-1)}^m}\multsp (\gamma \cdot \Delta ).{{\gamma }^5}\multsp {{(\Delta \cdot p)}^{m-1}}
}

The inverse operation is UnDeclareNonCommutative.

\dispSFinmath{
\MathBegin{MathArray}{l}
\Muserfunction{Twist4GluonOperator}[
   \{\Mvariable{oa},\multsp \Mvariable{ob},\multsp \Mvariable{oc},\multsp \Mvariable{od}\},  \\
\noalign{\vspace{0.5ex}}
   \hspace{1.em} \{\Mvariable{p1},\Mvariable{la1},\Mvariable{a1}\},\{\Mvariable{p2},\Mvariable{la2},\Mvariable{a2}\},
    \{\Mvariable{p3},\Mvariable{la3},\Mvariable{a3}\},\{\Mvariable{p4},\Mvariable{la4},\Mvariable{a4}\}]\\
\MathEnd{MathArray}
}

\dispSFinmath{
\MathBegin{MathArray}[p]{l}
\big(p_{2}^{\Mvariable{la1}}\multsp {{\Delta }^{\Mvariable{la2}}}\multsp \Delta \cdot {p_1}-
     {g^{\Mvariable{la1}\Mvariable{la2}}}\multsp \Delta \cdot {p_2}\multsp \Delta \cdot {p_1}+
     {{\Delta }^{\Mvariable{la1}}}\multsp p_{1}^{\Mvariable{la2}}\multsp \Delta \cdot {p_2}-
     {{\Delta }^{\Mvariable{la1}}}\multsp {{\Delta }^{\Mvariable{la2}}}\multsp {p_1}\cdot {p_2}\big)\multsp   \\
\noalign{\vspace{
   0.760417ex}}
\hspace{2.em} \big(p_{4}^{\Mvariable{la3}}\multsp {{\Delta }^{\Mvariable{la4}}}\multsp \Delta \cdot {p_3}-
      {g^{\Mvariable{la3}\Mvariable{la4}}}\multsp \Delta \cdot {p_4}\multsp \Delta \cdot {p_3}+
      {{\Delta }^{\Mvariable{la3}}}\multsp p_{3}^{\Mvariable{la4}}\multsp \Delta \cdot {p_4}-
      {{\Delta }^{\Mvariable{la3}}}\multsp {{\Delta }^{\Mvariable{la4}}}\multsp {p_3}\cdot {p_4}\big)\multsp
    {{\delta }_{\Mvariable{a1}\Mvariable{od}}}\multsp {{\delta }_{\Mvariable{a2}\Mvariable{oc}}}\multsp
    {{\delta }_{\Mvariable{a3}\Mvariable{ob}}}\multsp {{\delta }_{\Mvariable{a4}\Mvariable{oa}}}+  \\
\noalign{\vspace{0.760417ex}}
   \hspace{1.em} \big(p_{2}^{\Mvariable{la1}}\multsp {{\Delta }^{\Mvariable{la2}}}\multsp \Delta \cdot {p_1}-
     {g^{\Mvariable{la1}\Mvariable{la2}}}\multsp \Delta \cdot {p_2}\multsp \Delta \cdot {p_1}+
     {{\Delta }^{\Mvariable{la1}}}\multsp p_{1}^{\Mvariable{la2}}\multsp \Delta \cdot {p_2}-
     {{\Delta }^{\Mvariable{la1}}}\multsp {{\Delta }^{\Mvariable{la2}}}\multsp {p_1}\cdot {p_2}\big)\multsp   \\
\noalign{\vspace{
   0.760417ex}}
\hspace{2.em} \big(p_{4}^{\Mvariable{la3}}\multsp {{\Delta }^{\Mvariable{la4}}}\multsp \Delta \cdot {p_3}-
      {g^{\Mvariable{la3}\Mvariable{la4}}}\multsp \Delta \cdot {p_4}\multsp \Delta \cdot {p_3}+
      {{\Delta }^{\Mvariable{la3}}}\multsp p_{3}^{\Mvariable{la4}}\multsp \Delta \cdot {p_4}-
      {{\Delta }^{\Mvariable{la3}}}\multsp {{\Delta }^{\Mvariable{la4}}}\multsp {p_3}\cdot {p_4}\big)\multsp
    {{\delta }_{\Mvariable{a1}\Mvariable{oc}}}\multsp {{\delta }_{\Mvariable{a2}\Mvariable{od}}}\multsp
    {{\delta }_{\Mvariable{a3}\Mvariable{ob}}}\multsp {{\delta }_{\Mvariable{a4}\Mvariable{oa}}}+  \\
\noalign{\vspace{0.760417ex}}
   \hspace{1.em} \big(p_{3}^{\Mvariable{la1}}\multsp {{\Delta }^{\Mvariable{la3}}}\multsp \Delta \cdot {p_1}-
     {g^{\Mvariable{la1}\Mvariable{la3}}}\multsp \Delta \cdot {p_3}\multsp \Delta \cdot {p_1}+
     {{\Delta }^{\Mvariable{la1}}}\multsp p_{1}^{\Mvariable{la3}}\multsp \Delta \cdot {p_3}-
     {{\Delta }^{\Mvariable{la1}}}\multsp {{\Delta }^{\Mvariable{la3}}}\multsp {p_1}\cdot {p_3}\big)\multsp   \\
\noalign{\vspace{
   0.760417ex}}
\hspace{2.em} \big(p_{4}^{\Mvariable{la2}}\multsp {{\Delta }^{\Mvariable{la4}}}\multsp \Delta \cdot {p_2}-
      {g^{\Mvariable{la2}\Mvariable{la4}}}\multsp \Delta \cdot {p_4}\multsp \Delta \cdot {p_2}+
      {{\Delta }^{\Mvariable{la2}}}\multsp p_{2}^{\Mvariable{la4}}\multsp \Delta \cdot {p_4}-
      {{\Delta }^{\Mvariable{la2}}}\multsp {{\Delta }^{\Mvariable{la4}}}\multsp {p_2}\cdot {p_4}\big)\multsp
    {{\delta }_{\Mvariable{a1}\Mvariable{od}}}\multsp {{\delta }_{\Mvariable{a2}\Mvariable{ob}}}\multsp
    {{\delta }_{\Mvariable{a3}\Mvariable{oc}}}\multsp {{\delta }_{\Mvariable{a4}\Mvariable{oa}}}+  \\
\noalign{\vspace{0.760417ex}}
   \hspace{1.em} \big(p_{3}^{\Mvariable{la1}}\multsp {{\Delta }^{\Mvariable{la3}}}\multsp \Delta \cdot {p_1}-
     {g^{\Mvariable{la1}\Mvariable{la3}}}\multsp \Delta \cdot {p_3}\multsp \Delta \cdot {p_1}+
     {{\Delta }^{\Mvariable{la1}}}\multsp p_{1}^{\Mvariable{la3}}\multsp \Delta \cdot {p_3}-
     {{\Delta }^{\Mvariable{la1}}}\multsp {{\Delta }^{\Mvariable{la3}}}\multsp {p_1}\cdot {p_3}\big)\multsp   \\
\noalign{\vspace{
   0.760417ex}}
\hspace{2.em} \big(p_{4}^{\Mvariable{la2}}\multsp {{\Delta }^{\Mvariable{la4}}}\multsp \Delta \cdot {p_2}-
      {g^{\Mvariable{la2}\Mvariable{la4}}}\multsp \Delta \cdot {p_4}\multsp \Delta \cdot {p_2}+
      {{\Delta }^{\Mvariable{la2}}}\multsp p_{2}^{\Mvariable{la4}}\multsp \Delta \cdot {p_4}-
      {{\Delta }^{\Mvariable{la2}}}\multsp {{\Delta }^{\Mvariable{la4}}}\multsp {p_2}\cdot {p_4}\big)\multsp
    {{\delta }_{\Mvariable{a1}\Mvariable{ob}}}\multsp {{\delta }_{\Mvariable{a2}\Mvariable{od}}}\multsp
    {{\delta }_{\Mvariable{a3}\Mvariable{oc}}}\multsp {{\delta }_{\Mvariable{a4}\Mvariable{oa}}}+  \\
\noalign{\vspace{0.760417ex}}
   \hspace{1.em} \big(p_{4}^{\Mvariable{la1}}\multsp {{\Delta }^{\Mvariable{la4}}}\multsp \Delta \cdot {p_1}-
     {g^{\Mvariable{la1}\Mvariable{la4}}}\multsp \Delta \cdot {p_4}\multsp \Delta \cdot {p_1}+
     {{\Delta }^{\Mvariable{la1}}}\multsp p_{1}^{\Mvariable{la4}}\multsp \Delta \cdot {p_4}-
     {{\Delta }^{\Mvariable{la1}}}\multsp {{\Delta }^{\Mvariable{la4}}}\multsp {p_1}\cdot {p_4}\big)\multsp   \\
\noalign{\vspace{
   0.760417ex}}
\hspace{2.em} \big(p_{3}^{\Mvariable{la2}}\multsp {{\Delta }^{\Mvariable{la3}}}\multsp \Delta \cdot {p_2}-
      {g^{\Mvariable{la2}\Mvariable{la3}}}\multsp \Delta \cdot {p_3}\multsp \Delta \cdot {p_2}+
      {{\Delta }^{\Mvariable{la2}}}\multsp p_{2}^{\Mvariable{la3}}\multsp \Delta \cdot {p_3}-
      {{\Delta }^{\Mvariable{la2}}}\multsp {{\Delta }^{\Mvariable{la3}}}\multsp {p_2}\cdot {p_3}\big)\multsp
    {{\delta }_{\Mvariable{a1}\Mvariable{oc}}}\multsp {{\delta }_{\Mvariable{a2}\Mvariable{ob}}}\multsp
    {{\delta }_{\Mvariable{a3}\Mvariable{od}}}\multsp {{\delta }_{\Mvariable{a4}\Mvariable{oa}}}+  \\
\noalign{\vspace{0.760417ex}}
   \hspace{1.em} \big(p_{4}^{\Mvariable{la1}}\multsp {{\Delta }^{\Mvariable{la4}}}\multsp \Delta \cdot {p_1}-
     {g^{\Mvariable{la1}\Mvariable{la4}}}\multsp \Delta \cdot {p_4}\multsp \Delta \cdot {p_1}+
     {{\Delta }^{\Mvariable{la1}}}\multsp p_{1}^{\Mvariable{la4}}\multsp \Delta \cdot {p_4}-
     {{\Delta }^{\Mvariable{la1}}}\multsp {{\Delta }^{\Mvariable{la4}}}\multsp {p_1}\cdot {p_4}\big)\multsp   \\
\noalign{\vspace{
   0.760417ex}}
\hspace{2.em} \big(p_{3}^{\Mvariable{la2}}\multsp {{\Delta }^{\Mvariable{la3}}}\multsp \Delta \cdot {p_2}-
      {g^{\Mvariable{la2}\Mvariable{la3}}}\multsp \Delta \cdot {p_3}\multsp \Delta \cdot {p_2}+
      {{\Delta }^{\Mvariable{la2}}}\multsp p_{2}^{\Mvariable{la3}}\multsp \Delta \cdot {p_3}-
      {{\Delta }^{\Mvariable{la2}}}\multsp {{\Delta }^{\Mvariable{la3}}}\multsp {p_2}\cdot {p_3}\big)\multsp
    {{\delta }_{\Mvariable{a1}\Mvariable{ob}}}\multsp {{\delta }_{\Mvariable{a2}\Mvariable{oc}}}\multsp
    {{\delta }_{\Mvariable{a3}\Mvariable{od}}}\multsp {{\delta }_{\Mvariable{a4}\Mvariable{oa}}}+  \\
\noalign{\vspace{0.760417ex}}
   \hspace{1.em} \big(p_{2}^{\Mvariable{la1}}\multsp {{\Delta }^{\Mvariable{la2}}}\multsp \Delta \cdot {p_1}-
     {g^{\Mvariable{la1}\Mvariable{la2}}}\multsp \Delta \cdot {p_2}\multsp \Delta \cdot {p_1}+
     {{\Delta }^{\Mvariable{la1}}}\multsp p_{1}^{\Mvariable{la2}}\multsp \Delta \cdot {p_2}-
     {{\Delta }^{\Mvariable{la1}}}\multsp {{\Delta }^{\Mvariable{la2}}}\multsp {p_1}\cdot {p_2}\big)\multsp   \\
\noalign{\vspace{
   0.760417ex}}
\hspace{2.em} \big(p_{4}^{\Mvariable{la3}}\multsp {{\Delta }^{\Mvariable{la4}}}\multsp \Delta \cdot {p_3}-
      {g^{\Mvariable{la3}\Mvariable{la4}}}\multsp \Delta \cdot {p_4}\multsp \Delta \cdot {p_3}+
      {{\Delta }^{\Mvariable{la3}}}\multsp p_{3}^{\Mvariable{la4}}\multsp \Delta \cdot {p_4}-
      {{\Delta }^{\Mvariable{la3}}}\multsp {{\Delta }^{\Mvariable{la4}}}\multsp {p_3}\cdot {p_4}\big)\multsp
    {{\delta }_{\Mvariable{a1}\Mvariable{od}}}\multsp {{\delta }_{\Mvariable{a2}\Mvariable{oc}}}\multsp
    {{\delta }_{\Mvariable{a3}\Mvariable{oa}}}\multsp {{\delta }_{\Mvariable{a4}\Mvariable{ob}}}+  \\
\noalign{\vspace{0.760417ex}}
   \hspace{1.em} \big(p_{2}^{\Mvariable{la1}}\multsp {{\Delta }^{\Mvariable{la2}}}\multsp \Delta \cdot {p_1}-
     {g^{\Mvariable{la1}\Mvariable{la2}}}\multsp \Delta \cdot {p_2}\multsp \Delta \cdot {p_1}+
     {{\Delta }^{\Mvariable{la1}}}\multsp p_{1}^{\Mvariable{la2}}\multsp \Delta \cdot {p_2}-
     {{\Delta }^{\Mvariable{la1}}}\multsp {{\Delta }^{\Mvariable{la2}}}\multsp {p_1}\cdot {p_2}\big)\multsp   \\
\noalign{\vspace{
   0.760417ex}}
\hspace{2.em} \big(p_{4}^{\Mvariable{la3}}\multsp {{\Delta }^{\Mvariable{la4}}}\multsp \Delta \cdot {p_3}-
      {g^{\Mvariable{la3}\Mvariable{la4}}}\multsp \Delta \cdot {p_4}\multsp \Delta \cdot {p_3}+
      {{\Delta }^{\Mvariable{la3}}}\multsp p_{3}^{\Mvariable{la4}}\multsp \Delta \cdot {p_4}-
      {{\Delta }^{\Mvariable{la3}}}\multsp {{\Delta }^{\Mvariable{la4}}}\multsp {p_3}\cdot {p_4}\big)\multsp
    {{\delta }_{\Mvariable{a1}\Mvariable{oc}}}\multsp {{\delta }_{\Mvariable{a2}\Mvariable{od}}}\multsp
    {{\delta }_{\Mvariable{a3}\Mvariable{oa}}}\multsp {{\delta }_{\Mvariable{a4}\Mvariable{ob}}}+  \\
\noalign{\vspace{0.760417ex}}
   \hspace{1.em} \big(p_{4}^{\Mvariable{la1}}\multsp {{\Delta }^{\Mvariable{la4}}}\multsp \Delta \cdot {p_1}-
     {g^{\Mvariable{la1}\Mvariable{la4}}}\multsp \Delta \cdot {p_4}\multsp \Delta \cdot {p_1}+
     {{\Delta }^{\Mvariable{la1}}}\multsp p_{1}^{\Mvariable{la4}}\multsp \Delta \cdot {p_4}-
     {{\Delta }^{\Mvariable{la1}}}\multsp {{\Delta }^{\Mvariable{la4}}}\multsp {p_1}\cdot {p_4}\big)\multsp   \\
\noalign{\vspace{
   0.760417ex}}
\hspace{2.em} \big(p_{3}^{\Mvariable{la2}}\multsp {{\Delta }^{\Mvariable{la3}}}\multsp \Delta \cdot {p_2}-
      {g^{\Mvariable{la2}\Mvariable{la3}}}\multsp \Delta \cdot {p_3}\multsp \Delta \cdot {p_2}+
      {{\Delta }^{\Mvariable{la2}}}\multsp p_{2}^{\Mvariable{la3}}\multsp \Delta \cdot {p_3}-
      {{\Delta }^{\Mvariable{la2}}}\multsp {{\Delta }^{\Mvariable{la3}}}\multsp {p_2}\cdot {p_3}\big)\multsp
    {{\delta }_{\Mvariable{a1}\Mvariable{od}}}\multsp {{\delta }_{\Mvariable{a2}\Mvariable{oa}}}\multsp
    {{\delta }_{\Mvariable{a3}\Mvariable{oc}}}\multsp {{\delta }_{\Mvariable{a4}\Mvariable{ob}}}+  \\
\noalign{\vspace{0.760417ex}}
   \hspace{1.em} \big(p_{4}^{\Mvariable{la1}}\multsp {{\Delta }^{\Mvariable{la4}}}\multsp \Delta \cdot {p_1}-
     {g^{\Mvariable{la1}\Mvariable{la4}}}\multsp \Delta \cdot {p_4}\multsp \Delta \cdot {p_1}+
     {{\Delta }^{\Mvariable{la1}}}\multsp p_{1}^{\Mvariable{la4}}\multsp \Delta \cdot {p_4}-
     {{\Delta }^{\Mvariable{la1}}}\multsp {{\Delta }^{\Mvariable{la4}}}\multsp {p_1}\cdot {p_4}\big)\multsp   \\
\noalign{\vspace{
   0.760417ex}}
\hspace{2.em} \big(p_{3}^{\Mvariable{la2}}\multsp {{\Delta }^{\Mvariable{la3}}}\multsp \Delta \cdot {p_2}-
      {g^{\Mvariable{la2}\Mvariable{la3}}}\multsp \Delta \cdot {p_3}\multsp \Delta \cdot {p_2}+
      {{\Delta }^{\Mvariable{la2}}}\multsp p_{2}^{\Mvariable{la3}}\multsp \Delta \cdot {p_3}-
      {{\Delta }^{\Mvariable{la2}}}\multsp {{\Delta }^{\Mvariable{la3}}}\multsp {p_2}\cdot {p_3}\big)\multsp
    {{\delta }_{\Mvariable{a1}\Mvariable{oa}}}\multsp {{\delta }_{\Mvariable{a2}\Mvariable{od}}}\multsp
    {{\delta }_{\Mvariable{a3}\Mvariable{oc}}}\multsp {{\delta }_{\Mvariable{a4}\Mvariable{ob}}}+  \\
\noalign{\vspace{0.760417ex}}
   \hspace{1.em} \big(p_{3}^{\Mvariable{la1}}\multsp {{\Delta }^{\Mvariable{la3}}}\multsp \Delta \cdot {p_1}-
     {g^{\Mvariable{la1}\Mvariable{la3}}}\multsp \Delta \cdot {p_3}\multsp \Delta \cdot {p_1}+
     {{\Delta }^{\Mvariable{la1}}}\multsp p_{1}^{\Mvariable{la3}}\multsp \Delta \cdot {p_3}-
     {{\Delta }^{\Mvariable{la1}}}\multsp {{\Delta }^{\Mvariable{la3}}}\multsp {p_1}\cdot {p_3}\big)\multsp   \\
\noalign{\vspace{
   0.760417ex}}
\hspace{2.em} \big(p_{4}^{\Mvariable{la2}}\multsp {{\Delta }^{\Mvariable{la4}}}\multsp \Delta \cdot {p_2}-
      {g^{\Mvariable{la2}\Mvariable{la4}}}\multsp \Delta \cdot {p_4}\multsp \Delta \cdot {p_2}+
      {{\Delta }^{\Mvariable{la2}}}\multsp p_{2}^{\Mvariable{la4}}\multsp \Delta \cdot {p_4}-
      {{\Delta }^{\Mvariable{la2}}}\multsp {{\Delta }^{\Mvariable{la4}}}\multsp {p_2}\cdot {p_4}\big)\multsp
    {{\delta }_{\Mvariable{a1}\Mvariable{oc}}}\multsp {{\delta }_{\Mvariable{a2}\Mvariable{oa}}}\multsp
    {{\delta }_{\Mvariable{a3}\Mvariable{od}}}\multsp {{\delta }_{\Mvariable{a4}\Mvariable{ob}}}+  \\
\noalign{\vspace{0.760417ex}}
   \hspace{1.em} \big(p_{3}^{\Mvariable{la1}}\multsp {{\Delta }^{\Mvariable{la3}}}\multsp \Delta \cdot {p_1}-
     {g^{\Mvariable{la1}\Mvariable{la3}}}\multsp \Delta \cdot {p_3}\multsp \Delta \cdot {p_1}+
     {{\Delta }^{\Mvariable{la1}}}\multsp p_{1}^{\Mvariable{la3}}\multsp \Delta \cdot {p_3}-
     {{\Delta }^{\Mvariable{la1}}}\multsp {{\Delta }^{\Mvariable{la3}}}\multsp {p_1}\cdot {p_3}\big)\multsp   \\
\noalign{\vspace{
   0.760417ex}}
\hspace{2.em} \big(p_{4}^{\Mvariable{la2}}\multsp {{\Delta }^{\Mvariable{la4}}}\multsp \Delta \cdot {p_2}-
      {g^{\Mvariable{la2}\Mvariable{la4}}}\multsp \Delta \cdot {p_4}\multsp \Delta \cdot {p_2}+
      {{\Delta }^{\Mvariable{la2}}}\multsp p_{2}^{\Mvariable{la4}}\multsp \Delta \cdot {p_4}-
      {{\Delta }^{\Mvariable{la2}}}\multsp {{\Delta }^{\Mvariable{la4}}}\multsp {p_2}\cdot {p_4}\big)\multsp
    {{\delta }_{\Mvariable{a1}\Mvariable{oa}}}\multsp {{\delta }_{\Mvariable{a2}\Mvariable{oc}}}\multsp
    {{\delta }_{\Mvariable{a3}\Mvariable{od}}}\multsp {{\delta }_{\Mvariable{a4}\Mvariable{ob}}}+  \\
\noalign{\vspace{0.760417ex}}
   \hspace{1.em} \big(p_{3}^{\Mvariable{la1}}\multsp {{\Delta }^{\Mvariable{la3}}}\multsp \Delta \cdot {p_1}-
     {g^{\Mvariable{la1}\Mvariable{la3}}}\multsp \Delta \cdot {p_3}\multsp \Delta \cdot {p_1}+
     {{\Delta }^{\Mvariable{la1}}}\multsp p_{1}^{\Mvariable{la3}}\multsp \Delta \cdot {p_3}-
     {{\Delta }^{\Mvariable{la1}}}\multsp {{\Delta }^{\Mvariable{la3}}}\multsp {p_1}\cdot {p_3}\big)\multsp   \\
\noalign{\vspace{
   0.760417ex}}
\hspace{2.em} \big(p_{4}^{\Mvariable{la2}}\multsp {{\Delta }^{\Mvariable{la4}}}\multsp \Delta \cdot {p_2}-
      {g^{\Mvariable{la2}\Mvariable{la4}}}\multsp \Delta \cdot {p_4}\multsp \Delta \cdot {p_2}+
      {{\Delta }^{\Mvariable{la2}}}\multsp p_{2}^{\Mvariable{la4}}\multsp \Delta \cdot {p_4}-
      {{\Delta }^{\Mvariable{la2}}}\multsp {{\Delta }^{\Mvariable{la4}}}\multsp {p_2}\cdot {p_4}\big)\multsp
    {{\delta }_{\Mvariable{a1}\Mvariable{od}}}\multsp {{\delta }_{\Mvariable{a2}\Mvariable{ob}}}\multsp
    {{\delta }_{\Mvariable{a3}\Mvariable{oa}}}\multsp {{\delta }_{\Mvariable{a4}\Mvariable{oc}}}+  \\
\noalign{\vspace{0.760417ex}}
   \hspace{1.em} \big(p_{3}^{\Mvariable{la1}}\multsp {{\Delta }^{\Mvariable{la3}}}\multsp \Delta \cdot {p_1}-
     {g^{\Mvariable{la1}\Mvariable{la3}}}\multsp \Delta \cdot {p_3}\multsp \Delta \cdot {p_1}+
     {{\Delta }^{\Mvariable{la1}}}\multsp p_{1}^{\Mvariable{la3}}\multsp \Delta \cdot {p_3}-
     {{\Delta }^{\Mvariable{la1}}}\multsp {{\Delta }^{\Mvariable{la3}}}\multsp {p_1}\cdot {p_3}\big)\multsp   \\
\noalign{\vspace{
   0.760417ex}}
\hspace{2.em} \big(p_{4}^{\Mvariable{la2}}\multsp {{\Delta }^{\Mvariable{la4}}}\multsp \Delta \cdot {p_2}-
      {g^{\Mvariable{la2}\Mvariable{la4}}}\multsp \Delta \cdot {p_4}\multsp \Delta \cdot {p_2}+
      {{\Delta }^{\Mvariable{la2}}}\multsp p_{2}^{\Mvariable{la4}}\multsp \Delta \cdot {p_4}-
      {{\Delta }^{\Mvariable{la2}}}\multsp {{\Delta }^{\Mvariable{la4}}}\multsp {p_2}\cdot {p_4}\big)\multsp
    {{\delta }_{\Mvariable{a1}\Mvariable{ob}}}\multsp {{\delta }_{\Mvariable{a2}\Mvariable{od}}}\multsp
    {{\delta }_{\Mvariable{a3}\Mvariable{oa}}}\multsp {{\delta }_{\Mvariable{a4}\Mvariable{oc}}}+  \\
\noalign{\vspace{0.760417ex}}
   \hspace{1.em} \big(p_{4}^{\Mvariable{la1}}\multsp {{\Delta }^{\Mvariable{la4}}}\multsp \Delta \cdot {p_1}-
     {g^{\Mvariable{la1}\Mvariable{la4}}}\multsp \Delta \cdot {p_4}\multsp \Delta \cdot {p_1}+
     {{\Delta }^{\Mvariable{la1}}}\multsp p_{1}^{\Mvariable{la4}}\multsp \Delta \cdot {p_4}-
     {{\Delta }^{\Mvariable{la1}}}\multsp {{\Delta }^{\Mvariable{la4}}}\multsp {p_1}\cdot {p_4}\big)\multsp   \\
\noalign{\vspace{
   0.760417ex}}
\hspace{2.em} \big(p_{3}^{\Mvariable{la2}}\multsp {{\Delta }^{\Mvariable{la3}}}\multsp \Delta \cdot {p_2}-
      {g^{\Mvariable{la2}\Mvariable{la3}}}\multsp \Delta \cdot {p_3}\multsp \Delta \cdot {p_2}+
      {{\Delta }^{\Mvariable{la2}}}\multsp p_{2}^{\Mvariable{la3}}\multsp \Delta \cdot {p_3}-
      {{\Delta }^{\Mvariable{la2}}}\multsp {{\Delta }^{\Mvariable{la3}}}\multsp {p_2}\cdot {p_3}\big)\multsp
    {{\delta }_{\Mvariable{a1}\Mvariable{od}}}\multsp {{\delta }_{\Mvariable{a2}\Mvariable{oa}}}\multsp
    {{\delta }_{\Mvariable{a3}\Mvariable{ob}}}\multsp {{\delta }_{\Mvariable{a4}\Mvariable{oc}}}+  \\
\noalign{\vspace{0.760417ex}}
   \hspace{1.em} \big(p_{4}^{\Mvariable{la1}}\multsp {{\Delta }^{\Mvariable{la4}}}\multsp \Delta \cdot {p_1}-
     {g^{\Mvariable{la1}\Mvariable{la4}}}\multsp \Delta \cdot {p_4}\multsp \Delta \cdot {p_1}+
     {{\Delta }^{\Mvariable{la1}}}\multsp p_{1}^{\Mvariable{la4}}\multsp \Delta \cdot {p_4}-
     {{\Delta }^{\Mvariable{la1}}}\multsp {{\Delta }^{\Mvariable{la4}}}\multsp {p_1}\cdot {p_4}\big)\multsp   \\
\noalign{\vspace{
   0.760417ex}}
\hspace{2.em} \big(p_{3}^{\Mvariable{la2}}\multsp {{\Delta }^{\Mvariable{la3}}}\multsp \Delta \cdot {p_2}-
      {g^{\Mvariable{la2}\Mvariable{la3}}}\multsp \Delta \cdot {p_3}\multsp \Delta \cdot {p_2}+
      {{\Delta }^{\Mvariable{la2}}}\multsp p_{2}^{\Mvariable{la3}}\multsp \Delta \cdot {p_3}-
      {{\Delta }^{\Mvariable{la2}}}\multsp {{\Delta }^{\Mvariable{la3}}}\multsp {p_2}\cdot {p_3}\big)\multsp
    {{\delta }_{\Mvariable{a1}\Mvariable{oa}}}\multsp {{\delta }_{\Mvariable{a2}\Mvariable{od}}}\multsp
    {{\delta }_{\Mvariable{a3}\Mvariable{ob}}}\multsp {{\delta }_{\Mvariable{a4}\Mvariable{oc}}}+  \\
\noalign{\vspace{0.760417ex}}
   \hspace{1.em} \big(p_{2}^{\Mvariable{la1}}\multsp {{\Delta }^{\Mvariable{la2}}}\multsp \Delta \cdot {p_1}-
     {g^{\Mvariable{la1}\Mvariable{la2}}}\multsp \Delta \cdot {p_2}\multsp \Delta \cdot {p_1}+
     {{\Delta }^{\Mvariable{la1}}}\multsp p_{1}^{\Mvariable{la2}}\multsp \Delta \cdot {p_2}-
     {{\Delta }^{\Mvariable{la1}}}\multsp {{\Delta }^{\Mvariable{la2}}}\multsp {p_1}\cdot {p_2}\big)\multsp   \\
\noalign{\vspace{
   0.760417ex}}
\hspace{2.em} \big(p_{4}^{\Mvariable{la3}}\multsp {{\Delta }^{\Mvariable{la4}}}\multsp \Delta \cdot {p_3}-
      {g^{\Mvariable{la3}\Mvariable{la4}}}\multsp \Delta \cdot {p_4}\multsp \Delta \cdot {p_3}+
      {{\Delta }^{\Mvariable{la3}}}\multsp p_{3}^{\Mvariable{la4}}\multsp \Delta \cdot {p_4}-
      {{\Delta }^{\Mvariable{la3}}}\multsp {{\Delta }^{\Mvariable{la4}}}\multsp {p_3}\cdot {p_4}\big)\multsp
    {{\delta }_{\Mvariable{a1}\Mvariable{ob}}}\multsp {{\delta }_{\Mvariable{a2}\Mvariable{oa}}}\multsp
    {{\delta }_{\Mvariable{a3}\Mvariable{od}}}\multsp {{\delta }_{\Mvariable{a4}\Mvariable{oc}}}+  \\
\noalign{\vspace{0.760417ex}}
   \hspace{1.em} \big(p_{2}^{\Mvariable{la1}}\multsp {{\Delta }^{\Mvariable{la2}}}\multsp \Delta \cdot {p_1}-
     {g^{\Mvariable{la1}\Mvariable{la2}}}\multsp \Delta \cdot {p_2}\multsp \Delta \cdot {p_1}+
     {{\Delta }^{\Mvariable{la1}}}\multsp p_{1}^{\Mvariable{la2}}\multsp \Delta \cdot {p_2}-
     {{\Delta }^{\Mvariable{la1}}}\multsp {{\Delta }^{\Mvariable{la2}}}\multsp {p_1}\cdot {p_2}\big)\multsp   \\
\noalign{\vspace{
   0.760417ex}}
\hspace{2.em} \big(p_{4}^{\Mvariable{la3}}\multsp {{\Delta }^{\Mvariable{la4}}}\multsp \Delta \cdot {p_3}-
      {g^{\Mvariable{la3}\Mvariable{la4}}}\multsp \Delta \cdot {p_4}\multsp \Delta \cdot {p_3}+
      {{\Delta }^{\Mvariable{la3}}}\multsp p_{3}^{\Mvariable{la4}}\multsp \Delta \cdot {p_4}-
      {{\Delta }^{\Mvariable{la3}}}\multsp {{\Delta }^{\Mvariable{la4}}}\multsp {p_3}\cdot {p_4}\big)\multsp
    {{\delta }_{\Mvariable{a1}\Mvariable{oa}}}\multsp {{\delta }_{\Mvariable{a2}\Mvariable{ob}}}\multsp
    {{\delta }_{\Mvariable{a3}\Mvariable{od}}}\multsp {{\delta }_{\Mvariable{a4}\Mvariable{oc}}}+  \\
\noalign{\vspace{0.760417ex}}
   \hspace{1.em} \big(p_{4}^{\Mvariable{la1}}\multsp {{\Delta }^{\Mvariable{la4}}}\multsp \Delta \cdot {p_1}-
     {g^{\Mvariable{la1}\Mvariable{la4}}}\multsp \Delta \cdot {p_4}\multsp \Delta \cdot {p_1}+
     {{\Delta }^{\Mvariable{la1}}}\multsp p_{1}^{\Mvariable{la4}}\multsp \Delta \cdot {p_4}-
     {{\Delta }^{\Mvariable{la1}}}\multsp {{\Delta }^{\Mvariable{la4}}}\multsp {p_1}\cdot {p_4}\big)\multsp   \\
\noalign{\vspace{
   0.760417ex}}
\hspace{2.em} \big(p_{3}^{\Mvariable{la2}}\multsp {{\Delta }^{\Mvariable{la3}}}\multsp \Delta \cdot {p_2}-
      {g^{\Mvariable{la2}\Mvariable{la3}}}\multsp \Delta \cdot {p_3}\multsp \Delta \cdot {p_2}+
      {{\Delta }^{\Mvariable{la2}}}\multsp p_{2}^{\Mvariable{la3}}\multsp \Delta \cdot {p_3}-
      {{\Delta }^{\Mvariable{la2}}}\multsp {{\Delta }^{\Mvariable{la3}}}\multsp {p_2}\cdot {p_3}\big)\multsp
    {{\delta }_{\Mvariable{a1}\Mvariable{oc}}}\multsp {{\delta }_{\Mvariable{a2}\Mvariable{ob}}}\multsp
    {{\delta }_{\Mvariable{a3}\Mvariable{oa}}}\multsp {{\delta }_{\Mvariable{a4}\Mvariable{od}}}+  \\
\noalign{\vspace{0.760417ex}}
   \hspace{1.em} \big(p_{4}^{\Mvariable{la1}}\multsp {{\Delta }^{\Mvariable{la4}}}\multsp \Delta \cdot {p_1}-
     {g^{\Mvariable{la1}\Mvariable{la4}}}\multsp \Delta \cdot {p_4}\multsp \Delta \cdot {p_1}+
     {{\Delta }^{\Mvariable{la1}}}\multsp p_{1}^{\Mvariable{la4}}\multsp \Delta \cdot {p_4}-
     {{\Delta }^{\Mvariable{la1}}}\multsp {{\Delta }^{\Mvariable{la4}}}\multsp {p_1}\cdot {p_4}\big)\multsp   \\
\noalign{\vspace{
   0.760417ex}}
\hspace{2.em} \big(p_{3}^{\Mvariable{la2}}\multsp {{\Delta }^{\Mvariable{la3}}}\multsp \Delta \cdot {p_2}-
      {g^{\Mvariable{la2}\Mvariable{la3}}}\multsp \Delta \cdot {p_3}\multsp \Delta \cdot {p_2}+
      {{\Delta }^{\Mvariable{la2}}}\multsp p_{2}^{\Mvariable{la3}}\multsp \Delta \cdot {p_3}-
      {{\Delta }^{\Mvariable{la2}}}\multsp {{\Delta }^{\Mvariable{la3}}}\multsp {p_2}\cdot {p_3}\big)\multsp
    {{\delta }_{\Mvariable{a1}\Mvariable{ob}}}\multsp {{\delta }_{\Mvariable{a2}\Mvariable{oc}}}\multsp
    {{\delta }_{\Mvariable{a3}\Mvariable{oa}}}\multsp {{\delta }_{\Mvariable{a4}\Mvariable{od}}}+  \\
\noalign{\vspace{0.760417ex}}
   \hspace{1.em} \big(p_{3}^{\Mvariable{la1}}\multsp {{\Delta }^{\Mvariable{la3}}}\multsp \Delta \cdot {p_1}-
     {g^{\Mvariable{la1}\Mvariable{la3}}}\multsp \Delta \cdot {p_3}\multsp \Delta \cdot {p_1}+
     {{\Delta }^{\Mvariable{la1}}}\multsp p_{1}^{\Mvariable{la3}}\multsp \Delta \cdot {p_3}-
     {{\Delta }^{\Mvariable{la1}}}\multsp {{\Delta }^{\Mvariable{la3}}}\multsp {p_1}\cdot {p_3}\big)\multsp   \\
\noalign{\vspace{
   0.760417ex}}
\hspace{2.em} \big(p_{4}^{\Mvariable{la2}}\multsp {{\Delta }^{\Mvariable{la4}}}\multsp \Delta \cdot {p_2}-
      {g^{\Mvariable{la2}\Mvariable{la4}}}\multsp \Delta \cdot {p_4}\multsp \Delta \cdot {p_2}+
      {{\Delta }^{\Mvariable{la2}}}\multsp p_{2}^{\Mvariable{la4}}\multsp \Delta \cdot {p_4}-
      {{\Delta }^{\Mvariable{la2}}}\multsp {{\Delta }^{\Mvariable{la4}}}\multsp {p_2}\cdot {p_4}\big)\multsp
    {{\delta }_{\Mvariable{a1}\Mvariable{oc}}}\multsp {{\delta }_{\Mvariable{a2}\Mvariable{oa}}}\multsp
    {{\delta }_{\Mvariable{a3}\Mvariable{ob}}}\multsp {{\delta }_{\Mvariable{a4}\Mvariable{od}}}+  \\
\noalign{\vspace{0.760417ex}}
   \hspace{1.em} \big(p_{3}^{\Mvariable{la1}}\multsp {{\Delta }^{\Mvariable{la3}}}\multsp \Delta \cdot {p_1}-
     {g^{\Mvariable{la1}\Mvariable{la3}}}\multsp \Delta \cdot {p_3}\multsp \Delta \cdot {p_1}+
     {{\Delta }^{\Mvariable{la1}}}\multsp p_{1}^{\Mvariable{la3}}\multsp \Delta \cdot {p_3}-
     {{\Delta }^{\Mvariable{la1}}}\multsp {{\Delta }^{\Mvariable{la3}}}\multsp {p_1}\cdot {p_3}\big)\multsp   \\
\noalign{\vspace{
   0.760417ex}}
\hspace{2.em} \big(p_{4}^{\Mvariable{la2}}\multsp {{\Delta }^{\Mvariable{la4}}}\multsp \Delta \cdot {p_2}-
      {g^{\Mvariable{la2}\Mvariable{la4}}}\multsp \Delta \cdot {p_4}\multsp \Delta \cdot {p_2}+
      {{\Delta }^{\Mvariable{la2}}}\multsp p_{2}^{\Mvariable{la4}}\multsp \Delta \cdot {p_4}-
      {{\Delta }^{\Mvariable{la2}}}\multsp {{\Delta }^{\Mvariable{la4}}}\multsp {p_2}\cdot {p_4}\big)\multsp
    {{\delta }_{\Mvariable{a1}\Mvariable{oa}}}\multsp {{\delta }_{\Mvariable{a2}\Mvariable{oc}}}\multsp
    {{\delta }_{\Mvariable{a3}\Mvariable{ob}}}\multsp {{\delta }_{\Mvariable{a4}\Mvariable{od}}}+  \\
\noalign{\vspace{0.760417ex}}
   \hspace{1.em} \big(p_{2}^{\Mvariable{la1}}\multsp {{\Delta }^{\Mvariable{la2}}}\multsp \Delta \cdot {p_1}-
     {g^{\Mvariable{la1}\Mvariable{la2}}}\multsp \Delta \cdot {p_2}\multsp \Delta \cdot {p_1}+
     {{\Delta }^{\Mvariable{la1}}}\multsp p_{1}^{\Mvariable{la2}}\multsp \Delta \cdot {p_2}-
     {{\Delta }^{\Mvariable{la1}}}\multsp {{\Delta }^{\Mvariable{la2}}}\multsp {p_1}\cdot {p_2}\big)\multsp   \\
\noalign{\vspace{
   0.760417ex}}
\hspace{2.em} \big(p_{4}^{\Mvariable{la3}}\multsp {{\Delta }^{\Mvariable{la4}}}\multsp \Delta \cdot {p_3}-
      {g^{\Mvariable{la3}\Mvariable{la4}}}\multsp \Delta \cdot {p_4}\multsp \Delta \cdot {p_3}+
      {{\Delta }^{\Mvariable{la3}}}\multsp p_{3}^{\Mvariable{la4}}\multsp \Delta \cdot {p_4}-
      {{\Delta }^{\Mvariable{la3}}}\multsp {{\Delta }^{\Mvariable{la4}}}\multsp {p_3}\cdot {p_4}\big)\multsp
    {{\delta }_{\Mvariable{a1}\Mvariable{ob}}}\multsp {{\delta }_{\Mvariable{a2}\Mvariable{oa}}}\multsp
    {{\delta }_{\Mvariable{a3}\Mvariable{oc}}}\multsp {{\delta }_{\Mvariable{a4}\Mvariable{od}}}+  \\
\noalign{\vspace{0.760417ex}}
   \hspace{1.em} \big(p_{2}^{\Mvariable{la1}}\multsp {{\Delta }^{\Mvariable{la2}}}\multsp \Delta \cdot {p_1}-
     {g^{\Mvariable{la1}\Mvariable{la2}}}\multsp \Delta \cdot {p_2}\multsp \Delta \cdot {p_1}+
     {{\Delta }^{\Mvariable{la1}}}\multsp p_{1}^{\Mvariable{la2}}\multsp \Delta \cdot {p_2}-
     {{\Delta }^{\Mvariable{la1}}}\multsp {{\Delta }^{\Mvariable{la2}}}\multsp {p_1}\cdot {p_2}\big)\multsp   \\
\noalign{\vspace{
   0.760417ex}}
\hspace{2.em} \big(p_{4}^{\Mvariable{la3}}\multsp {{\Delta }^{\Mvariable{la4}}}\multsp \Delta \cdot {p_3}-
     {g^{\Mvariable{la3}\Mvariable{la4}}}\multsp \Delta \cdot {p_4}\multsp \Delta \cdot {p_3}+
     {{\Delta }^{\Mvariable{la3}}}\multsp p_{3}^{\Mvariable{la4}}\multsp \Delta \cdot {p_4}-
     {{\Delta }^{\Mvariable{la3}}}\multsp {{\Delta }^{\Mvariable{la4}}}\multsp {p_3}\cdot {p_4}\big)\multsp
   {{\delta }_{\Mvariable{a1}\Mvariable{oa}}}\multsp {{\delta }_{\Mvariable{a2}\Mvariable{ob}}}\multsp
   {{\delta }_{\Mvariable{a3}\Mvariable{oc}}}\multsp {{\delta }_{\Mvariable{a4}\Mvariable{od}}}\\
\MathEnd{MathArray}
}

\dispSFoutmath{
\{ \}
}


\dispSFinmath{
\Mfunction{Options}[\Mvariable{Uncontract}]
}

\dispSFoutmath{
\{\Mvariable{Dimension}\rightarrow \Mvariable{Automatic},\Mvariable{DimensionalReduction}\rightarrow \Mvariable{False},
    \Mvariable{Pair}\rightarrow \{\},\Mvariable{Unique}\rightarrow \Mvariable{True}\}
}

\dispSFinmath{
\Mvariable{t1}\multsp =\multsp \Muserfunction{LeviCivita}[\mu ,\nu ][p,q]
}

\dispSFinmath{
{{\epsilon }^{\mu \nu pq}}
}

\dispSFoutmath{
\Muserfunction{Uncontract}[\Mvariable{t1},\multsp p]
}

\Subsection*{Upper}

\Subsubsection*{Description}

Upper may be used inside LorentzIndex to indicate an contravariant LorentzIndex.

See also: { }LorentzIndex, Lower, Contract1.

\Subsection*{UVPart { }***unfinished*** (examples missing)}

\Subsubsection*{Description}

UVPart[exp, q] discards ultraviolet finite integrals { }(q \(=\) integration momentum).

\Subsubsection*{Examples}

\dispSFinmath{
{{\epsilon }^{\mu \nu \$AL\$8091(1)q}}\multsp {p^{\$AL\$8091(1)}}
}

\Subsection*{Vectors}

\Subsubsection*{Description}

Vectors is an option for FORM2FeynCalc. Its default setting is Automatic. It may be set to a list, if the FORM-file does not contain a
  V(ectors) statement.

See also: { }FORM2FeynCalc.

\Subsection*{WriteOut}

\Subsubsection*{Description}

WriteOut is an option for OneLoop. { }If set to True, the result of OneLoop will be written to a file called "name.res", where name is
  the first argument of OneLoop.

See also: { }OneLoop.

\Subsection*{WriteOutPaVe}

\Subsubsection*{Description}

WriteOutPaVe is an option for PaVeReduce and OneLoopSum. If set to a string, the results of all Passarino-Veltman PaVe's are stored in
  files with names generated from this string and the arguments of PaVe.

See also: { }PaVeReduce, PaVe, OneLoopSum.

\Subsection*{Write2}

\Subsubsection*{Description}

Write2[file, val1 \(=\) expr1, val2 \(=\) expr2, ...] writes the settings val1 \(=\) expr1, val2 \(=\) expr2 in sequence followed by a
  newline, to the specified output file. Setting the option FormatType of Write2 to FortranForm results in FORTRAN syntax output.

The continuation character for FORTRAN output can be controlled by changing \${}FortranContinuationCharacter.

\dispSFinmath{
\Mvariable{t2}=\Muserfunction{DiracSlash}[p]
}

\dispSFoutmath{
\gamma \cdot p
}

\dispSFinmath{
\Muserfunction{Uncontract}[\Mvariable{t2},\multsp p]
}

\dispSFoutmath{
{{\gamma }^{\$AL\$8092(1)}}\multsp {p^{\$AL\$8092(1)}}
}

\dispSFinmath{
\Muserfunction{Uncontract}[\Mvariable{t1},\multsp p,q]
}

\dispSFoutmath{
{{\epsilon }^{\mu \nu \$AL\$8094(1)\$AL\$8093(1)}}\multsp
   {q^{\$AL\$8093(1)}}\multsp {p^{\$AL\$8094(1)}}
}

See also:  Isolate, PaVeReduce.

\Subsubsection*{Examples}

\dispSFinmath{
\Muserfunction{Uncontract}[\Muserfunction{ScalarProduct}[p,q],\multsp q]
}

\dispSFoutmath{
p\cdot q
}

\dispSFinmath{
\Muserfunction{Uncontract}[\Muserfunction{ScalarProduct}[p,q],q,\Mvariable{Pair}\rightarrow \Mvariable{All}]
}

This writes the assignment r\(=\)t to a file.

\dispSFinmath{
{p^{\$AL\$8096(1)}}\multsp {q^{\$AL\$8096(1)}}
}

This shows the contents of the file.

\dispSFinmath{
\Muserfunction{Uncontract}[\Muserfunction{ScalarProduct}[p,q]\RawWedge 2,q,\Mvariable{Pair}\rightarrow \Mvariable{All}]
}

\dispSFoutmath{
{p^{\$AL\$8097(1)}}\multsp {q^{\$AL\$8097(1)}}\multsp {p^{\$AL\$8097(2)}}
   \multsp {q^{\$AL\$8097(2)}}
}

\dispSFinmath{
\Mfunction{Clear}[\Mvariable{t1},\Mvariable{t2}]
}

\dispSFinmath{
\Muserfunction{DeclareNonCommutative}[x]
}

\dispSFoutmath{
\Muserfunction{DataType}[x,\Mvariable{NonCommutative}]
}

\dispSFinmath{
\Mvariable{True}
}

\dispSFinmath{
\Muserfunction{UnDeclareNonCommutative}[x]
}

\dispSFoutmath{
\Muserfunction{DataType}[x,\Mvariable{NonCommutative}]
}

\dispSFinmath{
\Mvariable{True}
}

This is how to write out the expression t2 in Fortran format.

\dispSFinmath{
\Muserfunction{DeclareNonCommutative}[y,z]
}

\dispSFinmath{
\Muserfunction{DataType}[y,z,\Mvariable{NonCommutative}]
}

\dispSFoutmath{
\{\Mvariable{True},\Mvariable{True}\}
}

\dispSFinmath{
\Muserfunction{UnDeclareNonCommutative}[y,z]
}

\Subsection*{XYT}

\Subsubsection*{Description}

XYT[exp, x,y] transforms { }\(\Muserfunction{DataType}[y,z,\Mvariable{NonCommutative}]\) away ...

\Subsection*{ZeroMomentumInsertion}

\Subsubsection*{Description}

ZeroMomentumInsertion is an option of FeynRule, Twist2GluonOperator and Twist2QuarkOperator.

See also: { }FeynRule, Twist2GluonOperator, Twist2QuarkOperator.

\Subsection*{Zeta2}

\Subsubsection*{Description}

Zeta2 denotes Zeta[2]. N[Zeta[2]] evaluates to N[Zeta[2]].

See also: SimplifyPolyLog.

\Subsubsection*{Examples}

\dispSFinmath{
\{\Mvariable{True},\Mvariable{True}\}
}

\dispSFoutmath{
\{ \}
}


\dispSFoutmath{
\Mfunction{FullForm}[\$FortranContinuationCharacter]
}

\dispSFinmath{
"\&"
}

\dispSFoutmath{
\Mfunction{Options}[\Mvariable{Write2}]
}

\Subsection*{\${}Abbreviations}

\Subsubsection*{Description}

\${}Abbreviations are a list of string substitution rules used when generating names for storing intermediate results. It is used by
  OneLoop and PaVeReduce.The elements of the list should be of the form "name" \(\rightarrow \) "abbreviation".

\dispSFinmath{
\MathBegin{MathArray}{l}
\{\Mvariable{D0Convention}\rightarrow 0,\Mvariable{FinalSubstitutions}\rightarrow \{\},
    \Mvariable{FormatType}\rightarrow \Mvariable{InputForm},\Mvariable{FortranFormatDoublePrecision}\rightarrow \Mvariable{True},  \\
   \noalign{\vspace{0.666667ex}}
\hspace{1.em} \Mvariable{PageWidth}\rightarrow 62,\Mvariable{PostFortranFile}\rightarrow ,
    \Mvariable{PreFortranFile}\rightarrow ,\Mvariable{StringReplace}\rightarrow \{\}\}\\
\MathEnd{MathArray}
}

\dispSFoutmath{
\Mfunction{Attributes}[\Mvariable{Write2}]
}

See also:  Abbreviation, OneLoop, PaVeReduce, WriteOut, WriteOutPaVe.

\Subsection*{\${}AL}

\Subsubsection*{Description}

\${}AL is the head for dummy indices which may be introduced by Uncontract.

\dispSFinmath{
\{\Mvariable{HoldRest}\}
}

\dispSFoutmath{
t=\Mfunction{Collect}\big[{{\big({{(a-c)}^2}+{{(a-b)}^2}\big)}^2},a,\Mvariable{Factor}\big]
}

\dispSFinmath{
4\multsp {a^4}-8\multsp (b+c)\multsp {a^3}+8\multsp ({b^2}+c\multsp b+{c^2})\multsp {a^2}-4\multsp (b+c)\multsp ({b^2}+{c^2})\multsp a+
   {{({b^2}+{c^2})}^2}
}

\dispSFoutmath{
\Mvariable{tempfilename}=\Mfunction{ToString}[\$SessionID]<>".s";
}

\dispSFinmath{
\Muserfunction{Write2}[\Mvariable{tempfilename},r=t];
}

\dispSFinmath{
\MathBegin{MathArray}{l}
\Mvariable{TableForm}[  \\
\noalign{\vspace{0.5ex}}
\hspace{1.em} \Mfunction{ReadList}[
    \Mfunction{If}[\$OperatingSystem==="MacOS",":",""]<>\Mvariable{tempfilename},\Mvariable{String}]]\\
\MathEnd{MathArray}
}

\dispSFoutmath{
\MathBegin{MathArray}[c]{l}
  \Mvariable{r\multsp =\multsp (\multsp 4*a\RawWedge 4\multsp -\multsp 8*a\RawWedge 3*(b\multsp +\multsp
    c)\multsp -\multsp 4*a*(b\multsp +\multsp c)*(b\RawWedge 2\multsp +\multsp c\RawWedge 2)\multsp +\multsp } \\
  \multsp (b\RawWedge
    2\multsp +\multsp c\RawWedge 2)\RawWedge 2\multsp +\multsp 8*a\RawWedge 2*(b\RawWedge 2\multsp +\multsp b*c\multsp +\multsp
    c\RawWedge 2) \\
  \multsp \multsp \multsp \multsp \multsp \multsp \multsp );
  \MathEnd{MathArray}
}

\dispSFinmath{
\Mfunction{DeleteFile}[\Mfunction{If}[\$OperatingSystem==="MacOS",":",""]<>\Mvariable{tempfilename}]
}

\Subsection*{\${}BreitMaison}

\Subsubsection*{Description}

The Breitenlohner-Maison \(\Mvariable{t2}=x+\Muserfunction{Isolate}[t,a,\Mvariable{IsolateNames}\rightarrow W]\) scheme is currently not supported
by the Dirac algebra functions.Use Tracer if you need it (PHI supplies the functions FCToTracer and
  TracerToFC).

See also:  Tr, \${}Larin, \${}West.

\Subsection*{\${}Color}

\Subsubsection*{Description}

\${}Color is False by default. If set to True, some special variables will be colored.

\Subsection*{\${}Covariant}

\Subsubsection*{Description}

The boolean setting of \${}Covariant determines whether lorentz indices are displayed as lower indices (True) or as upper ones (False).

\dispSFinmath{
4\multsp {a^4}-8\multsp W(1)\multsp {a^3}+8\multsp W(3)\multsp {a^2}-4\multsp W(1)\multsp W(2)\multsp a+{{W(2)}^2}+x
}

\dispSFoutmath{
\Muserfunction{Write2}[\Mvariable{tempfilename},r=\Mvariable{t2}];
}

\dispSFinmath{
\MathBegin{MathArray}{l}
\Mvariable{TableForm}[  \\
\noalign{\vspace{0.5ex}}
\hspace{1.em} \Mfunction{ReadList}[
    \Mfunction{If}[\$OperatingSystem==="MacOS",":",""]<>\Mvariable{tempfilename},\Mvariable{String}]]\\
\MathEnd{MathArray}
}

\dispSFoutmath{
\MathBegin{MathArray}[c]{l}
  \Mvariable{W[1]\multsp =\multsp (b\multsp +\multsp c} \\
  \multsp \multsp \multsp \multsp \multsp \multsp
    \multsp ); \\
  \Mvariable{W[2]\multsp =\multsp (b\RawWedge 2\multsp +\multsp c\RawWedge 2} \\
  \multsp \multsp \multsp \multsp
    \multsp \multsp \multsp ); \\
  \Mvariable{W[3]\multsp =\multsp (b\RawWedge 2\multsp +\multsp b*c\multsp +\multsp c\RawWedge
    2} \\
  \multsp \multsp \multsp \multsp \multsp \multsp \multsp ); \\
  \Mvariable{r\multsp =\multsp (\multsp 4*a\RawWedge 4\multsp
    +\multsp x\multsp -\multsp 8*a\RawWedge 3*HoldForm[W[1]]\multsp -\multsp 4*a*HoldForm[W[1]]*} \\
  \multsp \multsp
    HoldForm[W[2]]\multsp +\multsp HoldForm[W[2]]\RawWedge 2\multsp +\multsp 8*a\RawWedge 2*HoldForm[W[3]] \\
  \multsp \multsp \multsp
    \multsp \multsp \multsp \multsp );
  \MathEnd{MathArray}
}

\dispSFinmath{
\Mfunction{DeleteFile}[\Mfunction{If}[\$OperatingSystem==="MacOS",":",""]<>\Mvariable{tempfilename}]
}

\dispSFinmath{
\Muserfunction{Write2}[\Mvariable{tempfilename},r=\Mvariable{t2},\Mvariable{FormatType}\rightarrow \Mvariable{FortranForm}];
}

\dispSFoutmath{
\MathBegin{MathArray}{l}
\Mvariable{TableForm}[  \\
\noalign{\vspace{0.5ex}}
\hspace{1.em} \Mfunction{ReadList}[
    \Mfunction{If}[\$OperatingSystem==="MacOS",":",""]<>\Mvariable{tempfilename},\Mvariable{String}]]\\
\MathEnd{MathArray}
}

\dispSFinmath{
\MathBegin{MathArray}[c]{l}
  \multsp \multsp \multsp \multsp \multsp \multsp \multsp \multsp W(1D0)=\multsp W(1D0) \\
  \multsp \multsp
    \multsp \multsp \multsp \multsp \multsp \multsp W(2D0)=\multsp W(2D0) \\
  \multsp \multsp \multsp \multsp \multsp \multsp \multsp
    \multsp W(3D0)=\multsp W(3D0) \\
  \multsp \multsp \multsp \multsp \multsp \multsp \multsp \multsp r\multsp =\multsp 4D0*a**4\multsp
    +\multsp x\multsp -\multsp 8D0*a**3*W(1D0)\multsp -\multsp  \\
  \multsp \multsp \multsp \multsp \multsp \&\multsp \multsp
    4D0*a*W(1D0)*W(2D0)\multsp +\multsp W(2D0)**2\multsp +\multsp 8D0*a**2*W(3D0) \\
  \multsp \multsp \multsp \multsp \multsp \multsp
    \multsp \multsp \multsp \multsp \multsp \multsp \multsp \multsp \multsp \multsp \multsp \multsp
  \MathEnd{MathArray}
}

\Subsection*{\${}FCS}

\Subsubsection*{Description}

\${}FCS is a list of functions with a short name. E.g. GA[nu] can be used instead of DiracGamma[nu].

\dispSFinmath{
\MathBegin{MathArray}{l}
\Mfunction{DeleteFile}[\Mfunction{If}[\$OperatingSystem==="MacOS",":",""]<>\Mvariable{tempfilename}];  \\
   \noalign{\vspace{0.5ex}}\Mfunction{Clear}[t,\Mvariable{t2},\Mvariable{tempfilename}];\\
\MathEnd{MathArray}
}

\dispSFoutmath{
{{(x\multsp y)}^m}
}

\Subsection*{\${}FCT}

\Subsubsection*{Description}

Setting the global variable \${}FCT to True changes the default typesetting rules for the noncommutative multiplication operator Dot[].

\dispSFinmath{
\Mvariable{Zeta2}
}

\dispSFinmath{
\zeta (2)
}

\dispSFoutmath{
\Mfunction{N}[\Mvariable{Zeta2}]
}

\dispSFinmath{
1.64493
}

\dispSFinmath{
\Muserfunction{SimplifyPolyLog}[\pi \RawWedge 2]
}

\dispSFoutmath{
6\multsp \zeta (2)
}

\Subsection*{\${}FeynCalcStuff}

\Subsubsection*{Description}

\${}FeynCalcStuff is the list of availabe stuff in FeynCalc.

\Subsection*{\${}FeynCalcVersion}

\Subsubsection*{Description}

\${}FeynCalcVersion is a string that represents the version of FeynCalc.

\dispSFinmath{
\$Abbreviations//\Mfunction{StandardForm}
}

\dispSFoutmath{
\MathBegin{MathArray}{l}
\{\RawWedge \rightarrow ,*\rightarrow ,\{\rightarrow ,[\rightarrow ,\}\rightarrow ,]\rightarrow ,/\rightarrow
    ,  \\
\noalign{\vspace{0.5ex}}\rightarrow ,  \\
\noalign{\vspace{0.5ex}}\rightarrow ,\multsp \rightarrow ,,\multsp \rightarrow ,
   \Mvariable{AxialVector}\rightarrow \Mvariable{AV},\Mvariable{Fermion}\rightarrow F,\Mvariable{Momentum}\rightarrow ,  \\
   \noalign{\vspace{0.5ex}}
\hspace{1.em} \Mvariable{Pair}\rightarrow ,\Mvariable{ParticleMass}\rightarrow m,
   \Mvariable{PseudoScalar}\rightarrow \Mvariable{PS},\Mvariable{RenormalizationState}\rightarrow ,  \\
\noalign{\vspace{0.5ex}}
   \hspace{1.em} \Mvariable{Scalar}\rightarrow S,\Mvariable{SmallVariable}\rightarrow \Mvariable{sma},
    \Mvariable{Subscript}\rightarrow \Mvariable{su},\Mvariable{Vector}\rightarrow V\}\\
\MathEnd{MathArray}
}

\Subsection*{\${}FortranContinuationCharacter}

\Subsubsection*{Description}

\${}FortranContinuationCharacter is the continuation character used in Write2.

\dispSFinmath{
\$AL
}

\dispSFoutmath{
\$AL
}

See also:  Write2.

\Subsection*{\${}Gauge}

\Subsubsection*{Description}

\${}Gauge(\(=\) 1/xi) is a constant specifying the gauge fixing parameter of QED in Lorentz gauge. { }The usual choice is Feynman gauge,
  \${}Gauge\(=\)1. Notice that \${}Gauge is used by some functions, the option Gauge by others.

See also:  Gauge.

\Subsection*{\${}Larin}

\Subsubsection*{Description}

If set to True, the Larin-Gorishny-Atkyampo-DelBurgo-scheme for \(\Muserfunction{Uncontract}[\Muserfunction{ScalarProduct}[p,q],q,\Mvariable{Pair}\rightarrow
\Mvariable{All}]\) in {\itshape D}-dimensions is used, i.e. before evaluating traces (but after moving \({p^{\$AL\$8098(1)}}\multsp
{q^{\$AL\$8098(1)}}\) anticommuting in all dimensions to the right of the Dirac string) a product { }gamma[mu].gamma5 is substituted
to { }-I/6
  Eps[mu,al,be,si] gamma[al,be,si], where all indices live in {\itshape D}-dimensions now. Especially the Levic-Civita tensor is taken to be {\itshape
D}-dimensional, i.e., contraction of two Eps's results in {\itshape D}'s. This has (FOR ONE AXIAL-VECTOR-CURRENT ONLY, it is not so clear if this
scheme also works for more than one fermion line involving \(\$AL=\mu ;\)) the same effect as the Breitenlohner-Maison-'t Hooft-Veltman scheme.

\dispSFinmath{
\Muserfunction{Uncontract}[\Muserfunction{ScalarProduct}[p,q],q,\Mvariable{Pair}\rightarrow \Mvariable{All}]
}

\dispSFoutmath{
{p^{\mu \$8099(1)}}\multsp {q^{\mu \$8099(1)}}
}

See also: { }Tr, \${}BreitMaison, \${}West.

\Subsection*{\${}LimitTo4}

\Subsubsection*{Description}

\${}LimitTo4 is a global variable with default setting True. If set to False, no limit Dimension \(\rightarrow \) 4 is performed after
  tensor integral decomposition.

\dispSFinmath{
\$AL=.;
}

\dispSFoutmath{
{{\gamma }_5}
}

See also:  PaVe, PaVeReduce, OneLoop.

\Subsection*{\${}LorentzIndices}

\Subsubsection*{Description}

\${}LorentzIndices is a global variable. If set to True the dimension of LorentzIndex is displayed as an index.

\dispSFinmath{
\$Covariant
}

\dispSFinmath{
\Mvariable{False}
}

\dispSFoutmath{
\Muserfunction{GA}[\mu ]
}

\dispSFinmath{
{{\gamma }^{\mu }}
}

\dispSFoutmath{
\$Covariant=\Mvariable{True};
}

\dispSFinmath{
\Muserfunction{GA}[\mu ]
}

\dispSFoutmath{
{{\gamma }_{\mu }}
}

\dispSFinmath{
\$Covariant=\Mvariable{False};
}

\dispSFoutmath{
\$FCS
}

\dispSFinmath{
\MathBegin{MathArray}{l}
\{\Mvariable{CDr},\Mvariable{FAD},\Mvariable{FC},\Mvariable{FCE},\Mvariable{FCI},\Mvariable{FDr},\Mvariable{FI},
    \Mvariable{FV},\Mvariable{FVD},\Mvariable{GA},\Mvariable{GA5},\Mvariable{GGV},\Mvariable{GP},  \\
\noalign{\vspace{0.666667ex}}
   \hspace{1.em} \Mvariable{GS},\Mvariable{GSD},\Mvariable{LC},\Mvariable{LCD},\Mvariable{MT},\Mvariable{MTD},\Mvariable{QGV},
    \Mvariable{QO},\Mvariable{SD},\Mvariable{SOD},\Mvariable{SP},\Mvariable{SPC},\Mvariable{SPD},\Mvariable{SPL},\Mvariable{PMV}\}\\
   \MathEnd{MathArray}
}

\Subsection*{\${}MIntegrate}

\Subsubsection*{Description}

\${}MIntegrate is a global list of integrations done by Mathematica inside OPEIntDelta.

\dispSFinmath{
\$FCT\multsp =\multsp \Mvariable{True};
}

\dispSFoutmath{
\Muserfunction{GA}[\mu ]\multsp .\multsp \Muserfunction{GA}[\nu ]
}

See also:  OPEIntDelta.

\Subsection*{\${}MU}

\Subsubsection*{Description}

\${}MU is the head for dummy indices which may be introduced by Chisholm (and evtl. Contract and DiracReduce).

\dispSFinmath{
{{\gamma }^{\mu }}\multsp {{\gamma }^{\nu }}
}

\dispSFoutmath{
\$FCT\multsp =\multsp \Mvariable{False};
}

See also:  Chisholm.

\Subsection*{\${}MemoryAvailable}

\Subsubsection*{Description}

\${}MemoryAvailable is { }a global variable which is set to an integer {\itshape n}, where {\itshape n} is the available amount of main memory in
MB. The default is 128. It should be increased if possible. The higher \${}MemoryAvailable can
  be, the more intermediate steps do not have to be repeated by FeynCalc.

\dispSFinmath{
\Muserfunction{GA}[\mu ]\multsp .\multsp \Muserfunction{GA}[\nu ]
}

\dispSFoutmath{
{{\gamma }^{\mu }}.{{\gamma }^{\nu }}
}

\Subsection*{\${}NonComm}

\Subsubsection*{Description}

\${}NonComm contains a list of all non-commutative heads present.

\dispSFinmath{
\$FeynCalcVersion
}

\dispSFoutmath{
5.0.0beta1
}

See also:  NonCommQ, NonCommutative.

\Subsection*{\${}PairBrackets}

\Subsubsection*{Description}

\${}PairBrackets determines whether brackets are drawn around scalar products in the notebook interface.

\dispSFinmath{
\$FortranContinuationCharacter
}

\dispSFoutmath{
\&
}

\dispSFinmath{
{{\gamma }_5}
}

\dispSFoutmath{
{{\gamma }_5}
}

\dispSFinmath{
{{\gamma }_5}
}

\dispSFoutmath{
\$Larin
}

\dispSFinmath{
\Mvariable{False}
}

\dispSFoutmath{
\$LimitTo4
}

\dispSFinmath{
\Mvariable{True}
}

\dispSFoutmath{
\$LorentzIndices=\Mvariable{True};
}

\Subsection*{\${}SpinorMinimal}

\Subsubsection*{Description}

\${}SpinorMinimal is a global switch for an additional simplification attempt in DiracSimplify for more than one Spinor-line. The default
  is False, since otherwise it costs too much time.

\dispSFinmath{
\Muserfunction{MetricTensor}[\alpha ,\beta ,\Mvariable{Dimension}\rightarrow n]
   \Muserfunction{DiracMatrix}[\alpha ,\Mvariable{Dimension}\rightarrow n]
}

\dispSFoutmath{
{{\gamma }^{{{\alpha }_n}}}\multsp \Muserfunction{MetricTensor}(\alpha ,\beta ,\Mvariable{Dimension}\rightarrow n)
}

See also:  DiracSimplify.

\Subsection*{\${}VeryVerbose}

\Subsubsection*{Description}

\${}VeryVerbose is a global variable with default setting 0. If set to 1, 2, ..., less and more intermediate comments and informations
  are displayed during calculations.

\dispSFinmath{
\%//\Mfunction{StandardForm}
}

\dispSFoutmath{
\Muserfunction{DiracGamma}[\Muserfunction{LorentzIndex}[\alpha ,n],n]\multsp
   \Muserfunction{MetricTensor}[\alpha ,\beta ,\Mvariable{Dimension}\rightarrow n]
}

\dispSFinmath{
\Muserfunction{MTD}[\alpha ,\beta ]\multsp \Muserfunction{FVD}[p,\beta ]\Muserfunction{GAD}[\mu ]//\Muserfunction{FCI}
}

\dispSFinmath{
{{\gamma }^{{{\mu }_D}}}\multsp {g^{\alpha \beta }}\multsp {p^{{{\beta }_D}}}
}

\dispSFinmath{
\%//\Mfunction{StandardForm}
}

See also:  \${}V0.

\Subsection*{\${}West}

\Subsubsection*{Description}

If \${}West is set to True (which is the default), traces involving more than 4 Dirac matrices and \(\MathBegin{MathArray}{l}
\Muserfunction{DiracGamma}[\Muserfunction{LorentzIndex}[\mu ,D],D]\multsp
   \Muserfunction{Pair}[\Muserfunction{LorentzIndex}[\alpha ,D],\Muserfunction{LorentzIndex}[\beta ,D]]\multsp   \\
   \noalign{\vspace{0.5ex}}
\hspace{1.em} \Muserfunction{Pair}[\Muserfunction{LorentzIndex}[\beta ,D],\Muserfunction{Momentum}[p,D]]\\
   \MathEnd{MathArray}\) are calculated recursively according to formula (A.5) from Comp. Phys. Comm 77 (1993) 286-298, which is based on the Breitenlohner
  Maison \(\$LorentzIndices=\Mvariable{False};\)-scheme.

See examples in Tr.

\Commentary{Evaluation time and memory usage}

\dispSFinmath{
\$MIntegrate
}

\dispSFoutmath{
\{\}
}

\dispSFinmath{
\$MU
}

\dispSFoutmath{
\$MU
}

\dispSFinmath{
\$MemoryAvailable
}

\dispSFoutmath{
256
}

\dispSFinmath{
\$NonComm
}

\dispSFoutmath{
\MathBegin{MathArray}{l}
\{\Mvariable{ChiralityProjector},\Mvariable{DiracGamma},\Mvariable{DiracGammaT},\Mvariable{DiracMatrix},
    \Mvariable{DiracSigma},\Mvariable{DiracSlash},\Mvariable{Spinor},\Mvariable{GA},\Mvariable{GSD},  \\
\noalign{\vspace{
   0.666667ex}}
\hspace{1.em} \Mvariable{GS},\Mvariable{LeftPartialD},\Mvariable{LeftRightPartialD2},\Mvariable{LeftRightPartialD},
   \Mvariable{PartialD},\Mvariable{QuantumField},\Mvariable{RightPartialD},  \\
\noalign{\vspace{0.666667ex}}
\hspace{1.em} \Mvariable{Sp
    inorUBar},\Mvariable{SpinorU},\Mvariable{SpinorVBar},\Mvariable{SpinorV},\Mvariable{SUNT},\Mvariable{FieldStrength},
   \Mvariable{QuarkGluonVertex},\Mvariable{QuarkPropagator},  \\
\noalign{\vspace{0.666667ex}}
\hspace{1.em} C,
   \Mvariable{Twist2QuarkOperator},\Mvariable{OPESum},\Mvariable{CovariantD},\ScriptCapitalU ,
   \Mvariable{HighEnergyPhysics`Phi`Objects`Private`plus},\ScriptU ,\chi ,  \\
\noalign{\vspace{0.666667ex}}
\hspace{1.em} {{\chi }_-},
    {{\chi }_+},\Mvariable{UFMinus},\Mvariable{UFPlus},\Mvariable{UGamma},\Mvariable{UMatrix},\Mvariable{USmall},
    \Mvariable{Twist2AlienOperator},\Mvariable{Twist2CounterOperator}\}\\
\MathEnd{MathArray}
}

\Section*{Functions (experimental)}

\Subsection*{DoPolarizationSums}

\Subsubsection*{Description}

***EXPERIMENTAL***\\
DoPolarizationSums[exp] sums over vector polarizations for expressions with a factor of the form\\
Pair[LorentzIndex[rho1\_{}], Momentum[Polarization[p\_{}, -I]]]*\\
Pair[LorentzIndex[rho2\_{}], Momentum[Polarization[p\_{}, I]]].

See also:  Polarization, Uncontract.

\Subsubsection*{Examples}

In the following (and indeed everywhere else within FeynCalc), notice that Lorentz indices written as super or subscripts are not to be
  taken as such. Instead, by convention, when two indices are contracted one is always lower and the other upper.

FeynCalc uses the following normalization of Polarization vectors:

\dispSFinmath{
\$PairBrackets
}

\dispSFoutmath{
\Mvariable{False}
}

\dispSFinmath{
\Muserfunction{SP}[p,q]\Muserfunction{SP}[r,s]
}

\dispSFoutmath{
p\cdot q\multsp r\cdot s
}

DoPolarizationSums is chosen in such a way as to be consistent with this normalization and with \(\$PairBrackets=\Mvariable{True}\)\(\Mvariable{True}\)\(=\)\(\Muserfunction{SP}[p,q]\Muserfunction{SP}[r,s]\):

We can choose e.g. the following polarization vectors, labeled with a subscript, \{\((p\cdot q)\multsp (r\cdot s)\),\(\$PairBrackets=\Mvariable{False}\),\(\Mvariable{False}\),\(\$SpinorMinimal\)\}
\(=\) \{(\(\ImaginaryI \),0,0,0), (0,\(\ImaginaryI \),0,0), (0,0,\(\ImaginaryI \),0), (0,0,0,\(\ImaginaryI \))\}.

\dispSFinmath{
\Mvariable{False}
}

\dispSFinmath{
\$VeryVerbose
}

\dispSFinmath{
0
}

\dispSFoutmath{
\$VeryVerbose\multsp =\multsp 3;
}

\dispSFinmath{
\$VeryVerbose=0;
}

\dispSFinmath{
{{\gamma }_5}
}

\(341.105\)\(\Mfunction{MemoryInUse}[]\)\(=\)\(24424832\):

\dispSFinmath{
\Mfunction{MaxMemoryUsed}[]
}

\dispSFoutmath{
25668048
}

\(\Mfunction{TimeUsed}[]\)\(182.8\)\(=\)\(\MathBegin{MathArray}{l}
\Muserfunction{Pair}[\Muserfunction{LorentzIndex}[\mu ],
     \Muserfunction{Momentum}[\Muserfunction{Polarization}[\Mvariable{p4},-\ImaginaryI ]]]\multsp   \\
\noalign{\vspace{0.5ex}}
   \hspace{1.em} \Muserfunction{Pair}[\Muserfunction{LorentzIndex}[\mu ],
    \Muserfunction{Momentum}[\Muserfunction{Polarization}[\Mvariable{p4},\ImaginaryI ]]]\\
\MathEnd{MathArray}\):

\dispSFinmath{
\varepsilon _{\mu }^{*}({p_4})\multsp {{\varepsilon }_{\mu }}({p_4})
}

\dispSFoutmath{
\%//\Muserfunction{Contract}
}

\dispSFinmath{
-1
}

\dispSFoutmath{
\underline{\sum }
}

\dispSFinmath{
{{\varepsilon }_{\mu }}({{"p"}_4})\multsp {{({{\varepsilon }^{\nu }})}_{}}({{"p"}_4})
}

\dispSFoutmath{
-{{{g_{\mu }}}^{\NoBreak \nu }}
}

\dispSFinmath{
{e_1}
}

\dispSFoutmath{
{e_2}
}

\dispSFinmath{
{e_3}
}

\dispSFoutmath{
{e_4}
}

\dispSFinmath{
\Mfunction{Clear}[\Mvariable{ee},\Mvariable{tt}]
}

\dispSFoutmath{
\Muserfunction{ev}[\Mvariable{i\_}]:=\Mfunction{ReplacePart}[\{0,0,0,0\},\imag ,i];
}

\dispSFinmath{
\{\Muserfunction{ev}[1],\Muserfunction{ev}[2],\Muserfunction{ev}[3],\Muserfunction{ev}[4]\}
}

\Subsection*{FeynmanDoIntegrals ***unfinished***}

\Subsubsection*{Description}

***EXPERIMENTAL***\\
\hspace*{0.5ex} FeynmanDoIntegrals[exp, moms, vars] attempts to evaluate integrals over Feynman parameters vars in an expression exp as
  produced e.g. with FeynmanReduce. The variables given must be atomic quantities (AtomQ). If vars is omitted all variables in the
  integrals will be integrated. If vars is given, only the variables in vars will be integrated. moms is a list of all external momenta.
  The integrals in exp must be given in the form Integratedx[x, low, up].int, where x is the integration variable, low and up are the
  integration limits and int the integrand. The interval [low,up] is assumed to include integration bounds put by possible DeltaFunctions
  and moreover it is assumed that up \(\geq \) 0 and that up \(>\) low. The two options FCIntegrate and FCNIntegrate determine which
  integration will be applied to integrals judged respectively analytically and numerically doable. This judging is a very rough one.
  Using (TimedIntegrate[\#{}\#{}, Integrate\(\rightarrow \)Integrate]\&{}) or (TimedIntegrate[\#{}\#{}, Integrate\(\rightarrow
  \)NIntegrate]\&{}) as the setting of one or both allows for finer judging. Those that are judged neither analytically nor numerically
  doable are left unevaluated, but can of course be attempted evaluated by a simple sustitution. Beside the explicit options of
  FeynmanDoIntegrals options of the integration functions specified (FCIntegrate and FCNIntegrate) may be given and are passed on to
  these.

See also:  FeynmanReduce, FeynmanParametrize1.

\dispSFinmath{
\Bigg(\NoBreak \MathBegin{MathArray}[c]{cccc}
  \ImaginaryI &0&0&0 \\
  0&\ImaginaryI &0&0 \\
  0&0&\ImaginaryI &0 \\
  0&0&0&\ImaginaryI

  \MathEnd{MathArray}\NoBreak \Bigg)
}

\dispSFoutmath{
{{{e_i}}^{\mu }}
}

\Subsubsection*{Examples}

\dispSFinmath{
\Muserfunction{ee}[\Mvariable{i\_}][\Mvariable{mu\_}]:=\Muserfunction{ev}[i][[\mu]];
}

\Subsection*{FeynmanParametrize1 ***unfinished***}

\Subsubsection*{Description}

***EXPERIMENTAL***\\
\hspace*{0.5ex} FeynmanParametrize1[exp,k,Method-\(>\)Denominator] introduces Feynman parameters for all one-loop integrals in exp (k
  \(=\) integration momentum) using formula (11.A.1) from "The Quantum Theory of Fields" vol. 1 by Steven Weinberg. {
  }FeynmanParametrize1[exp,k,Method-\(>\)Exp] introduces Feynman parameters for all one-loop integrals in exp (k \(=\) integration
  momentum) using 1/(A-I eps) \(=\) I Integrate[Exp[-I x (A-I eps)],\{x,0,Infinity\},
  Assumptions-\(>\)\{Arg[A]\(=\)\(=\)0,Arg[eps]\(=\)\(=\)0\}]. In this case, when the option Integrate is set to True, odd factors of
  k-tensors are dropped and even factors are replaced according to Itzykson\&{}Zuber (8-117).

\dispSFinmath{
\Muserfunction{tt}[\Mvariable{i\_},\Mvariable{j\_},\Mvariable{mu\_},\Mvariable{nu\_}]:=
   \Muserfunction{ee}[i][\mu]\Muserfunction{ee}[j][\nu]
}

\dispSFoutmath{
{{{e_i}}^{\mu }}
}

\Subsubsection*{Examples}

\dispSFinmath{
{{{e_j}}^{\nu }}
}

\Subsection*{FeynmanReduce ***unfinished***}

\Subsubsection*{Description}

***EXPERIMENTAL***\\
\hspace*{0.5ex} FeynmanReduce[exp,params] takes a Feynman parameterized expression exp (as e.g. generated with FeynmanParametrize1) and a
  list of Feynman parameters as input and attempts to simplify the expression. If no parameters are given, Integratedx variables in the
  expression will be used. Currently, reduction of exponentials is implemented. This will work on terms of the form
  E\(\RawWedge\)p1[a,b,c,...]*p2[a,b,c,...], where p1 and p2 are fractions of polynomials in the Feynman parameters a,b,c,... If the
  option Expand is set to True, FeynmanReduce will attempt to bring the expression exp into a sum of such terms and operate on the terms
  one by one.

\dispSFinmath{
\underline{\sum }
}

\dispSFoutmath{
\multsp {{{e_i}}^{\mu }}{e_{j,\nu }}
}

\Subsubsection*{Examples}

\dispSFinmath{
-{{{g_{\mu }}}^{\NoBreak \nu }}
}

Other objects that are considered to be ok or under active development:

\dispSFinmath{
\Mfunction{Table}\big[\sum _{i=1}^{4}\Muserfunction{tt}[i,i,\mu,\nu],\{\mu,1,4\},\{\nu,1,4\}\big]
}

\dispSFoutmath{
\Bigg(\NoBreak \MathBegin{MathArray}[c]{cccc}
  -1&0&0&0 \\
  0&-1&0&0 \\
  0&0&-1&0 \\
  0&0&0&-1
  \MathEnd{MathArray}\NoBreak \Bigg)
}

Objects that are under consideration for being discarded:

\dispSFinmath{
\underline{\sum }
}

\dispSFoutmath{
\multsp {{{e_i}}^{\mu }}{e_{j,\nu }}
}

\dispSFinmath{
-{{{g_i}}^j}
}

\dispSFoutmath{
\Mfunction{Table}\big[\sum _{\mu=1}^{4}\Muserfunction{tt}[i,j,\mu,\mu],\{i,1,4\},\{j,1,4\}\big]
}

\dispSFinmath{
\Bigg(\NoBreak \MathBegin{MathArray}[c]{cccc}
  -1&0&0&0 \\
  0&-1&0&0 \\
  0&0&-1&0 \\
  0&0&0&-1
  \MathEnd{MathArray}\NoBreak \Bigg)
}



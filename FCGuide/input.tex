\section{Input Functions}

The \fc syntax for metric tensors, four-vectors, Dirac matrices, etc.,
is such that the positioning of the arguments automatically defines the nature of the variables. %You can only enter algebraic objects; there
%is no possibility of working with explicit components of vectors
%or with specific Dirac matrices like $\gamma_0$ or $\gamma_2$.
No declaration of Lorentz indices or four-vectors has
to be made.
Covariant and contravariant Lorentz indices are treated on an equal basis,
assuming summation over (twice) repeated indices, where each pair of
indices are assumed to be co- and contravariant respectively.
Since explicit components of Lorentz tensors are not used, this
convention causes no problems.
The ambiguity in the sign of the Levi-Civita tensor can be
fixed by an option of the function \mb{DiracTrace} and
\mb{EpsChisholm}, see page \pageref{input}.

The input functions are macros that are immediately translated into
the internal representation of \fc. But these sometimes lengthy
internal representations are difficult to recognize when working
interactively. Therefore, some effort has been put into making the output
of \fc look as much as possible like typeset formulas.
The typesetting of \fc depends on the input/output interface in use.

\begin{itemize}
\item{\bf Terminal interface.} Per default \fc sets the global \mma variable
\mb{\$PrePrint} to \mb{FeynCalcForm},
forcing the display to be in a more readable manner.
It is important to realize that the display given by \mb{FeynCalcForm} is
different from the input syntax. Thus, you may not give as input,
for example, a scalar product in the result of a trace calculation
by copying the form you see on the screen. To have the output be the
internal representation, set \mb{\$PrePrint=.}. To return to formatted output,
set \mb{\$PrePrint = FeynCalcForm}.
\item{\bf Notebook interface.} \fc per default automatically sets the
\mma default output format type to \mb{TraditionalForm}. Because \fc defines
the \mb{TraditionalForm}-look of its objects, the output looks very much like
typeset formulas. Again, the screen output cannot be copied and used
as input. If the \mma default output format type
\footnote{The \mma default output format type is set using the
menu 'Cell' $\rightarrow$ 'Default Output Format Type'.}
is set to
%\mb{OutputForm},
\mb{InputForm} or \mb{StandardForm}, the
screen output is the internal representation of \fc and can be copied
and used as input. To obtain the same output as with a terminal interface,
set the \mma default output format type to \mb{OutputForm} and set
\mb{\$PrePrint = FeynCalcForm}. To return to formatted output, set
the \mma default output format type to \mb{TraditionalForm}.
\end{itemize}

In the rest of this guide the examples given will use the notebook interface.

The internal structure of \fc
is explained in section \ref{inside} and in the
reference part \ref{rg}. Usually it is not necessary to know it.
Notice simply that there are three layers of \fc: The input functions,
an output representation and the internal representation.

In order to do calculations in the framework of $D$-dimensional
regularization, \fc can be used for  calculations
in 4 or $D$
%or $D-4$
dimensions.
The dimension of  metric tensors, Dirac matrices and  four-vectors
is specified by setting the option \mb{Dimension} of the corresponding
input functions.
The default is four. If \fc is  used for interactive
$D$-dimensional calculations, you should change the option of the
input functions.
Note that Lorentz indices, momenta and
Dirac matrices carry their dimension
locally, i.e., for calculating a Dirac trace in $D$ dimensions the
dimension of the Dirac matrices has to be set accordingly, thus
there is no need of a dimension option for e.g. the function \mb{DiracTrace}.

\subsection{Lorentz Tensors and Scalar Products}

\otabtwo{
\mbs{FourVector[{\sl p}, $\mu$]} &  four-vector  $p^{\mu}$ \cr
\mbs{LeviCivita[$\mu$, $\nu$, $\rho$, $\sigma$]} & Levi-Civita tensor $\varepsilon^{\mu \nu \rho \sigma}$  \cr
\mbs{LeviCivita[$\mu$, $\nu$][{\sl p, q}]} & short form for $\varepsilon^{\mu \nu \rho \sigma} \, p_\rho q_\sigma $ \cr
\mbs{MetricTensor[$\mu$, $\nu$]} & metric tensor $g^{\mu \nu}$ \cr
\mbs{ScalarProduct[{\sl p},\ {\sl q}]} & $p \cdot q$ \cr
} {Basic Lorentz tensor input functions.}

\otabtwo{
\mbs{FV[{\sl p}, $\mu$], \; FVD[{\sl p}, $\mu$]} &  4- and D-dimensionsal $p^{\mu}$ \cr
\mbs{LC[$\mu$, $\nu$, $\rho$, $\sigma$],
     LCD[$\mu$, $\nu$, $\rho$, $\sigma$]} & 4- and D-dimensional Levi-Civita tensor $\varepsilon^{\mu \nu \rho \sigma}$  \cr
\mbs{MT[$\mu$, $\nu$], \; MTD[$\mu$, $\nu$]} & 4- and D-dimensional metric tensor $g^{\mu \nu}$ \cr
\mbs{SP[{\sl p},\ {\sl q}], SPD[{\sl p},\ {\sl q}]} & 4- and D-dimensional $p \cdot q$ \cr
} {Short input forms for basic Lorentz tensor input functions.}

\otabthree{
{\sl option name} & {\sl default value} & \cr
\hhline
\mb{Dimension} & \mb{4} &  number of space-time dimensions \cr
} {Option for \mb{FourVector}, \mb{MetricTensor}, \mb{LeviCivita} and  \mb{ScalarProduct}.}

Internally, the input given with \mb{FourVector}, \mb{MetricTensor}, \mb{LeviCivita} and
\mb{ScalarProduct} is represented in terms of the lower level functions \mb{Pair}, \mb{Momentum} and \mb{LorentzIndex} (this is elaborated in section \ref{inside}).

The following examples show the input syntax and the typesetting of \fc.

\beom
\dtog{<<HighEnergyPhysics`FeynCalc`}{}{
First we have to load \fc.
}
\domtog{ScalarProduct[p, q]}{ {\sl p}.{\sl q}
}{
Enter a scalar product $p\cdot q$.
This is the typeset output.
}
\domtog{MetricTensor[$\mu$, $\nu$]}{\ $g^{\mu \nu}$\hfil\\
}{
This is the input for $g^{\mu \nu}$.
}
\domtog{MetricTensor[$\mu$, $\nu$, Dimension $\Rule$ D]}{\ $g^{\mu \nu}$\hfil\\
}{
A metric tensor $g^{\mu \nu}$ in $D$ dimensions
is entered in this way.
The number of dimensions is suppressed in the screen
output
(for notebook output. For terminal output,
the number of dimensions is displayed as an index,
%\mb{\subscript{\rm \mb{g}}{\rm D}[mu, nu]}).}
\mb{g$_{\rm \mb{D}}$[mu, nu]}).}
\domtog{FourVector[p,\ $\mu$]}{\ $p_{\mu}$\hfil\\
}{
A four-dimensional $p_{\mu}$.
}
\domtog{FourVector[p\ -\ 2\ q,\ $\mu$]}{\ $(p - 2\, q)_{\mu}$\hfil\\
}{
You may also enter a linear combination of
four-vectors: $(p - 2\, q)_{\mu}$.
}
\domtog{FourVector[q,\ $\mu$,\ Dimension\ $\Rule$\ D]}{\ $q_\mu$\hfil\\
}{
This is a $D$-dimensional $q_{\mu}$.
}
\domtog{PolarizationVector[k,\ $\mu$]}{\ $\varepsilon_{\mu}(k)$\hfil\\
}{
A polarization vector $\varepsilon_{\mu}(k)$ is a special four-vector.
}
\domtog{Conjugate[PolarizationVector[k,\ $\mu$]]}{\ $\varepsilon_{\mu}^{*}(k)$\hfil\\
}{
This is how to enter a $\varepsilon_{\mu}^{*}(k)$.
}
\enom

Polarization vectors are represented in a special way (see page \pageref{input}).
The transversality condition $\varepsilon(k) \cdot k=0$ is automatically
fulfilled (see page \pageref{polcon}).
%Levi-Civita tensors and
Polarization vectors are defined
in \fc in four dimensions only. For \mb{FourVector}, \mb{MetricTensor}, \mb{LeviCivita},
and  \mb{ScalarProduct} other dimensions may be specified with the
option \mb{Dimension}.

\otabtwo{
\mbs{PolarizationVector[{\sl k},\ {\sl mu}]} & $\varepsilon_{\mu}(k)$\cr
\mbs{Conjugate[ PolarizationVector[{\sl k},\ {\sl mu}]]} & $\varepsilon^{*}_{\mu}(k)$\cr
}{The input for a polarization vector and polarization.}

\subsection{Dirac Matrices}

A Dirac matrix $\gamma^{\mu}$ is entered like \mb{DiracMatrix[$\mu$]}.
For  $\bps = p_{\mu} \, \gamma^{\mu}$ you may use
\mb{DiracSlash[p]}. Products of Dirac matrices or slashes are
entered either by adding subsequent arguments, i.e.,
\mb{DiracMatrix[{\sl mu},\ {\sl nu},\ ...]} and
\mb{DiracSlash[{\sl p},\ {\sl q},\ ...]}, or by multiplying the
\mb{DiracMatrix} and \mb{DiracSlash} with the \mma \mb{Dot}, ``\mb{.}''.

\otabtwo{
\mbs{DiracMatrix[{\sl mu},\ {\sl nu},\ ...]} & Dirac matrices
$\;\gamma^{\mu}\;\gamma^{\nu}\ldots$ \cr
\mbs{DiracSlash[p,\ q,\ ...]} & Feynman slashes $ \; \bps \, \bqs\ldots$ \cr
\mbs{DiracMatrix[5]} & $\gamma^{5}$ \cr
\mbs{ChiralityProjector[+1]} & $\omega_{+} = (1+\gamma^{5})/2$\cr
\mbs{DiracMatrix[6]} & $\gamma^{6} = (1+\gamma^{5})/2$\cr
\mbs{ChiralityProjector[-1]} & $\omega_{-} = (1-\gamma^{5})/2$ \cr
\mbs{DiracMatrix[7]} & $\gamma^{7} = (1-\gamma^{5})/2$ \cr
}{Various input functions for Dirac matrices.}

The \mma \mb{Dot}, ``\mb{.}'', is used in the input as noncommutative
%associative
multiplication operator for the objects \mb{DiracMatrix},
\mb{DiracSlash}, \mb{Spinor}, \mb{LeptonSpinor}, \mb{QuarkSpinor}
and \mb{SUNT}.
The ``\mb{.}'' may also be used as a delimiter instead of the ``\mb{,}''
within the functions \mb{DiracMatrix}, \mb{DiracSlash} and
\mb{SUNT}.

In the output with \mb{FeynCalcForm} the ``\mb{.}''
as noncommutative multiplication operator
is suppressed.

\beom
\domtog{(DiracSlash[p\ +\ q]\ +\ m)\ .\  DiracMatrix[$\mu$] }{\ $( m\ +\ \gamma\  \cdot \ (p\ +\ q) )\ \gamma^\mu$ \hfil\\
}{
This is how you enter $ ( \bps +\bqs + m )\, \gamma^{\mu}$.
}
\domtog{DiracMatrix[$\mu,\ 5,\ \nu,\ 6,\ \rho,\ 7$]}{\ $\gamma^{\mu}\,\gamma^{5}\,\gamma^{\nu}\,\gamma^{6}\,\gamma^{\rho}\,
\gamma^{7}$\hfil\\
}{
Here is $\gamma^{\mu}\,\gamma^{5}\,\gamma^{\nu}\,\gamma^{6}\,\gamma^{\rho}\,
\gamma^{7}$\,.
}
%\domtog{{DiracMatrix[$\mu\ .\ 5\ .\ \nu\ .\ 6\ .\ \rho\ .\ 7$]\\
%}}{\ $\gamma^{\mu}\,\gamma^{5}\,\gamma^{\nu}\,\gamma^{6}\,\gamma^{\rho}\,
%\gamma^{7}$\hfil\\
%}{
%This is an alternative input variety.
%Note that the spaces around the ``.'' are
%essential when $ \gamma^{5}, \gamma^{6}$ or $\gamma^{7}$
%are involved.
%}
\domtog{DiracSlash[2b, a, 2(d - c), 6q - 3p] }{\ $2 \gamma\cdot b\;\; \gamma\cdot a\;\; 2\, ( \gamma\cdot d -  \gamma\cdot c)\;\; (6 \, \gamma\cdot q - 3 \;\;  \gamma\cdot p)$]\hfil\\
}{
This is a four-dimensional product:\\
$2 \bs\, \as\, 2\, (\ds - \cs)\, (6 \,\qs - 3 \, \ps)$.
}
%\domtog{{DiracSlash[(2\ b)\ .\ a\ .\ (2\ (d\ -\ c))\ . \  (6\ q\ -\ 3\ p)] \\
%}}{\ 12\ gs[b]\ gs[a]\ gs[c\ -\ d]\ gs[p\ -\ 2\ q]\hfil\\
%}{
%Here is an alternative input with \mb{Dot}.
%Commutative products have to be grouped
%with parentheses here.
%}
\domtog{DiracSlash[Polarization[k]]}{\ $\gamma\cdot \varepsilon(k)$\hfil\\
}{
Slashed polarization vectors $\es(k)$ are entered in this
way.
}
\enom

\otabthree{
{\sl option name} & {\sl default value} &\cr
\hhline
\mb{Dimension} & \mb{4} &  space-time dimension \cr
} {Option for \mb{DiracMatrix} and  \mb{DiracSlash}.}

As setting for \mb{Dimension}, a \mma \mb{Symbol}
{\sl dim}, or {\sl dim}-4 are possible.

\beom
\domtog{DiracMatrix[$\mu$,\ Dimension\ $\Rule$\ D]}{\ $\gamma^{\mu}$\hfil\\
}{
Entering of a D-dimensional  $\gamma^{\mu}$.
The number of dimensions is suppressed in the screen
output (for notebook output).
}
\domtog{DiracMatrix[mu,\ mu,\ Dimension\ $\Rule$\ D\ -\ 4]}{\ $\gamma^{\mu} \gamma^{\mu}$\hfil\\
}{
This is  $\gamma^{\mu} \,  \gamma^{\mu}$ in $D-4$ dimensions.
}
\enom

\subsection{Spinors}

In \fc spinors are entered with the four functions \mb{SpinorU}, \mb{SpinorUBar}, \mb{SpinorV}, \mb{SpinorVBar}. Products of spinors with spinors or Dirac matrices (Fermion chains) are built with the \mma \mb{Dot}, ``\mb{.}''.
%The only difference between \mb{QuarkSpinor} and \mb{LeptonSpinor}
%is that Gell-Mann matrices (see section \ref{sun}) do not commute with quark spinors.

\otabtwo{
\mbs{SpinorU[{\sl p},\ {\sl m}]} & $u(p,m)$ \cr
\mbs{SpinorUBar[{\sl p},{\sl m}]} & $\overline{u}(p,m)$ \cr
\mbs{SpinorV[{\sl p},\ {\sl m}]} & $v(p,m)$\cr
\mbs{SpinorVBar[{\sl p},{\sl m}]} & $\overline{v}(p,m)$ \cr
} {The input functions for fermionic spinors.}

\beom
\domtog{SpinorUBar[p,\ m]\ .\ DiracSlash[p]}{\ $\overline{u}(p, m) \;\; \gamma\cdot p$\hfil\\
}{
This is $\overline{u}(p, m) \, \bps$.
}
\domtog{DiracSlash[p]\ .\ SpinorU[p,\ m]}{\ $\gamma\cdot p \;\; u(p,m)$\hfil\\
}{
On the right-hand side of $\bps$ the spinor,
$u(p,m)$ is used.
}
\domtog{SpinorVBar[p,\ m]\ .\ DiracSlash[q,\ p]}{\ $\overline{v}(p,m) \;\; \gamma\cdot q \,\, \gamma\cdot p$\hfil\\
}{
This is $\overline{v}(p,m) \, \qs \, \bps$.
}
\domtog{DiracSlash[p]\ .\ SpinorV[p,\ m]}{\ $\gamma\cdot p \;\; v(p,m)$\hfil\\
}{
Here we have $\gamma\cdot p \, v(p,m)$.
}
\domtog{SpinorU[p,\ 0]}{\ $u(p)$\hfil\\
}{
Enter a spinor obeying the massless Dirac equation.
The 0 is not displayed in the output.
}
\domtog{SpinorU[p]}{\ $u(p)$\hfil\\
}{
You may also omit the second argument for
a massless spinor.
}
\enom

%If you want to calculate polarized amplitudes at tree level, a third
%argument must be given to \mb{Spinor}.
%
%%
%\otabtwo{
%\mbs{Spinor[{\sl p},\ {\sl m},\ LeftHanded]} & a left-handed spinor \cr
%\mbs{Spinor[{\sl p},\ {\sl m},\ RightHanded]} & a right-handed spinor \cr
%\mbs{Spinor[{\sl p},\ {\sl m},\ {\sl n}]} & spinor with external $n$
%($n \cdot p = 0$)
%\cr} {Polarized spinors.}
%
%See the section on tree-level calculations
%for the different effects of the above specifications.

\subsection{SU$(N$) Matrices and Structure Constants}\label{sun}

%\fc is not a package especially designed for QCD calculations. But
%some basic features for Gell-Mann matrices are included.

\fc provides functions for the input of SU$(N$) matrices and the corresponding
structure constants. More specialized input functions in SU(2) and SU(3)
(Gell-Mann and Pauli matrices) are provided by the subpackage \fphi.

\otabtwo{
\mb{SUNT[{\sl a}]} & the {\sl a}'th generator, $T_a$, of SU($N$) in the fundamental representation \cr
} {Input of SU($N$) matrices.}

For noncommutative multiplication of SU($N$) matrices, use the \mma \mb{Dot}, ``\mb{.}''.

\beom
\domtog{SUNT[a]\ .\ SUNT[b]\ .\ SUNT[c]}{\ $T_a\, T_b\, T_c$\hfil\\
}{
This is $T_a\, T_b\, T_c$.
}
\domtog{SUNT[a,\ b,\ c] }{\ $T_a \, T_b \, T_c$\hfil\\
}{
The above can also be entered like this.
}
\domtog{SUNT[a,\ b,\ c]\ SUNDelta[c,\ d]}{\ $T_a \, T_b \, T_c \; \delta_{c\, d}$\hfil\\
}{
This is how you enter $T_a \, T_b \, T_c \; \delta_{c\, d}$.
}
\enom

\otabtwo{
\mbs{SUNDelta[i,j]} & Kronecker delta function in SU($N$) \cr
\mbs{SUNF[i,j,k]} & antisymmetric structure constants of SU($N$) \cr
\mbs{SUND[i,j,k]} & symmetric structure constants of SU($N$) \cr
} {Input of SU($N$) structure constants.}

\otabthree{
{\sl option name} & {\sl default value} &\cr
\hhline
\mb{Explicit} & \mb{True} & replace \mb{SUNF} or \mb{SUND} by traces \cr
} {An option for the structure constants \mb{SUNF} and \mb{SUND}.}

\beom
\domtog{SUNF[a,\ b,\ c]}{\ $f_{a\,b\,c}$\hfil\\
}{
This is how you enter $f_{a\,b\,c}$.
}
\domtog{SUND[a,\ b,\ c]}{\ $d_{a\,b\,c}$\hfil\\
}{
This is how you enter $d_{a\,b\,c}$.
}
\domtog{SUNF[a,\ b,\ c,\ Explicit\ $\Rule$\ True]}{\ $2\ \ComplexI\; (\,{\rm tr}(T_a\, T_c\, T_b)\, -\, {\rm tr}(T_a\, T_b\, T_c)\,)$\hfil\\
}{
They can be expressed in terms of the generating matrices (see section \ref{gelltrace} for a discussion of the trace).
}
\domtog{SUNF[a, b, c, Explicit $\Rule$ True]\ + \
SUNF[a, c, b, Explicit $\Rule$ True]\; //\; Expand}{\ 0\hfil\\
}{
Antisymmetry,
$f_{a\,c\,b}\,= -f_{a\,b\,c}$.  }
\domtog{SUND[a, b, c, Explicit $\Rule$ True]\ - \ SUND[a, c, b, Explicit $\Rule$ True]\; //\; Expand}{\ 0\hfil\\ }{Symmetry,
$d_{a\,c\,b}\,= d_{a\,b\,c}$.}
\enom

\subsection{Denominators of Propagators}
The denominators of propagators are entered in a special way.
Each one-loop Feynman amplitude has factors of the form
$$
df = \frac {1}{[q^{2}-m_{0}^{2}]\, [(q+p_{1})^{2}-m_{1}^{2}] \,
             [(q+p_{2})^{2}-m_{2}^{2}] \; \ldots
                     }
$$
with $q$ as loop momentum and the $p_{i}$ denoting linear combinations of the
external four-vectors.

\otabtwo{
\mbs{FeynAmpDenominator[\\ PropagatorDenominator[{\sl q},\ $\hbox{\sl m}_{0}$], \\
PropagatorDenominator[{\sl q}\ +\ {\sl p},\ $\hbox{\sl m}_{1}$],\ ...]}
&
$1/( [ q^{2} - m_0^{2} ] [(q+p)^{2} - m_1^{2}])\, \ldots$ \cr
\mbs{FAD[ \{{\sl q}, $\hbox{\sl m}_0 $\}, \{{\sl q-p,M}\}, \ldots ]}
&
$1/( [ q^{2} - m_0^{2} ] [(q+p)^{2} - m_1^{2}])\, \ldots$ \cr
} {Entering denominators of propagators.}

\beom
\domtog{FeynAmpDenominator[\ PropagatorDenominator[q,\ m1],\ PropagatorDenominator[q\ +\ p,\ m2]]}{\Frac{1}{( q^{2} - m_1^{2} ) ((q+p)^{2} - m_2^{2})}\hfil\\
}{
This is $1/( [ q^{2} - m_1^{2} ] [(q+p)^{2} - m_2^{2}])$.
}
\enom

\beom
\domtog{FAD[q, q-p, {k,M}]}{
\Frac{1}{ q^2 (q-p)^{2} (k^2 - M^{2})}\hfil\\
}{
This is $1/( [ q^{2} ] [(q-p)^{2}] [k^2 - M^{2}])$.
}
\enom

\subsection{Small Variables}

For radiative corrections in the standard model, fermionic masses
often can be neglected with respect to the gauge boson masses.
Since, however, some of the Passarino-Veltman integrals may be infrared divergent,
the fermionic (and photonic) small masses must remain as
arguments of them and cannot be set to 0.
In order to achieve this behavior, these small masses have to be
entered with  head \mb{SmallVariable}.
The idea is that every variable with head \mb{SmallVariable} evaluates
to $0$, unless it is an argument of a scalar integral.

You can avoid to explicitly wrap the head \mb{SmallVariable} around a mass by
using the option \mb{SmallVariables} of the function \mb{OneLoop}
(the function \mb{OneLoop} is described in section \ref{passvelt}).

\otabtwo{
\mbs{SmallVariable[{\sl m}]} & head of a small mass $m$, $m \ll M$
}{A head for small variables.}

%\beom
%\domtog{USpinor[p,\ Small[m]]}{\ u[p]\hfil\\
%}{
%In spinors, a mass with head \mb{Small} is replaced by 0.
%}
%\enom

\subsection{Quantum Fields}
\label{quantumFields}

\fc is meant to be a general framework for calculations in quantum field theory. Thus, \fc
defines notation not only for working with what comes out of a Feynman amplitude calculation, namely loop integrals, momenta, Dirac and SU($N$) structures, propagators, etc., but also for
the quantities needed to work out the input to such calculations, namely quantum fields and
lagrangians.

\otabtwo{
\mbs{QuantumField[{\sl par1}, {\sl par2}, ... , {\sl ftype}, \{{\sl lorind}\}, \{{\sl sunind}\}]} & denotes a quantum field of type {\sl ftype} with possible Lorentz indices {\sl lorind} and SU($N$) indices {\sl sunind}. The optional first arguments {\sl par1}, {\sl par2}, $\cdots$, are partial derivatives acting on the field \cr
\mbs{PartialD[{\sl $\mu$}]} & denotes a space-time derivative when given as first argument to \mb{QuantumField}
}{Input of quantum fields.}

\beom
\domtog{QuantumField[$\psi$]}{\ $\psi$\hfil\\
}{
A scalar field $\psi$.
}
\domtog{QuantumField[$\psi$, \{i\}]}{\ $\psi_i$\hfil\\
}{
If the field is considered as part of an SU($N$) multiplet and the SU($N$) index is $i$, this is the notation.
}
\domtog{QuantumField[$g$, \{$\nu$\}, \{i\}]}{\ $g_\mu^i$\hfil\\
}{
Some field $g$ with an SU($N$) index $i$ and a Lorentz index $\nu$.
}
\domtog{QuantumField[PartialD[$\mu$],$g$, \{$\nu$\}, \{i\}]}{\ $\partial_\mu g_\nu^i$\hfil\\
}{
This is the notation for the space-time derivative $\partial/\partial x^\mu$ of the same field.
}
\enom

\subsection{Propagators}

Several utility functions for input of propagators are available.

\otabtwo{
\mbs{GluonPropagator[{\sl p,  $\mu$,a,  $\nu$,b}]} & represents a gluon propagator $\Pi_{ab}^{\mu \nu}(p)$ \cr
\mbs{GP[{\sl p, 1, 2}]} & is a short form for \mbox{\mbs{GluonPropagator[{\sl p, li1,ci1, li2,ci2}]}}   \cr
\mbs{GhostPropagator[{\sl p, $\mu$,a, $\nu$,b}]} & represents a ghost propagator \cr
\mbs{GHP[{\sl p, 1,2 }]} & is a short form for \mbox{\mbs{GhostPropagator[{\sl p, li1,ci1, li2,ci2}]}}  \cr
\mbs{QuarkPropagator[{\sl p}]} & gives the massless quark propagator\cr
\mbs{QuarkPropagator[{\sl p,m}]} & gives the quark propagator with mass m\cr
\mbs{QP[{\sl p,m}]} & gives the quark propagator with mass m\cr
}{Input functions for gluon ghost and quark propagators.}

{\raggedleft
The options for \mbs{GluonPropagator} are:
}
\otabthree{
{\sl option name} & {\sl default value} & \cr
\hhline
\mb{Gauge} & \mb{1} &  gauge choice (1: Feynman)\cr
\mb{OPE} & \mb{0} &  whether to add the 2-gluon OPE Feynman rule\cr
\mb{CounterTerm} & \mb{False} &  whether to add counter terms\cr
\mb{CouplingConstant} & \mb{Gstrong} &  the coupling constant\cr
\mb{Dimension} & \mb{D} &  the space-time dimension\cr
}{Options for \mb{GluonPropagator}.}

\beom
\domtog{GluonPropagator[p,$\mu, a, \, \nu, b$ ]}{
$-\frac{\ImaginaryI \multsp {g^{\mu \nu }}\multsp
      {{\delta }_{a b}}}{{p^2}}
$
}{
Enter a Gluon propagator.
}
\enom

\beom
\domtog{GP[q,1,2]}{
$-\frac{\ImaginaryI \multsp {g^{i1 i2 }}\multsp
      {{\delta }_{c1 c2}}}{{p^2}}
$
}{
A convenient short way to enter a gluon propagator.
}
\enom

\beom
\domtog{GluonPropagator[p, \{$\mu$,a\}, \{$\nu$,b\}, Gauge->$\xi$]}{
$
\frac{\ImaginaryI \multsp \big(\frac{(1-\xi )\multsp ({{ p }^{\mu}})\multsp ({{ p }^{\nu}})}{{p^2}}-
 {g^{\mu \nu}}\big)\multsp {{\delta }_{a b}}}{{p^2}}
$
}{Specifying the general covariant gauge.}
\enom

\beom
\domtog{GluonPropagator[p, \{$\mu$,a\}, \{$\nu$,b\}, Gauge->$\{n, \alpha\}$]}{
$
\frac{\ImaginaryI \multsp \big(-{g^{\mu \nu}}-\frac{({{ p }^{\mu}})\multsp ({{ p }^{\nu}})
 \multsp {n^2}-\alpha \multsp ({{ n }^{\mu}})\multsp ({{ n }^{\nu}})\multsp {p^2}}{{{n
 \cdot p}^2}}+\frac{({{ p }^{\mu}})\multsp ({{ n }^{\nu}})+
 ({{ n }^{\mu}})\multsp ({{ p }^{\nu}})}{n \cdot p}\big)\multsp
 {{\delta }_{a b}}}{{p^2}}
$
}{Specifying an axial gauge.}
\enom

\beom
\domtog{GP[q,1,2, OPE->True]}{
$
-\frac{1}{2}\multsp \Omega \multsp ({{(-1)}^m}+1)\multsp
    {{\delta }_{\Mvariable{c1}\NoBreak \Mvariable{c2}}}\multsp
    (\Mfunction{O}_{\Mvariable{i1}\VeryThinSpace \Mvariable{i2}}^
     {\Mvariable{G2}}\Mfunction{(}p))\multsp
    {{\bigg(\frac{1}{{p^2}}\bigg)}^2}-
   \frac{\ImaginaryI \multsp
      {g^{\Mvariable{i1}\NoBreak \Mvariable{i2}}}\multsp
      {{\delta }_{\Mvariable{c1}\NoBreak \Mvariable{c2}}}}{{p^2}}
$
}{
A possible way of entering a gluon propagator and an OPE insertion.
}
\enom
The explicit form of the OPE-Operator can be obtained by applying the
\mb{Explicit} function.

\beom
\domtog{QuarkPropagator[p]}{
$-\frac{\ImaginaryI \multsp {\gamma \cdot p }}{p^2}
$
}{
Enter a quark propagator.
}
\enom

{\raggedleft The options for \mbs{QuarkPropagator} are:}
\otabthree{
{\sl option name}     & {\sl default value} & \cr
\hhline
\mb{CounterTerm}      & \mb{False}   &  whether to add counter terms\cr
\mb{CouplingConstant} & \mb{Gstrong} &  the coupling constant\cr
\mb{Dimension}        & \mb{D}       &  the space-time dimension\cr
\mb{Loop}             & \mb{0}       &  the loop order of the propagator \cr
\mb{OPE}              & \mb{False}   &  whether to add the 2-gluon OPE Feynman rule\cr
\mb{Polarization}     & \mb{0}       &  for OPE insertions this option is taken over \cr
}{Options for QuarkPropagator.}

\beom
\domtog{QuarkPropagator[p, Loop -> 1]}{
$
-\frac{\ImaginaryI \multsp {C_F} \multsp (2-\varepsilon )
\multsp g_{s}^{2}\multsp {S_n} \multsp \gamma \cdot p\multsp
      {{\big(-\frac{{p^2}}{{{\mu }^2}}\big)}^{\varepsilon /2}}}{\varepsilon }
$
}{
The 1-loop quark propagator.
}
\enom

\beom
\domtog{QuarkPropagator[p, OPE -> True]}{
$
-\frac{\ImaginaryI \multsp {C_F} \multsp (2-\varepsilon )
\multsp g_{s}^{2}\multsp {S_n} \multsp \gamma \cdot p\multsp
      {{\big(-\frac{{p^2}}{{{\mu }^2}}\big)}^{\varepsilon /2}}}{\varepsilon }
$
}{
The 1-loop quark propagator.
}
\enom


\subsection{Vertex functions}

\otabtwo{
\mbs{GluonVertex[{\sl p, $\mu$,a, q, $\nu$,b, k, $\lambda$,c}]} & represents a gluon vertex $V_{abc}^{\mu \nu \lambda}(p,q,k)$
 \cr
\mbs{GV[{\sl p,1, q,2, k,3}]} & is a short form for \mbox{\mbs{GluonVertex[{\sl p, i1,c1, q,i2,c2, k,i3,c3}]}}   \cr
}{Input functions for gluon vertex functions.}

\beom
\domtog{GluonVertex[{\sl p , $\mu$,a, q, $\nu$,b, k, $\lambda$,c}]\//\//Explicit}{
$
{g_s}\multsp ({p^{\lambda }}\multsp {g^{\mu \nu }}-{q^{\lambda }}\multsp {g^{\mu \nu }}-{g^{\lambda \nu }}\multsp {k^{\mu }}+
 {g^{\lambda \nu }}\multsp {q^{\mu }}+{g^{\lambda \mu }}\multsp {k^{\nu }}-{g^{\lambda \mu }}\multsp {p^{\nu }})\multsp {f_{abc}}
$
}{One way how to enter a 3-gluon Vertex.}
\enom

\beom
\domtog{GluonVertex[{\sl p , $\mu$,a, q, $\nu$,b, k, $\lambda$,c, r, $\sigma$,d}]\//\//Explicit}{
$
-\ImaginaryI \multsp g_{s}^{2}\multsp
 (({g^{\lambda \sigma }}\multsp {g^{\mu \nu }}-{g^{\lambda \mu }}\multsp {g^{\nu \sigma }})\multsp {f_{ad\Mvariable{u6}}}\multsp
 {f_{bc\Mvariable{u6}}}+
\\
(  {g^{\lambda \sigma }}\multsp {g^{\mu \nu }}-{g^{\lambda \nu }}\multsp {g^{\mu \sigma }})\multsp {f_{ac\Mvariable{u6}}}\multsp
 {f_{bd\Mvariable{u6}}}+({g^{\lambda \mu }}\multsp {g^{\nu \sigma }}-{g^{\lambda \nu }}\multsp {g^{\mu \sigma }})\multsp
 {f_{ab\Mvariable{u6}}}\multsp {f_{cd\Mvariable{u6}}})
$
}{One way how to enter a 4-gluon Vertex.}
\enom

\subsection{Twist 2 Operators}

$
\frac{1}{2}\multsp (1+{{(-1)}^m})\multsp g_{s}^{2}\multsp
 \big({f_{a\Mvariable{c3}d}}\multsp {f_{bc\Mvariable{c3}}}\multsp
 \big(\Mfunction{O}_{\Mvariable{la}\VeryThinSpace \nu\VeryThinSpace \mu\VeryThinSpace \Mvariable{si}}^{\Mvariable{G4}}\Mfunction{(
 }k,q,p,s)\big)-
\\ {f_{ac\Mvariable{c3}}}\multsp {f_{b\Mvariable{c3}d}}\multsp
 \big(\Mfunction{O}_{\mu\VeryThinSpace \Mvariable{la}\VeryThinSpace \nu\VeryThinSpace \Mvariable{si}}^{\Mvariable{G4}}\Mfunction{(}p,
 k,q,s)\big)-{f_{ab\Mvariable{c3}}}\multsp {f_{c\Mvariable{c3}d}}\multsp
 \big(\Mfunction{O}_{\mu\VeryThinSpace \nu\VeryThinSpace \Mvariable{la}\VeryThinSpace \Mvariable{si}}^{\Mvariable{G4}}\Mfunction{(}p,
 q,k,s)\big)\big)
$

\subsection{Propagators}

Several utility functions for input of propagators are available.

\otabtwo{
\mbs{GluonPropagator[{\sl p,  $\mu$,a,  $\nu$,b}]} & represents a gluon propagator $\Pi_{ab}^{\mu \nu}(p)$ \cr
\mbs{GP[{\sl p, 1, 2}]} & is a short form for \mbox{\mbs{GluonPropagator[{\sl p, li1,ci1, li2,ci2}]}}   \cr
\mbs{GhostPropagator[{\sl p, $\mu$,a, $\nu$,b}]} & represents a ghost propagator \cr
\mbs{GHP[{\sl p, 1,2 }]} & is a short form for \mbox{\mbs{GhostPropagator[{\sl p, li1,ci1, li2,ci2}]}}  \cr
\mbs{QuarkPropagator[{\sl p}]} & gives the massless quark propagator\cr
\mbs{QuarkPropagator[{\sl p,m}]} & gives the quark propagator with mass m\cr
\mbs{QP[{\sl p,m}]} & gives the quark propagator with mass m\cr
}{Input functions for gluon ghost and quark propagators.}

{\raggedleft
The options for \mbs{GluonPropagator} are:
}
\otabthree{
{\sl option name} & {\sl default value} & \cr
\hhline
\mb{Gauge} & \mb{1} &  gauge choice (1: Feynman)\cr
\mb{OPE} & \mb{0} &  whether to add the 2-gluon OPE Feynman rule\cr
\mb{CounterTerm} & \mb{False} &  whether to add counter terms\cr
\mb{CouplingConstant} & \mb{Gstrong} &  the coupling constant\cr
\mb{Dimension} & \mb{D} &  the space-time dimension\cr
}{Options for \mb{GluonPropagator}.}

\beom
\domtog{GluonPropagator[p,$\mu, a, \, \nu, b$ ]}{
$-\frac{\ImaginaryI \multsp {g^{\mu \nu }}\multsp
      {{\delta }_{a b}}}{{p^2}}
$
}{
Enter a Gluon propagator.
}
\enom

\beom
\domtog{GP[q,1,2]}{
$-\frac{\ImaginaryI \multsp {g^{i1 i2 }}\multsp
      {{\delta }_{c1 c2}}}{{p^2}}
$
}{
A convenient short way to enter a gluon propagator.
}
\enom

\beom
\domtog{GP[q,1,2, OPE->True]}{
$
-\frac{1}{2}\multsp \Omega \multsp ({{(-1)}^m}+1)\multsp
    {{\delta }_{\Mvariable{c1}\NoBreak \Mvariable{c2}}}\multsp
    (\Mfunction{O}_{\Mvariable{i1}\VeryThinSpace \Mvariable{i2}}^
     {\Mvariable{G2}}\Mfunction{(}p))\multsp
    {{\bigg(\frac{1}{{p^2}}\bigg)}^2}-
   \frac{\ImaginaryI \multsp
      {g^{\Mvariable{i1}\NoBreak \Mvariable{i2}}}\multsp
      {{\delta }_{\Mvariable{c1}\NoBreak \Mvariable{c2}}}}{{p^2}}
$
}{
A convenient short way to enter a gluon propagator and an OPE insertion.
}
\enom

\beom
\domtog{QuarkPropagator[p]}{
$-\frac{\ImaginaryI \multsp {\gamma \cdot p }}{p^2}
$
}{
Enter a quark propagator.
}
\enom

{\raggedleft The options for \mbs{QuarkPropagator} are:}
\otabthree{
{\sl option name}     & {\sl default value} & \cr
\hhline
\mb{CounterTerm}      & \mb{False}   &  whether to add counter terms\cr
\mb{CouplingConstant} & \mb{Gstrong} &  the coupling constant\cr
\mb{Dimension}        & \mb{D}       &  the space-time dimension\cr
\mb{Loop}             & \mb{0}       &  the loop order of the propagator \cr
\mb{OPE}              & \mb{False}   &  whether to add the 2-gluon OPE Feynman rule\cr
\mb{Polarization}     & \mb{0}       &  for OPE insertions this option is taken over \cr
}{Options for QuarkPropagator.}


\subsection{Vertex functions}

\otabtwo{
\mbs{GluonVertex[{\sl p, $\mu$,a, q, $\nu$,b, k, $\lambda$,c}]} & represents a gluon vertex $V_{abc}^{\mu \nu \lambda}(p,q,k)$
 \cr
\mbs{GV[{\sl p,1, q,2, k,3}]} & is a short form for \mbox{\mbs{GluonVertex[{\sl p, i1,c1, q,i2,c2, k,i3,c3}]}}   \cr
}{Input functions for gluon vertex functions.}


\subsection{Twist 2 Operators}


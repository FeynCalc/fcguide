\sectionmark{Preface}
\section*{Preface}

This book is for practitioners of calculations in perturbative QCD,
the Standard Model, Chiral Perturbation Theory and other quantum field theories.
The intention is to serve educational as well as research purposes.
This is reflected in the organization of the text:
The first part is a tutorial-like introduction with many small examples.
It is necessarily of a rather technical character and should prepare the
reader for doing concrete physics calculations.
The second part contains a variety of example calculations from textbook
to research level. The interested student will here find calculations 
from the literature, worked through to a detail not usual in textbooks.
The researcher will find calculations that should provide a starting point
for calculations in his or her own field of research.
The third part is a reference guide, listing all important functions,
along with smaller examples.

Over the years, the detailed course of and motivation for the development of \fc has
changed somewhat both in scope and focus. Reasons for this include the
different research interests of the authors, the relative slowness of \mma and the development of
multi-loop calculations with other scientific software packages - in
particular FORM and programs based on FORM \cite{Vermaseren:2000nd,
Hahn:1998yk}. While such packages are typically faster than \fc and thus suited
for very large scale calculations, we believe \fc has kept relevance by being a
relatively comprehensive and complete framework while also exhibiting the
user-friendliness and versatility of a pure \mma package.

None the less, some effort has been put in the development of \fc tools,
precisely for interacting with FORM.
These tools allow taking advantage of FORM's efficient handling of large expressions, which can be simply
unmanageable with \mma, while staying within the \mma system, which as a
general purpose computer algebra system, has an intuitive and flexible
user interface and built-in mathematical knowledge.

Another focus has been to explore other types of calculations than the evaluation of Feynman
diagrams, e.g. complex algebraic manipulations of quantum fields and lagrangians, traditionally done by hand only.
Such calculations have become feasible because of the steep increase of the
computing power of commodity hardware.

Despite the shifting motivations driving development, the central vision remains
largely the same:
\fc was conceived as and continues to be developed as a general-purpose, relatively easy to use,
modular toolbox for quantum field theory calculations.
Dirac algebra utilities, Lorentz and color algebra manipulation capabilities, 
functional derivation utilities, integral and other tables are encapsulated in
distinct functions which do well-defined operations.
Such basic functions can then be used on a higher level to build sophisticated model dependent
functions.

The most important general principles for quantum field theory calculations that the \fc project
strives to promote are:
%%SOFAR

\begin{itemize}

%
%\item{{\bf Standardization.} In order to reuse code, clear standards should be formulated,
%thoroughly documented and followed. The aim should be to reach common standards with
%other similar or related projects (using other programming languages) for fields, lagrangians,
%Lorentz-vectors, etc. in the end allowing easy comparison of calculations of the same process.}

\item{{\bf Systematic knowledge build-up.} In physics calculations, errors due to typos or
other trivial mistakes are not acceptable, therefore the code used should be as thoroughly
checked as possible by as many people as possible. This implies openness and adherence to
standards. It is proposed to build up a repository of lagrangians, amplitudes and integrals,
stored in a standardized electronic form.}

\item{{\bf Prototyping.} Electronic tools should be put to use for making it easy to
try out new quantum field theory models. This should be achieved by providing predefined
building blocks capturing the more general physics and the more common calculational tasks.}

\item{{\bf Checking.} Calculations should, when possible, be performed by different methods.
Thus, \fc should be used to check hand calculations or the results of other programs and
vice versa.}

\end{itemize}

\fc is an open-source project and contributions are warmly welcomed. These are
best done by checking out a working copy of code on
\href{http://github.org/FeynCalc/feyncalc}{Github} and then getting in touch via
the \href{http://www.feyncalc.org/}{website}. Though
much testing of the code has been done, there is absolutely no claim that \fc is
bug free. Users are encouraged to be sceptical about results, and when sure that
the program returns a wrong answer, to report it via the website.

\subsection*{Version History}

The roots of \fc go back to 1987. 
During a stay as a graduate student in Albuquerque, New Mexico, Rolf Mertig learned to program in Macsyma \cite{Drinkard:1981dr} from the experts Stanly Steinberg and Michael Wester. Back in Germany, 
elementary particle physicists needed automation of the calculation of Feynman diagrams of eletroweak 
processes to one loop. The algorithms were partially developed together with 
Ansgar Denner and Manfred B\ODoubleDot{}hm and implemented in a 
purely functional way by Rolf Mertig during the years 1987 - 1989. 

The basic idea was to have general functions for some of the more mechanical parts of the diagram calculations, generalizable tools for use in calculations of different 1-loop processes, especially 1\(\rightarrow \)2 and 2\(\rightarrow \)2 processes. These included tools for:

\begin{itemize}

\item{Lorentz algebra.
\({g^{\Mvariable{\mu \nu }}}\multsp {p_{\mu }}
\multsp \rightarrow \multsp {p^{\nu }}
\multsp ,\multsp \multsp\)
\({g^{\Mvariable{\alpha \beta }}}{g_{\Mvariable{\alpha \beta }}}
   \multsp \rightarrow \multsp D\multsp \cdots\),}

\item{Dirac algebra.
\({{\gamma }^{\alpha }}\multsp {{\gamma }^{\nu }}
{{\gamma }_{\alpha }}\multsp \rightarrow \multsp 
(2-n)\multsp {{\gamma }^{\nu }},\multsp \multsp \)
\(\overvar{v}{\_}(p,m)\multsp 
\bps\)  \(\rightarrow \) - m \(\overvar{v}{\_}(p,m)\multsp \cdots\),}

\item{Passarino-Veltman tensor integral decomposition.
Applications of these tools included complete 1-loop processes in the Standard Model,
\({{\rm e}^+}{{\rm e}^-}\multsp \rightarrow 
\multsp {\rm Z}\multsp {\rm H}\multsp \),  \(\multsp {{\rm e}^+}{{\rm e}^-}\multsp \rightarrow \)
\(\multsp {{\rm W}^+}{{\rm W}^-}\multsp \)
and the 2-loop photon self-energy in QED.}

\end{itemize}

At the end of 1989 several problems showed up with the Macsyma implementation. The purely
functional programming style proved to be difficult to debug and, in fact, inappropriate. The
rudimentary pattern matcher in Macsyma was not adequate. There was no way to incorporate new
functions easily into the whole Macsyma system, and no possibility of providing online
documentation. Furthermore, Macsyma's memory management ("garbage collection") was slow when
handling large expressions.

In early 1990 it became clear that \mma was a much more natural programming environment for \fc.

\subsubsection*{1990-1991 : The first version  of \fc in \mma}

In 1990 user-friendly  packages were built with extended automatic capabilities (\mb{OneLoop},
\mb{OneLoopSum}), and SU(3) algebra capabilities were added (\mb{SU3Simplify}).

In 1991 initial documentation was written and the program was made available on anonymous
ftp-servers: mathsource.wolfram.com and canc.can.nl.

Applications from 1990 - 1996  included:

\begin{itemize}

\item 1-loop 2 \(\rightarrow \)2 processes in the Standard
Model, such as:

\({\rm gg}\multsp \rightarrow \multsp 
{\rm t \overline{t}} \multsp , \multsp \multsp
{{\rm e}^+}{{\rm e}^-}\multsp \rightarrow \multsp {\rm Z}\multsp {\rm H}\multsp, \multsp \multsp
%{{\rm W}^+}{{\rm W}^-}\multsp \)\(\rightarrow \) \({{\rm W}^+}{{\rm W}^-}\multsp, \multsp \multsp
{\rm gg} \rightarrow {\rm t\bar{t}} \),

\item background field gauge calculations,

\item high-energy approximation of \(\multsp{{\rm e}^+}
{{\rm e}^-}\multsp \rightarrow \multsp {{\rm W}^+}{{\rm W}^-}\multsp \),

\item  2-loop Standard Model self-energies.

\end{itemize}

No real attempt was made to provide tools for tree-level calculations.
For this purpose other programs appeared, among them another  \mma package, \hip \cite{yeh}, developed at SLAC, 
CompHEP \cite{Pukhov:1999gg}, and of course, FORM \cite{Vermaseren:2000nd}.

\subsubsection*{1992-1995 : \fc 2.0-2.2, unification and simplification.}

During this period of development the SU(3) algebra was changed to SU($N$). Several tools for
automatic tree-level calculation were added, for example the function \mb{SquareAmplitude}.
(Unfortunately, the  documentation was not updated.) All sub-packages were put into one file
\(\big(\simeq {{10}^4}\) lines), "FeynCalc2.2beta.m" \footnote{This file is still available upon
request}.

\subsubsection*{1993-1996 : \fc 3.0, modularization, typesetting, Operator Product Expansion}

Due to the rapidly increasing amount of code, totalling 2.5 megabytes, \fc was
reorganized in a modular way. The definitions of each function were contained in
a package file which was loaded only on demand.

\begin{itemize}

\item
In these years many improvements were made to the code, driven by the work of the author
(Rolf Mertig) together with Willy van Neerven in perturbative QCD.

One effort was to build databases: Convolutions, integrals, tensor integral transformation
formulae, Feynman rules,
and Feynman parameterizations. 
Applications include 2-loop spin-independent and spin-dependent Altarelli-Parisi splitting
functions \cite{Altarelli:1977zs}.

\item The new typesetting capabilities of \mma 3 (TraditionalForm) were used to substantially
improve the look of the output.
Typesetting rules were added for e.g. \({\overvar{\partial }{\rightarrow}}_{\mu } \multsp \).

\item Code was written to allow new abstract data types, for example for noncommutative algebra
and for special integrals.

\item QCD tools for OPE were added.

\item Automatic Feynman rule derivation (by functional
derivation) was implemented in order to get special Feynman rules
for twist-2 (and higher) operators.

\end{itemize}

\subsubsection*{1997-2000 : \fc 4.0-4.1, QCD and OPE, ChPT, tables}

The package \tarcer, which was initially an independent project, was integrated
into \fc. \tarcer \cite{Mertig:1998vk} adds two-loop functionality for propagator-type integrals
using the recurrence relations of Tarasov \cite{Tarasov:1997kx}.

In 2000, Frederik Orellana, then a PhD student at the University of Zurich,
started contributing to the package.
He worked in Chiral Perturbation Theory \cite{Gasser:1984gg} (ChPT) and made some changes, as well
as adding code to \fc in order to support ChPT and effective theories in general, and interfacing
with \fa. Eventually, he integrated his package, PHI, into FeynCalc. The
ambition of PHI is to allow full automation of Feynman diagram calculations,
starting from a Lagrangian - as opposed to starting from a \fa model definition,
i.e. to automatize the generation of a \fa model and to allow the \fa output to
be directly used by \fc.
% Also, first implementations of the 'tHooft-Veltman formulae \cite{'tHooft:1978xw} for
% numerical evaluation of the one-loop integrals $B_0, C_0$ and $D_0$ were added.

\subsubsection*{2000-2014 : \fc 5-8, maintenance}

With both authors out of theoretical physics, no new developments were
undertaken, but the mailing list remained active and bugs were fixed.

\subsubsection*{2014 : \fc 9, cleanup, testing}

In 2014, Vladyslav Shtabovenko, has joined the team. He is working on effective
theories in quarkonium production as a PhD student at the Technical University
of Munich. To \fc, he has contributed work on Dirac gamma schemes and a unit
testing framework.

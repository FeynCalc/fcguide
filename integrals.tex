\section{Tensor and Scalar Integrals}

In this section is described the capabilities of \fc to reduce one-loop Feynman diagrams.
If is also discussed how to provide input for \fc and how to further process the output of \fc.

The methods and conventions implemented in \fc for the evaluation of one-loop diagrams are
described in \cite{ansgar} and \cite{feyncalc}. The usual Passarino-Veltman scheme for the
one-loop integrals is adapted to a large extent \cite{ansgar}.  The coefficient functions of the
tensor integrals are defined  similar to \cite{ansgar}, except that the Passarino-Veltman
integrals take internal masses squared  as arguments.

\fc can reduce all $n$-point integrals with $n\leq 4$ to scalar integrals $A_0, B_0, C_0$ and
$D_0$.

For the numerical evaluation of scalar integrals, several possibilities exist: 1) The program
FormF by M. Veltman. Unfortunately this only runs in CDC and 68000 assembler and is thus not a
realistic alternative. 2) The library \FF by G.J.~van Oldenborgh \cite{Ol91} written in Fortran
77. The library can be put to use with the \lpts package \cite{Hahn:1998yk} written by Thomas
Hahn, which features a Fortran, c++, and \mma interface. Loading the \mma package will simply
cause the functions $A_0, B_0, C_0$ and $D_0$ (and $E_0$ and $F_0$) to return numerical values
when given numerical arguments. This is the recommended procedure for obtaining numerical output
from \fc results. A more simple Fortran wrapper program to link FF and \fc is available from Rolf
Mertig. 3) The optional subpackage \fphi provides some functions for evaluating $B_0$ and $C_0$.
These are written purely in \mma but are not very well tested.

\subsection{Passarino-Veltman Integrals and Reduction of Coefficient Functions}
\label{passvelt}

The scalar integrals $A_0, B_0, C_0$ and $D_0$ are represented
in \fc as functions with all arguments consisting of scalar products or 
masses squared. Thus $A_0$ has one, $B_0$ three, $C_0$ six, and 
$D_0$ ten arguments. The symmetry properties of the arguments  are 
implemented, i.e., a standard representative of all possible 
argument permutations of each $ B_0, C_0$ and $D_0$ is returned.
For example, \mb{B0[pp,\ m2\phat 2,\ m1\phat 2]} \ra \mb{B0[pp,\ m1\phat 2,\ m2\phat 2]}, where 
\mb{pp} denotes the scalar product $(p\cdot p) = p^2$.

\otabtwo{
\mbs{A0[{\sl m02}]} &$(i\pi^2)^{-1} \int d^{D}q (q^{2}-m_0^{2})^{-1}$ \cr
\mbs{B0[{\sl p10},{\sl m02},{\sl m12}]} & 
$(i\pi^2)^{-1}\int  d^{D}q ([(q^{2}-m_0^{2}][(q+p_1)^{2}-m_1^{2}])^{-1}$ \cr
\mbs{DB0[{\sl p10},{\sl m02},{\sl m12}]} &
$\partial B_0(p_1^2,m_0^2,m_1^2)/\partial p_1^2$ \cr
\mbs{C0[{\sl p10},{\sl p12},{\sl p20},{\sl m02},{\sl m12}, {\sl m22}]} &
$(i\pi^2)^{-1} \int  d^{D}q 
([q^{2}-m_0^{2}][(q+p_1)^{2}-m_1^{2}][(q+p_2)^{2}-m_2^{2}])^{-1} $ \cr
\mbs{D0[{\sl p10},{\sl p12},{\sl p23},{\sl p30},{\sl p20},{\sl p13}, {\sl m02},{\sl m12},{\sl m22},{\sl m32}]} &
$(i\pi^2)^{-1} \int  d^{D}q ([q^{2}-m_0^{2}][(q+p_1)^{2}-m_1^{2}][(q+p_2)^{2}-m_2^{2}]
[(q+p_3)^{2}-m_3^{2}])^{-1} $ 
} {Scalar Passarino-Veltman functions $A_0$, $B_0$,$B_0'$, $C_0$ and $D_0$.}

In \fc and in the mathematical definitions given above,
the factor $(2\pi\mu)^{(4-D)}$ with the scaling variable $\mu$ is suppressed.
The convention for the scalar arguments is
{\sl pi0} $= p_i^2 \,\; \;${\sl pij} $= (p_i - p_j)^2, \; \; ${\sl mi2} $= m_i^2$.

\beom
\domtog{{B0[s,\ mz2,\ mw2]\\
}}{\ $B_0(s,\ {\rm mw2},\ {\rm mz2})$\hfil\\
}{
The $B_0$ function is symmetric in the mass arguments.
}
\domtog{{D[\%,\ s]\\
}}{\ DB0($s$,\ mw2,\  mz2)\hfil\\
}{
Taking the derivative with respect to the first argument yields
\mb{DB0}.
}
\enom

The tensor-integral decomposition is automatically done by
\fc when calculating one-loop amplitudes, 
but extra functions are provided to reduce the coefficients of the 
tensor-integral decomposition.

For fixing the conventions of the coefficient functions
the definitions of the tensor-integrals and the decomposition are given below.
In general the one-loop tensor integral is 
\[
T^{N}_{\mu_{1}\ldots\mu_{P}} (p_{1},\ldots,p_{N-1},m_{0},\ldots,m_{N-1})=
\frac{(2\pi\mu)^{4-D}}{i\pi^{2}}\int d^{D}\!q
\frac{q_{\mu _{1}}\cdots q_{\mu _{P}}}{
{\mathcal D}_{0}{\mathcal D}_{1}\cdots
{\mathcal D}_{N-1}
}
\]
with the denominator factors
\[
{\mathcal D}_{0}=
q^{2}-m_{0}^{2} \nonumber \qquad 
{\mathcal D}_{i}=
(q+p_{i})^{2}-m_{i}^{2}, \nonumber \qquad i=1,\ldots,N-1 \nonumber
\]
originating from the propagators in the Feynman diagram.
The $i \varepsilon $ part of the denominator factors is suppressed.
%Currently \fc is limited to $P\leq N\leq 4$ without $P=N=4$.

The tensor integral decompositions for the integrals that \fc can
do are listed below. The coefficient functions $B_i, B_{ij},
C_i, C_{ij}, C_{ijk}, D_i, D_{ij}, D_{ijk}$ and $D_{ijkl}$ 
are totally symmetric in their indices.

\[
\begin{array}{lll}
\hphantom{C_{\mu \nu \rho }} &&
\hphantom{\mbox{}(g_{\mu \nu }p_{1\rho }+g_{\nu \rho }p_{1\mu }+
g_{\mu \rho }p_{1\nu }) C_{001} + (g_{\mu \nu }p_{2\rho }+g_{\nu \rho }
p_{2\mu }+ g_{\mu \rho }p_{2\nu }) C_{002}}\\[-.7em]
B_{\mu } & = & p_{1\mu } B_{1} \\[1em]
B_{\mu \nu } & = & g_{\mu \nu } B_{00} + p_{1\mu } p_{1\nu } B_{11}
\end{array}
\]
\[
\begin{array}{lll}
C_{\mu } & = & p_{1\mu } C_{1} + p_{2\mu } C_{2}
= \disp\sum_{i=1}^{2} p_{i\mu} C_{i} \\[1.2em]
C_{\mu \nu } & = & g_{\mu \nu }C_{00} + p_{1\mu }p_{1\nu }C_{11} + p_{2\mu}
p_{2\nu }C_{22} +(p_{1\mu }p_{2\nu }+p_{2\mu }p_{1\nu }) C_{12} \\[1ex]
&=& g_{\mu \nu }C_{00} +
\disp\sum_{i,j=1}^{2} p_{i\mu} p_{j\nu }C_{ij} \\[1.2em]
C_{\mu \nu \rho } &=& \mbox{}(g_{\mu \nu }p_{1\rho }+g_{\nu \rho }p_{1\mu }+
g_{\mu \rho }p_{1\nu }) C_{001} + (g_{\mu \nu }p_{2\rho }+g_{\nu \rho }
p_{2\mu }+ g_{\mu \rho }p_{2\nu }) C_{002}  \\[1ex]
&&\mbox{} + p_{1\mu }p_{1\nu }p_{1\rho } C_{111}
+ p_{2\mu }p_{2\nu }p_{2\rho } C_{222} \\[1ex]
&&\mbox{} +(p_{1\mu }p_{1\nu }p_{2\rho }+
p_{1\mu }p_{2\nu }p_{1\rho }+
p_{2\mu }p_{1\nu }p_{1\rho }  ) C_{112} \\[1ex]
&&\mbox{} +(p_{2\mu }p_{2\nu }p_{1\rho } + p_{2\mu }p_{1\nu }p_{2\rho }+
p_{1\mu }p_{2\nu }p_{2\rho } ) C_{122} \\[1ex]
&=& \disp\sum_{i=1}^{2}(g_{\mu \nu }p_{i\rho }+g_{\nu \rho }p_{i\mu }
+ g_{\mu \rho }p_{i\nu }) C_{00i}
+ \disp \sum_{i,j,k=1}^{2} p_{i\mu} p_{j\nu }p_{k\rho}C_{ijk} \\[1ex]
%\end{array}
%\]
%\[
%\begin{array}{lll}
%\hphantom{C_{\mu \nu \rho }} &&
%\hphantom{\mbox{}(g_{\mu \nu }p_{1\rho }+g_{\nu \rho }p_{1\mu }+
%g_{\mu \rho }p_{1\nu }) C_{001} + (g_{\mu \nu }p_{2\rho }+g_{\nu \rho }
%p_{2\mu }+ g_{\mu \rho }p_{2\nu }) C_{002}}\\[0em]
D_{\mu } &=& \disp\sum_{i=1}^{3} p_{i\mu} D_{i} \\[1em]
D_{\mu \nu } &=& g_{\mu \nu }D_{00}
+ \disp\sum_{i,j=1}^{3} p_{i\mu} p_{j\nu }D_{ij} \\[1em]
D_{\mu \nu \rho } &=&
\disp\sum_{i=1}^{3}(g_{\mu \nu }p_{i\rho }+g_{\nu \rho }p_{i\mu }
+ g_{\mu \rho }p_{i\nu }) D_{00i}
+ \sum_{i,j,k=1}^{3} p_{i\mu} p_{j\nu }p_{k\rho}D_{ijk} \\[1.4em]
D_{\mu \nu \rho \sigma } &=&
(g_{\mu\nu}g_{\rho \sigma} + g_{\mu\rho}g_{\nu \sigma }
+ g_{\mu\sigma}g_{\nu\rho }) D_{0000}\\ [.8em]
&&\mbox{}+\disp\sum_{i,j=1}^{3}(g_{\mu \nu }p_{i\rho}p_{j\sigma}
+ g_{\nu \rho }p_{i\mu }p_{j\sigma} + g_{\mu \rho }p_{i\nu}p_{j\sigma}
+ g_{\mu \sigma}p_{i\nu}p_{j\rho }+ g_{\nu \sigma}p_{i\mu}p_{j\rho }
+ g_{\rho \sigma}p_{i\mu}p_{j\nu}) D_{00ij} \\[.8em]
&&\mbox{}+\disp\sum_{i,j,k,l=1}^{3} p_{i\mu} p_{j\nu
}p_{k\rho}p_{l\sigma}D_{ijkl}
\end{array}
\]

All coefficient functions and the scalar integrals
are summarized in one generic function, \mb{PaVe}.

\otabtwo{
\mbs{PaVe[{\sl i},\ {\sl j},\ ...,\ \{{\sl P10},\ {\sl P12},\ ...\},\ \{{\sl m02},\ {\sl m12},\ ...\}]}&
Passarino-Veltman coefficient functions
} {Passarino-Veltman coefficient functions of the tensor integral decomposition.}

The first set of arguments \mb{{\sl i},\ {\sl j},\ ...} are exactly those 
indices of the coefficient functions of the tensor integral decomposition.
If only a \mb{0} is given as first argument, the scalar integrals are 
understood. The last argument, the list of inner masses 
$m_0^2,\, m_1^2,\, ...$, determines whether a one-, two-,
three- or four-point function is meant. 
\mb{PaVe} is totally symmetric in the  \mb{{\sl i},\ {\sl j},\ ...} arguments.
The foremost argument is the list of scalar products of the $p_i$.
They are the same as defined above for the scalar $B_0$, $C_0$ and 
$D_0$ functions. For $A_0$ an empty list has to be given.

A certain set of special \mb{PaVe} shown in the following 
examples simplify to the usual notation.

To shorten the input squared masses are abbreviated with a suffix 2,
i.e., a mass $m^2$ is denoted by \mb{m2}.
The scalar quantity $p^2$ is entered as \mb{pp},
$p_i^2$ as \mb{pi0} and  $(p_i-p_j)^2$ as \mb{pij}.


\beom
\dtog{SetOptions[A0,\ A0ToB0\ $\Rule$\ False];\\
\ SetOptions[\{B1,\ B00,\ B11\},\ BReduce\ $\Rule$\ False];
}{}{
For the following examples some options are set.
}
\domtog{{PaVe[0,\ \{\},\ \{m02\}]\\
}}{\ $A_0$(m02)\hfil\\
}{
This is the scalar Passarino-\\
Veltman one-point function.
}
\domtog{{PaVe[0,\ \{pp\},\ \{m02,\ m12\}]\\
}}{\ $B_0$(pp,\ m02,\ m12)\hfil\\
}{
This is the two-point function \\
$B_0(p^2, m_0^2, m_1^2)$.
}
\domtog{{PaVe[1,\ \{pp\},\ \{m12,\ m22\}]\\
}}{\ $B_1$(pp,\ m12,\ m22)\hfil\\
}{ 
Here is the coefficient function \\ $B_1(p^2,m_0^2,m_1^2)$.
}
\domtog{{PaVe[0,0,\ \{pp\},\ \{m02,\ m12\}]\\
}}{\ $B_{00}$(pp,\ m02,\ m12)\hfil\\
}{
Coefficient functions of metric \\
tensors have two "0" indices for each $g^{\mu \nu}$.
}
\domtog{{PaVe[1,1,\ \{p10\},\ \{m12,\ m22\}]\\
}}{\ $B_{11}$(p10,\ m12,\ m22)\hfil\\
}{
This is $B_{11}(p_1^2,m_1,m_2)$, the coefficient 
function of $p_{1\mu} \,p_{1\nu}$ of $B_{\mu\nu}$.
}
\domtog{{PaVe[0,\ \{p10,\ p12,\ p20\},\ \{m12,\ m22,\ m32\}]\\
}}{\ $C_0$(p10,\ p12,\ p20,\ m12,\ m22,\ m32)\hfil\\
}{
\mb{PaVe} with one $0$ as first argument are scalar Passarino-Veltman integrals. 
}
\domtog{{PaVe[0,\ \{p10,\ p12,\ p23,\ p30,\ p20,\ p13\},\ \\
\{m12,\ m22,\ m32,\ m42\}]\\
}}{\ $D_0$(p10,\ p12,\ p23,\ p30,\ p20,\ p13,\ m12,\ m22,\ m32,\ m42)\hfil\\
}{
This is $D_0$ with 10 arguments.
}
\enom

\otabtwo{
\mbs{B1[{\sl p10},\ {\sl m02},\ {\sl m12}]} & coefficient function $B_1(p_1^2,m_0^2,m_1^2)$
 \cr
\mbs{B00[{\sl p10},\ {\sl m02},\ {\sl m12}]} & coefficient function 
$B_{00}(p_1^2,m_0^2,m_1^2)$\cr
\mbs{B11[{\sl p10},\ {\sl m02},\ {\sl m12}]} & coefficient function $
B_{11}(p_1^2,m_0^2,m_1^2)$
} {Two-point coefficient functions.}

The two-point coefficient functions can be reduced to lower-order 
ones.  For special arguments, also $B_0$ is expressed  in terms of $A_0$,
if the option \mb{BReduce} is set. Setting the option 
\mb{B0Unique} to \mb{True} simplifies $B_0(m^2,0,m^2) \rightarrow
2 + B_0(0,m^2,m^2)$ and $B_0(0,0,m^2)  \rightarrow 1  + B_0(0,m^2,m^2)$.

\otabthree{
{\sl option name} & {\sl default value} &\cr
\hhline
\mb{A0ToB0} & \mb{False} & express $A_0(m^2)$ by 
$m^2 (1 + B_0(0, m^2, m^2))$\cr
\mb{BReduce} & \mb{True}, \mb{False} for \mb{B0}& reduce \mb{B0}, \mb{B1}, \mb{B00}, \mb{B11} to \mb{A0} and \mb{B0} \cr
\mb{B0Unique} & \mb{False} & simplify $B_0(a,0,a)$ and $B_0(0,0,a)$
} {Reduction options for \mb{A0} and the two-point functions.}

\beom
\domtog{{B1[pp,\ m12,\ m22]\\
}}{\renewcommand{\clineskip}{36pt}-\Frac{B_0{\rm (pp,\ m12,\ m22)}}{2}\ +\ \\
\Frac{{\rm (-m12\ +\ m22)}\ (-B_0{\rm (0,\ m12,\ m22)}\ +\ B_0{\rm (pp,\ m12,\ m22)})}{{\rm 2\ pp}}\hfil\\
\renewcommand{\clineskip}{20pt}}{
The default is to reduce $B_1$ to $B_0$.
}
\enom

Arguments of two-point functions with head \mb{SmallVariable} are replaced 
by 0, if the other arguments have no head \mb{SmallVariable} 
and are nonzero. \mb{A0[0]} and \mb{A0[SmallVariable[m]\phat 2]} simplify to $0$.

\beom
\domtog{{B1[pp,\ SmallVariable[me2],\ m22]\\
}}{\renewcommand{\clineskip}{36pt}-\Frac{B_0{\rm (pp,\ 0,\ m22)}}{2}\ +\ \\
\Frac{{\rm m22}\ (-B_0{\rm (0,\ 0,\ m22)}\ +\ B_0{\rm (pp,\ 0,\ m22)})}{\rm 2\ pp}\hfil\\
\renewcommand{\clineskip}{20pt}}{
The small mass \mb{m}
is set to 0, since the other arguments are 
non-zero and not \mb{SmallVariable}.
}
\domtog{{B1[SmallVariable[me2],\ SmallVariable(me2),\ 0]\\
}}{\ -\Frac{1}{2}\ -\ \Frac{\rm B0(me2,\ 0,\ me2)}{2}\hfil\\}{
But in this case no arguments are replaced by 0.
\mb{SmallVariable} heads are not displayed int \mb{TraditionalForm}}
\domtog{\% // StandardForm\\
}{\ -\Frac{1}{2}\ -\ \Frac{\rm B0[SmallVariable[me2],\ 0,\ SmallVariable[me2]]}{2}\hfil\\}{
In \mb{StandardForm} we see them.
}
\enom

Any mass with head \mb{SmallVariable} is neglected if it appears in a
sum, but not as an argument of Passarino-Veltman (\mb{PaVe}) functions or 
\mb{PropagatorDenominator}.

\otabtwo{
\mbs{SmallVariable[{\sl var}]} & is a small (negligible) variable \cr
} {Head for small variables.}

\otabtwo{
\mbs{PaVeReduce[{\sl expr}]} & reduces coefficient functions \mb{PaVe} to 
\mb{A0}, \mb{B0},  \mb{C0}, \mb{D0} \cr
\mbs{KK[{\sl i}]} & abbreviations in \mb{HoldForm}  in the result of \mb{PaVeReduce}
} {Reduction function for Passarino-Veltman coefficient functions.}

Depending on the option \mb{BReduce} \mb{B1}, \mb{B00} and
\mb{B11} may also remain in the result of \mb{PaVeReduce}.

\otabthree{
{\sl option name} & {\sl default value} &\cr
\hhline
\mb{IsolateNames} & \mb{False} & use \mb{Isolate} with this option\cr
\mb{Mandelstam} & \mb{\{\}} & Mandelstam relation, e.g., 
\mb{\{s,\ t,\ u,\ 2\ mw\phat 2\}}
} {Options for \mb{PaVeReduce}.}

The function \mb{Isolate} is explained in section \ref{isolate}.

\beom
\domtog{{PaVeReduce[\ PaVe[2,\ \{SmallVariable[me2],\ mw2,\ t\},\{SmallVariable[me2],\ 0,\ mw2\}\ ]]\\
}}{\Frac{B_0{\rm (mw2,\ 0,\ mw2)}}{{\rm mw2}\ -\ t}\ -\ \Frac{B_0(t,\ 0,\ {\rm mw2})}{{\rm mw2}\ -\ t}\hfil\\
}{
Reduce $C_2(m_{\rm e}^{2},m_{\rm W}^{2},t,\ m_{\rm e}^2, 0 , m_{\rm  W}^2)$
to scalar integrals.
}
\domtog{{c12\ =\ PaVeReduce[\ PaVe[1,\ 2,\ \{s,\ m2,\ m2\},\ \{m2,\ m2,\ w2\}],\\
IsolateNames\ $\Rule$\ c\ ]\ \\
}}{\ $c(11)$\hfil\\
}{
Break down the  coefficient function 
$C_{12}(s,m^{2},m^{2},m^2,m^2,w^2)$.

This is the result in \mb{HoldForm}.
%The \mbs{K\[{\sl i}]} are given in \mb{HoldForm}.
}
\domtog{{c[11]\\
}}{\renewcommand{\clineskip}{36pt}\Frac{2\ {\rm m}2\ c(3)}{s\ c(6)}\ + \ \Frac{c(1)\ c(7)}{s\ \Superscript{c(6)}{2}}\ + \ \Frac{c(2)\ c(8)}{2\ \Superscript{c(6)}{2}}\ + \ \Frac{c(4)\ c(9)}{s\ c(6)}\ + \ \Frac{{\rm w}2\ c(5)\ c(10)}{\Superscript{c(6)}{2}}\ + \ \Frac{1}{2\ c(6)}\hfil\\
\renewcommand{\clineskip}{20pt}}{
The \mb{FullForm} of the \\
assignment to \mb{c12} is 
\mb{HoldForm[c[11]]}.
If you want to get the value of \mb{c[11]},
you can either type \mb{ReleaseHold[c12]} or \mb{c[11]} as 
done in the example.
}
\dtog{SetOptions[B1,\ BReduce\ $\Rule$\ True];}{}{Have $B_1$'s be reduced to $B_0$'s.}
\domtog{{Simplify\ /@\ Collect[FixedPoint[ReleaseHold,\ c12],\ {\_B0,\ \_C0}]\\
}}{\renewcommand{\clineskip}{36pt}\Frac{1}{2\ (4\ {\rm m}2\ -\ s)}\ +\ \Frac{({\rm m}2\ -\ {\rm {\rm w}}2)\ B_0(0,\ {\rm m}2,\ {\rm w}2)}{2\ {\rm m}2\ (4\ {\rm m}2\ -\ s)}\ -\ \\
\Frac{(8\ \Superscript{{\rm m}2}{2}\ \ -\ 10\ {\rm m}2\ {\rm w}2\ -\ 2\ {\rm m}2\ s\ +\ {\rm w}2\ s)\ B_0({\rm m}2,\ {\rm m}2,\ {\rm w}2)}{2\ {\rm m}2\ \Superscript{(4\ {\rm m}2\ -\ s)}{2}}\ \\
+\ \Frac{(4\ {\rm m}2\ -\ 6\ {\rm w}2\ -\ s)\ B_0(s,\ {\rm m}2,\ {\rm m}2)}{2\ \Superscript{(4\ {\rm m}2\ -\ s)}{2}}\ \\
+\ \Frac{{\rm w}2\ (8\ {\rm m}2\ -\ 3\ {\rm w}2\ -\ 2\ s)\ C_0({\rm m}2,\ {\rm m}2,\ s,\ {\rm m}2,\ {\rm w}2,\ {\rm m}2)}{\Superscript{(4\ {\rm m}2\ -\ s)}{2}}\hfil\\
\renewcommand{\clineskip}{20pt}}{
Repeated application of \\
\mb{ReleaseHold} 
reinserts all \mb{K}.
}
\domtog{{d122\ =\ PaVeReduce[\ \\
\ PaVe[1,\ 2,\ 2,\ \{SmallVariable[me2],\ mw2,\ mw2,\ SmallVariable[me2],\ s,\ t\},\\
\{0,\ SmallVariable[me2],\ 0,\ SmallVariable[me2]\}],\\
\ Mandelstam\ $\Rule$\ \{s,\ t,\ u,\ 2\ mw2\},\ IsolateNames\ $\Rule$\ f\ ]\ \\
}}{\ f[16]\hfil\\
}{
Take a coefficient function from $D_{\mu\, \nu\, \rho}$: \\
$D_{122}(m_e^{2}, m_w^{2}, m_w^{2}, m_e^{2}, s, t, 0, m_e^2\, , 0, m_e^2)$.
}
\dtog{{Write2[\ "d122.for",\ d122res\ =\ d122,\ \\
FormatType\ $\Rule$\ FortranForm\ ];\\
}}{}{
Write the result out into a Fortran file.
}
\dtog{!!d122.for\\
\\
\scriptsize\tt
        f(1)= B0(mw2,me2,0D0)\\
        f(2)= B0(s,0D0,0D0)\\
        f(3)= B0(t,me2,me2)\\
        f(4)= C0(mw2,mw2,t,me2,0D0,me2)\\
        f(5)= C0(mw2,s,me2,me2,0D0,0D0)\\
        f(6)= C0(t,me2,me2,me2,me2,0D0)\\
        f(7)= D0(mw2,mw2,me2,me2,t,s,me2,0D0,me2,0D0)\\
        f(8)= mw2 + s\\
        f(9)= 4*mw2 - t\\
        f(10)= mw2**2 - s*u\\
        f(11)= mw2 - s\\
        f(12)= 4*mw2**5-5*mw2**4*s-16*mw2**3*s**2+ \\
    \&  4*mw2**2*s**3 + 4*mw2*s**4 - mw2**4*u - \\
    \&  4*mw2**2*s**2*u+8*mw2*s**3*u+4*mw2**2*s*u**2+ \\
    \&  s**3*u**2 + s**2*u**3\\
        f(13)= mw2**2 - 4*mw2*s + 2*s**2 + s*u\\
        f(14)= 4*mw2**3-9*mw2**2*s+2*s**3-mw2**2*u- \\
    \&  4*mw2*s*u + 5*s**2*u + 3*s*u**2\\
        f(15)= 2*mw2**6-8*mw2**5*s+12*mw2**4*s**2-\\
    \&  8*mw2**3*s**3+2*mw2**2*s**4-2*mw2**5*t+\\
    \&  20*mw2**4*s*t-36*mw2**3*s**2*t+20*mw2**2*s**3*t-\\
    \&  2*mw2*s**4*t-6*mw2**3*s*t**2+6*mw2**2*s**2*t**2- \\
    \&  6*mw2*s**3*t**2+4*mw2*s**2*t**3-s**2*t**4\\
        f(16)= -f(8)/(2D0*f(9)*f(10)) - \\
    \&  (s**2*t**2*f(6)*f(11))/(2D0*f(10)**3) + \\
    \&  (s**3*t**2*f(7)*f(11))/(2D0*f(10)**3) + \\
    \&  (s**2*t*f(5)*f(11)**2)/f(10)**3 - \\
    \&  (f(1)*f(12))/(2D0*f(9)**2*f(10)**2*f(11)) + \\
    \&  (s*f(2)*f(13))/(2D0*f(10)**2*f(11)) + \\
    \&  (f(3)*f(8)*f(14))/(2D0*f(9)**2*f(10)**2) - \\
    \&  (f(4)*f(8)*f(15))/(2D0*f(9)**2*f(10)**3)\\
        d122res = f(16)
}{}{
This shows the resulting Fortran file.

The first abbreviations are always the scalar integrals.

The partially recursive definitions of the 
abbreviations are not fully optimized. 
%If you want to have a more optimized 
%Fortran output you may use the settings
%\mb{MacsymaForm} or \mb{MapleForm} of the option
%\mb{FormatType} for creating an output file
%with the syntax of the computer algebra systems 
%{\it Macsyma} and {\it Maple}, respectively.
%Then you can use the Fortran code generators
%of {\it Maple} or {\it Macsyma}. In some versions
%of {\it Macsyma} the excellent package 
%{\it Gentran} by B.~Gates and H.~van Hulzen is 
%available.

The function \mb{Write2} is explained in section \ref{misc}.

Note that the head \mb{SmallVariable} is eliminated in the 
Fortran output automatically.
}
\dtog{DeleteFile["d122.for"];}{}{Delete the output file.}
\enom
The Fortran code generated by \mb{Write2} should be checked with care.
All integer numbers (except 0 as argument of \mb{B0}, \mb{C0}, \mb{D0}) 
are translated to integers. This causes problems when translating 
variables with rational powers and must be corrected in the Fortran 
output by hand.

\subsection{Tables of Integrals}

\otabtwo{
\mbs{CheckDB[{\sl expr}, {\sl fil}]} & saves or retrieves 
{\sl expr} to/from a file {\sl fil} \cr
} {Storing and retrieving results.}

\otabthree{
{\sl option name} & {\sl default value} &\cr
\hhline
\mbs{Directory} & \mb{"fcdb"} & directory to use \cr
\mbs{NoSave} & \mb{False} & don't save to disk \cr
\mbs{ForceSave} & \mb{False} & force the expression to be evaluated even if the
file exists. The expression is also saved if \mb{NoSave} is set to \mb{False}\cr
\mbs{Check} & \mb{False} & check if the evaluation of the expression agrees with what
is loaded from disk and return \mb{True} or \mb{False}\cr
}{Options of \mb{CheckDB}.}

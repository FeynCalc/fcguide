\section{Introduction}

\fc is a collection of utilities for algebraic calculations in 
High Energy Physics. It is implemented as a \mma package; that is,
a number of extensions to the programming language \mma,
written in \mma. 
%In the section below are described some of the main new
%capabilities provided.

This book is for the practicioners of perturbative calculations in QCD, the Standard Model,
Chiral Perturbation Theory and similar theories.
The book serves as a technical manual of \fc, augmented by textbook and research example calculations.
The user is assumed to know Quantum Field Theory and have some understanding of \mma.
%It is strongly recommended to study the \mma book \cite{Mathematica} by
%Stephen Wolfram before starting with \fc.

The original idea of \fc was to provide convenient tools 
for radiative corrections in the Standard Model of particle physics. 
The input for \fc, the analytical expressions for the diagrams, can be 
entered by hand or can be taken directly from the output of the Feynman diagram generator \fa\ \cite{feynarts} .
The user can provide certain additional  information about the process 
under consideration, i.e. the kinematics and the choice of the standard 
matrix elements may be defined. Once this is done, \fc performs the 
algebraic calculations like tensor algebra, tensor integral decomposition and reduction,
yielding a polynomial in standard matrix elements,  special functions, kinematical 
variables and scalars suitable for further numerical evaluation.

\fc also provides calculator-like features. You enter a Dirac trace 
in a very similar notation to that used when doing calculations by hand, and 
get back an answer suitable 
for further manipulation. These features include basic operations like contraction of
tensors, simplification of products of Dirac matrices and trace calculation.

More complex algebraic operations have also been implemented, notably derivation
of polynomials in quantum fields with respect to such fields.

Over the years, the detailed course of and motivation for the development of \fc has
changed somewhat both in scope and focus. The reasons were  the different research
interests of the two authors, the relative slowness of \mma and the rapid development of
multi-loop calculations carried out with tools \cite{form, formcalc} more suited 
for very large scale calculations.
%3) the rapid development of the raw computing power of hardware. 
As a consequence the focus has moved away from providing cutting edge results of multi-loop calculations
in the Standard Model and in QCD. 

However, due to improvements in software and hardware it became possible
to explore other types of calculations, e.g. complex algebraic manipulations of quantum fields
and lagrangians, traditionally done by hand only.

Despite the motivations driving development, the central vision remains largely the same: 
\fc was conceived as and continues to be developed as a general-purpose, relatively easy to use,
modular toolbox for quantum field theory calculations. 
Dirac algebra utilities, Lorentz and Color algebra manipulation capabilities, 
functional derivation utilities, integral and other tables have been cleanly 
separated off in distinct functions which 
do well-defined operations.  
Such basic functions can then be used  on a higher level to build sophisticated model dependent functions.

In the following we list what we consider the most important goals of the \fc project:
%%SOFAR

\begin{itemize}

\item{{\bf Code reuse.} In calculations as complex as quantum field theory calculations,
the code used should be thoroughly checked. Reinventing the wheel for every calculation should be avoided.}

\item{{\bf Standardization.} In order to reuse code, clear standards should be formulated, thoroughly documented and followed. This should also make it easier to reach common standards with other similar or related projects (using other programming languages) for fields, lagrangians, Lorentz-vectors, etc. in the end allowing easy comparison of calculations of the same process.}

\item{{\bf Repository of lagrangians, amplitudes and integrals.} The agreement on standards allows the creation of a database of quantum field theory results stored in a standardized electronic form. We encourage users of \fc to contribute back the results they obtain.}

\item{{\bf Algorithm optimization.} The more people using (an implementation of) an algorithm, the better the chances of improving (the implementation of) it.}

\item{{\bf Fast prototyping.} Quickly trying out new quantum field theory models should be made easier by using the building blocks provided by a clear standardization as well as a set of functions for the most common calculational tasks.}

\item{{\bf Checks.} \fc can be used to check your hand calculations or the results of other programs.}

\end{itemize}

Notice in this connection that the standards put forth
in this document (and in the code) are what we consider to be the most general but
still useful standards
for doing quantum fields theory with \mma. This means that standards are
to be considered as relatively fixed, but not that we are unwilling to change them
given sufficiently good reasons.

From all of the above it should be clear that we consider interaction with users to be crucial
for the further evolving of \fc. Therefore, suggestions and in particular
code contributions are warmly welcomed. Smaller contributions can be emailed to
the mailing list address, feyncalc@feyncalc.org, or to one of the authors; 
larger contributions should be coordinated with the
developers who can be reached at the same email address.
Though much testing of the code has been done, there is absolutely no claim that
\fc is bug-free. You should be sceptical about the results, and when you are sure the
program  returns a wrong answer, you are encouraged to send email
to feyncalc@feyncalc.org.

\subsection{Version History}

The roots of \fc go back to 1987. 
During a stay as a graduate student in Albuquerque, New Mexico, Rolf Mertig learned to program in Macsyma \cite{Drinkard:1981dr} from the experts Stanly Steinberg and Michael Wester. Back to Germany, 
elementary particle physicists needed automation of the calculation of Feynman diagrams of eletroweak 
processes to one loop. The algorithms were partially developed together with 
Ansgar Denner and Manfred B\ODoubleDot{}hm and implemented in a 
purely functional way by Rolf Mertig during the years 1987 - 1989. 

The basic idea was to have general functions for some of the more mechanical parts of the diagram calculations, generalizable tools for use in calculations of different 1-loop processes, especially 1\(\rightarrow \)2 and 2\(\rightarrow \)2 processes. These included tools for:

\begin{itemize}

\item{Lorentz algebra.
\({g^{\Mvariable{\mu \nu }}}\multsp {p_{\mu }}
\multsp \rightarrow \multsp {p^{\nu }}
\multsp ,\multsp \multsp\)
\({g^{\Mvariable{\alpha \beta }}}{g_{\Mvariable{\alpha \beta }}}
   \multsp \rightarrow \multsp D\multsp \cdots\),}

\item{Dirac algebra.
\({{\gamma }^{\alpha }}\multsp {{\gamma }^{\nu }}
{{\gamma }_{\alpha }}\multsp \rightarrow \multsp 
(2-n)\multsp {{\gamma }^{\nu }},\multsp \multsp \)
\(\overvar{v}{\_}(p,m)\multsp 
\bps\)  \(\rightarrow \) - m \(\overvar{v}{\_}(p,m)\multsp \cdots\),}

\item{Passarino-Veltman tensor integral decomposition.
Applications of these tools included complete 1-loop processes in the Standard Model,
\({e^+}{e^-}\multsp \rightarrow 
\multsp Z\multsp H\multsp \),  \(\multsp {e^+}{e^-}\multsp \rightarrow \)
\(\multsp {W^+}{W^-}\multsp \)
and the 2-loop photon self-energy in QED.}

\end{itemize}

However, at the end of 1989 several problems showed up with the Macsyma implementation. The purely functional programming style proved to be difficult to debug and, in fact, inappropriate. The rudimentary pattern matcher in Macsyma was not useful. There was no way to incorporate new functions easily into the whole Macsyma system, and no possibility of providing online documentation. Furthermore, Macsyma's memory managment ("garbage collection") was slow when handling large expressions.

In early 1990 it became clear that \mma was a much more natural programming environment for \fc.

\subsubsection*{1990-1991 : The first version  of \fc in \mma}

In 1990, user-friendly  packages were built with extended automatic capabilities (\mb{OneLoop}, \mb{OneLoopSum}), and SU(3) algebra capabilities were added (\mb{SU3Simplify}).

In 1991 initial documentation was written and the program was made available on anonymous ftp-servers: mathsource.wolfram.com and canc.can.nl.

Applications from 1990 - 1996  included:

\begin{itemize}

\item 1-loop 2 \(\rightarrow \)2 processes in the Standard
Model, such as:

\({\rm gg}\multsp \rightarrow \multsp 
{\rm t \overline{t}} \multsp , \multsp \multsp
{{\rm e}^+}{{\rm e}^-}\multsp \rightarrow \multsp {\rm Z}\multsp {\rm H}\multsp, \multsp \multsp
%{{\rm W}^+}{{\rm W}^-}\multsp \)\(\rightarrow \) \({{\rm W}^+}{{\rm W}^-}\multsp, \multsp \multsp
{\rm gg} \rightarrow {\rm t\bar{t}} \),

\item background field gauge calculations,

\item high-energy approximation of \(\multsp{{\rm e}^+}
{{\rm e}^-}\multsp \rightarrow \multsp {{\rm W}^+}{{\rm W}^-}\multsp \),

\item  2-loop Standard Model self-energies.

\end{itemize}

No real attempt was made to provide tools for tree-level calculations. 
For this purpose other programs appeared, among them another  \mma package, \hip \cite{yeh}, developed at SLAC, 
CompHEP \cite{comphep}, and of course FORM \cite{form}.

\subsubsection*{1992-1995 : \fc 2.0-2.2, unification and simplification.}

During this period of development the SU(3) algebra was changed to SU($N$). Several tools for automatic tree-level calculation were added, for example the function \mb{SquareAmplitude}. (Unfortunately, the  documentation was not updated.) All sub-packages were put into one file \(\big(\simeq {{10}^4}\) lines), "FeynCalc2.2beta.m" \footnote{This file is still available upon request}.

\subsubsection*{1993-1996 : \fc 3.0, modularization, typesetting, Operator Product Expansion (OPE)}

Due to the rapidly increasing amount of code, \fc was reorganized in a modular way. The definitions of each function are contained in a package file which is loaded only on demand. For the maintenance of hundreds of packages, totalling 2.5 megabytes, software engineering was needed.

\begin{itemize}

\item
In these years many improvements were made to the code, driven by the work of the author (Rolf Mertig) 
together with Willy van Neerven in perturbative Quantum Chromo Dynamics (QCD).

One effort was to build data bases: Convolutions, integrals, tensor integral transformation formulae, Feynman rules, 
and Feynman parameterizations. 
Applications include 2-loop spin-independent and spin-dependent Altarelli-Parisi splitting functions \cite{spinsplit}.

\item The new typesetting capabilities of \mma 3 (TraditionalForm) were used to substantially improve the look of the output. Typesetting rules were added for e.g. \({\overvar{\partial }{\rightarrow}}_{\mu } \multsp \).

\item Code was written to allow new abstract data types, for example for noncommutative algebra and for special integrals.

\item QCD tools for the Operator Product Expansion  (OPE)
were added.

\item Automatic Feynman rule derivation (with functional
differentation) was implemented in order to get special Feynman rules
for twist-2 (and higher) operators.

\end{itemize}

\subsubsection*{1997-2000 : \fc 4.0-4.1, QCD and OPE, ChPT, tables}

Another was the package \tarcer, which was initially an independent project, but then integrated into \fc.  \tarcer \cite{Mertig:1998vk} adds two-loop functionality for propagator-type integrals using the recurrence relations of Tarasov \cite{Tarasov:1997kx}.

In 2000 maintenance of the package was taken over by Frederik Orellana. He worked in Chiral Perturbation Theory \cite{Gasser:1984gg} (ChPT) and made some changes, as well as adding code to \fc in order to support ChPT and effective theories in general, and interfacing with \fa. Also, first implementations of the 'tHooft-Veltman formulae \cite{'tHooft:1978xw} for numerical evaluation of the one-loop integrals $B_0, C_0$ and $D_0$ were added.

\subsection{Installation}

\begin{itemize}

\item{Download the file "HighEnergyPhysics-5.0.tar.gz" or "HighEnergyPhysics-5.0.zip"}

\item{If you already have an old installation and have costumized the file "FCConfig.m", back it up.}

\item{Unpack the archive in any of the following places:

\begin{itemize}

\item{Anywhere on your file system. If the location is not on the Mathematica \mb{\$Path} you will have to load
FeynCalc by \mb{<</myfullpathname/HighEnergyPhysics/FeynCalc.m}} 

\item One of the Mathematica "Applications" directories: Under {\sc unix}/{\sc linux} this is, for Mathematica 4.2 or higher,
"~/.Mathematica/Applications/", where the tilde refers to you home directory.
For other operating systems evaluate \mb{\$UserAddOnsDirectory} or \mb{\$AddOnsDirectory} in \mma
\end{itemize}
}

\item{If you've made any customizations in the configuration file "FCConfig.m", merge them from the file you've moved away into the new file.}

\item{Start \mma.}

\item{Choose 'Rebuild Help Index' from the 'Help' menu.}

\item Load \fc with \mb{<<HighEnergyPhysics`FeynCalc`} or by \mb{<<HighEnergyPhysics`FeynCalc`}

\end{itemize}




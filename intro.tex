\chaptermark{The FeynCalc Framework}
\chapter{The \fc Framework}

\section{Introduction}
\label{intro}

\fc is a \mma package that provides a framework for doing calculations
in quantum field theory and, within this framework, a collection of utilities
for the more common as well as more specialized tasks
in such calculations.

%The original idea of \fc was to provide convenient tools for
One typical task is the calculation of
radiative corrections in the Standard Model of particle physics. 
The input for \fc, the analytical expressions for the diagrams, can be 
entered by hand or can be taken directly from the output of the Feynman diagram generator \fa\ \cite{feynarts} .
The user can provide certain additional  information about the process 
under consideration, i.e. the kinematics and the choice of the standard 
matrix elements may be defined. Once this is done, \fc performs the 
algebraic calculations like tensor algebra, tensor integral decomposition and reduction,
yielding a polynomial in standard matrix elements,  special functions, kinematical 
variables and scalars suitable for further numerical evaluation.

\fc also provides calculator-like features.
These features include basic operations like contraction of
tensors, simplification of products of Dirac matrices and trace calculation.
For instance, a Dirac trace is entered
in a notation very similar to that used when doing calculations by hand, and 
an answer suitable for further manipulation is returned.

More complex algebraic operations have also been implemented, notably derivation
of polynomials in quantum fields with respect to such fields.

\fc is installed by following the steps below.

\begin{itemize}

\item{Download the file "HighEnergyPhysics-5.0.tar.gz" or "HighEnergyPhysics-5.0.zip"}

\item{If you already have an old installation and have costumized the file "FCConfig.m", back it up.}

\item{Unpack the archive in any of the following places:

\begin{itemize}

\item{Anywhere on your file system. If the location is not on the Mathematica \mb{\$Path} you will have to load
FeynCalc by \mb{<</myfullpathname/HighEnergyPhysics/FeynCalc.m}} 

\item One of the Mathematica "Applications" directories: Under {\sc unix}/{\sc linux} this is, for Mathematica 4.2 or higher,
"~/.Mathematica/Applications/", where the tilde refers to you home directory.
For other operating systems evaluate \mb{\$UserAddOnsDirectory} or \mb{\$AddOnsDirectory} in \mma
\end{itemize}
}

\item{If you've made any customizations in the configuration file "FCConfig.m", merge them from the file you've moved away into the new file.}

\item{Start \mma.}

\item{Choose 'Rebuild Help Index' from the 'Help' menu.}

\item Load \fc with \mb{<<HighEnergyPhysics`FeynCalc`} or by \mb{<<HighEnergyPhysics`FeynCalc`}

\end{itemize}




\section{Miscellaneous Functions}
\label{misc}

\subsection{Low Level Dirac Algebra Functions}
\label{diracLow}

There are a number of lower level functions for doing Dirac algebra. 
Some are used by the functions described in
section \ref{diracalg}, some may be useful to perform more controlled
calculations. For completeness, they are described here.

\otabtwo{
\mbs{DiracGammaCombine[{\sl expr}]} & combines sums, \mb{DiracGamma[Momentum[{\sl a}]] + 
        DiracGamma[Momentum[{\sl b}]] + ...}, in {\sl expr} into \mb{DiracGamma[
    Momentum[{\sl a} + {\sl b} + ..]]} \cr
\mbs{DiracGammaExpand[{\sl expr}]} & expands all \mb{DiracGamma[
    Momentum[{\sl a} + {\sl b} + ..]]} in {\sl expr} into\mb{ DiracGamma[Momentum[{\sl a}]] + 
        DiracGamma[Momentum[{\sl b}]] + ... }\cr
\mbs{DiracGammaT[{\sl x}]} & the transpose of \mb{DiracGamma[{\sl x}]} \cr
\mbs{DiracSigma[{\sl a}, {\sl b}]} & $i/2$ ({\sl a b} -{\sl  b a}) in 4 dimensions. 
     {\sl a} and {\sl b} must have head \mb{DiracGamma}, \mb{DiracMatrix} or \mb{DiracSlash} \cr
\mbs{DiracSigmaExplicit[{\sl expr}]} & inserts in {\sl expr} the definition of \mb{DiracSigma} \cr
\mbs{DiracSpinor[{\sl p}, {\sl m}, {\sl ind}]} & a Dirac spinor for a fermion with momentum {\sl p} and mass {\sl m} and indices {\sl ind} \cr
\mbs{EpsChisholm[{\sl expr}]} & substitutes for a gamma matrix contracted with a Levi-Civita tensor (\mb{Eps}) the Chisholm identity \cr
\mbs{ChisholmSpinor[{\sl expr}]} & uses the Chisholm identity on a Dirac gamma matrix between spinors \cr
\mbs{ChisholmSpinor[{\sl expr}, {\sl i}]} & uses the Chisholm identity on the first or second part of a spinor chain, depending on whether {\sl i} is 1 or 2 \cr
} {Low level Dirac algebra functions.}

\subsection{Functions for Polynomial Manipulations}

Unfortunately, at least with \mma 2.0, the built-in functions \mb{Factor}, \mb{Collect}
and \mb{Together} are either not general enough for the purposes needed in 
\fc or consume too much CPU time for certain polynomials
with rational coefficients. The \fc extensions \mb{Factor2}, \mb{Collect2} 
and \mb{Combine} should be used instead when manipulating the rational 
polynomials emerging from \mb{OneLoop} or \mb{PaVeReduce}. 

\otabtwo{
\mbs{Collect2[{\sl expr}, {\sl x}]} & group together powers of terms containing $x$\cr
\mbs{Collect2[{\sl expr}, \{$\hbox{\sl x}_{1}$, $\hbox{\sl x}_{2}$, ...\}]} & group together powers of 
terms containing $x_1, x_2,  ...$ \cr
\mbs{Combine[{\sl expr}]} & put terms in a sum over a common denominator \cr
\mbs{Factor2[{\sl expr}]} &  factor a polynomial in a canonical way
} {Modifications of \mb{Collect}, \mb{Together} and \mb{Factor}.}

\beom
\domtog{{Collect2[ (b - 2) d f[x] + f[x] + c p[y] + p[y], \{f, y\}]\\
}}{ $(b d - 2 d +1) f(x) + (c + 1) p(y)$\hfil\\
}{
This collects \mb{f[x]} and \mb{p[y]}.
%The default setting is not to expand terms like $(b-2) d$.
}
%\domtog{{Collect2[3\ B0[pp,m1,m2]\ (m\phat 2 + s) + B0[pp,m1,m2]\ s, \\
%B0,\ Expanding $\Rule$ True]\\
%}}{($3\ m^2\, +\, 4\, s)\ B_0({\rm pp},\, {\rm m1},\, {\rm m2})$\hfil\\
%}{
%With the setting \\ \mb{ProductExpand $\Rule$ True} the product $3 (s + m^2)$
%gets expanded.
%}
\domtog{{Combine[(a - b) (c - d)/e\ +\ g]\\
}}{\Frac{(a\, -\, b)\, (c\, -\, d)\ +\, e\, g}{e}\hfil\\
}{
This puts terms over a common denominator without expanding the numerator.
}
\domtog{{test\ =\ (a - b) x\ +\ (b - a) y\\
}}{ $(a - b)\, x\, +\, (-a + b)\, y$\hfil\\
}{
Consider a generic polynomial. 
}
\domtog{{Map[Factor, test]\\
}}{ $(a - b)\, x\, +\, (-a + b)\, y$\hfil\\
}{
Under \mma 2.0, mapping \mb{Factor} on the summands shows that no canonical factorization 
of integers in sums takes place.
}
\domtog{{Map[Factor2, test]\\
}}{ $(a - b)\, x\, -\, (a - b)\, y$\hfil\\
}{
If a canonical factorization is desired, 
you may use \mb{Factor2} instead.
}
\domtog{{Factor[test]\\
}}{ $(-a + b)\, (-x + y)$\hfil\\
}{
This is the overall factorization of \mb{Factor}.
}
\domtog{{Factor2[test]\\
}}{ $(a - b)\, (x - y)$\hfil\\
}{
Here the convention used by \\
\mb{Factor2} becomes clear:
The first term in a subsum gets a positive prefactor.
}
\enom

\otabthree{
{\sl option name} & {\sl default value} \cr
\hhline
\mb{ProductExpand} & \mb{False} & expand products 
} {An expansion controlling  option for \mb{Collect2} and \mb{Combine}.}

\subsection{An Isolating Function for Automatically Introducing Abbreviations}
\label{isolate}

\otabtwo{
\mbs{Isolate[{\sl expr}]} & if \mb{Length[{\sl expr}]} > 0, substitute an abbreviation \mb{KK[{\sl i}]} for {\sl expr} \cr
\mbs{Isolate[{\sl expr},\ \{$\hbox{\sl x}_{1}$,\ $\hbox{\sl x}_{2}$,\ ...\}]} & substitute abbreviations for subsums 
free of $x_1, x_2, ...$ \cr
} {A function for isolating variables by introducing abbreviations for common
subexpressions.}

\otabthree{
{\sl option name} & {\sl default value}\cr
\hhline
\mb{IsolateNames} & \mb{KK} & head of the abbreviations \cr
\mb{IsolateSplit} & \mb{442} & a limit beyond which \mb{Isolate} splits 
the expression into two sums
} {Options for \mb{Isolate} and \mb{Collect2}.}

\beom
\domtog{{Isolate[a + b] \\
}}{ ${\rm KK}(1)$\hfil\\
}{
\mb{Isolate} with one argument introduces a single \mb{K[{\sl i}]} as abbreviation.
}
\domtog{{test = Isolate[(a + b) f + (c + d) f + e, f]\\
}}{ $e + f {\rm KK}(1) + f {\rm KK}(2)$\hfil\\
}{
Here $f$ gets isolated with \mb{KK[1]}  and \mb{KK[2]} 
replacing the bracketed subsums.
}
\domtog{{FullForm[test]\\
}}{
%Out[\the\promptcount]//FullForm=} \\
Plus[e, Times[f, HoldForm[KK[1]]], Times[f, HoldForm[KK[2]]]]\hfil\\
}{
Looking at the \mb{FullForm} reveals that the \mb{KK[{\sl i}]} are 
given in \mb{HoldForm}.
}
\domtog{{\{ KK[1], test, ReleaseHold[test] \}\\
}}{ $\{a\, +\, b,\, e\, +\, f\, {\rm KK}(1)\, +\, f\, {\rm KK}(2)\, e\, +\, (a + b)\, f\, +\, (c + d)\, f\}$\hfil\\
}{ 
Asking for \mb{KK[1]} returns its value, but in \mb{test} 
\mb{KK[1]} is held.
} 
\domtog{{Isolate[a[z]\ (b + c (y + z)) + d[z] (y + z), \{a, d\}, \\
IsolateNames $\Rule$ g] \\
}}{ $d(z)\, g(1) + a(z)\, g(2)$\hfil\\
}{
For the term \\
\mb{(b\, +\, c\, (y + z))}\\
a single abbreviation \mb{g[2]} is 
returned.
}
\domtog{{und[x\_\_]\ :=\ Isolate[Plus[x], IsolateNames $\Rule$ h]/;\\
FreeQ2[\{x\}, \{a, d\}];\\
 (a[z] (b + c (y + z)) + d[z] (y + z))/.\\
 Plus $\Rule$ und /. und $\Rule$ Plus\\
}}{ $d(z)\, h(1) + a(z)\, h(2)$\hfil\\
}{
With the help of an additional \\
function it is easy to introduce identical abbreviations for each 
subsum.

This trick of selectively substituting functions for 
sums is also quite useful in special reordering 
of polynomials.
}
\domtog{{ReleaseHold[\%]\\
}}{ $(y + z)\, d(z)\, + a(z)\, (b + c\, h(1))$\hfil\\
}{
Here it is clear that \mb{h[1]} = (y + z) is part of
\mb{h[2]}.
}
\domtog{{Isolate[ a - b - c - d - e, IsolateNames $\Rule$ l, IsolateSplit $\Rule$ 15 ]\\
}}{ $l(2)$\hfil\\
}{
Decreasing the option \\
\mb{IsolateSplit} significantly 
forces \mb{Isolate} to 
use more than one \mb{l} for \mb{a - b - c - d - e}.
}
\domtog{{\{l[2], l[1]\}\\
}}{ $\{-c - d - e + l(1), a - b\}$\hfil\\
}{
This shows the values of the \mb{l}.
}
\enom

The importance of \mb{Isolate} is significant, since it gives you a means
to handle very big expressions. The use of \mb{Isolate} on
big polynomials before writing them to a file is especially recommended.

A subtle issue of the option \mb{IsolateSplit}, whose setting  refers 
to the "size" of an expression as returned by \\
\mb{Length[Characters[ToString[FortranForm[{\sl expr}]]]]}, 
is that it inhibits too many continuation lines in the Fortran file when writing out 
expressions involving \mb{KK[i]} with \mb{Write2}. See Section \ref{write2} for the
usage of \mb{Write2}.

You may want to change \mb{IsolateSplit} when working with big rational polynomials
in order to optimize the successive (\mb{FixedPoint}) application of 
functions like \mb{Combine[ReleaseHold[\#]]\&}.

\subsection{An Extension of FreeQ and Two Other Useful Functions}
\otabtwo{
\mbs{FreeQ2[{\sl expr}, \{$\hbox{\sl f}_{1}$, $\hbox{\sl f}_{2}$, ...\}]} & 
yields \mb{True} if $expr$ does not contain any occurrence of
$f_1$, $f_2$, \ldots \cr
\mbs{NumericalFactor[{\sl expr}]} &  gives the numerical factor of $expr$ \cr
\mbs{PartitHead[{\sl expr}, {\sl h}]} & returns a list \mb{\{{\sl a}, {\sl h}\mb{[}b\mb{]}\}} with $a$ free of expressions with head $h$
} {Three useful functions.}

\beom
\domtog{{FreeQ2[w\phat 2 + m\phat 2 B0[pp, m1, m2], \{w, B0\}] \\
}}{ False\hfil\\
}{
\mb{FreeQ2} is an extension of \\
\mb{FreeQ}, allowing a list as second argument.
}
\domtog{{NumericalFactor[-137\ x]\\
}}{ $-137$\hfil\\
}{
This gives the numerical factor of the expression.
}
\domtog{{PartitHead[f[m]\ (s - u), f]\\
}}{ $\{s - u, f(m)\}$\hfil\\
}{
The action of \mb{PartitHead} on a product is to split it apart.
}
\domtog{{PartitHead[s\phat 2 + m\phat 2 - f[m], f]\\
}}{$\{m^2\, +\, s^2,\, -f(m)\}$\hfil\\
}{
A sum gets separated into \\
subsums.
}
\enom

\subsection{Writing Out to Mathematica and Fortran%, Macsyma and Maple
}
\label{write2}

The \mma functions \mb{Write} and \mb{Save} may on rare occasions create files 
that cannot  be read in again correctly. What sometimes happens is that, for 
example, a division operator is written out as the first character of a line.
When the file is loaded again, \mma may simply ignore that part  of the file
before this character.
You can easily work around this bug 
by editing the file and wrapping  
the expressions with brackets ``( )''. Since this is a cumbersome
procedure for lots of expressions, it has been automatized in 
\mb{Write2} for writing out expressions. 
If you use \mb{Save}, you should check 
the output file for correctness.

An option can be given to \mb{Write2} allowing the 
output to be written in \mb{FortranForm}
%, \mb{MacsymaForm} or \mb{MapleForm}
. This facility is elementary and mostly limited to the 
type of polynomials usually encountered in \fc. 

At least with \mma 2.0, notice:
The \mb{FortranForm} option should not be used if the expression to be 
written out contains a term $(-x)^n$, where $n$ is a symbol, (such terms are never
produced by \fc, unless they already contained in the input).
The \mma \mb{FortranForm} is also incapable of recognizing some elementary 
functions like \mb{Exp[x]}.
%Therefore if you want to generate more elaborate Fortran code  you may want 
%to use Maple, which provides an optimization option when 
%translating Maple code to Fortran, or 
%the most sophisticated package for Fortran code generation from 
%computer algebra systems: Gentran (by B.~Gates and H.~van Hulzen),
%which is available in some Macsyma versions. To this end the \mb{Write2} option \mb{FormatType} may be set to \mb{MacsymaForm} or \mb{MapleForm}.
%You may achieve a rudimentary optimization by applying \mb{Isolate} on 
%the expression to be written out. With the option setting \mb{FortranForm} 
%and \mb{InputForm} the function \mb{Write2} recognizes variables in 
%\mb{HoldForm} and writes these out with their definitions first.

\otabtwo{
\mbs{Write2[{\sl fi}, {\sl x} = {\sl y}]} &
write the setting $x = y$ into a file \mb{{\sl fi}}\cr
\mbs{Write2[{\sl fi}, {\sl x} = {\sl y}, {\sl a} = {\sl b}, ...]} &
write the setting $x = y$ into a file \mb{{\sl fi}} \cr
} {A modification of \mb{Write}.}

\otabthree{
{\sl option name} & {\sl default value} \cr
\hhline
\mb{FormatType} & \mb{InputForm} & language in which to write the output \cr
\mb{D0Convention} & \mb{0} & which convention to use for the scalar integrals 
} {Options for \mb{Write2}.} 

With the default setting \mb{0} of \mb{D0Convention} the arguments of the 
scalar integrals are not changed when written into a Fortran program.
Another possible setting is \mb{1}, which  interchanges the fifth and sixth arguments
of \mb{D0} and writes the mass arguments of \mb{A0}, \mb{B0}, \mb{C0} and 
\mb{D0} (without squares).

\otabtwo{
\mb{InputForm} & write out in \mma form \cr
\mb{FortranForm} & write out in Fortran form \cr
%\mb{MacsymaForm} & write out in Macsyma form \cr
%\mb{MapleForm}  & write out in Maple form
} {The four possible settings of the option \mb{FormatType} for \mb{Write2}.}

%The output possibilities for the two other computer algebra systems 
%Macsyma and Maple are very limited: square brackets in 
%\mma are changed to round brackets,  additional replacements are for
%\mb{MacsymaForm} \mb{\{ Pi $\Rule$ \%pi, I $\Rule$ \%i, = $\Rule$ : \}} and for 
%\mb{MapleForm} \mb{\{ = $\Rule$ := \}}.

%The reasons to include these interface abilities are to utilize the 
%Fortran optimization possibility of Maple
%and the excellent package Gentran for Fortran code generation.

\beom
\domtog{{tpol = z + Isolate[Isolate[2 Pi I + f[x] (a - b), f]]\\
}}{ $z + {\rm KK}(4)$\hfil\\
}{
Create a test polynomial.
}
\dtog{{Write2["test.m", test = tpol];\\
}}{}{
The default is to write out in \mma \mb{InputForm}.
}
\domtog{{!!test.m\\
}}{\tt 
KK[3] = (a - b\\
);\\
KK[4] = ((2*I)*Pi + f[x]*HoldForm[KK[3]]\\
);\\
test = z + HoldForm[KK[4]]
}{
Show the content of the file.
}
\dtog{DeleteFile["test.m"];}{}{Clean up.}
\dtog{{Write2["test.for", test = tpol, \\
FormatType\ $\Rule$\ FortranForm];\\
}}{}{
This writes a file in Fortran format.
}
\domtog{{!!test.for\\
}}{\tt KK(3) = a - b\\
KK(4)\ = (0,2)*Pi + f(x)*KK(3)\\
test = z + KK(4)\\
\hfil\\
}{
Show the content of the Fortran file.
}
%\dtog{{Write2["test.mac", test = tpol, \\
%FormatType $\Rule$ MacsymaForm];\\
%}}{}{
%This writes a file in Macysma format.
%}
%\domtog{{!!test.mac\\
%}}{\tt test : ( 2*\%i*\%pi + z + (a - b)*f(x) )\$\hfil\\
%}{
%Here the Macsyma format can be seen.
%}
%\dtog{{Write2["test.map", test = tpol, \\
%FormatType $\Rule$ MapleForm];\\
%}}{}{
%Here we get a file with Maple conventions.
%}
%\domtog{{!!test.map\\
%}}{\tt test := ( 2*I*Pi + z + (a - b)*f(x) );\hfil\\
%}{
%The Maple conventions are only slightly different.
%}
\dtog{DeleteFile /@ {"test.m", "test.for"};}{}{Clean up.}
\enom

\subsection{More on Levi-Civita Tensors}

\otabtwo{
\mbs{EpsEvaluate[{\sl expr}]} & evaluate Levi-Civita \mb{Eps}
} {A function for total antisymmetry and linearity with respect to 
\mb{Momentum} \hspace{0.1mm}.}

\beom
\domtog{{Contract[LeviCivita[$\mu$, $\nu$, $\rho$, $\sigma$] FourVector[p + q, $\sigma$], EpsContract $\Rule$ False]\\
}}{ $\epsilon^{\mu, \nu, \rho, p + q}$\hfil\\
}{
This is $\varepsilon^{\mu\, \nu\, \rho\, \sigma} (p+q)_{\sigma} =
          \varepsilon^{\mu\, \nu\, \rho\, (p+q)}$.  
}
\domtog{{EpsEvaluate[\%]\\
}}{ $\epsilon^{\mu, \nu, \rho, p}\, +\, \epsilon^{\mu, \nu, \rho, q}$\hfil\\
}{
In this way linearity is enforced:
$\varepsilon^{\mu \, \nu\,  \rho\,  (p+q)} = 
 \varepsilon^{\mu \, \nu\,  \rho\,  p} + \varepsilon^{\mu\, \nu\, \rho\, q}$.
}
%\domtog{{LeviCivita[a, b, c, d] /. d $\Rule$ c\\
%}}{ eps[a, b, c, c]\hfil\\
%}{
%This does not evaluate directly to 0.
%}
%\domtog{{EpsEvaluate[\%]\\
%}}{ 0\hfil\\
%}{
%Like this you can always evaluate $\varepsilon$'s.
%}
\enom

\otabtwo{
\mbs{EpsChisholm[{\sl expr}]} &  utilize  $\gamma _{\mu} \varepsilon ^{ \mu \nu \rho \sigma } =
 + i \,( \gamma ^{ \nu} \gamma ^{ \rho } \gamma ^{ \sigma }
 - g^{ \nu \rho }  \gamma ^{ \sigma } -  g^{ \rho \sigma } \gamma ^{\nu}
 +  g^{ \nu  \sigma } \gamma ^{ \rho })  \gamma ^{5}$
} {A function for the  Chisholm identity.}

\otabthree{
{\sl option name} & {\sl default value} \cr
\hhline
\mb{LeviCivitaSign} & -1 & sign convention for the $\varepsilon$-tensor
}{An option for \mb{EpsChisholm}.}

\beom
\domtog{EpsChisholm[
  LeviCivita[$\alpha$, $\beta$, $\gamma$, $\delta$] DiracMatrix[$\mu$,
$\alpha$] // FCI]
}{$\ComplexI \gamma^\mu . \gamma^\beta . \gamma^\gamma . \gamma^\delta . \gamma^5 - 
\ComplexI \gamma^\mu . \gamma^\delta . \gamma^5 . g^{\beta\gamma} + 
\ComplexI \gamma^\mu . \gamma^\gamma . \gamma^5 . g^{\beta\delta} - 
\ComplexI \gamma^\mu . \gamma^\beta . \gamma^5 . g^{\gamma\delta}$
}{ 
Use the Chisholm identity for\\
$\varepsilon^{a \, b \, c\, d} \, \gamma^{\mu} \, \gamma^{a}$.
}
\enom

\otabtwo{
\mbs{ Eps[{\sl a},\ {\sl b},\ {\sl c},\ {\sl d}]} & internal representation of Levi-Civita tensors, where the arguments must have head \mb{LorentzIndex} or \mb{Momentum}
} {
The internal function for Levi-Civita tensors, see also section \ref{int}.
}
\beom
%\dtog{{\$PrePrint=.\\
%}}{}{
%This clears \mb{\$PrePrint}.
%}
\domtog{{LeviCivita[$\mu$, $\nu$, $\rho$, $\sigma$] // StandardForm
}}{ Eps[LorentzIndex[$\mu$], LorentzIndex[$\nu$],LorentzIndex[$\rho$], LorentzIndex[$\sigma$]]\hfil\\
}{
This shows the internal structure.
}
\domtog{{Contract[\%\ FourVector[p, s]] // StandardForm
}}{ Eps[LorentzIndex[$\mu$], LorentzIndex[$\nu$],LorentzIndex[$\rho$], Momentum[p]]\hfil\\
}{
Contracting $\varepsilon^{\mu\,\nu\,\rho\,\sigma} \, p^{\sigma}$ 
yields $\varepsilon^{\mu\,\nu\,\rho\,p}$.

In the internal representation the $p$ has the head \mb{Momentum} wrapped 
around it.
}
\enom

\subsection{Simplifications of Expressions with Mandelstam Variables}
\label{trick}

\otabtwo{
\mbs{TrickMandelstam[{\sl expr}, \{{\sl s}, {\sl t}, {\sl u}, {\sl m1}\phat 2 + {\sl m2}\phat 2 + {\sl m3}\phat 2 + {\sl m4}\phat 2\}]}&
simplifies $expr$ to the shortest form removing $s, t$ or $u$ in
each sum using $s+t+u=m_1^{2} + m_2^{2} + m_3^{2} + m_4^{2}$
} {For tricky Mandelstam variable substitution.}

\beom
\domtog{{TrickMandelstam[(s + t - u) (2\ mw2 - t - u),
\{s, t, u, 2\ mw2\}]\\
}}{$2\, s\, ({\rm mw2} - u)$\\
}{
This is an easy example \\
of simplifications done by \mb{TrickMandelstam}.
}
\domtog{{TrickMandelstam[m\phat 2\ s - s\phat 2 + m\phat 2 t - s t + m\phat 2 u - s u,
\{s,\ t,\ u,\ 2\ m\phat 2\}]\\
}}{ $2\, m^2 (m^2\, -\, s)$
}{
The result is always given in a factorized form.
}
\enom

\subsection{Manipulation of Propagators}

\otabtwo{
\mbs{FeynAmpDenominatorCombine[{\sl expr}]} & {\ \\ expands {\sl expr} w.r.t. to \mb{FeynAmpDenominator} and  combines products of \mb{FeynAmpDenominator} into one \mb{FeynAmpDenominator}} \cr
\mbs{FeynAmpDenominatorSimplify[{\sl expr}]} & {\ \\ simplifies each  PropagatorDenominator in a canonical way} \cr
\mbs{FeynAmpDenominatorSimplify[{\sl expr}, {sl q}]} & {\ \\ includes some translation of momenta} \cr
\mbs{FeynAmpDenominatorSimplify[{\sl expr}, {sl q1}, {sl q2}]} & {\ \\ additionally removes 2-loop integrals with no mass scale} \cr
\mbs{FeynAmpDenominatorSplit[{\sl expr}]} & {\ \\ splits every \mb{FeynAmpDenominator[a,b, ...]} into \mb{FeynAmpDenominator[a]*FeynAmpDenominator[b] ...}} \cr
\mbs{FeynAmpDenominatorSplit[{\sl expr}, {sl q}]} & {\ \\ splits every \mb{FeynAmpDenominator[a,b, ...]} into a product of two, one with {sl q} and other momenta, the other without {sl q}} \cr
\mbs{PropagatorDenominatorExplicit[{\sl expr}]} & {\ \\ writes out \mb{FeynAmpDenominator[a,b, ...]} explicitly as a fraction} \cr
}{Functions for manipulating propagators.}

\subsection{Polarization Sums}

\otabtwo{
\mbs{PolarizationSum[{\sl mu}, {\sl nu}]} & $-g^{\mu \nu}$ \cr
\mbs{PolarizationSum[{\sl mu}, {\sl nu}, {\sl k}]} & $-g^{\mu \nu} + k^{\mu}\,k^{\nu}/k^2$ \cr
\mbs{PolarizationSum[{\sl mu}, {\sl nu}, {\sl k}, {\sl n}]} & 
$-g^{\mu \nu} -   
k_{\mu} k_{\nu} n^{2}/(k\cdot n)^{2}
+ (n_{\mu}  k_{\nu} +  n_{\nu}  k_{\mu})/(k \cdot n)$ 
}{Polarization sums.}

\beom
\domtog{{PolarizationSum[$\mu$, $\nu$, k]
}}{
\Frac{k^\mu\, k^\nu}{k^2} \ -$g^{\mu \nu}$
}{
This is the polarization sum for massive bosons:
$\Sigma \varepsilon_{\mu} \varepsilon^{*}_{\nu}\, =\, -g^{\mu \nu} + k^\mu \, k^\nu/k^2$.
}
\domtog{{PolarizationSum[$\mu$, $\nu$, k, p1 - p2]\\
}}{\renewcommand{\clineskip}{36pt} -$g^{\mu \nu}$\ 
-\ \Frac{k^\mu k^\nu\, {\rm p1}^2}{\Superscript{(k \cdot p1\, -\, k \cdot {\rm p2})}{2}}\ 
+\ \Frac{2 k^\mu k^\nu\, {\rm p1 p2}}{\Superscript{(k \cdot p1\, -\, k \cdot {\rm p2})}{2}}\ 
-\ \Frac{k^\mu k^\nu\, {\rm p2}^2}{\Superscript{(k \cdot p1\, -\, k \cdot {\rm p2})}{2}}\ 
+\ \Frac{({\rm p1}\, -\, {\rm p2})^\mu\, k^\nu}{k {\rm p1}\, -\, k {\rm p2}}
+\ \Frac{k^\mu\, ({\rm p1}\, -\, {\rm p2})^\nu}{k {\rm p1}\, -\, k {\rm p2}}
\renewcommand{\clineskip}{20pt}}{
Here the polarization sum for gluons is given; 
with external momentum $n=p_1 -p_2$.
}
\enom

\subsection{Permuting the Arguments of the Four-Point Function}

The arguments of the Passarino-Veltman four-point function \mb{D0}
are permuted by \fc into an alphabetical order. If you want a specific  
permutation of the 24 possible ones, you can use the 
function \mb{PaVeOrder}.

\otabtwo{
\mbs{PaVeOrder[{\sl expr}]}&
order the arguments of \mb{C0} and \mb{D0} in $expr$ 
canonically.
}{An ordering function for $C_0$ and  $D_0$.}

\otabthree{
{\sl option name} & {\sl default value} \cr
\hhline
\mb{PaVeOrderList} & \mb{\{\}} & order $D_0$ according to \mbs{\{..., {\sl a}, {\sl b}, ... \}}
} {An ordering option for \mb{PaVeOrder}.}
With the default setting of \mb{PaVeOrder} a canonical ordering is chosen.

\beom
\domtog{{PaVeOrder[D0[me2, me2, mw2, mw2, t, s, me2, 0, me2, 0],\\
PaVeOrderList\ $\Rule$\ \{me2, me2, 0, 0\}]\\
}}{$D_0$(me2, $s$, mw2, $t$, mw2, me2, me2, 0, 0, me2)\hfil\\
}{
It is sufficient to supply a subset of the arguments.
}
\domtog{{PaVeOrder[D0[a, b, c, d, e, f, m12, m22, m32, m42],\\
PaVeOrderList\ $\Rule$\ \{f, e\}]\\
}}{$D_0$($a, d, c, b, f, e$, m22, m12, m42, m32)\hfil\\
}{
This interchanges $f$ and $e$. 
}
\domtog{{PaVeOrder[D0[a, b, c, d, e, f, m12, m22, m32, m42]\ + \ D0[me2, me2, mw2, mw2, t, s, me2, 0, me2, 0],\ PaVeOrderList\ $\Rule$\ \{ \{me2, me2, 0, 0\}, \{f, e\}\}]\\
}}{$D_0$($a, d, c, b, f, e$, m22, m12, m42, m32) + 
$D_0$(me2, $s$, mw2, $t$, mw2, me2, me2, 0, 0, me2)\hfil\\
}{
This shows how to permute several $D_0$.
}
\enom


\chaptermark{Reference Guide for FeynCalc}
\section{Reference Guide for FeynCalc}
\label{rg}

\fname{A0}
\fheading{\mb{A0[{\sl m}\phat 2]}}
\fusage{
\mb{A0[{\sl m}\phat 2]} is the scalar Passarino-Veltman one-point
function.
}
\fnotes{
{\tiny $\blacksquare$}\  \mb{A0[{\sl m}\phat 2]} is an abbreviation for 
$A_{0}(m^2) = -i \pi^{-2} \int d^{D} q\,(2\mu\pi)^{4-D}
                       \,[q^{2}-m^{2}]^{-1}$.
{\tiny $\blacksquare$}\  
\mb{A0[0]} and  \mb{A0[Small[{\sl mass}\phat 2]]} give 0.
{\tiny $\blacksquare$}\  
The following option can be given:
\ftabthree
{
A0ToB0 & True & replace $A_0(m^2)$ by $m^2 (1+B_0(0,m^2,m^2))$
}

\seepage
}

\ffinish

\fname{A0ToB0}
\fheading{\mb{A0ToB0}}
\fusage{
\mb{A0ToB0} is an option for \mb{A0}. If set to \mb{True}, $A_0(m^2)$ 
is replaced by $m^2 (1 + B_0(0,m^2,m^2))$.
}
\fnotes{
\seepage
}

\ffinish

\fname{B0}
\fheading{\mb{B0[{\sl pp},$\hbox{\sl m}_{1}$\phat 2,$\hbox{\sl m}_{2}$\phat 2]}}
\fusage{
\mb{B0[{\sl pp},$\hbox{\sl m}_{1}$\phat 2,$\hbox{\sl m}_{2}$\phat 2]} is the scalar  Passarino-Veltman two-point 
function.
}
\fnotes{
{\tiny $\blacksquare$}\  \mb{B0[{\sl pp},$\hbox{\sl m}_{1}$\phat 2,$\hbox{\sl m}_{2}$\phat 2]} is an abbreviation for
$B_{0}(p^{2},m_{1}^2,m_{2}^2)= -i \pi^{-2} \int
                           d^{D} q\,(2\mu\pi)^{4-D} \,
               ([q^{2}-m_{1}^{2}] [(q+p)^{2}-m_{2}^{2}])^{-1}$. {\tiny $\blacksquare$}\  \mb{B0} is symmetric in its second and third argument. {\tiny $\blacksquare$}\  The following options can be given:
\ftabthree
{
\mb{BReduce} & \mb{False} & reduce \mb{B0} in special cases 
to \mb{A0} \cr
\mb{B0Unique} & \mb{False} & replace $B_0(a,0,a)$ by 
$(2+B_0(0,a,a))$ and $B_0(0,0,a)$ by $(1+B_0(0,a,a))$.
}

\seepage
}

\ffinish


\fname{B00}
\fheading{\mb{B00[{\sl pp},$\hbox{\sl m}_{1}$\phat 2,$\hbox{\sl m}_{2}$\phat 2]}}
\fusage{
\mb{B00[{\sl pp},$\hbox{\sl m}_{1}$\phat 2,$\hbox{\sl m}_{2}$\phat 2]} is the coefficient of  $g^{\mu \nu}$ of the 
tensor integral decomposition  of $B^{\mu \nu}$.
}
\fnotes{
{\tiny $\blacksquare$}\  The following option can be given:
\ftabthree
{
\mb{BReduce} & \mb{True} & reduce \mb{B00}  to \mb{B1} and \mb{A0}
} {\tiny $\blacksquare$}\  If \mb{BReduce} is set to \mb{True} the following simplification holds:

$B_{00}(a,b,c) = 1/6 (A_0(b) + (B_1(a,b,c) (a-c+b) + a/3 B_0(a,b,c) +
1/6 (a + b - c/3) ))$.

\seepage
}

\ffinish

\fname{B0Unique}
\fheading{\mb{B0Unique}}
\fusage{
\mb{B0Unique} is an option for \mb{B0}. 
}
\fnotes{
{\tiny $\blacksquare$}\  Sometimes it is useful to set this option to \mb{True},  since  only then 
all simplifications between different $B_0$ occur.

\seepage
}

\ffinish



\fname{B1}
\fheading{\mb{B1[pp,$\hbox{\sl m}_{1}$\phat 2,$\hbox{\sl m}_{2}$\phat 2]}}
\fusage{
\mb{B1[{\sl pp},$\hbox{\sl m}_{1}$\phat 2,$\hbox{\sl m}_{2}$\phat 2]} is the coefficient of  $p^{\mu}$ of the
tensor integral decomposition  of $B^{\mu}$.
       }
\fnotes{
{\tiny $\blacksquare$}\  The following option can be given:
\ftabthree
{
\mb{BReduce} & \mb{True} & reduce \mb{B1}  to \mb{B0} and \mb{A0}
} {\tiny $\blacksquare$}\  If a variable of $B_1$ has head \mb{Small} and at least one other 
variable does not (and is different from 0), the variable with head \mb{Small}
is set to 0. {\tiny $\blacksquare$}\  If the option \mb{BReduce} is set to \mb{True}, \mb{B1} simplifies in 
the following way, where $a,b,c$ are $m^2$ with no head \mb{Small}.
The same simplifications are performed if instead of 0 a 
\mb{Small} variable is supplied as an argument.

$B_1(a,b,c)=1/2 ( A_0(b) - A_0(b) - (a-c+b) B_0(a,b,c))$.

$B_1(a,b,b)=-1/2 B_0(a,b,b)$.

$B_1(a,a,0) = - 1/2 B_0(a,a,0) - 1/2$.

$B_1(a,0,a) = 1/2 - 1/2 B_0(a,0,m)$.
 
$B_1(0,0,a) = - 1/2 B_0(0,0,a) + 1/4$.

$B_1(0,a,0) = - 1/2 B_0(0,a,0) - 1/4$.

\seepage
}

\ffinish

\fname{B11}
\fheading{\mb{B11[{\sl pp},$\hbox{\sl m}_{1}$\phat 2,$\hbox{\sl m}_{2}$\phat 2]}}
\fusage{
\mb{B11[{\sl pp},$\hbox{\sl m}_{1}$\phat 2,$\hbox{\sl m}_{2}$\phat 2]} is the coefficient of  $p^{\mu}\,p^{\nu}$ of the
tensor integral decomposition  of $B^{\mu \nu}$.
}
\fnotes{
{\tiny $\blacksquare$}\  The following option can be given:
\ftabthree{
\mb{BReduce} & \mb{True} & reduce \mb{B11}  to \mb{B1} and \mb{A0}
}
 {\tiny $\blacksquare$}\  If the option \mb{BReduce} is set to \mb{True}, \mb{B11} simplifies in
the following way, where $a,b,c$ are $m^2$ with no head \mb{Small}.

$B_{11}(a,b,c) = 1/(3 a) (A_0(b) - 2 (a - c + b) B_1(a,b,c) -
a B_0(a,b,c) - 1/2(a + b - c/3) )$.

$B_{11}(0,a,a) = 1/3 B_0(0,a,a)$.

\seepage
}

\ffinish

\fname{BReduce}
\fheading{\mb{BReduce}}
\fusage{
\mb{BReduce} is an option for \mb{B0}, \mb{B00}, \mb{B1} and \mb{B11}, 
determining whether reductions to lower-order \mb{A} and \mb{B} are done.
       }
\fnotes{
{\tiny $\blacksquare$}\  The default setting of \mb{BReduce} is \mb{True}.

\seepage
}
\ffinish


\fname{CancelQ2}
\fheading{\mb{CancelQ2}}
\fusage{\mb{CancelQ2}  is an option for \mb{OneLoop}.
If set to \mb{True}, cancellation
of all $q^2$ with the first propagator via
$q^{2} \rightarrow ( (q^{2}-m^{2}) + m^{2} )$ is performed, where 
$q$ denotes the integration momentum.
}
\fnotes{
{\tiny $\blacksquare$}\  With the default \mb{True} the translation of the integration momentum 
in the lower order Passarino Veltman functions is done such that
the third mass argument of the 4-point integral is put at position 1.

\seepage
}

\ffinish


\fname{CancelQP}
\fheading{\mb{CancelQP}}
\fusage{\mb{CancelQP}  is an option for \mb{OneLoop}.
If set to \mb{True}, cancellation
of $q \cdot p$ with propagators 
is performed, where $q$ denotes the integration momentum.
}
\fnotes{
\seepage
}

\ffinish

\fname{ChiralityProjector}
\fheading{\mb{ChiralityProjector[+1]}}
\fusage{\mb{ChiralityProjector[+1]} is an alternative input for
$\omega_{+} = \frac{1}{2} (1 + \gamma^{5})$.
\mb{ChiralityProjector[-1]} denotes 
$\omega_{-} = \frac{1}{2} (1 - \gamma^{5})$. 
}
\fnotes{
{\tiny $\blacksquare$}\  \mb{ChiralityProjector[1]} is identical to \mb{DiracMatrix[6]}.
\mb{ChiralityProjector[-1]} is identical to \mb{DiracMatrix[7]}.
The internal representation is \mb{DiracGamma[6]} and 
\mb{DiracGamma[7]}.

\seepage
 {\tiny $\blacksquare$}\  See also: \mb{DiracMatrix}, \mb{DiracGamma}.
}
\ffinish
 
\fname{C0}
\fheading{\mb{C0}}
\fusage{
\mb{C0[$\hbox{\sl p}_{10}$,$\hbox{\sl p}_{12}$,$\hbox{\sl p}_{20}$,$\hbox{\sl m}_{1}$\phat 2,$\hbox{\sl m}_{2}$\phat 2,$\hbox{\sl m}_{3}$\phat 2]} is the scalar  
Passarino-Veltman three-point function.
The first three arguments of \mb{C0} are the scalar products
$p_{10} = p_{1}^{2}$, $p_{12} = (p_{1}-p_{2})^{2}$, $p_{20} = p_{2}^{2}$.
       }
\fnotes{
{\tiny $\blacksquare$}\  \mb{C0[$\hbox{\sl p}_{10}$,$\hbox{\sl p}_{12}$,$\hbox{\sl p}_{20}$,$\hbox{\sl m}_{1}$\phat 2,$\hbox{\sl m}_{2}$\phat 2,$\hbox{\sl m}_{3}$\phat 2]} is an abbreviation for 
$C_0= -i \pi^{-2} \int
d^{D} q\,(2\mu\pi)^{4-D} \,
([q^{2}-m_{1}^{2}] [(q+p_1)^{2}-m_{2}^{2}]
 [(q+p_2)^{2}-m_{3}^{2}])^{-1}
$. {\tiny $\blacksquare$}\  A standard representative of the six equivalent argument permutations is 
chosen.

\seepage
 {\tiny $\blacksquare$}\  See also: \mb{PaVeOrder}.
}

\ffinish

\fname{Collect2}
\fheading{\mb{Collect2[{\sl expr},\ {\sl x}]}}
\fusage{\mb{Collect2[{\sl expr},\ {\sl x}]} collects together terms 
which are not free of any occurrence of $x$.

\mb{Collect2[{\sl expr},\ \{$\hbox{\sl x}_{1}$,\ $\hbox{\sl x}_{2}$,\ ...\}]} collects together terms 
which are not free of $x_1$, $x_2$, \ldots. 
}
\fnotes{
{\tiny $\blacksquare$}\  The following option can be given.
\ftabthree
{
\mb{ProductExpand} & \mb{False} & expand products in $expr$ 
free of $x$ ($x_1, x_2$, ...)\cr
\mb{IsolateHead} & \mb{False} & 
isolate with respect to \mb{\{$\hbox{\sl x}_{1}$,\ $\hbox{\sl x}_{2}$,\ ...\}}\cr
\mb{IsolateSplit} & \mb{442} & the limit before \mb{Isolate} splits 
}

\seepage {\tiny $\blacksquare$}\  See also: \mb{Isolate}.
}

\ffinish

\fname{Combine}
\fheading{\mb{Combine[{\sl expr}]}}
\fusage{\mb{Combine[{\sl expr}]} puts terms in a sum over a common denominator.
\mb{Combine} is similar to \mb{Together}, but works better on certain
polynomials with rational coefficients.} 
\fnotes{
{\tiny $\blacksquare$}\  The following option can be given:
\ftabthree{
\mb{ProductExpand} & \mb{False} & expand products
}

\seepage
}

\ffinish

\fname{CombineGraphs}
\fheading{\mb{CombineGraphs}}
\fusage{\mb{CombineGraphs}  is an option for \mb{OneLoopSum}.
}
\fnotes{
{\tiny $\blacksquare$}\  The utilization of this option may speed up \fc.
But depending on the available memory there is a turning point where 
\fc~ becomes quite inefficient when forced to calculate too may complicated
diagrams at once.

\seepage
}

\ffinish

\fname{Contract}
\fheading{\mb{Contract}}
\fusage{
\mb{Contract[{\sl expr}]} contracts pairs of Lorentz indices in \mb{expr}.
       }

\fnotes{
{\tiny $\blacksquare$}\  For the contraction of two Dirac matrices \mb{DiracSimplify} has to be used. {\tiny $\blacksquare$}\  The result of \mb{Contract} is not fully expanded.  {\tiny $\blacksquare$}\  The following options can be given:
\ftabthree
{
\mb{EpsContract}   & \mb{False}  & contract products of 
                  Levi-Civita (Eps) tensors via the determinant 
                  formula\cr
\mb{Expanding} & \mb{True} & expand all sums containing Lorentz indices \cr
\mb{Factoring}     & \mb{False} & factor the result canonically
}

{\tiny $\blacksquare$}\  Examples:


%\mb{Contract[\ MetricTensor[al,\ al]\ ]} \ra 4.
%
%\mb{Contract[\ MetricTensor[al,\ al,\ Dimension$\rightarrow$D\ ]} \ra $D$.
%
%\mb{Contract[\ MetricTensor[al,\ be]\ MetricTensor[al,\ si]\ ]} \ra \mb{g[be,\ si]}.
%
%\mb{Contract[\ MetricTensor[al,\ be]\ FourVector[p,\ al]\ ]} \ra \mb{p[be]}.
%\mb{Contract[\ DiracMatrix[mu]\ MetricTensor[mu,\ al]\ ]} \ra \mb{ga[al]}. 
%
%\mb{Contract[\ DiracMatrix[mu]\ FourVector[p,\ mu]\ ]} \ra \mb{gs[p]}.
%\mb{Contract[\ FourVector[p\ +\ q,\ mu]\ MetricTensor[mu,\ nu]\ ]} \ra 
%\mb{(p\ +\ q)[nu]}.
\mb{Contract[\ FourVector[p,\ mu]\phat 2\ ]} \ra \mb{p.p}. 

\mb{Contract[\ FourVector[p\ +\ q,\ mu]\ FourVector[p'\ +\ q',\ mu]\ ]// \\
ExpandScalarProduct} \ra \mb{p.p'\ +\ p.q'\ +\ q.p'\ +\ q.q'}.

\mb{Contract[\ LeviCivita[m,\ n,\ r,\ a]\ LeviCivita[m,\ n,\ r,\ b],\ EpsContract\ $\rightarrow$\ True\ ]} \ra - \mb{6\ g[a,\ b]}.

\mb{Contract[\ Pair[\ LorentzIndex[mu,\ Dim],\ LorentzIndex[mu,\ Dim]\ ]\ ]} \ra 
\mb{Dim}.

\mb{Contract[\ Pair[\ LorentzIndex[mu,\ D\ -\ 4],\ LorentzIndex[mu]\ ]\ ]} \ra \mb{4}.

\mb{Contract[\ Pair[\ LorentzIndex[mu,\ D\ -\ 4],\ LorentzIndex[mu]\ ]\ ]} \ra \mb{0}.

\mb{Contract[\ Pair[\ LorentzIndex[mu,\ D\ -\ 4],\ LorentzIndex[mu,\ D\ -\ 4]\ ]\ ]} \ra 
\mb{-4\ +\ D}.

%Clear \mb{\$PrePrint} for the next examples (\mb{\$PrePrint=.}).
\mb{Contract[\ f[\ \_\_\_,\ LorentzIndex[mu],\ \_\_\_\ ]\ MetricTensor[mu,\ al]\ ]}
\ra \mb{\ f[\ \_\_\_,\ LorentzIndex[al],\ \_\_\_\ ]}.

\mb{Contract[\ f[\ \_\_\_,\ LorentzIndex[mu],\ \_\_\_\ ]\ FourVector[p,\ mu]\ ]}
\ra \mb{\ f[\ \_\_\_,\ FourVector[p],\ \_\_\_\ ]}.

\mb{Contract[\ DiracMatrix[mu,\ Dimension\ $\rightarrow$\ D]\ MetricTensor[mu,\ al]\ ]} \ra \\
\mb{DiracGamma[LorentzIndex[nu]]}.

\mb{Contract[\ DiracGamma[LorentzIndex[mu,\ D-4],\ D-4]\ MetricTensor[mu,\ al]\ ]} \ra
\mb{0}.

{\tiny $\blacksquare$}\  If big expressions are contracted and substitutions for the
resulting scalar products are made, it is best not to do these
replacements after contraction, but to set the values of the relevant 
scalar products {\it before} invoking \mb{Contract}; in this way the 
intermediate expression swell is minimized.

\seepage

{\tiny $\blacksquare$}\  See also: \mb{ExpandScalarProduct}.  
}

\ffinish

\fname{D0}
\fheading{\mb{D0}}
\fusage{
\mb{D0[$\hbox{\sl p}_{10}$,\ $\hbox{\sl p}_{12}$,\ $\hbox{\sl p}_{23}$,\ $\hbox{\sl p}_{30}$,\ $\hbox{\sl p}_{20}$,\ $\hbox{\sl p}_{13}$,\ $\hbox{\sl m}_{1}$\phat 2,\ $\hbox{\sl m}_{2}$\phat 2,\ $\hbox{\sl m}_{3}$\phat 2,\ $\hbox{\sl m}_{4}$\phat 2]}  is the scalar Passarino-Veltman four-point function. 
The first six arguments of \mb{D0} are the scalar products
$ p_{10} = p_{1}^{2},\;p_{12} = (p_{1}-p_{2})^{2},\; 
p_{23} = (p_{2}-p_{3})^{2},\; p_{30} = p_{3}^{2},\;
p_{20} = p_{2}^{2},\;
p_{13} = (p_{1}-p_{3})^{2}.$
}
\fnotes{
{\tiny $\blacksquare$}\  \mb{D0[$\hbox{\sl p}_{10}$,\ $\hbox{\sl p}_{12}$,\ $\hbox{\sl p}_{23}$,\ $\hbox{\sl p}_{30}$,\ $\hbox{\sl p}_{20}$,\ $\hbox{\sl p}_{13}$,\ $\hbox{\sl m}_{1}$\phat 2,\ $\hbox{\sl m}_{2}$\phat 2,\ $\hbox{\sl m}_{3}$\phat 2,\ $\hbox{\sl m}_{4}$\phat 2]} is an abbreviation for 
$D_0= -i \pi^{-2} \int
d^{4} q\,
([q^{2}-m_{1}^{2}] [(q+p_1)^{2}-m_{2}^{2}]
 [(q+p_2)^{2}-m_{3}^{2}] [(q+p_{3})^{2}-m_{4}^{2}])^{-1}
 $.

\seepage {\tiny $\blacksquare$}\  See also: \mb{PaVe}, \mb{PaVeOrder}.
}

\ffinish

\fname{D0Convention}
\fheading{\mb{D0Convention}}
\fusage{
\mb{D0Convention} is an option for \mb{Write2}.
Possible settings are $0$ or $1$. With the last setting the fifth and sixth arguments
of \mb{D0} are interchanged and all (internal) mass arguments of the scalar
Passarino Veltman integrals are given square free.
}
\fnotes{
\seepage
{\tiny $\blacksquare$}\  See also: \mb{Write2}, \mb{D0}.
}

\ffinish

\fname{DB0}
\fheading{\mb{DB0}}
\fusage{
\mb{DB0[$\hbox{\sl p}_{10}$,\ $\hbox{\sl m}_{0}$\phat 2,\ $\hbox{\sl m}_{1}$\phat 2]} is the derivative 
$\partial B_0(p^2, m_0^2, m_1^2) / \partial p^2$ of the two-point function
$B_0$.
}
\fnotes{
\seepage
{\tiny $\blacksquare$}\  See also: \mb{B0}.
}

\ffinish

\fname{DenominatorOrder}
\fheading{\mb{DenominatorOrder}}
\fusage{
\mb{DenominatorOrder} is an option for \mb{OneLoop}, if set to
\mb{True} the \mb{PropagatorDenominator} in 
\mb{FeynAmpDenominator} will be ordered in a standard way.
}
\fnotes{
{\tiny $\blacksquare$}\  You may want to set this option to \mb{False} when checking hand calculations.

\seepage
       }

\ffinish

\fname{Dimension}
\fheading{\mb{Dimension}}
\fusage{
\mb{Dimension} is an option for  
\mb{DiracMatrix}, \mb{DiracSlash}, \mb{FourVector}, 
\mb{MetricTensor}, \mb{OneLoop} and \mb{ScalarProduct}.
       }
\fnotes{
{\tiny $\blacksquare$}\  The setting of \mb{Dimension} may be \mb{4}, \mb{dim} or \mb{dim-4}, 
where \mb{dim} must be a \mma \mb{Symbol}\hspace{0.2mm}.

\seepage
}

\ffinish


\fname{DiracGamma}
\fheading{\mb{DiracGamma[{\sl x},\ {\sl optdim}]}}
\fusage{
\mb{DiracGamma[{\sl x},\ {\sl optdim}]} is the head of all Dirac matrices and
Feynman slashes $\bps$ ($=\gamma _{\mu} p^{\mu}$) in the internal 
representation. 

A four-dimensional Dirac  matrix 
$\gamma _{\mu}$  is \mb{DiracGamma[\ LorentzIndex[{\sl mu}]\ ]}, 
a four-dimensional Feynman slash is  \mb{DiracGamma[\ Momentum[{\sl p}]\ ]}.

$\gamma _{5}$ is represented as {\tt DiracGamma[5]},  the helicity projectors
$\gamma _{6}  = (1+\gamma _{5})/2 $ and $\gamma _{7}  = (1-\gamma _{5})/2 $ 
as \mb{DiracGamma[6]} and \mb{DiracGamma[7]} respectively.

For other than four dimensions an additional argument is necessary:
\mb{DiracGamma[\ LorentzIndex[{\sl mu},\ {\sl Dim}],\ {\sl Dim}\ ]} and  
\mb{DiracGamma[\ Momentum[{\sl q},\ {\sl Dim}],\ {\sl Dim}\ ]}.
}
\fnotes{
{\tiny $\blacksquare$}\  For standard input \mb{DiracMatrix} and \mb{DiracSlash} are more convenient. {\tiny $\blacksquare$}\  Note that \mb{DiracGamma[{\sl exp},\ {\sl Dim}]} projects out the smaller
dimension of the objects \mb{exp} and \mb{Dim}.
There are special relationships if \mb{Dim} takes the form
\mb{D-4}, e.g., \mb{DiracGamma[\ LorentzIndex[{\sl mu},\ {\sl D}-4],{\sl 4}\ ]} 
\ra \mb{0}.

\seepage
}

\ffinish

\fname{DiracMatrix}
\fheading{\mb{DiracMatrix[{\sl mu}]}} 
\fusage{
\mb{DiracMatrix[{\sl mu}]} is an input function for a Dirac matrix. 
A product of Dirac matrices $\gamma^{\mu} \gamma^{\nu} \gamma^{\rho} \, 
\ldots$ is entered as \mb{DiracMatrix[\ mu,\ nu,\ ro,\ ...\ ]} or equivalently as 
\mb{DiracMatrix[mu]\ .\ DiracMatrix[nu]\ .\ DiracMatrix[ro]\ .\ ...} 

$\gamma ^{5} $ may be entered as \mb{DiracMatrix[5]}, 
$\gamma ^{6}=(1+\gamma ^{5})/2$ as \mb{DiracMatrix[6]} 
and $\gamma ^{7}=(1-\gamma ^{5})/2$ as \mb{DiracMatrix[7]}
}
\fnotes{
{\tiny $\blacksquare$}\  The following option can be given:
\ftabthree
{
\mb{Dimension} & 4 & space-time dimension
}
{\tiny $\blacksquare$}\  $\gamma ^{5}$, $\gamma ^{6}$ and $\gamma ^{7}$ are defined purely in four dimensions.
\seepage
{\tiny $\blacksquare$}\  See also: \mb{DiracGamma}, \mb{DiracSlash}, \mb{ChiralityProjector}, \mb{\$BreitMaison}.
}

\ffinish


\fname{DiracOrder}
\fheading{\mb{DiracOrder[{\sl expr},\ {\sl orderlist}]}}
\fusage{
\mb{DiracOrder[{\sl expr},\ {\sl orderlist}]}  orders the Dirac matrices in 
\mb{expr} according to $orderlist$.

\mb{DiracOrder[{\sl expr}]}  orders the Dirac matrices in
\mb{expr} alphabetically.
}
\fnotes{{\tiny $\blacksquare$}\  $\gamma_{5}, \gamma_{6}$ and  $\gamma_{7}$ are not ordered;
use {\tt DiracSimplify} to push them all to the right. {\tiny $\blacksquare$}\  Example: \mb{DiracOrder[\ DiracSlash[a,\ q,\ a,\ p]\ ]} \ra
      \mb{-2\ a.a\ p.q\ +\ 2\ a.q\ gs[a]\ .\ gs[p]\ +\ a.a\ gs[p]\ .\ gs[q]}.

\mb{DiracOrder[\ DiracSlash[p],\ DiracSlash[q],\ \{q,p\}\ ]} \ra 
\mb{2\ p.q\ -\ gs[q]\ .\ gs[p]}. {\tiny $\blacksquare$}\  This function is just the
implementation of the anticommutator relation for Dirac  matrices.

\seepage

See also: \mb{DiracSimplify}.
}

\ffinish


\fname{DiracSimplify}
\fheading{\mb{DiracSimplify[{\sl expr}]}}
\fusage{\mb{DiracSimplify[{\sl expr}]}  simplifies products of Dirac matrices 
in \mb{expr}. Double Lorentz indices and four vectors are contracted.
The Dirac equation is applied.
All \mb{DiracMatrix[5]}, \mb{DiracMatrix[6]} and \mb{DiracMatrix[7]} are 
moved to the right. 
The order of the Dirac matrices is not changed.
}
\fnotes{
{\tiny $\blacksquare$}\  $(\ps - m)\, u(p) = 0$, $\;(\ps + m)\, v(p) = 0\,\;$
and  $\, \; \overline{u}(p)(\ps-m)=0$, $\;\overline{v}(p)(\ps+m)=0\,$ are 
represented by:
\mb{DiracSimplify[\ (DiracSlash[p]\ -\ m)\ .\ Spinor[p,\ m]\ ]} \ra 0,
\mb{DiracSimplify[\ (DiracSlash[p]\ +\ m)\ .\ Spinor[-p,m]\ ]} \ra 0,
\mb{DiracSimplify[\ Spinor[p,\ m]\ .\ (DiracSlash[p]\ -\ m\ )\ ]} \ra 0,
\mb{DiracSimplify[\ Spinor[-p,m]\ .\ (DiracSlash[p]\ +\ m\ )\ ]} \ra 0

{\tiny $\blacksquare$}\  Examples:

\mb{DiracSimplify[\ DiracMatrix[mu,\ mu]\ ]} \ra 4.

  \mb{DiracSimplify[\ DiracSlash[p,\ p]\ ]} \ra \mb{p.p}.
  
  \mb{DiracSimplify[\ DiracMatrix[al,\ be,\ al]\ ]} \ra \mb{-2\ ga[be]}.

  \mb{DiracSimplify[\ DiracMatrix[al,\ be,\ al,\ Dimension$\rightarrow$D]\ ]//Factor} \ra 
  \mb{(2-D)\ ga[be]}.

  \mb{DiracSimplify[\ DiracSlash[p],\ (DiracSlash[-q]+m),\ DiracSlash[p]\ ]} \ra 
  \mb{gs[q]\ p.p\ -\ 2\ gs[p]\ p.q\ +\ p.p\ m}.

  \mb{DiracSimplify[\ DiracMatrix[5,\ mu]\ ]} \ra \mb{-ga[mu]\ .\ ga[5]}.

  \mb{DiracSimplify[\ DiracMatrix[6,\ mu]\ ]} \ra \mb{ga[mu]\ .\ ga[7]}.

\seepage {\tiny $\blacksquare$}\  See also: \mb{DiracOrder}.
}

\ffinish

\fname{DiracSlash}
\fheading{\mb{DiracSlash}}
\fusage{\mb{DiracSlash[{\sl p}]} is an input function for a Feynman slash
$\bps= \gamma^{\mu} p_{\mu}$. A product of slashes may be entered by 
\mb{DiracSlash[p,\ q,\ ...]} or 
\mb{DiracSlash[p],\ DiracSlash[q],\ ...} .
}
\fnotes{
{\tiny $\blacksquare$}\  The following option can be given:
\ftabthree
{
Dimension & 4 & space-time dimension 
} {\tiny $\blacksquare$}\  The internal representation of a four-dimensional \mb{DiracSlash[p]} is
\mb{DiracGamma[\ Momemtum[{\sl p}]\ ]},  a \mb{D}-dimensional
\mb{DiracSlash[\ {\sl p},\ {\sl Dimension}\ $\rightarrow$\ D\ ]} is transformed into
\mb{DiracGamma[\ Momemtum[p,\ D],\ D]}.

\seepage {\tiny $\blacksquare$}\  See also: \mb{DiracGamma}, \mb{DiracMatrix}.
}

\ffinish

\fname{DiracTrace}
\fheading{\mb{DiracTrace}}
\fusage{\mb{DiracTrace[{\sl expr}]}  is the head of a  Dirac trace. 
Whether the trace is  evaluated depends on the option
\mb{DiracTraceEvaluate}.  
The argument \mb{expr} may be a product of Dirac matrices or slashes
separated by ``.''.
}
\fnotes{
{\tiny $\blacksquare$}\  The following options can be given:
\ftabthree
{
\mb{DiracTraceEvaluate}  &  \mb{False}  & evaluating the trace \cr
\mb{LeviCivitaSign}      &  \mb{-1}     & sign convention for the 
$\varepsilon$-tensor \cr
\mb{Factoring}           &  \mb{False}  & factor the result \cr
\mb{Mandelstam}          &  \mb{\{\}}          & 
if set to \mb{\{{\sl s},\ {\sl t},\ {\sl u},\ $\hbox{\sl m}_{1}$\phat 2\ +\ $\hbox{\sl m}_{2}$\phat 2\ +\ $\hbox{\sl m}_{3}$\phat 2\ +\ $\hbox{\sl m}_{4}$\phat 2\}}, 
\mb{TrickMandelstam} will be used \cr
\mb{PairCollect}         &  \mb{True}   & collect the results 
with respect to products of \mb{Pair}
}

{\tiny $\blacksquare$}\  Examples:

\mb{DiracTrace[\ DiracMatrix[al,\ be]\ ]} \ra \mb{4\ g[al,\ be]}.

\mb{DiracTrace[\ DiracSlash[p,\ q]\ ]} \ra \mb{4\ p.q}.

\mb{DiracTrace[\ DiracMatrix[mu,\ al,\ be,\ mu,\ Dimension$\rightarrow$D]\ ]} \ra 
\mb{4\ D\ g[al,\ be]}.

\mb{DiracTrace[\ DiracMatrix[a,\ b,\ c,\ d,\ 5]\ ]} \ra \mb{-4\ I\ Eps[a,\ b,\ c,\ d]}.

\mb{DiracTrace[\ MetricTensor[al,\ be]\ DiracMatrix[si,\ al,\ ro,\ si]\ ]} \ra 
\mb{16\ g[be,\ ro]}.

\mb{DiracTrace[\ DiracSlash[p\ -\ q]\ .\ (DiracSlash[q]\ +\ m)\ +\ DiracSlash[k,\ 2\ p]\ ]}
\ra \mb{8\ k.p\ +\ 4\ p.q\ -\ 4\ q.q}. 

With
\mb{PP=DiracSlash[p']}; \mb{P=DiracSlash[p]}; \mb{MU=DiracMatrix[mu]}; 
\mb{K=DiracSlash[k]}; \mb{NU=DiracMatrix[nu]}:
\mb{DiracTrace[\ (PP+m).MU.(P+K+m).NU.(P+m).NU.(P+K+m).MU/16\ ]/.m\phat 2$\rightarrow$m2/.m\phat 4$\rightarrow$m4}
\ra \mb{4\ m4\ +\ k.p\ (4\ m2\ +\ 2\ k.p')\ +\ k.p'\ (-4\ m2\ +\ 2\ p.p)\ +\ k.k\ (4\ m2\ -\ p.p')\ -\ 3\ m2\ p.p'\ +\ p.p\ p.p'}.

{\tiny $\blacksquare$}\   To  replace scalar products two possibilities exist:
 either set the corresponding \mb{ScalarProduct} 
 {\sl before} calculating the trace:
 \mb{ScalarProduct[p,\ q]\ =\ t/2;} (\mb{DiracTrace[p\ .\ q]}) \ra 
\mb{2\ t}, which is preferable, or substitute afterwards:  
(\mb{DiracTrace[\ DiracSlash[a,\ b]\ ]/.ScalarProduct[a,\ b]$\rightarrow$s/2}) \ra \mb{2\ s}.

\seepage {\tiny $\blacksquare$}\  See also: \mb{DiracMatrix}, \mb{DiracSlash}, 
\mb{ScalarProduct}, \mb{TrickMandelstam}.
 }

 \ffinish

\fname{DiracTraceEvaluate}
\fheading{\mb{DiracTraceEvaluate}}
\fusage{\mb{DiracTraceEvaluate}  is an option for \mb{DiracTrace}.
If set to \mb{False},\mb{DiracTrace} remains unevaluated.
}
\fnotes{
{\tiny $\blacksquare$}\  The resaon for this option is that \mb{OneLoop} needs traces in unevaluated
form.

\seepage
}
 
\ffinish


\fname{Eps}
\fheading{\mb{Eps[{\sl a},\ {\sl b},\ {\sl c},\ {\sl d}]}}
\fusage{\mb{Eps[{\sl a},\ {\sl b},\ {\sl c},\ {\sl d}]} is the head of the totally antisymmetric 
four-dimensional epsilon (Levi-Civita) tensor. 
The \mb{{\sl a},\ {\sl b},\ {\sl c},\ {\sl d}} {\it must} have head \mb{LorentzIndex} or \mb{Momentum}.
}
\fnotes{
{\tiny $\blacksquare$}\  All entries are transformed to four dimensions. {\tiny $\blacksquare$}\  For user-friendly input of an \mb{Eps} just with \mb{LorentzIndex}'
use \mb{LeviCivita}. {\tiny $\blacksquare$}\  For $\varepsilon ^{ \mu \nu \rho \sigma } q_{ \sigma }$ the
compact notation $\varepsilon ^{ \mu \nu \rho q}$ is used, i.e.:
\mb{Eps[LorentzIndex[mu],\ LorentzIndex[nu],\ LorentzIndex[ro],\ Momentum[q]]}. {\tiny $\blacksquare$}\  \mb{Eps} is just a head not having any functional
properties. In order to exploit linearity ($\varepsilon ^{ \mu \nu \rho
(p+q)} \rightarrow \varepsilon ^{ \mu \nu \rho p} +
\varepsilon ^{ \mu \nu \rho q}$)  and the total antisymmetric property
 of the Levi-Civita tensor use \mb{EpsEvaluate}.

\seepage
}


\ffinish

\fname{EpsChisholm}
\fheading{\mb{EpsChisholm[{\sl expr}]}}
\fusage{\mb{EpsChisholm[{\sl expr}]} substitutes for a gamma matrix contracted 
with a Levi Civita tensor (\mb{Eps}) the Chisholm identity:
 $\gamma _{\mu} \varepsilon ^{ \mu \nu \rho \sigma } =
 + i \,( \gamma ^{ \nu} \gamma ^{ \rho } \gamma ^{ \sigma }
 - g^{ \nu \rho }  \gamma ^{ \sigma } -  g^{ \rho \sigma } \gamma ^{\nu}
 +  g^{ \nu  \sigma } \gamma ^{ \rho })  \gamma ^{5}$.
}
\fnotes{
{\tiny $\blacksquare$}\  With the option \mb{LeviCivitaSign} the sign of the right hand side of
the equation above can be altered. {\tiny $\blacksquare$}\  The following option can be given:
\ftabthree
{
\mb{LeviCivitaSign} & -1& sign convention 
}

\seepage
}

\ffinish

\fname{EpsContract}
\fheading{\mb{EpsContract}}
\fusage{\mb{EpsContract} is an option for \mb{Contract} specifying whether 
Levi-Civita tensors \mb{Eps} will be contracted, i.e., products 
of two  \mb{Eps} are replaced via the determinant formula.
}
\fnotes{
\seepage
}

\ffinish

\fname{EpsEvaluate}
\fheading{\mb{EpsEvaluate[{\sl expr}]}}
\fusage{\mb{EpsEvaluate[{\sl expr}]} applies total antisymmetry and
linearity (with respect to \mb{Momentum}) to all Levi-Civita tensors
(\mb{Eps}) in $expr$.
}
\fnotes{
\seepage
}
 
\ffinish

\fname{EvaluateDiracTrace}
\fheading{\mb{EvaluateDiracTrace[{\sl expr}]}}
\fusage{\mb{EvaluateDiracTrace[{\sl expr}]} evaluates \mb{DiracTrace} in 
$expr$.}
\fnotes{
\seepage
}

\ffinish

\fname{Expanding} 
\fheading{\mb{Expanding}}
\fusage{\mb{Expanding} 
is an option for \mb{Contract} specifying whether expansion
will be done in \mb{Contract}. If set to \mb{False}, not all 
Lorentz indices might get contracted.
}
\fnotes{
\seepage
}

\ffinish

\fname{ExpandScalarProduct}
\fheading{\mb{ExpandScalarProduct[{\sl expr}]}}
\fusage{\mb{ExpandScalarProduct[{\sl expr}]} 
expands scalar products in {\it expr}.
}
\fnotes{
{\tiny $\blacksquare$}\  Example: 

\mb{ExpandScalarProduct[\ ScalarProduct[\ p-q,r+2\ s\ ]\ ]} \ra
\mb{p.r\ +\ 2\ p.s\ -\ q.r\ -\ 2\ q.s}.

{\tiny $\blacksquare$}\ At the interal level \mb{ExpandScalarProduct}  expands actually 
everything with head \mb{Pair}. {\tiny $\blacksquare$}\  Since a four-vector has head \mb{Pair} internally also, 
\mb{ExpandScalarProduct[\ FourVector[p\ -\ 2\ q,\ mu]\ ]} \ra \mb{p[mu]\ -\ 2\ q[mu]}.

\seepage
}

\ffinish

\fname{Factor2}
\fheading{\mb{Factor2[{\sl expr}]}}
\fusage{\mb{Factor2[{\sl expr}]} factors a polynomial in a standard way.
\mb{Factor2} works better than \mb{Factor} on polynomials involving 
rationals with sums in the denominator.
}
\fnotes{
{\tiny $\blacksquare$}\  In general it is better to use \mb{Factor2} in \mma~ 2.0.

\seepage
}

\ffinish

\fname{Factoring}
\fheading{\mb{Factoring}}
\fusage{\mb{Factoring} is an option for 
\mb{Contract}, \mb{DiracTrace}, \mb{DiracSimplify} and \mb{OneLoop}.
If set to \mb{True} the result will be factored, using \mb{Factor2}.
}
\fnotes{
\seepage
}

\ffinish

\fname{FeynAmp}
\fheading{\mb{FeynAmp[{\sl name},\ {\sl amp},\ {\sl q}]}}
\fusage{\mb{FeynAmp[{\sl name},\ {\sl amp},\ {\sl q}]} is the head of the Feynman amplitude 
given by \fa. The first argument \mb{name} is for bookkeeping, \mb{amp} is 
the analytical expression for the amplitude, and \mb{q} is the integration 
variable. In order to calculate the amplitute replace \mb{FeynAmp} by  
\mb{OneLoop}. In the output of \fa\ $name$ has the head \mb{GraphName}.
}
\fnotes{
\seepage
{\tiny $\blacksquare$}\  See also: \mb{GraphName}, Guide to \fa.
}

\ffinish

\fname{FeynAmpDenominator}
\fheading{\mb{FeynAmpDenominator[PropagatorDenominator[...],\ PropagatorDenominator[...],\ ...]}}
\fusage{\mb{FeynAmpDenominator[\ PropagatorDenominator[...],\ PropagatorDenominator[...],\ ...]} is the head of the denominators of the propagators, 
i.e., \mb{FeynAmpDenominator[x]} is the representation of $1/x$.
}
\fnotes{
{\tiny $\blacksquare$}\  Example:

$1/([q^{2}-m_{1}^{2}] [(q+p_{1})^{2}-m_{2}^{2}])$
is represented as \mb{FeynAmpDenominator[\ PropagatorDenominator[{\sl q},{\sl m1}],\ \\
PropagatorDenominator[{\sl q}+p1,{\sl m2}]\ ]}. 

\seepage {\tiny $\blacksquare$}\  See also: \mb{PropagatorDenominator}.
}

\ffinish

\fname{FeynAmpList} 
\fheading{\mb{FeynAmpList[{\sl info}\ ...\ ][{\sl FeynAmp}[\ ...\ ],\ {\sl FeynAmp}[\ ...\ ],\ \ ...]}} 
\fusage{\mb{FeynAmpList[{\sl info}\ ...\ ]{\sl FeynAmp}[\ ...\ ],\ {\sl FeynAmp}[\ ...\ ],\ \ ...]} 
is the head of a list of \mb{FeynAmp} in the result of \fa. 
}
\fnotes{
\seepage
{\tiny $\blacksquare$}\  See also: Guide to \fa.
}
 
\ffinish

%\fname{FeynCalc}
%\fheading{\mb{FeynCalc[\\\ ]}}
%\fusage{\mb{?FeynCalc} gives on-line help.}
%
%\ffinish

\fname{FeynCalcForm}
\fheading{\mb{FeynCalcForm[{\sl expr}]}}
\fusage{\mb{FeynCalcForm[{\sl expr}]} changes the printed output 
of {\it expr} to an easy to read form.
The default setting of \mb{\$PreRead} is \mb{\$PreRead\ =\ FeynCalcForm}, which 
forces to display everything after applying \mb{FeynCalcForm}.
}
\fnotes{
{\tiny $\blacksquare$}\  \mb{Small}, \mb{Momentum} and \mb{LorentzIndex} are set to \mb{Identity}
by \mb{FeynCalcForm}. {\tiny $\blacksquare$}\  \mb{PaVe} are abbreviated. {\tiny $\blacksquare$}\  The action of \mb{FeynCalcForm} is:  
\mb{DiracMatrix[al]} \ra \mb{ga[al]}.

\mb{DiracSlash[p]} \ra \mb{gs[p]}.

\mb{FeynAmpDenominator[\ PropagatorDenominator[q,m1],\ PropagatorDenominator[q+p,m2]\ ]} \ra \mb{1/(q\phat 2\ \ -\ m1\phat 2)\ ((p\ +\ q)\phat 2\ -\ \ m2\phat 2)}.

\mb{FourVector[p,\ mu]} \ra \mb{p[mu]}.

\mb{FourVector[p-q,\ mu]} \ra \mb{(p\ -\ q)[mu]}.

\mb{GellMannMatrix[a]} \ra \mb{la[a]}.

\mb{GellMannTrace[x]} \ra \mb{tr[x]}.

\mb{DiracTrace[x]} \ra \mb{tr[x]}.

\mb{LeviCivita[a,\ b,\ c,\ d]} \ra \mb{eps[a,\ b,\ c,\ d]}.

\mb{MetricTensor[mu,\ nu]} \ra \mb{g[mu\ nu]}.

\mb{Momentum[\ Polarization[p]\ ]} \ra \mb{ep[p]}.

\mb{Conjugate[PolarizationVector[k,\ mu]]} \ra \mb{ep(*)[k,\ mu]}.

\mb{PolarizationVector[p,\ mu]} \ra \mb{ep[p][mu]}.

\mb{ScalarProduct[p,\ q]} \ra \mb{p.q}.
 
\mb{Spinor[p,\ m]} \ra \mb{u[p,\ m]}.

\mb{Spinor[-p,\ m]} \ra \mb{v[p,\ m]}.

\mb{QuarkSpinor[p,\ m]} \ra \mb{u[p,\ m]}.

\mb{QuarkSpinor[-p,\ m]} \ra \mb{v[p,\ m]}.

\mb{SU3F[i,\ j,\ k]} \ra \mb{f[i,\ j,\ k]}.

\seepage
}

\ffinish

\fname{FinalSubstitutions}
\fheading{\mb{FinalSubstitutions}}
\fusage{\mb{FinalSubstitutions} is an option for 
\mb{OneLoop} and \mb{OneLoopSum}. 
All substitutions given to this opton will be performed at 
the end of the calculation.
}
\fnotes{
{\tiny $\blacksquare$}\  Example: \mb{FinalSubstitutions} \ra \mb{\{mw\phat 2\ $\rightarrow$\ mw2,\ B0\ $\rightarrow$\ B0R\}}.

\seepage
}

\ffinish

\fname{FourVector}
\fheading{\mb{FourVector[{\sl p},\ {\sl mu}]}}
\fusage{\mb{FourVector[{\sl p},\ {\sl mu}]}  is the input for a four
vector $p_{\mu}$. 
}
\fnotes{
{\tiny $\blacksquare$}\  \mb{FourVector[{\sl p},\ {\sl mu}]} is directly transformed to the internal
representation: \mb{Pair[Momentum[{\sl p}],\ LorentzIndex[{\sl mu}]]}. {\tiny $\blacksquare$}\  The following option can be given:
\ftabthree
{
\mb{Dimension} & 4 & space-time dimension
}

\seepage {\tiny $\blacksquare$}\  See also: \mb{LorentzIndex}, \mb{Momentum}, \mb{Pair}.
}

\ffinish

\fname{FreeQ2}
\fheading{\mb{FreeQ2[{\sl expr},\ \{{\sl form1},\ {\sl form2},\ ...\}]}} 
\fusage{\mb{FreeQ2[{\sl expr},\ \{{\sl form1},\ {\sl form2},\ ...\}]}
yields \mb{True} if $expr$ does not contain any occurrence of 
$form1$, $form2$, \ldots . \mb{FreeQ2[{\sl expr},\ {\sl form}]} is the 
same as \mb{FreeQ[{\sl expr},\ \{{\sl form1},\ {\sl form2},\ ...\}]}.
}
\fnotes{
{\tiny $\blacksquare$}\  Stephen Wolfram pointed out that you can use alternatively 
\mb{FreeQ[{\sl expr},\ {\sl form1}\ ||\ \ {\sl form2}\ ||\ ...]}.

\seepage
}

\ffinish

\fname{GellMannMatrix}
\fheading{\mb{GellMannMatrix[{\sl a}]}}
\fusage{\mb{GellMannMatrix[{\sl a}]}  is the Gell-Mann matrix $\lambda_a$. 
A product of Gell-Mann matrices may be entered as
\mb{GellMannMatrix[a,\ b,\ ...]} or 
as \mb{GellmannMatrix[a]\ .\ GellMannMatrix[b]\ .\ ...}.
\mb{GellMannMatrix[1]} denotes the unit-matrix in color space.
}
\fnotes{
\seepage
{\tiny $\blacksquare$}\  See also: \mb{SU3Delta}\hspace{0.2mm}, \mb{SU3F}, \mb{GellMannTrace}.
}

\ffinish

\fname{GellMannTrace}
\fheading{\mb{GellMannTrace[{\sl expr}]}}
\fusage{\mb{GellMannTrace[{\sl expr}]}  calculates the trace of $expr$.
All color indices should occur twice and $expr$ must be a product 
of \mb{SU3F}, \mb{SU3Delta} and \mb{GellMannMatrix}.
}
\fnotes{
{\tiny $\blacksquare$}\  The Cvitanovic algorithm is used. {\tiny $\blacksquare$}\  Examples: \mb{[GellMannTrace[GellMannMatrix[i.i]]\ } \ra 16.
\mb{GellMannTrace[GellMannMatrix[a\ .\ b\ .\ c]\ SU3F[a,\ b,\ c]]} \ra 48 I.
\mb{GellMannTrace[GellMannMatrix[a\ .\ c\ .\ e\ .\ d]\ SU3F[a,\ b,\ e]\ SU3F[b,\ c,\ d]]} \ra 0.
\mb{GellMannTrace[GellMannMatrix[1]]} \ra 3.

\seepage
}

\ffinish


\fname{GraphName}
\fheading{\mb{GraphName[{\sl a},\ {\sl b},\ {\sl c},\ {\sl d}]}} 
\fusage{\mb{GraphName[{\sl a},\ {\sl b},\ {\sl c},\ {\sl d}]} is the first argument of 
\mb{FeynAmp} given by \fa. It may be used also as first argument of 
\mb{OneLoop}. The arguments \mb{{\sl a},\ {\sl b},\ {\sl c},\ {\sl d}} indicate information 
of the graph under consideration.
}
\fnotes{
\seepage}

\ffinish

\fname{InitialSubstitutions}
\fheading{\mb{InitialSubstitutions}}
\fusage{\mb{InitialSubstitutions}  is an option for \mb{PaVeReduce} 
and \mb{OneLoop}. All substitutions hereby indicated will be performed at the 
beginning of the calculation. Energy momentum conservation 
may be especially indicated in the setting.
}
\fnotes{
{\tiny $\blacksquare$}\  Example: \mb{InitialSubstitutions} \ra 
\mb{\{CW\phat 2\ $\rightarrow$\ 1\ -\ SW\phat 2,\ k2\ $\rightarrow$\ -\ k1\ +\ p1\ +\ p2\ \ \}}.

\seepage
}

\ffinish
\fname{Isolate}
\fheading{\mb{Isolate[{\sl expr},\ \{$\hbox{\sl x}_{1}$,\ $\hbox{\sl x}_{2}$,\ ...\ \}}}
\fusage{\mb{Isolate[{\sl expr},\ \{$\hbox{\sl x}_{1}$,\ $\hbox{\sl x}_{2}$,\ ...\ \}]} 
 substitutes \mb{K[{\sl i}]} for all subsums in $expr$
which are free of any occurrence of \mb{$\hbox{\sl x}_{1}$,\ $\hbox{\sl x}_{2}$,\ ...}, 
 if \mb{Length[{\sl expr}]>0}.  

\mb{Isolate[{\sl expr}]} substitutes  an  
abbreviation \mb{K[{\sl i}]} in \mb{HoldForm} for
$expr$, if \mb{Length[{\sl expr}]>0}.
}
\fnotes{
{\tiny $\blacksquare$}\  The following options can be given:
\ftabthree
{
\mb{IsolateHead} & \mb{K} & the head of the abbreviations \cr
\mb{IsolateSplit} & \mb{442} & a limit beyound which \mb{Isolate} splits 
the expression in two sums 
} {\tiny $\blacksquare$}\  \mb{IsolateSplit} is the maximum of the characters of the \mb{FortranForm}
of the expression being isolated by \mb{Isolate}. 
The default setting inhibits \mb{Write2} from producing too many 
continuation lines when writing out in \mb{FortranForm}. {\tiny $\blacksquare$}\  The result of \mb{Isolate} can always be recovered by 
\mb{MapAll[\ ReleaseHold,\ result\ ]}. {\tiny $\blacksquare$}\  Example:
\mb{Isolate[a[z]\ (b+c)\ +\ d[f]\ (x+y),\ \{a,d\}\ ]} \ra \mb{a[z]\ K[1]\ +\ d[f]\ K[2]}.

\seepage {\tiny $\blacksquare$}\   See also: \mb{Write2}.
}

\ffinish

\fname{IsolateHead}
\fheading{\mb{IsolateHead}}
\fusage{\mb{IsolateHead} is an option for \mb{Isolate}.}
\fnotes{
\seepage 
}

\ffinish

\fname{IsolateSplit}
\fheading{\mb{IsolateSplit}}
\fusage{\mb{IsolateSplit} is an option for \mb{Isolate}.}
\fnotes{ 

\seepage 
}

\ffinish

\fname{K}
\fheading{\mb{K[{\sl i}]}}
\fusage{\mb{K[{\sl i}]} are abbreviations which may result from 
\mb{PaVeReduce}, depending on the option \mb{IsolateHead}. 
The \mb{K[{\sl i}]} are returned in \mb{HoldForm} and may be recovered 
by \mb{ReleaseHold}.
}
\fnotes{ 
\seepage 
}
\ffinish

\fname{LeptonSpinor}
\fheading{\mb{LeptonSpinor[{\sl p},\ {\sl mass}]}}
\fusage{\mb{LeptonSpinor[{\sl p},\ {\sl mass}]} specifies a Dirac spinor.
Which of the spinors $u, v, \overline{u}$ or $\overline{v}$
is understood, depends on the sign of the mass argument and
the relative position of \mb{DiracSlash[p]}:
\mb{LeptonSpinor[{\sl p},\ {\sl mass}]}  is that spinor which yields
\mb{mass*LeptonSpinor[{\sl p},\ {\sl mass}]} if the Dirac equation is applied to 
\mb{DiracSlash[{\sl p}]\ .\ LeptonSpinor[{\sl p},\ {\sl mass}]} or 
\mb{LeptonSpinor[{\sl p},\ {\sl mass}]\ .\ DiracSlash[{\sl p}]}.

If a spinor is multiplied by a Dirac matrix or another spinor, the multiplication
operator "." must be used. 
}
\fnotes{
\seepage
{\tiny $\blacksquare$}\  
See also: \mb{DiracSimplify}.
}

\ffinish

\fname{LeviCivita}
\fheading{\mb{LeviCivita[{\sl mu},\ {\sl nu},\ {\sl ro},\ {\sl si}]}}
\fusage{\mb{LeviCivita[{\sl mu},\ {\sl nu},\ {\sl ro},\ {\sl si}]}   is an input function
for the totally antisymmetric Levi-Civita tensor.
}
\fnotes{
{\tiny $\blacksquare$}\  \mb{LeviCivita[mu,\ nu,\ ro,\ si]} transforms to the internal 
representation \mb{Eps[\ LorentzIndex[mu],\ LorentzIndex[nu],\ LorentzIndex[ro],\\
LorentzIndex[si]\ ]}. {\tiny $\blacksquare$}\  For simplification of Levi-Civita tensors use \mb{EpsEvaluate}.

\seepage {\tiny $\blacksquare$}\  See also: \mb{Eps}, \mb{EpsEvaluate}.
}

\ffinish


\fname{LeviCivitaSign}
\fheading{\mb{LeviCivitaSign}}
\fusage{\mb{LeviCivitaSign} is an option for \mb{DiracTrace}.
The possible settings are (+1) or (-1). This option determines the 
sign convention of the result of 
$tr(\,\gamma^{a} \, \gamma^{b} \,\gamma^{c} \,\gamma^{d} \,\gamma^{5} \,)$.}
\fnotes{
\seepage
}

\ffinish
\fname{LorentzIndex}
\fheading{\mb{LorentzIndex[{\sl mu}]}}
\fusage{\mb{LorentzIndex[{\sl mu},\ {\sl optdim}]} is the head of Lorentz indices.
The internal representation of a four-dimensional $\mu $ is
\mb{LorentzIndex[mu]}. For other than four dimensions enter
\mb{LorentzIndex[{\sl mu},\ {\sl dim}]}.
}
\fnotes{
{\tiny $\blacksquare$}\  \mb{LorentzIndex[{\sl mu},\ 4]} simplifies to \mb{LorentzIndex[mu]}.

\seepage
}

\ffinish

\fname{MacsymaForm}
\fheading{\mb{MacsymaForm}}
\fusage{\mb{MacsymaForm} is an option for \mb{FormatType} in \mb{Write2}.
}
\fnotes{
\seepage
{\tiny $\blacksquare$}\  See also: \mb{Write2}, \mb{MapleForm}.
}

\ffinish

\fname{Mandelstam}
\fheading{\mb{Mandelstam}}
\fusage{\mb{Mandelstam} is an option for
\mb{DiracTrace}, \mb{OneLoop} and \mb{TrickMandelstam}.
A typical setting is 
\mb{Mandelstam\ $\rightarrow$\ \{s,\ t,\ u,\ m1\phat 2\ +\ m2\phat 2\ +\ m3\phat 2\ +\ m4\phat 2\}},
which stands for  $s+t+u=m_{1}^{2} + m_{2}^{2} + m_{3}^{2} +  m_{4}^{2}$.
}
\fnotes{
\seepage
}

\ffinish

\fname{MapleForm}
\fheading{\mb{MapleForm}}
\fusage{\mb{MapleForm} is an option for \mb{Write2}.
}
\fnotes{
\seepage {\tiny $\blacksquare$}\  See also: \mb{Write2}, \mb{MacsymaForm}.
}

\ffinish

%\fname{MassiveBoson}
%\fheading{\mb{MassiveBoson}}
%\fusage{\mb{MassiveBoson} is an optional argument setting for
%\mb{PolarizationVector} and \mb{Polarization}.
%If given as argument it indicates that the polarization vector corresponds
%to a Boson with non-zero mass.
%}
%\fnotes{
%{\tiny $\blacksquare$}\  %
%%\seepage
%}
%\ffinish

\fname{MetricTensor}
\fheading{\mb{MetricTensor[{\sl mu},\ {\sl nu}]}}
\fusage{\mb{MetricTensor[{\sl mu},\ {\sl nu}]} is the input for a metric tensor 
\mb{g}$^{\mu \nu}$. 
}
\fnotes{
{\tiny $\blacksquare$}\  The following option can be given:
\ftabthree
{
\mb{Dimension} & 4 & space-time dimension
}

%The internal representation of
%\mb{MetricTensor[{\sl mu},\ {\sl nu},\ {\sl Dimension}\ $\rightarrow$\ dim]} is
%\mb{Pair[\ LorentzIndex[{\sl mu},\ {\sl dim}],\ LorentzIndex[{\sl nu},\ {\sl dim}]\ ]}.
%
\seepage {\tiny $\blacksquare$}\  See also: \mb{Contract}, \mb{LorentzIndex}, \mb{Pair}.
}

\ffinish

\fname{Momentum}
\fheading{\mb{Momentum[{\sl p},\ {\sl optdim}]}}
\fusage{\mb{Momentum[{\sl p},\ {\sl optdim}]} is the head of a momentum in the 
internal representation. 
A four-dimensional momentum \mb{p} is represented by \mb{Momentum[p]}.
For other than four dimensions an extra argument must be given:
\mb{Momentum[{\sl q},\ {\sl dim}]}.
}
\fnotes{
{\tiny $\blacksquare$}\  
\mb{Momentum[{\sl p},\ {\sl 4}]} is automatically transformed to \mb{Momentum[{\sl p}]}.

\seepage
}

\ffinish

\fname{NumericalFactor}
\fheading{\mb{NumericalFactor[{\sl expr}]}}
\fusage{\mb{NumericalFactor[{\sl expr}]} gives the numerical factor of $expr$.
}
\fnotes{
\seepage
}

\ffinish

\fname{OneLoop}
\fheading{\mb{OneLoop[{\sl name},\ {\sl q},\ {\sl amplitude}]}}
\fusage{\mb{OneLoop[{\sl name},\ {\sl q},\ {\sl amplitude}]}  calculates the 
one-loop Feynman diagram \mb{amplitude}. The argument $q$ denotes the
integration variable, i.e., the loop momentum. 
%
%For applying the method of dimensional regularization the default option 
%of \mb{Dimension}, \mb{D}, may be used. You may however enter four-dimensional
%\mb{MetricTensor}, \mb{FourVector} and \mb{PropagatorDenominator}. 
%\mb{OneLoop} performs the necessary extensions to \mb{D} dimensions automatically.
}
\fnotes{
{\tiny $\blacksquare$}\  The following options can be given:
\ftabthree
{
\mb{ReduceToScalars}   &  \mb{False}  & reduce to $B_{0}, C_{0},D_{0}$ \cr
\mb{DenominatorOrder} & \mb{False} & order the entries of \mb{FeynAmpDenominator} \cr 
\mb{Dimension} & \mb{True} & dimension of integration \cr
\mb{FinalSubstitutions} & \mb{\{\}} & substitutions done at the end of the 
calculation \cr
\mb{Factoring} & \mb{False}  &  factor the result \cr
\mb{InitialSubstitutions} & \mb{\{\}} & substitutions done at the beginning 
of the calculation \cr
\mb{Mandelstam} & \mb{\{\}} &  indicate the Mandelstam relation \cr 
\mb{Prefactor} &  \mb{1} & additional prefactor of the amplitude \cr 
\mb{CancelQ2} & \mb{True} & cancel $q^{2}\,$ \cr
\mb{CancelQP} & \mb{False} & cancel $q\cdot p\,$ \cr
\mb{ReduceGamma}  & \mb{False} & eliminate $\gamma_6$ and
 $\gamma_7$ \cr
\mb{SmallVariables} & \mb{\{\}} &  a list of masses, which will get wrapped 
around the head \mb{Small} \cr
%VectorBosonPolarization & {} & choose a specific polarization \cr
\mb{WriteOut} &  \mb{True} &  write out a result file \mb{name.m} 
} {\tiny $\blacksquare$}\  Energy momentum conservation may be given as rule of \mb{InitialSubstitutions}.

\seepage
}

\ffinish


\fname{OneLoopResult}
\fheading{\mb{OneLoopResult[{\sl name}]}}
\fusage{\mb{OneLoopResult[{\sl name}]} is the variable in the result file written out by
\mb{OneLoop} to which the corresponding result is assigned. $name$ is 
constructed from the first argument of \mb{OneLoop}.
}
\fnotes{
\seepage
}

\ffinish

\fname{OneLoopSum}
\fheading{\mb{OneLoopSum[\ FeynAmp[...],\ FeynAmp[...],\ ...\ ]}}
\fusage{\mb{OneLoopSum[\ FeynAmp[...],\ FeynAmp[...],\ ...\ ]} calculates 
a list of Feynman amplitudes by  replacing \mb{FeynAmp} step by step by 
\mb{OneLoop} and sums the result.
}
\fnotes{
{\tiny $\blacksquare$}\  The  following options can be given:
\ftabthree
{
\mb{CombineGraphs} & \mb{\{\}} & which amplitudes to sum before invoking \mb{OneLoop} 
\cr
\mb{FinalSubstitutions} & \mb{\{\}} & substitutions done at the end of 
the calculation \cr
\mb{IsolateHead} & \mb{K} & \mb{Isolate} the result \cr
\mb{Mandelstam} & \mb{\{\}} & use the Mandelstam relation \cr
\mb{Prefactor} & \mb{1}  &  multiply the result by a pre-factor \cr
\mb{ReduceToScalars} & \mb{True} & reduce the summed result to scalar integrals \cr
\mb{SelectGraphs}& \mb{All} & which amplitudes to select \cr
\mb{WriteOutPaVe} & \mb{False} & write out the reduced \mb{PaVe}
}
Possible settings for \mb{CombineGraphs} and \mb{SelectGraphs} are lists of integers.
For indicating a range of graphs also a list \mb{\{i,\ j\}} instead of a single integer 
may be provided.

\seepage
}

\ffinish

\fname{Pair}
\fheading{\mb{Pair[{\sl a},\ {\sl b}]}}
\fusage{\mb{Pair[{\sl a},\ {\sl b}]} is the head of a special pairing
used in the internal representation.
The arguments {\it a} and {\it b} may have 
heads \mb{LorentzIndex} or \mb{Momentum}. 
If both {\it a} and {\it b} have head \mb{LorentzIndex}, 
the metric tensor is understood. 
If {\it a} and {\it b} have head \mb{Momentum}, a scalar product is meant.
If one of  {\it a} and {\it b} has head
\mb{LorentzIndex} and the other head \mb{Momentum}, 
a Lorentz vector $p^{\mu}$ is understood.
}
\fnotes{
{\tiny $\blacksquare$}\  \mb{Pair} has only one functional definition: any integers multiplied with
{\it a} or {\it b} will be pulled out.

\seepage  {\tiny $\blacksquare$}\  See also: \mb{FourVector}, \mb{LorentzIndex}, \mb{MetricTensor},
\mb{ExpandScalarProduct}, \mb{ScalarProduct}.
}

\ffinish

\fname{PairCollect}
\fheading{\mb{PairCollect}}
\fusage{\mb{PairCollect} is an option for \mb{Contract}. }
\fnotes{
\seepage {\tiny $\blacksquare$}\  See also: \mb{Pair}, \mb{ScalarProduct}, \mb{Momentum}.
}

\ffinish

\fname{PaVe}
\fheading{\mb{PaVe[{\sl i},\ {\sl j},\ ...,\ {\sl plist},\ {\sl mlist}]}} 
\fusage{\mb{PaVe[{\sl i},\ {\sl j},\ ...,\ {\sl plist},\ {\sl mlist}]} denotes the 
Passarino-Veltman integrals. The length of the mass list $mlist$ 
indicates if a one-, two-, three- or four-point integral is understood.
The first set of arguments $i,j, \ldots$
signifies that the coefficient of $p_{i}^{\mu} \, p_{j}^{\nu}, \ldots$
of the tensor integral decomposition is meant,
where $p_{0}^{\mu} \, p_{0}^{\nu} = g^{\mu \nu}$.
Joining {\it plist} and {\it mlist} gives
the same conventions as for \mb{A0}, \mb{B0}, \mb{C0} and \mb{D0}.
}
\fnotes{
{\tiny $\blacksquare$}\  For the corresponding arguments of \mb{PaVe} 
the special cases \mb{A0}, \mb{B0}, \mb{C0}, \mb{D0}, \mb{B1},  
\mb{B00}, \mb{B11} are returned.

\seepage
}

\ffinish

%\fname{PaVeBr}
%\fheading{\mb{PaVeBr[{\sl i},\ {\sl j},\ ...,\ {\sl plist},\ {\sl mlist}]}}
%\fusage{\mb{PaVeBr[{\sl i},\ {\sl j},\ ...,\ {\sl plist},\ {\sl mlist}]} is like \mb{PaVe},
%but breaks one level down, resulting in \mb{PaVe} with lower-order 
%indices.
%}
%
%\ffinish

\fname{PaVeOrder}
\fheading{\mb{PaVeOrder[{\sl expr}]}}
\fusage{\mb{PaVeOrder[{\sl expr}]} 
brings all arguments of \mb{C0} and \mb{D0} into a canical order.
}
\fnotes{
{\tiny $\blacksquare$}\  The following option can be given:
 \ftabthree
    {
PaVeOrderList & \mb{\{\}} & order according to a list of arguments
}

\seepage
}

\ffinish

\fname{PaVeOrderList}
\fheading{\mb{PaVeOrderList}}
\fusage{\mb{PaVeOrderList} is an option for \mb{PaVeOrder} allowing to 
specify a specific order of the \mb{D0} functions.
}
\fnotes{
{\tiny $\blacksquare$}\  Possible settings are a sublist of the arguments of a \mb{D0}, or 
a list of such lists.

\seepage
}

\ffinish


\fname{PaVeReduce}
\fheading{\mb{PaVeReduce[{\sl expr}]}} 
\fusage{\mb{PaVeReduce[{\sl expr}]} reduces Passarino-Veltman integrals \mb{PaVe} to 
scalar integrals \mb{B0}, \mb{C0} and \mb{D0}, depending on the option \mb{BReduce} 
  eventually also \mb{A0}, \mb{B1}, \mb{B00} and \mb{B11}.
   }
    \fnotes{
{\tiny $\blacksquare$}\      The class of invariant  Passarino-Veltman integrals which
    can be currently reduced consists of
    all coefficients of the Lorentz invariant decomposition of
    $B^{\mu},  B^{\mu \nu}, C^{\mu}, C^{\mu \nu},  C^{\mu \nu \rho},
    D^{\mu}, D^{\mu \nu}, D^{\mu \nu \rho}, D^{\mu \nu \rho \sigma}$. {\tiny $\blacksquare$}\      The following options can be given:
    \ftabthree
    {
    IsolateHead& \mb{False}  & use \mb{Isolate}\cr
    Mandelstam    & \mb{\{\}} &  Mandelstam relation, e.g., \mb{\{s,\ t,\ u,\ 2\ mw\phat 2\}}
    }

\seepage {\tiny $\blacksquare$}\  See also: \mb{BReduce}, \mb{K}, \mb{PaVe}.
}

\ffinish


\fname{Polarization}
\fheading{\mb{Polarization[{\sl p},\ {\sl optargs}]}}
\fusage{\mb{Polarization[{\sl p},\ {\sl optarg}]} is the head of a 
polarization momentum$\varepsilon(p)$. \mb{Polarization} must 
always occur inside \mb{Momentum}.  The full internal representation of
$\varepsilon(p)$ is \mb{Momentum[\ Polarization[p]\ ]}. 
With this notation transversality of polarization vectors is provided.
}
\fnotes{
{\tiny $\blacksquare$}\  %The optional arguments are the same as in \mb{PolarizationVector}.
\mb{Polarization[{\sl p},\ -1]} stands for the complex conjugate 
$\varepsilon(p)^{*}$.

\seepage {\tiny $\blacksquare$}\  See also: \mb{PolarizationVector}.
}

\ffinish

\fname{PolarizationSum}
\fheading{\mb{PolarizationSum[{\sl mu},\ {\sl nu},\ ...]}}
\fusage{\mb{PolarizationSum[{\sl mu},\ {\sl nu},\ ...]}  defines different  
polarization sums. 
}
\fnotes{
{\tiny $\blacksquare$}\  \mb{PolarizationSum} does not calculate any polarization sum,
it is just an abbreviation function. {\tiny $\blacksquare$}\  \mb{PolarizationSum[{\sl mu},\ {\sl nu}]}$ = -g_{\mu \nu}$. {\tiny $\blacksquare$}\  \mb{PolarizationSum[{\sl mu},\ {\sl nu},\ {\sl k}]}$ = -g_{\mu \nu} + k_{\mu} k_{\nu}/k^{2}$. {\tiny $\blacksquare$}\  \mb{PolarizationSum[{\sl mu},\ {\sl nu},\ {\sl k},\ {\sl n}]}$ = -g_{\mu \nu}
- k_{\mu} k_{\nu} n^{2}/(k\cdot n)^{2}
+ (n_{\mu}  k_{\nu} +  n_{\nu}  k_{\mu})/(k \cdot n)$,
where $n_{\mu}$ denotes an external four vector.

\seepage
}

\ffinish

\fname{PolarizationVector}
\fheading{\mb{PolarizationVector[{\sl p},\ {\sl mu}]}}
\fusage{\mb{PolarizationVector[{\sl p},\ {\sl mu}]} is an input function for a
polarization vector $\varepsilon(k)^{\mu}$.
}
\fnotes{
{\tiny $\blacksquare$}\  \mb{Conjugate\{PolarizationVector[{\sl p},\ {\sl mu}]\}} is the input for 
$\varepsilon_{\mu}^{*}(k)$. {\tiny $\blacksquare$}\  The internal representation of \mb{PolarizationVector[{\sl p},\ {\sl mu}]} 
is \mb{Pair[\ Momentum[Polarization[{\sl p}]],\ LorentzIndex[{\sl mu}]\ ]}. {\tiny $\blacksquare$}\  The internal representation of
\mb{Conjugate\{PolarizationVector[{\sl p},\ {\sl mu}]\}} is
\mb{Pair[\ Momentum[Polarization[{\sl p},-1]],\ LorentzIndex[{\sl mu}]\ ]}.

\seepage {\tiny $\blacksquare$}\  See also: \mb{Polarization}.
}

\ffinish

\fname{Prefactor}
\fheading{\mb{Prefactor}}
\fusage{\mb{Prefactor} is an option for \mb{OneLoop} and \mb{OneLoopSum}.
If set as option of
\mb{OneLoop}, the amplitude is multiplied by  
\mb{Prefactor} before calculation; if
   specified as option of \mb{OneLoopSum}, 
it appears in the final result as
   a global factor.
}
\fnotes{
{\tiny $\blacksquare$}\  A possible setting is $1/(1-D)$ for calculating the transvere part
of self energies. The option \mb{Dimension} of \mb{OneLoop} must then
be set to $D$.

\seepage
} 

\ffinish


\fname{ProductExpand}
\fheading{\mb{ProductExpand}}
\fusage{\mb{ProductExpand} is an option for \mb{Collect2} and 
\mb{Combine}.
}
\fnotes{
\seepage
}

\ffinish

\fname{PropagatorDenominator} 
\fheading{\mb{PropagatorDenominator[{\sl q},\ {\sl m}]}}
\fusage{\mb{PropagatorDenominator[{\sl q},\ {\sl m}]}  is the denominator of a 
propagator, i.e.,  ($q^{2} - m^{2}$). 

\mb{PropagatorDenominator[{\sl q}]} evaluates to 
\mb{PropagatorDenominator[{\sl q},\ 0]}.
}
\fnotes{
{\tiny $\blacksquare$}\  If {\it q} is supposed to be D-dimensional enter:
\mb{PropagatorDenominator[\ Momentum[q,D],\ m]}. {\tiny $\blacksquare$}\  \mb{PropagatorDenominator} must always
occur only as an argument of \mb{FeynAmpDenominator}.

\seepage
}

\ffinish


\fname{QuarkSpinor}
\fheading{\mb{QuarkSpinor[{\sl p},\ {\sl mass}]}}
\fusage{\mb{QuarkSpinor[{\sl p},\ {\sl mass}]} specifies a Dirac spinor.
Which of the spinors $u, v, \overline{u}$ or $\overline{v}$
is understood, depends on the sign of the mass argument and
the relative position of \mb{DiracSlash[p]}:
\mb{QuarkSpinor[{\sl p},\ {\sl mass}]}  is that spinor which yields
\mb{mass*QuarkSpinor[{\sl p},\ {\sl mass}]} if the Dirac equation is applied to 
\mb{DiracSlash[{\sl p}]\ .\ QuarkSpinor[{\sl p},\ {\sl mass}]} or 
\mb{QuarkSpinor[{\sl p},\ {\sl mass}]\ .\ DiracSlash[{\sl p}]}.

If a spinor is multiplied by a Dirac matrix or another spinor, the multiplication
operator "." must be used. 
}
\fnotes{
\seepage {\tiny $\blacksquare$}\  See also: \mb{DiracSimplify}.
}

\ffinish

\fname{ReduceGamma}
\fheading{\mb{ReduceGamma}}
\fusage{\mb{ReduceGamma} is an option 
for \mb{OneLoop} determining
whether $\gamma_6$ and $\gamma_7$ are removed by their 
definitions $1/2 (1 + \gamma_5)$ and $1/2 (1 - \gamma_5)$.
}
\fnotes{
{\tiny $\blacksquare$}\  This option may be needed for certain standard matrixelements.
 
\seepage
}

\ffinish

\fname{ReduceToScalars}
\fheading{\mb{ReduceToScalars}}
\fusage{\mb{ReduceToScalars}  is an option for \mb{OneLoop} and
\mb{OneLoopSum} that specifies whether the result is reduced to
scalar integrals.
}
\fnotes{
{\tiny $\blacksquare$}\  Depending on the option \mb{BReduce} the Passarino-Veltman functions \mb{B1}, \mb{B11}, 
\mb{B00} and \mb{B11} may also remain.

\seepage {\tiny $\blacksquare$}\  See also: \mb{PaVeReduce}.
}

\ffinish

\fname{ScalarProduct}
\fheading{\mb{ScalarProduct[{\sl p},\ {\sl q}]}}
\fusage{\mb{ScalarProduct[{\sl p},\ {\sl q}]} is the input for a scalar product.
}
\fnotes{
{\tiny $\blacksquare$}\  The following option can be given:
\ftabthree
{
\mb{Dimension} & 4 & space-time dimension
} {\tiny $\blacksquare$}\  The internal representation is: \mb{Pair[\ Momentum[p],\ Momentum[q]\ ]}. {\tiny $\blacksquare$}\  Scalar products may be set, e.g., \mb{ScalarProduct[a,\ b]\ =\ c};
but \mb{a} and \mb{b} must not contain sums.

\seepage {\tiny $\blacksquare$}\  See also \mb{ExpandScalarProduct}.
}

\ffinish

\fname{SelectGraphs}
\fheading{ \mb{SelectGraphs}}
\fusage{\mb{SelectGraphs} is an option for \mb{OneLoopSum}.
The default setting is \mb{All}. 
It may be set to a list indicating that only a subclass of 
all graphs supplied to \mb{OneLoopSum} should be calculated.
}
\fnotes{
\seepage
}
\ffinish

\fname{SetMandelstam}
\fheading{
\mb{SetMandelstam[{\sl s},\ {\sl t},\ {\sl u},\ $\hbox{\sl p}_{1}$,\ $\hbox{\sl p}_{2}$,\ $\hbox{\sl p}_{3}$,\ $\hbox{\sl p}_{4}$,\ $\hbox{\sl m}_{1}$,\ $\hbox{\sl m}_{2}$,\ $\hbox{\sl m}_{3}$,\ $\hbox{\sl m}_{4}$]}} 
\fusage{
\mb{SetMandelstam[{\sl s},\ {\sl t},\ {\sl u},\ $\hbox{\sl p}_{1}$,\ $\hbox{\sl p}_{2}$,\ $\hbox{\sl p}_{3}$,\ $\hbox{\sl p}_{4}$,\ $\hbox{\sl m}_{1}$,\ $\hbox{\sl m}_{2}$,\ $\hbox{\sl m}_{3}$,\ $\hbox{\sl m}_{4}$]} 
defines the Mandelstam variables $s=(p_1 + p_2)^2$, $t=(p_1 + p_3)^2$,
$u=(p_1 + p_4)^2$ and sets the $p_i$ on-shell: $p_i^{2}=m_i^{2}$, where 
the $p_i$ satisfy $p_1 + p_2 + p_3 + p_4 = 0$.
}
\fnotes{
{\tiny $\blacksquare$}\  If $p_3 = - k_1$ and $p_4 = -k_2$, i.e., $p_1 + p_2  = k_1 + k_2$,
the input is: 
\mb{SetMandelstam[s,\ t,\ u,\ $\hbox{\sl p}_{1}$,\ $\hbox{\sl p}_{2}$,\ -\ $\hbox{\sl k}_{1}$,\ -\ $\hbox{\sl k}_{2}$,\ $\hbox{\sl m}_{1}$,\ $\hbox{\sl m}_{2}$,\ $\hbox{\sl m}_{3}$,\ $\hbox{\sl m}_{4}$]}

\seepage
}

\ffinish

\fname{SetStandardMatrixElements}
\fheading{
\mb{SetStandardMatrixElements[\{{\sl sma1}\ $\rightarrow$\ {\sl abb1}\},\ \{{\sl sma2}\ $\rightarrow$\ {\sl abb2}\},\ ...\ ,{\sl enmomcon}]}} 
\fusage{\mb{SetStandardMatrixElements[\{{\sl sma1}\ $\rightarrow$\ {\sl abb1}\},\ \{{\sl sma2}\ $\rightarrow$\ {\sl abb2}\},\ ...\ ,{\sl enmoconrule}\ ]}
defines the standard matrix elements $sma1, sma2, ..$ (e.g., 
\mb{Spinor[p1]\ .\ DiracSlash[k]\ .\ Spinor[p2]}), as 
\mb{StandardMatrixelement[{\sl abb1}]}, \mb{StandardMatrixelement[{\sl abb2}]}, ... . 
The last argument {\it enmomcon}
defines energy momentum conservation; 
e.g., $enmomcon=$ \mb{\{k2\ $\rightarrow$\ p1\ +\ p2\ -\ k1\}}.
}
\fnotes{
{\tiny $\blacksquare$}\  \mb{SetStandardMatrixElements}  should be invoked only once for a whole
process. It is most conveniently used in a separate specification
batch file. In this file also the settings 
of the scalar products should be done either directly and/or  with 
\mb{SetMandelstam} before applying \mb{SetStandardMatrixElements}. {\tiny $\blacksquare$}\  It is not necessary to predefine standard matrix elements.

\seepage {\tiny $\blacksquare$}\  See also: \mb{SetMandelstam}, \mb{StandardMatrixelement}. 
}

\ffinish

\fname{Small}
\fheading{\mb{Small[{\sl m}]}} 
\fusage{\mb{Small[{\sl me}]} is the head of a small mass $me$. 
The effect is that masses with this head are set to zero, if they occur outside a 
Passarino-Veltman function.
}
\fnotes{
\seepage
}

\ffinish

\fname{SmallVariables}
\fheading{\mb{SmallVariables}} 
\fusage{\mb{SmallVariables} is an option for \mb{OneLoop}.
}
\fnotes{
{\tiny $\blacksquare$}\  The setting of \mb{SmallVariables} is a list containing masses 
which are small compared to others.
If present the photon mass should always be listed.

\seepage
}

\ffinish

\fname{Spinor}
\fheading{\mb{Spinor[{\sl p},\ {\sl mass}]}}
\fusage{\mb{Spinor[{\sl p},\ {\sl mass}]} specifies a Dirac spinor.
Which of the spinors $u, v, \overline{u}$ or $\overline{v}$
is understood, depends on the sign of the mass argument and
the relative position of \mb{DiracSlash[p]}:
\mb{Spinor[{\sl p},\ {\sl mass}]}  is that spinor which yields
\mb{mass*Spinor[{\sl p},\ {\sl mass}]} if the Dirac equation is applied to 
\mb{DiracSlash[{\sl p}]\ .\ Spinor[{\sl p},\ {\sl mass}]} or 
\mb{Spinor[{\sl p},\ {\sl mass}]\ .\ DiracSlash[{\sl p}]}.

If a spinor is multiplied by a Dirac matrix or another spinor, the multiplication
operator "." must be used. 
}
\fnotes{
\seepage
{\tiny $\blacksquare$}\  See also: \mb{DiracSimplify}.
}

\ffinish

%\fname{SquareAmplitude}
%\fheading{\mb{SquareAmplitude[{\sl amps}]}}
%\fusage{\mb{SquareAmplitude[{\sl amp}]} squares the 
%tree level amplitudes \mb{amps}. If polarization vectors with 
%corresponding specifications --- \mb{LightBoson}/\mb{MassiveBoson} ---
%are present, the polarization sum is carried out. 
%}
%\fnotes{
%{\tiny $\blacksquare$}\  The following option can be given:
%\ftabthree{
%\mb{VectorBosonPolarization} & \mb{\{\}} & choose a specific vector boson polarization 
%}
%
%If the spinors have a third argument the following rules are applied:
%\mb{Spinor[{\sl p},\ {\sl m},\ LeftHanded]\ Spinor[{\sl p},\ {\sl m},\ LeftHanded]} \ra 
%$(\bps+m) (1-\gamma_5)/2$.
%
%\mb{Spinor[{\sl p},\ {\sl m},\ RightHanded]\ Spinor[{\sl p},\ {\sl m},\ RightHanded]} \ra 
%$(\bps+m) (1+\gamma_5)/2$.
%
%\mb{Spinor[{\sl p},\ {\sl m},\ {\sl n}]\ Spinor[{\sl p},\ {\sl m},\ {\sl n}]} \ra 
%$(\bps+m) ( (1+\gamma_5)\,n)/2$, where $n$ denotes an external 
%four-momentum.
%
%In some cases it is better to first apply \mb{DiracSimplify} to 
%$amps$. 
%}
%
%\ffinish

\fname{StandardMatrixElement}
\fheading{\mb{StandardMatrixElement[...]}}
\fusage{\mb{StandardMatrixElement[...]}  is the head of the standard
matrix elements. 
The standard matrix elements are a basis for the Feynman amplitude
which contain the spinor structure and, eventually, the 
dependence on the polarization vectors.

If \mb{SetStandardMatrixElements} has been used to define abbreviations 
for the non commutative and/or scalar-product structure, the arguments of 
\mb{StandardMatrixElement} are these abbreviations itself. 
Without invoking \mb{SetStandardMatrixElements} the arguments of 
\mb{StandardMatrixElement} contain the basis for the amplitude,
i.e., the  non commutative structure and/or scalar-product structure. 
}
\fnotes{
\seepage
}

\ffinish

\fname{SU3Delta}
\fheading{\mb{SU3Delta[{\sl a},\ {\sl b}]}}
\fusage{\mb{SU3Delta[{\sl a},\ {\sl b}]} is the Kronecker $\delta$ with color indices 
$a$ and $b$.
}
\fnotes{
{\tiny $\blacksquare$}\  \mb{SU3Delta[i,\ i]} \ra 8.

\seepage {\tiny $\blacksquare$}\  See also: \mb{SU3F}, \mb{GellMannMatrix}, \mb{GellMannTrace}.
}

\ffinish

\fname{SU3F}
\fheading{\mb{SU3F[{\sl a},\ {\sl b},\ {\sl c}]}}
\fusage{\mb{SU3F[{\sl a},\ {\sl b},\ {\sl c}]} are the structure constants $f_{abc}$ of SU(3).}
\fnotes{
{\tiny $\blacksquare$}\  Only algebraic properties are implemented. {\tiny $\blacksquare$}\  The following option can be given:
\ftabthree{
SU3FToTraces & \mb{True} & replace $f_{abc}$ by 
$\frac{i}{4}\,(tr(\lambda_a \lambda_c\lambda_b) - 
tr(\lambda_a \lambda_b \lambda_c))$.
}

\seepage {\tiny $\blacksquare$}\  See also: \mb{SU3Delta}, \mb{GellMannMatrix}, \mb{GellMannTrace}.
}

\ffinish

\fname{SU3FToTraces}
\fheading{\mb{SU3FToTraces}}
\fusage{\mb{SU3FToTraces} is an option for \mb{SU3F}.
}
\fnotes{
\seepage
}

\ffinish

\fname{Tr}
\fheading{\mb{Tr[{\sl expr}]}} 
\fusage{\mb{Tr[{\sl expr}]} calculates the Dirac trace of $expr$ directly.
\mb{Tr} is identical to \mb{DiracTrace} up to the default setting
of \mb{DiracTraceEvaluate}.}
\fnotes{
{\tiny $\blacksquare$}\  The following options can be given:
\ftabthree
{
\mb{DiracTraceEvaluate}  &  \mb{True}  & evaluating the trace \cr
\mb{LeviCivitaSign}      &  \mb{-1}     & sign convention for the
$\varepsilon$-tensor \cr
\mb{Factoring}           &  \mb{False}  & factor the result \cr
\mb{Mandelstam}          &  \mb{\{\}}          &
if set to \mb{\{{\sl s},\ {\sl t},\ {\sl u},\ $\hbox{\sl m}_{1}$\phat 2\ +\ $\hbox{\sl m}_{2}$\phat 2\ +\ $\hbox{\sl m}_{3}$\phat 2\ +\ $\hbox{\sl m}_{4}$\phat 2\}},
\mb{TrickMandelstam} will be used \cr
\mb{PairCollect}         &  \mb{True}   & collect the results 
with respect to products of \mb{Pair}
} {\tiny $\blacksquare$}\  See also: \mb{DiracTrace}.

\seepage
}

\ffinish

\fname{TrickMandelstam}
\fheading{\mb{TrickMandelstam[{\sl expr},\ \{{\sl s},\ {\sl t},\ {\sl u},\ $\hbox{\sl m}_{1}$\phat 2\ +\ $\hbox{\sl m}_{2}$\phat 2\ +\ $\hbox{\sl m}_{3}$\phat 2\ +\ $\hbox{\sl m}_{4}$\phat 2\}]}}
\fusage{\mb{TrickMandelstam[{\sl expr},\ \{{\sl s},\ {\sl t},\ {\sl u},\ $\hbox{\sl m}_{1}$\phat 2\ +\ $\hbox{\sl m}_{2}$\phat 2\ +\ $\hbox{\sl m}_{3}$\phat 2\ +\ $\hbox{\sl m}_{4}$\phat 2\}]} 
simplifies all sums in {\it expr} in such a way that one of the 
Mandelstam variables $s, t$ or $u$ is eliminated by the
relation $s+t+u=m_{1}^2+m_{2}^2+m_{3}^2+m_{4}^2$.
The trick is that the resulting sum has the shortest number of terms. 
}
\fnotes{
{\tiny $\blacksquare$}\  Example: \mb{TrickMandelstam[\ s\ +\ t\ +\ u,\ \{s,t,u,2\ mw\phat 2\}\ ]} \ra 
\mb{2\ mw\phat 2}.

\seepage
 }

\ffinish

%\fname{VectorBosonPolarization} 
%\fheading{
%\mb{VectorBosonPolarization}}
%\fusage{\mb{VectorBosonPolarization} is an option for 
%%\mb{OneLoop} and 
%\mb{SquareAmplitude}. For a process with four external momenta all polarization
%vectors of vector bosons can be expressed by the four-momenta.
%}
%\fnotes{
%Possible settings are:
%\ftabtwo
%{
%\mb{\{$\hbox{\sl p}_{i}$,\ "||",\ $\hbox{\sl p}_{1}$,\ $\hbox{\sl p}_{2}$,\ $\hbox{\sl p}_{3}$,\ $\hbox{\sl p}_{4}$\}}  &  parallel polarization for 
%$\varepsilon(p_i)$ \cr 
%\mb{\{$\hbox{\sl p}_{i}$,\ "O",\ $\hbox{\sl p}_{1}$,\ $\hbox{\sl p}_{2}$,\ $\hbox{\sl p}_{3}$,\ $\hbox{\sl p}_{4}$\}}  &  orthogonal polarization for 
%$\varepsilon(p_i)$ \cr 
%\mb{\{$\hbox{\sl p}_{i}$,\ "L",\ $\hbox{\sl p}_{1}$,\ $\hbox{\sl p}_{2}$,\ $\hbox{\sl p}_{3}$,\ $\hbox{\sl p}_{4}$\}}  &  longitudinal polarization for 
%$\varepsilon(p_i)$ \cr 
%\mb{\{$\hbox{\sl p}_{i}$,\ "+",\ $\hbox{\sl p}_{1}$,\ $\hbox{\sl p}_{2}$,\ $\hbox{\sl p}_{3}$,\ $\hbox{\sl p}_{4}$\}}  &  helicity +1 polarization for 
%$\varepsilon(p_i)$ \cr 
%\mb{\{$\hbox{\sl p}_{i}$,\ "-",\ $\hbox{\sl p}_{1}$,\ $\hbox{\sl p}_{2}$,\ $\hbox{\sl p}_{3}$,\ $\hbox{\sl p}_{4}$\}}  &  helicity -1 polarization for 
%$\varepsilon(p_i)$
%}
%with $\sum p_i = 0$, $p_1$ and $p_2$ incoming and $p_3$ and $p_4$ outgoing particles.
%}
%
%\ffinish

\fname{Write2}
\fheading{\mb{Write2[{\sl channel},\ $\hbox{\sl val}_{1}$\ =\ $\hbox{\sl expr}_{1}$,\ $\hbox{\sl val}_{2}$\ =\ $\hbox{\sl expr}_{2}$,\ ...]}}
\fusage{\mb{Write2[{\sl channel},\ $\hbox{\sl val}_{1}$\ =\ $\hbox{\sl expr}_{1}$,\ $\hbox{\sl val}_{2}$\ =\ $\hbox{\sl expr}_{2}$,\ ...]}
writes the settings $val_1 = expr_1$, $val_2 = expr_2$ in sequence   
followed by a newline, to the specified output channel.
}

\fnotes{
{\tiny $\blacksquare$}\  The following options can be given:
\ftabthree{
\mb{FormatType} & \mb{InputForm} & in which language to write out to the result 
file \cr
\mb{DOConvention} & \mb{0} & convention for scalar Passarino Veltman integrals
}
Other possible settings for \mb{FormatForm} are 
\mb{FortranForm}, \mb{MacsymaForm} and
\mb{MapleForm}.  {\tiny $\blacksquare$}\  Be careful on the ouput when generating Fortran files.
There might be problems like powers of ratios of integers which you 
have to correct by hand. {\tiny $\blacksquare$}\  If the expressions contain variables in \mb{HoldForm}, their values are 
written out first, if the output language is \mma or Fortran.

\seepage
}

\ffinish 

\fname{WriteOut}
\fheading{\mb{WriteOut}}
\fusage{\mb{WriteOut} is an option for \mb{OneLoop},
\mb{OneLoopSum}.
}
\fnotes{
{\tiny $\blacksquare$}\  Possible settings are:
\ftabtwo{
\mb{True} &  write output into a file $name$, where $name$ is the first argument 
of \mb{OneLoop}\cr
\mb{False} &  write no output \cr
\mb{"/usr/hep/weinberg/"} &  write result files in your local directory 
}

\seepage
}

\ffinish

\fname{WriteOutPaVe}
\fheading{\mb{WriteOutPaVe}}
\fusage{\mb{WriteOutPave} is an option for \mb{OneLoopSum}. 
If set to a string the reduced \mb{PaVe} are written into the file
indicated.}
\fnotes{
\seepage
}

\ffinish


\fname{\$BreitMaison}
\fheading{\mb{\$BreitMaison}}
\fusage{\mb{\$BreitMaison} is a global variable determining whether
the Breitenlohner-Maison scheme is used. 
}
\fnotes{
{\tiny $\blacksquare$}\  The default setting is \mb{False}, which implies the ''naive'' 
$\gamma_5$ prescription: $\gamma_5$ is assumed to anticommute with
Dirac matrices in all dimensions. {\tiny $\blacksquare$}\  If \mb{\$BreitMaison} is set to \mb{True}, $\gamma_5$ will anticommute 
with the four-dimensional part of a Dirac matrix and commute with 
the (D-4)-dimensional part. Reset \mb{\$PrePrint} for experimenting 
with the Breitenlohner-Maison scheme.
\mb{\$BreitMaison} must be set to \mb{True} before loading \fc.  The command is : \mb{FeynCalc`\$BreitMaison\ =\ True}. {\tiny $\blacksquare$}\  Not every operation has been 
tested thoroughly, therefore beware!

\seepage
}

\ffinish

\fname{\$MemoryAvailable}
\fheading{\mb{\$MemoryAvailable}}
\fusage{\mb{\$MemoryAvailable} is a global variable which is set to an integer 
$n$, where $n$ is the available amount of main memory in MB. 
The default is 8. It should be increased if possible. 
The higher \mb{\$MemoryAvailable} can be, 
the more intermediate steps do not have to be repeated by \fc.
}
\fnotes{
\seepage
}

\ffinish

\fname{\$VeryVerbose}
\fheading{\mb{\$VeryVerbose}}
\fusage{\mb{\$VeryVerbose} is a global variable with default setting 
\mb{0}. If set to \mb{1,\ 2,\ ...}, more and more 
intermediate comments and informations
are displayed during calculations.
}
\fnotes{
\seepage
}

\ffinish

